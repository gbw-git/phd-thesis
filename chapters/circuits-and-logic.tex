%!TEX root = ../main/thesis.tex
\documentclass[../main/thesis.tex]{subfiles}
\begin{document}
We have thus far shown how to translate formulas of $\FPR$ into equivalent
families of circuits. In this section we show how to translate families of
circuits into equivalent formulas -- completing the proof of the main theorem of
this paper. More explicitly, in this section we show how to construct for every
$P$-uniform family of transparent symmetric rank-circuits a corresponding $\FPR$
formula defining the same query. In Section~\ref{sec:evaluating-circuits} we
develop the theory needed for this construction and in
Section~\ref{sec:translating-formulas-to-FPR} we use this theory to explicitly
construct the formula.

\subsection {Evaluating Circuits}\label{sec:evaluating-circuits}
In this subsection we let $\mathcal{C} = (C_n)_{n \in \mathbb{N}}$ denote a
fixed family of polynomial-size symmetric $(\BB,\rho)$-circuits with unique
labels that compute a $q$-ary query. Let $n_0$ and $k$ be the constants in the
statement of Lemma~\ref{lem:row-column-supports}, and fix an $n > n_0$ and a
$\rho$-structure $\mathcal{A}$ with universe $U$ of size $n$. Let $C_n = \langle
G, \Omega, \Sigma, \Lambda, L \rangle$.

We first introduce some notation. We are often interested in the case where two
injections can be combined into a single injection over their combined domains.
We say that two injections $f$ and $q$ are \emph{compatible} if we can define an
injection over the union of their domains that agrees with both $f$ and $q$ on
their respective domains. We formalise this notion in the following definition.

\begin{definition}
	Let $f \in Y^{\underline{X}}$ and $q : Z^{\underline{W}}$. We say that $f$ is
  \emph{compatible} with $q$, and we write $f \sim q$, if for all $a \in X \cap
  W$, $f(a) = q(a)$ and for all $a \in X \setminus W$ and $b \in W \setminus X$,
  $f(a) \neq q(b)$.
\end{definition}

It is also useful to have some notation for combining two compatible functions.
Let $f \in Y^{\underline{X}}$ and $q : Z^{\underline{W}}$ be compatible
injections. Define the combination $(f | q): X \cup W \rightarrow Y \cup Z$ by

\begin{align*}
	(f \vert q) (x) =
	\begin{cases}
    f (x) & x \in X  \\
    q (x) & x \in Y.
	\end{cases}
\end{align*}

We also introduce some other notation for combining functions. Let $f: X
\rightarrow Y$ be an injection and $q: X \rightarrow Z$ be a function. We let
$q_f : \img(f) \rightarrow Z$ be defined by $q_f(a) := q \circ f^{-1}(a)$ for
all $a \in \img(f)$.


Let $\gamma \in [n]^{\underline{U}}$. We first show that the evaluation of a
gate $g$ in $C_n$ depends only on those elements $\gamma$ maps to $\consp(g)$.

\begin{lem}
	Let $g$ be a gate in $C_n$. Let $\eta \in U^{\underline{\consp(g)}}$ and
  $\gamma_1, \gamma_2 \in [n]^{\underline{U}}$ such that $\gamma^{-1}_1 \sim
  \eta$ and $\gamma^{-1}_2 \sim \eta$. Then $L^{\gamma_1 \mathcal{A}}(g)$ and
  $L^{\gamma_2 \mathcal{A}}(g)$ are isomorphic.
	\label{lem:support-determines-evaluation}
\end{lem}

\begin{proof}
	We have that there exists a unique $\pi \in \sym_n$ such that $\pi \gamma_1 =
  \gamma_2$. Moreover, since $\gamma^{-1}_1$ and $\gamma^{-1}_2$ are both
  consistent with $\eta$, it follows that $\pi$ must fix $\consp(g)$ pointwise.
  From the definition of a support, we have that $\pi g = g$, and so $L(g)$ is
  isomorphic to $\pi L(g)$. Thus there exists $\lambda \in \aut(g)$ such that
  such that $\pi L(g) = L(g) \lambda$, and so for all $x \in \ind(g)$,
	\begin{align*}
		L^{\gamma_1 \mathcal{A}}(g) (x) & = C_n[\gamma_1 \mathcal{A}](L(g)(x))                    \\
                                    & = C_n[\pi \gamma_1 \mathcal{A}][\pi L(g)(x)]            \\
                                    & = C_n[\gamma_2 \mathcal{A}][L(g)(\lambda(a,b))] \\
                                    & = L^{\gamma_2 \mathcal{A}}(g) (\lambda (a,b)),                 
	\end{align*}
	and it follows that $L^{\gamma_1 \mathcal{A}}(g)$ and $L^{\gamma_2
    \mathcal{A}}(g)$ are isomorphic.
\end{proof}

We associate with each gate $g$ in $C_n$ a set $\Gamma_g:= \{\gamma \in
[n]^{\underline{U}} : C[\gamma \mathcal{A}](g) = 1 \}$. We also associate with
each gate $g$ in $C_n$ a set $\EV_g \subseteq U^{\underline{\consp(g)}}$
consisting of all assignments to the support of $g$ for which $g$ evaluates to
true, i.e.\ $\EV_g := \{ \eta \in U^{\underline{\consp(g)}} : \exists \gamma \in
\Gamma_g \, ,\eta \sim \gamma^{-1}\}$. It follows from
Lemma~\ref{lem:support-determines-evaluation} that $\Gamma_g$ is entirely
determined by $\EV_g$. In other words, the set of all bijections that cause $g$
to evaluate to true is entirely determined by a set of assignments to the
support of $g$. It is important to note that, from the support theorem and
choice of $n$, each $\eta \in \EV_g$ has a constant-size domain, and as such
$\EV_g$ gives us a succinct way of encoding $\Gamma_g$.

In the remainder of this subsection we show how to define $\EV_g$ recursively
for each internal gate $g$ in $C_n$, i.e.\ how to define $\EV_g$ given $\{\EV_h
: h \in H_g \}$. In Section~\ref{sec:translating-formulas-to-FPR} we show that
there is a formula $\theta \in \FP^{\BB}$ implementing this recursive definition. We
then use $\theta$, and the fixed-point operator, to define an $\FPR$-formula $Q$
that evaluates the output gates of the appropriate-size circuit for a given
input structure, and hence defines the query computed by the family $(C_n)_{n
  \in \nats}$. In fact, the recursive definition of $\EV_g$ is handled
differently depending on if $g$ is a constant, relational, $\AND$, $\NAND$,
$\OR$, or $\RANK$ gate, and so we define a different $\FPR$-formula in each
case. We then use a disjunction over cases in order to define $\theta$. We note
that Anderson and Dawar~\cite{AndersonD17} define formulas of $\FPC$, and hence
$\FPR$, that handle every case except where $g$ is a rank gate. As such, we need
only consider the latter case.

The argument in~\cite{AndersonD17} works by establishing a bijection (for any
fixed $\gamma$) between the set of possible assignments to $\consp(g)$ and the
orbit of $g$. This bijection moreover takes $\EV_g$ to the collection of gates
in $\orb(g)$ that evaluate to true on input $\gamma\mathcal{A}$. This means that
in order to count the number of inputs to $g$ that evaluate to true, it is
enough to count $\EV_h$ for all children $h$ of $g$ and divide by an appropriate
factor. This has the advantage that the computation is then independent of any
choice of $\gamma$ and can be done inductively on the structure of the circuit.
This works for all symmetric functions, where $g$ can be evaluated by counting
the number of its children that evaluate to true. For matrix-symmetric gates, we
need to do more. In particular, we want to use the bijections between
$U^{\underline{\consp(h)}}$ and $\orb(h)$ to recover from $\EV_h$ information
about the matrix structure that is input to $g$. This involves a careful
analysis of how permutations that move $\consp(h)$ shift the row and column
indices in $L^{-1}(h)$. We develop this machinery next.

For the remainder of the section we fix a rank gate $g$ in $C_n$. For $\gamma
\in [n]^{\underline{U}}$ we let $L^{\gamma} = L^{\gamma \mathcal{A}}(g)$. We let
$A \times B := \ind(g)$. We informally think of $A$ as the set indexing rows and
$B$ as the set indexing columns. For each $h \in H_g$ we let $\row(h) :=
L(g)^{-1}(h)(1)$ and $\column(h) := L(g)^{-1}(h)(2)$. For $a \in \universe{g}$
we let $\consp(a) := \consp_{\spstab{g}}(a)$, $\orb(a) = \orb_{\spstab{g}}(a)$
and $\stab(a) = \stab_{\spstab{g}}(a)$.

We now describe the construction, for any $\eta \in U^{\underline{\consp(g)}}$,
of a matrix $M$ such that for any $\gamma \in [n]^{\underline{U}}$ with
$\gamma^{-1} \sim \eta$, the rank of $M$ is equal to the rank of $L^{\gamma}$.
This allows us to decide the membership of $\eta$ in $\EV_g$ where $g$ is a rank
gate, i.e.\ $\Sigma(g) = \RANK^r_p$, by defining $M$ for $\eta$, computing the
rank of $M$ over the field of characteristic $p$, and then checking if this
result is less than or equal to the threshold value $r$. We will show in the
next subsection that the construction of $M$, and hence the construction of
$\EV_g$, can be done in $\FPR$.

For the remainder of the subsection we fix $\eta \in U^{\underline{\consp(g)}}$.
For a gate $h \in H_g$, let $A_h := \{\vec{x} \in U^{\underline{\consp(h)}} :
\eta \sim \vec{x}\}$ be the set of assignments to the support of $h$ that are
compatible with $\eta$. We should also like to consider sets of assignments to
the supports of elements in the universe of $g$. For each $s \in \universe{g}$
let $A_s := \{\vec{x} \in U^{\underline{\consp(s)}} : \eta \sim \vec{x}\}$.

We now define the matrix $M$. We begin by defining the index sets for $M$. Let
$R^{\min} := \{\min (\orb(\row(h))) : h \in H_g\}$ and $C^{\min} := \{ \min
(\orb (\column(h))) : h \in H_g\}$, and let
\begin{align*}
	I := \{(i, \vec{x}): i \in R^{\min}, \vec{x} \in A_i\}, 
\end{align*}
and
\begin{align*}
	J := \{(j, \vec{y}): j \in C^{\min}, \vec{y} \in A_j\}. 
\end{align*}
The idea is that we can think of the rows and columns of $M$ as divided into
their orbits under the action of $\stab(\consp(g))$, with each orbit indexed by
the minimal index in $A$ (or $B$, respectively) that appears in it. Within an
orbit, we index the rows and colums instead by the elements of $A_i$ and $A_j$,
implicitly using the bijection between these sets and the orbits of $\row(h)$
and $\column(h)$.

We should like to associate with each index $((i, \vec{x}), (j, \vec{y}))$ a
gate $h$, and then somehow use the assignments $\vec{x}$ and $\vec{y}$ to
construct assignments to the row and column supports of $h$. The matrix at this
index is one if, and only if, the combination of these assignments causes $h$ to
evaluate to true. It remains to show how to select an appropriate gate $h$ and
ensure that the constructed assignments to its row and column supports are
consistent. We do this by shifting the domain of $\vec{y}$ such that the
compatibility conditions are met, and then finding a gate $h$ with row and
column supports equal to domains of the assignments $\vec{x}$ and the shifted
version of $\vec{y}$, respectively. We formalise this reasoning in
Lemma~\ref{lem:permutation-row-column}.

\begin{lem}
	\label{lem:permutation-row-column}
	For any $(i, \vec{x}) \in I$ and $(j, \vec{y}) \in J$, if $n > \max{n_0, 2k}$
  then there exists $\vec{c} \in [n]^{\underline{\consp(j)}}$ such that
  $\vec{y}_{\vec{c}} \sim \eta$ and $\vec{x} \sim \vec{y}_{\vec{c}}$.
\end{lem}

\begin{proof}
  Let us first speak informally about what is required in the definition of
  $\vec{c}$. The idea is to define $\vec{c}$ so that it shifts the domain of
  $\vec{y}$ such that $\vec{y}_{\vec{c}}$ is compatible with $\eta$ and
  $\vec{x}$. As such, we need to define $\vec{c}$ such that, first,
  $\vec{y}_{\vec{c}}$ agrees with $\eta$ on $\consp(g)$, second,
  $\vec{y}_{\vec{c}}$ and $\vec{x}$ agree on the intersection of $\consp(i)$ and
  the image of $\vec{c}$, and, third, elements outside the domain of $\eta$ and
  $\vec{x}$ are mapped by $\vec{y}_{\vec{c}}$ to elements outside of their
  combined range. In order to satisfy the first requirement we define $\vec{c}$
  to be the identity on $\consp(g)$. In order to satisfy the second requirement,
  we define $\vec{c}$ such that whenever $\vec{x}$ and $\vec{y}$ agree, i.e.\
  there exists $b \in \consp(i)$, $a \in \consp(j)$ such that $\vec{x}(b) =
  \vec{y}(a)$, then $\vec{c}$ `moves' the domain to be in accord with this
  agreement, i.e.\ $\vec{c}(a) = b = \vec{x}^{-1}(\vec{y}(a))$. In order to
  satisfy the third requirement we define $\vec{c}$ such that if $x \in
  \img(\vec{y})$ and $x$ is not in the image of either $\vec{x}$ or $\vec{\eta}$
  then $\vec{c}$ `moves' $\vec{y}^{-1}(x)$ outside of the domain of either
  $\vec{x}$ or $\eta$.

  We now define $\vec{c} \in [n]^{\underline{\consp(j)}}$ formally. Let $u_1,
  \ldots , u_k$ be the first $k$ elements of $[n] \setminus (\consp(g) \cup
  \consp(i))$. Let $T = \{a \in \consp(j) \setminus \consp(g) : \vec{y}(a) \in
  \vec{x}(\consp(i) \setminus \consp(g))\}$, and let $\vec{c} \in
  [n]^{\underline{\consp(j)}}$ be defined by
	\[
		\vec{c} (a) =
		\begin{cases}
			a                        & a \in \consp(g) \cap \consp(j)                                     \\
			\vec{x}^{-1}(\vec{y}(a)) & a \in T                                                            \\
			u_{i} & \text{$a$ is the $i$th element of $(\consp(j) \setminus (\consp(g)
        \cup T)))$}.
		\end{cases}
	\]
	It is easy to see that $\vec{y}_{\vec{c}} \sim \eta$. It remains to show that
  $\vec{x} \sim \vec{y}_{\vec{c}}$. Let $c = \img(\vec{c})$. Let $z \in
  \consp(i) \cap c$ and $z' = \vec{c}^{-1}(z)$. Suppose $z' \in \consp(g)\cap
  \consp(j) \cap \consp(i)$. Then we have $\vec{x}(z) = \vec{x}(z') = \eta (z')
  = \vec{y}(z') = \vec{y}_{\vec{c}}(z)$. By a similar argument, if $z' \in
  \consp(g) \cap \consp(j) \setminus \consp(i)$ we have $\vec{x}(z) \neq
  \vec{y}_{\vec{c}}(z)$. Suppose $z' \in T$. Then $\vec{c}(z') =
  \vec{x}^{-1}(\vec{y}(z'))$ which gives us that $\vec{y}_{\vec{c}}(z) =
  \vec{y}(z') = \vec{x}(\vec{c}(z')) = \vec{x}(z)$. We finally note that $z'
  \notin (\consp(j) \setminus (\consp(g) \cup T))$ as $\{u_1 , \ldots , u_k\}
  \cap \consp(i) = \emptyset$. This completes the intersection component of
  compatibility.
		
	Let $z \in \consp(i) \setminus c$ and $w \in c \setminus \consp(i)$. Let $w' =
  \vec{c}^{-1}(w)$. Suppose $w' \in \consp(g) \cap \consp(j)$. Then from the
  fact that $w \notin \consp(i)$ we have that $\vec{y}_{\vec{c}}(w) \neq
  \vec{x}(z)$ (using a similar argument as in the above case). Suppose $w' \in
  T$. Then there exists $b \in \consp(i) \setminus \consp(g)$ such that
  $\vec{y}(w') = \vec{x}(b)$. It follows that $w = \vec{c}(w') =
  \vec{x}^{-1}\vec{y} (w') = b$, which is a contradiction as $w \notin
  \consp(i)$ by assumption, and we conclude $w' \notin T$. Suppose finally that
  $w' \in \consp(j) \setminus (\consp(g) \cup T)$. It follows that for all $b
  \in \consp(i) \setminus \consp(g)$ we have that $\vec{x}(b) \neq
  \vec{y}_{\vec{c}}(w)$. It remains to check the result for $b \in (\consp(i)
  \setminus c) \setminus (\consp(i) \setminus \consp(g)) = (\consp(g) \cap
  \consp(i)) \setminus c$. But then $b \notin \consp(j)$, and so $\vec{x}(b) =
  \eta (b)$. But $w' \notin \consp(g)$, and so $\vec{x}(b) = \eta(b) \neq
  \vec{y}(w') = \vec{y}(w)$. The result follows.
\end{proof}

For any $(i, \vec{x}) \in I$ and $(j, \vec{y}) \in J$, let $\vec{c}$ be the
function from Lemma~\ref{lem:permutation-row-column} and let $\sigma_{c} \in
\spstab{g}$ be any permutation such that for all $a \in \consp(j)$, $\sigma_{c}
a = \vec{c}(a)$. We have from Lemma~\ref{lem:support-determine-action} that the
action of $\sigma_{c}$ on $j$ is well defined. Let $h := L(g)(i, \sigma_c j)$.
Let $\vec{w}$ be the restriction of $(\vec{x} \vert \vec{y}_{\vec{c}})$ to the
support of $h$. We define the matrix $M : I \times J \rightarrow \{0,1\}$ by

\begin{align*}
	M((i , \vec{x}), (j, \vec{y})) := \vec{w} \in \EV_h. 
\end{align*}

Note that we could equivalently define $\vec{w}$ as that element of $A_h$ such
that $\vec{w}$ is compatible with $\vec{x}$ and $\vec{y}_{\vec{c}}$. We make use
of this observation in the next subsection.

Let $f \in U^{\underline{\consp(x)}}$ and $\gamma \in [n]^{\underline{U}}$, then
we let $\Pi^{\gamma}_{f}$ denote any permutation in $\spstab{g}$ such that
$\Pi^{\gamma}_f (a) = \gamma (f(a))$ for all $a \in \consp(x)$. Note that from
Lemma~\ref{lem:support-determine-action}, $\Pi^{\gamma}_f(x)$ is well-defined
independently of the particular choice of permutation.

We also note that, for a fixed gate $h \in H_g$, the mapping $\vec{z} \mapsto
\Pi^{\gamma}_{\vec{z}}(h)$, for $\vec{z} \in A_h$, allows us to establish a
correspondence between compatible assignments to the support of $h$ and the
orbit of $h$. Lemma~\ref{lem:translate-EV-circuits} formalises this observation.

\begin{lem}
	Let $\gamma\in [n]^{\underline{U}}$ and $h \in C_n$. Then $\vec{z} \in \EV_h$
  if, and only if, $C_n[\gamma \mathcal{A}](\Pi^{\gamma}_{\vec{z}} (h)) = 1$.
  \label{lem:translate-EV-circuits}
\end{lem}
\begin{proof}
  From the definition of $\EV_h$, $\vec{z} \in \EV_h$ if, and only if, there
  exists $\delta \in [n]^{\underline{U}}$ such that $\delta^{-1} \sim \eta$,
  $\delta^{-1} \sim \vec{z}$, and $C_n[\delta \mathcal{A}](h) = 1$.

  We now make a few observations. First, we have $((\Pi^{\gamma}_{\vec{z}})^{-1}
  \gamma)^{-1} = \gamma^{-1}\Pi^{\gamma}_{\vec{z}} \sim \eta$, as
  $\Pi^{\gamma}_{\vec{z}} \in \spstab{g}$. Second, we have that for any $a \in
  \consp(h)$, $\Pi^{\gamma}_{\vec{z}}(a) = \gamma (\vec{z}(a))$ and so
  $\gamma^{-1}\Pi^{\gamma}_{\vec{z}} = ((\Pi^{\gamma}_{\vec{z}})^{-1}
  \gamma)^{-1} \sim \vec{z}$. Third, we have from
  Lemma~\ref{lem:support-determines-evaluation} that for any $\delta \in
  [n]^{\underline{U}}$ such that $\delta^{-1} \sim \eta$ and $\delta^{-1} \sim
  \vec{z}$, $C_n[\delta \mathcal{A}](h) =
  C_n[((\Pi^{\gamma}_{\vec{z}})^{-1}\gamma) \mathcal{A}](h) = C_n[\gamma
  \mathcal{A}](\Pi^{\gamma}_{\vec{z}}h)$.

  Putting these observations together we have that $\vec{z} \in EV_h$ if, and
  only if, there exists $\delta \in [n]^{\underline{U}}$ such that $\delta^{-1}
  \sim \eta$, $\delta^{-1} \sim \vec{z}$ and $C_n[\delta \mathcal{A}](h) = 1$,
  if, and only if, $C_n[\gamma \mathcal{A}](\Pi^{\gamma}_{\vec{z}}h) = 1$.
\end{proof}

Let $\gamma \in [n]^{\underline{U}}$ such that $\eta \sim \gamma^{-1}$. Let
$\alpha^{\gamma}: I \rightarrow A$ and $\beta^{\gamma}: J \rightarrow B$ be
defined by $\alpha^{\gamma} (i, \vec{x}) := \Pi^{\gamma}_{\vec{x}}(i)$ and
$\beta^{\gamma} (j, \vec{y}) := \Pi^{\gamma}_{\vec{y}}(j)$, respectively. We now
prove a number of technical lemmas that ultimately allow us to to prove that the
matrix $M$ quotiented by an appropriate equivalence relation is isomorphic to
$L^{\gamma}$. We use $\alpha^{\gamma}$ and $\beta^{\gamma}$ to construct the
isomorphism. We first show that $\alpha^{\gamma}$ and $\beta^{\gamma}$ are
surjective.

\begin{lem} 
	For any bijection $\gamma \in [n]^{\underline{U}}$ such that $\gamma^{-1} \sim
  \eta$ both $\alpha^{\gamma}$ and $\beta^{\gamma}$ are surjective.
  \label{lem:alpha-beta-surjective}
\end{lem}
\begin{proof}
	We show that $\alpha^{\gamma}$ is surjective, with the same result for
  $\beta^{\gamma}$ following similarly. Let $q \in A$ and let $i = \min
  (\orb_{g} (q))$. Then there exists $\sigma \in \spstab{g}$ such that $\sigma
  (i) = q$. Let $\vec{x} := \gamma^{-1} (\sigma (\id_{\consp(i)}))$. Notice that
  for $a \in \consp(i)$ we have that $\vec{x}(a) = \gamma^{-1} (\sigma (a))$,
  and since $\gamma^{-1} \sigma \sim \eta$, it follows that $\vec{x} \in A_i$.
		
	For $a \in \consp(i)$ we have $\Pi^{\gamma}_{\vec{x}} (a) = \gamma
  (\vec{x}(a)) = \gamma (\gamma^{-1} (\sigma (a))) = \sigma (a)$. From
  Lemma~\ref{lem:support-determine-action} it follows that $\alpha(i, \vec{x}) =
  \sigma(i) = q$.
\end{proof}

The following lemma allows us to factor a permutation through $\alpha^{\gamma}$
and $\beta^{\gamma}$.

\begin{lem}
  \label{lem:alpha-and-gamma}
	Let $(i,\vec{x}) \in I$ and $(j, \vec{y}) \in J$. Let $\gamma \in
  [n]^{\underline{U}}$ be a bijection such that $\gamma^{-1} \sim \eta$ and $\pi
  \in \spstab{g}$. Then $\pi \alpha^{\gamma}(i, \vec{x}) = \alpha^{\pi
    \gamma}(i, \vec{x})$ and $\pi \beta^{\gamma}(j, \vec{y}) = \beta^{\pi
    \gamma}(j, \vec{y})$.
\end{lem}
\begin{proof}
	We have that $\pi \alpha^{\gamma}(i, \vec{x}) = \pi \Pi^{\gamma}_{\vec{x}}(i)$
  and $(\pi \Pi^{\gamma}_{\vec{x}}(\id_{\consp(i)}) = \pi \cdot \gamma (\vec{x})
  = \Pi^{\pi \gamma}_{\vec{x}}(\id_{\consp(i)})$. Since $\Pi^{\gamma}_{\vec{x}}$
  and $\Pi^{\pi \gamma}_{\vec{x}}$ are in $\spstab{g}$, it follows from Lemma
  ~\ref{lem:support-determine-action}, that $\pi \alpha^{\gamma}(i, \vec{x}) =
  \pi \Pi^{\gamma}_{\vec{x}} (i) = \Pi^{\pi \gamma}_{\vec{x}}(i) = \alpha^{\pi
    \gamma}(i, \vec{x})$. Similarly, $\pi \beta^{\gamma}(j, \vec{y}) =
  \beta^{\pi \gamma} (j, \vec{y})$.
\end{proof}

The following result applies Lemma~\ref{lem:alpha-and-gamma} to the evaluation
of gates in the circuit.

\begin{lem}
	\label{lem:alpha-ind-gamma}
	Let $(i,\vec{x}) \in I$ and $(j, \vec{y}) \in J$. Let $\gamma_1, \gamma_2 \in
  [n]^{\underline{U}}$ be such that $\gamma^{-1}_1 \sim \eta$ and $\gamma^{-1}_2
  \sim \eta$. Let $\mathcal{A}$ be a structure. Then $C_n[\gamma_1 \mathcal{A}]
  (L(\alpha^{\gamma_1}(i, \vec{x}), \beta^{\gamma_1}(j, \vec{y}))) =
  C_n[\gamma_2 \mathcal{A}] (L(\alpha^{\gamma_2}(i, \vec{x}),
  \beta^{\gamma_2}(j, \vec{y})))$.
\end{lem}
\begin{proof}
	We note that there exists $\pi \in \sym_n$ such that $\gamma_1 = \pi
  \gamma_2$. Moreover, since $\gamma^{-1}_1$ and $\gamma^{-1}_2$ are both
  consistent with $\eta$, it follows that $\pi \in \spstab{g}$. We then have
  that
	\begin{align*}
		C_n[\gamma_1 \mathcal{A}](L(\alpha^{\gamma_1}(i, \vec{x}), \beta^{\gamma_1}(j,
		\vec{y})) & = C_n[\pi \gamma_1 \mathcal{A}](\pi L(\alpha^{\gamma_1}(i, \vec{x}), 
                \beta^{\gamma_1}(j, \vec{y})) \\
		          & = C_n[\pi \gamma_1 \mathcal{A}](L(\pi                                
                \alpha^{\gamma_1}(i, \vec{x}), \pi \beta^{\gamma_1}(j, \vec{y}))\\
		          & = C_n[\pi                                                            
                \gamma_1 \mathcal{A}](L(\alpha^{\pi \gamma_1}(i, \vec{x}), \pi \beta^{\pi
                \gamma_1}(j, \vec{y})\\
		          & = C_n[\gamma_2 \mathcal] (L(\alpha^{\gamma_2}(i,                     
                \vec{x}), \beta^{\gamma_2}(j, \vec{y})))\\
	\end{align*}The third equality follows from Lemma \ref{lem:alpha-and-gamma}.
\end{proof}

In the definition of the matrix $M$, we define, for a given $(i, \vec{x}) \in I$
and $(j, \vec{y}) \in J$, a gate $h \in H_g$ and an assignment to its support.
We use this gate-assignment pair to define the value of the matrix at the point
$((i, \vec{x}), (j, \vec{y}))$. The following result is used to show that
$\alpha$ and $\beta$ can be used to determine the gate $h$, an important step in
developing the required isomorphism.

\begin{lem}
  \label{lem:defining-h-from-IJ}
  Let $i \in A$, $j \in B$, $\vec{x} \in A_i$, $\vec{y} \in B_j$. Let $\sigma_1,
  \sigma_2 \in \spstab{g}$, $\vec{s}_1 \in [n]^{\underline(\consp(i))}$, and
  $\vec{s}_2 \in [n]^{\underline{\consp(j)}}$ be such that $\vec{s}_1$ and
  $\vec{s}_2$ are the restrictions of $\sigma_1$ and $\sigma_2$ to $\consp(i)$
  and $\consp(j)$, respectively, and $\vec{x}_{s_1}$ and $\vec{y}_{s_2}$ are
  compatible. Let $h := L(g)(\sigma_1 \cdot i, \sigma_2 \cdot j)$. Let $\vec{z}
  \in A_h$ be the restriction of $(\vec{x}_{\vec{s}_1} \vert
  \vec{y}_{\vec{s}_2})$ to $\consp(h)$ and $\gamma' :=
  (\Pi^{\gamma}_{\vec{z}})^{-1} \gamma$. Then $\Pi^{\gamma'}_{\vec{x}}(i) =
  \sigma_1 \cdot i = \row (h)$ and $\Pi^{\gamma'}_{\vec{y}}(j) = \sigma_2 \cdot
  j = \column (h)$.
\end{lem}
\begin{proof}

  We prove the result for the $\sigma_1$ case. The other case follows similarly.
  We have from Lemma \ref{lem:support-determine-action} that it is sufficient to
  show that for all $a \in \consp(i)$, $\Pi^{\gamma'}_{\vec{x}} (a) = \sigma_1
  (a)$. Let $a \in \consp(i)$. We have from Lemmas~\ref{lem:support-containment}
  and~\ref{lem:support-mapping} that $\sigma_1 (a) \in \consp(\row(h)) \subseteq
  \consp(\row(h))\cup \consp(\column(h)) = \consp(g) \cup \consp(h)$, and thus
  $\sigma_1(a) \in \consp(g)$ or $\sigma_1(a) \in \consp(h)$.
  
  If $\sigma_1(a) \in \consp(g)$, then, since $\Pi^{\gamma}_{\vec{z}}$ and
  $\sigma_1$ in $\spstab{g}$ and $\gamma^{-1} \sim \eta$ and hence $\gamma' \sim
  \eta$, we have that $\Pi^{\gamma'}_{\vec{x}} \in \spstab{g}$ and so
  $\sigma_1(a) = a = \Pi^{\gamma'}_{\vec{x}}(a)$.

  If $\sigma_1(a) \in \consp(h)$ then, since $\sigma_1(a) \in \consp(\row(h))$,
  $\Pi^{\gamma}_{\vec{z}}(\sigma_1(a)) = \gamma(\vec{z}(\sigma_1(a))) =
  \gamma(\vec{x}_{\vec{s}_1}(\sigma_1(a))) = \gamma (\vec{x}(a))$. Thus
  $\sigma_1 (a) = (\Pi^{\gamma}_{\vec{z}})^{-1}\gamma (\vec{x}(a)) =
  \Pi^{\gamma'}_{\vec{x}}(a)$. The result follows.
\end{proof}

We combine the above lemmas in order to show that $\alpha^{\gamma}$ and
$\beta^{\gamma}$ together define a surjective homomorphism from $M$ to
$L^{\gamma}$.

\begin{thm}
  If $(i, \vec{x})\in I$; $(j, \vec{y})\in J$, and $\gamma\in
  [n]^{\underline{U}}$ is such that $ \gamma^{-1} \sim \eta$, then
  $M((i,\vec{x}), (j, \vec{y})) = L^{\gamma}(\alpha^{\gamma}(i, \vec{x}),
  \beta^{\gamma}(j, \vec{y}))$.
	\label{lem:ML-equal-elements}
\end{thm}
\begin{proof}
	Let $\vec{c}$ be the vector defined in Lemma \ref{lem:permutation-row-column}
  and $\sigma_c$ be any permutation in $\spstab{g}$ such that the restriction of
  $\sigma_c$ to $\consp(j)$ equals $\vec{c}$.
  
  Let $\vec{w}$ be the restriction of $(\vec{x} \vert \vec{y}_{\vec{c}})$ to
  $\consp(h)$, let $\gamma' = (\Pi^{\gamma}_{\vec{w}})^{-1} \gamma$, and let $h
  = L(g)(i, \sigma_c(j))$. Then
	\begin{align*}
		M((i, \vec{x}), (j, \vec{y}))
    & = \vec{w} \in \EV_h                                    \\
    & = C_n[\gamma \mathcal{A}] (\Pi^{\gamma}_{\vec{w}} (h)) \\
    & = C_n[\gamma' \mathcal{A}] (h)                                                             \\
    & = C_n[\gamma' \mathcal{A}](L(\row(h), \column (h)))                                        \\
    & = C_n[\gamma' \mathcal{A}](L(\alpha^{\gamma'}(i, \vec{x}), \beta^{\gamma'}(j, \vec{y})))   \\
    & = C_n[\gamma \mathcal{A}](L(\alpha^{\gamma}(i, \vec{x}), \beta^{\gamma}(j, \vec{y})))     \\
    & = L^{\gamma}(\alpha^{\gamma}(i, \vec{x}), \beta^{\gamma}(j, \vec{y}))                      
	\end{align*}
	The second equality follows from Lemma~\ref{lem:translate-EV-circuits}. The
  fifth equality follows from Lemma~\ref{lem:defining-h-from-IJ}. The sixth
  equality follows from Lemma~\ref{lem:alpha-ind-gamma}.
\end{proof}

We have shown that, but for injectivety, $\alpha^{\gamma}$ and $\beta^{\gamma}$,
witness an isomorphism between $M$ and $L^{\gamma}$. We should thus like to show
that $\alpha^{\gamma}$ and $\beta^{\gamma}$ are also injective, and hence
complete the proof. However, it can be shown that $\alpha^{\gamma}(i, \vec{x}) =
\alpha^{\gamma}(i, \vec{x}')$ if, and only if, there exists $\pi \in
\stab_{\consp(g)}(i)$ such that $\vec{x} = \vec{x}'\pi$ see
Lemma~\ref{lem:functions-mutual-equivalence}) and, as such, $\alpha^{\gamma}$
and $\beta^{\gamma}$ are not, in general, injective.

We instead define a new matrix $M_{\equiv}$ by quotienting $M$ by an appropriate
equivalence relation and show that the liftings of $\alpha^{\gamma}$ and
$\beta^{\gamma}$ to the equivalence classes of $\equiv$ witness an isomorphism
between $M_\equiv$ and $L^{\gamma}$. We first define the equivalence relation
$\equiv$.

Let $s \in \universe{g}$ and let $\vec{x}, \vec{x}' \in A_s$ and we say that
$\vec{x}$ and $\vec{x}'$ are \emph{mutually stable} on $s$ if there exists $\pi
\in \stab(s)$ such that $\vec{x}= \vec{x}'\pi$. We also say that two
permutations $\sigma, \sigma' \in \spstab{g}$ are \emph{mutually stable} on $s$
if there exists $\pi \in \stab(s)$ such that $\sigma (\id_{\consp(s)}) = \sigma'
\pi (\id_{\consp(s)}) $. We note that mutual stability is an equivalence
relation on $A_s$ (and $\spstab{g}$), and we denote the equivalence of two
vectors $\vec{x}, \vec{x}' \in A_s$ by $\vec{x} \equiv_s \vec{x}'$ and the
equivalence of two permutations $\sigma, \sigma' \in \spstab{g}$ by $\sigma
\equiv_s \sigma'$).

We now define the matrix $M_{\equiv}$ and the liftings of $\alpha^{\gamma}$ and
$\beta^{\gamma}$.

Let $I_{\equiv} := \{(i, [\vec{x})]_{\equiv_i}) : (i, \vec{x}) \in I\}$ and
$J_\equiv := \{(j, [\vec{y}]_{\equiv_j}) : (j, \vec{y})\}$. Let $M_{\equiv} :
I_{\equiv} \times J_{\equiv} \rightarrow \{0,1\}$ be defined by $M_\equiv ((i,
[\vec{x}]_\equiv), (j, [\vec{y}]_\equiv)) := M((i,\vec{x}), (j, \vec{y}))$, for
$(i, [\vec{x}]_{\equiv}) \in I_\equiv$ and $(j, [\vec{y}]_{\equiv}) \in
J_\equiv$.

We show in Lemma~\ref{lem:matrix-quot-well-defined} that this matrix is
well-defined and in Lemma~\ref{lem:alpha-beta-mutal-equivalence} that
$\alpha^{\gamma}$ and $\beta^{\gamma}$ are constant on mutual equivalence
classes, and as such may be lifted to act on equivalence classes.

We first prove the claim that those permutations that agree on $s \in
\universe{g}$ are exactly those that are mutually stable on $s$.

\begin{lem}
	Let $s \in \universe{g}$ and let $\sigma, \sigma' \in \stab(\consp(g))$. Then
  $\sigma(s) = \sigma' (s)$ if, and only if, $\sigma \equiv_s \sigma'$.
	\label{lem:functions-mutual-equivalence}
\end{lem}
\begin{proof}
	Suppose $\sigma(s) = \sigma'(s)$. Then $\pi := (\sigma')^{-1}\sigma \in
  \stab(s)$ and $\sigma = \sigma' \pi$. Suppose there exists $\pi \in \stab(s)$
  such that for all $a \in \consp(g)$, $\sigma (a) = \sigma' \pi (a)$. From
  Lemma~\ref{lem:support-determine-action} we have that $\sigma (i) = \sigma'
  \pi (i)$, and so, since $\pi \in \stab(s)$, $\sigma(i) = \sigma' \pi (i) =
  \sigma' (i)$.
\end{proof}

The following lemma gives us that $\alpha^{\gamma}$ and $\beta^{\gamma}$ are
constant on classes of mutually stable assignments.

\begin{lem}
	Let $((i, \vec{x}), (j, \vec{y})), ((i, \vec{x}'), (j, \vec{y}')) \in I \times
  J$ and let $\gamma \in [n]^{\underline{U}}$. If $\vec{x} \equiv_i \vec{x}'$
  then $\alpha^{\gamma}(i, \vec{x}) = \alpha^{\gamma}(i, \vec{x}')$ and if
  $\vec{y} \equiv_j \vec{y}'$ then $\beta^{\gamma}(j, \vec{y}) =
  \beta^{\gamma}(j, \vec{y}')$.
	\label{lem:alpha-beta-mutal-equivalence}
\end{lem}
\begin{proof}
	From mutual stability there exists $\sigma \in \stab(i)$ such that $\vec{x}' =
  \vec{x} \sigma$. We have that $\alpha^{\gamma}(i, \vec{x})$ stabilises
  $\consp(g)$. Also, for all $a \in \consp(i)$, $\Pi^{\gamma}_{\vec{x}} (\sigma
  (a)) = \gamma (\vec{x}(\sigma (a))) = \gamma (\vec{x}'(a)) =
  \Pi^{\gamma}_{\vec{x}'}(a)$. It follows from Lemma
  \ref{lem:functions-mutual-equivalence} that $\alpha^{\gamma}(i,\vec{x}) =
  \Pi^{\gamma}_{\vec{x}} (i) = \Pi^{\gamma}_{\vec{x}'}(i) = \alpha^{\gamma}(i,
  \vec{x}')$. The result follows similarly for $\beta$.
\end{proof}

We define the liftings of $\alpha^{\gamma}$ and $\beta^{\gamma}$ as follows. Let
$\alpha^{\gamma}_{\equiv} : I_\equiv \rightarrow A$ and $\beta^{\gamma}_\equiv:
J_\equiv \rightarrow B$ by $\alpha^{\gamma}_{\equiv}(i, [\vec{x}]) :=
\alpha^{\gamma} (i, \vec{x})$ and $\beta^{\gamma}_{\equiv}(j, [\vec{y}]) =
\beta^{\gamma} (j, \vec{y})$, for all $(i, \vec{x}) \in I_{\equiv}$ and $(j,
\vec{y}) \in J_\equiv$. Lemma~\ref{lem:alpha-beta-mutal-equivalence} gives us
that these liftings are well defined.

The following lemma gives us that $M_\equiv$ is well defined.

\begin{lem}
	Let $((i, \vec{x}), (j, \vec{y})), ((i, \vec{x}'), (j, \vec{y}')) \in I \times
  J$ be such that $\vec{x} \equiv_i \vec{x}'$ and $\vec{y} \equiv_j \vec{y}'$,
  then $M((i, \vec{x}), (j, \vec{y})) = M((i, \vec{x}'), (j, \vec{y}'))$.
	\label{lem:matrix-quot-well-defined}
\end{lem}
\begin{proof}
	\begin{align*}
		M((i, \vec{x}),(j, \vec{y})) & = L^{\gamma}(\alpha^{\gamma}(i, \vec{x}), \beta^{\gamma}(j, \vec{y})    \\
		                             & = L^{\gamma}(\alpha^{\gamma}(i, \vec{x}'), \beta^{\gamma}(j, \vec{y}')) \\
		                             & = M((i, \vec{x}'), (j, \vec{y}'))                                       
	\end{align*}
	The second equality follows from Lemma \ref{lem:alpha-beta-mutal-equivalence}.
\end{proof}

We now prove the required isomorphism.

\begin{thm}
	Let $\gamma \in [n]^{\underline{U}}$ such that $\gamma^{-1} \sim \eta$. Then
  $L^{\gamma}$ is isomorphic to $M_{\equiv}$.
	\label{thm:LM-equivalence}
\end{thm}
\begin{proof}
	Let $(i, [\vec{x}]) \in I_\equiv$ and $(j, [\vec{y}]) \in J_\equiv$. Then,
  from Lemmas~\ref{lem:matrix-quot-well-defined}
  and~\ref{lem:ML-equal-elements}, we have that $M_\equiv ((i, [\vec{x}]), (j,
  [\vec{y}])) = M ((i, \vec{x}), (j, \vec{y})) = L^{\gamma}(\alpha^{\gamma}(i,
  \vec{x}), \beta^{\gamma}(j, \vec{y}))$.
		
	Moreover, from Lemma \ref{lem:alpha-beta-mutal-equivalence} we can lift
  $\alpha^\gamma$ to $I_\equiv$ and $\beta^{\gamma}$ to $J_\equiv$. It remains
  to show that $\alpha^\gamma_{\equiv}$ and $\beta^{\gamma}_{\equiv}$ are
  bijections. We prove the result for $\alpha^{\gamma}_{\equiv}$, with the proof
  for $\beta^\gamma_\equiv$ following similarly.
		
	We first note that $\alpha^{\gamma}_{\equiv}$ is surjective as both
  $\alpha^{\gamma}$ (see Lemma \ref{lem:alpha-beta-surjective}) and the lifting
  function (i.e.\ the quotient map from $I$ to $I_\equiv$) are surjective.
		
	Suppose $\alpha^{\gamma}_\equiv((i, [\vec{x}])) = \alpha^{\gamma}_\equiv((i',
  [\vec{x}']))$, and so $\Pi^{\gamma}_{\vec{x}}(i) =
  \Pi^{\gamma}_{\vec{x}'}(i')$. But then $i$ and $i'$ are in the same orbit and
  so, from the definition of $I_{\equiv}$, we have $i = i'$. Thus
  $\Pi^{\gamma}_{\vec{x}}(i) = \Pi^{\gamma}_{\vec{x}'}(i)$, and so from Lemma
  \ref{lem:functions-mutual-equivalence} there exists $\sigma \in \stab(i)$ such
  that for all $a \in \consp(i)$ we have that $\Pi^{\gamma}_{\vec{x}}(a) =
  \Pi^{\gamma}_{\vec{x}'} (\sigma (a))$. But then $\Pi^{\gamma}_{\vec{x}}(a) =
  \gamma (\vec{x}(a)) = \Pi^{\gamma}_{\vec{x}'}(\sigma (a)) = \gamma (\vec{x}'
  (\sigma (a))$ and, since $\gamma$ is a bijection, it follows that $\vec{x}(a)
  = \vec{x}' \sigma (a)$. Thus $\vec{x} \equiv_i \vec{x}'$, and so $[\vec{x}] =
  [\vec{x}']$. We thus have that $\alpha^{\gamma}_\equiv$ and
  $\beta^{\gamma}_\equiv$ are injections, and the result follows.
\end{proof}

The final result in this subsection establishes that, while $M$ and $L^{\gamma}$
are not in general isomorphic, $M$, $M_\equiv$ and $L^{\gamma}$ all have the
same rank.

\begin{lem}
	Let $\gamma \in U^{\underline{n}}$ be such that $\gamma^{-1} \sim \eta$ and
  let $p \in \nats$ be prime. Then $\rank_p (M) = \rank_p (M_\equiv) = \rank_p
  (L^{\gamma})$.
  \label{lem:rank-triple-equivilence}
\end{lem}
\begin{proof}
	Let $f : \{0,1\}^{J_\equiv} \rightarrow \{0,1\}^{J}$ be defined such that for
  all $\vec{b} \in {0,1}^{J_\equiv}$, $f(\vec{b}) (j, \vec{y}) := \vec{b}(j,
  [\vec{y}])$ for all $(j, \vec{y}) \in J$. We have that $f$ is injective as,
  for $\vec{b}_1, \vec{b}_2 \in \{0,1\}^{J_\equiv}$, if $f(\vec{b}_1) =
  f(\vec{b}_2)$ then $\vec{b}_1 (j, [\vec{y}]) = f(\vec{b}_1)(j, \vec{y}) =
  f(\vec{b}_2)(j, \vec{y}) = \vec{b}_2(j, [\vec{y}])$, for all $(j, \vec{y}) \in
  J$. Also $f$ is linear as, for $\lambda \in \ff_p$ and $\vec{b}_1, \vec{b}_2
  \in \{0,1\}^{J_\equiv}$, $f(\lambda \vec{b}_1) (j, \vec{y}) = (\lambda
  \vec{b}_1)(j, [\vec{y}]) = \lambda (\vec{b}_1 (j, [\vec{y}])) = \lambda (f(
  \vec{b}_1) (j, \vec{y}))$, and $ f(\vec{b}_1 + \vec{b}_2) (j, \vec{y}) =
  (\vec{b}_1 + \vec{b}_2) (j, [\vec{y}]) = \vec{b}_1(j, [\vec{y}]) +
  \vec{b}_2(j, [\vec{y}]) = f(\vec{b}_1)(j, \vec{y}) + f(\vec{b}_2)(j,
  \vec{y})$, for all $(j, \vec{y}) \in J$.
		
	From the construction of $M_\equiv$ that $f$ maps row vectors in $M_\equiv$ to
  row vectors in $M$. Let $\vec{b}'$ be a row vector in $M$. Then there exists
  $\vec{b} \in \{0,1\}^{J_\equiv}$ defined by $\vec{b} (j, [\vec{y}]) :=
  \vec{b}' (j, \vec{y})$ (Lemma~\ref{lem:matrix-quot-well-defined} gives us that
  this function is well defined). It follows that $f(\vec{b}) = \vec{b}'$, and
  so $f$ is a surjective mapping from the row vectors of $M_\equiv$ to the row
  vectors of $M$.
		
	It follows from the fact that $f$ is an injective linear map and that $f$ maps
  rows in $M_\equiv$ to rows in $M$ that if $B$ is a maximal independent set of
  row vectors in $M_\equiv$ then $f(B)$ is an independent set of row vectors in
  $M$. We now show that in this case $f(B)$ is also maximal. Suppose $f(B)$ is
  not a maximal independent set of row vectors in $M$, then there exists a row
  vector $\vec{b}'$ in $M$ such that $\vec{b}' \notin f(B)$ and $f(B) \cup
  \{b'\}$ is independent. But since $f$ maps rows in $M_\equiv$ to rows in $M$
  surjectively, it follows that $B \cup \{f^{-1}(b') \}$ must be an independent
  set of rows in $M_\equiv$. But since $B$ is maximal, it follows that
  $f^{-1}(b') \in B$ and so $b' \in f(B)$, a contradiction. We thus have that
  $f(B)$ is a maximal independent set of rows in $M$ and, since $f$ is
  injective, we have that $\rank_p (M) = \vert f(B) \vert = \vert B \vert =
  \rank_p (M_\equiv) = \rank_p(L^{\gamma})$. (The final equality follows from
  Theorem~\ref{thm:LM-equivalence}).
\end{proof}

In the next subsection we encode the definition and evaluation of $M$ for some
gate $g$ as a formula of $\FPR$. We use this formula, and the least fixed point
operator, to define an $\FPR$ formula that evaluates the entire circuit.

\subsection{Translating to Formulas of FPR}
\label{sec:translating-formulas-to-FPR}
We let $\mathcal{C} = (C_n)_{n \in \mathbb{N}}$ denote a fixed $\PT$-uniform
family of transparent symmetric $(\RB, \rho)$-circuits defining a $q$-ary query.
Our aim in this subsection is to define an $\FPR$-formula $Q$ such that for any
$\rho$-structure $\mathcal{A}$ of size $n$, the $q$-ary query defined by $C_n$
for the input $\mathcal{A}$ is defined by $Q$ when interpreted in $\mathcal{A}$.
For the rest of this section fix $n \in \nats$ and $C_n := \langle G, \Sigma,
\Omega, \Lambda, L \rangle \in \mathcal{C}$.

We have from the Immerman-Vardi theorem~\cite{Immerman198686, Vardi:1982} that
any polynomial-time algorithm can be converted into an equivalent
$\FP[\leq]$-interpretation. Suppose $\phi'(\vec{x}_1, \ldots , \vec{x}_m)$ is an
$\FP[\leq]$-formula, where each $\vec{x}_i$ is a $t$-tuple of variables for some
$t \in \nats$. We can construct, using a simple syntactic substitution, an
equivalent $\FPC[\emptyset]$-formula $\phi(x_1, \ldots , x_m)$ such that all the
variables that appear in $\phi$ are number variables and for any structure
$\mathcal{A}$ of size $n$, for each $i \in [m]$, and $\vec{a}_i \in [n]^{t}$,
$\langle [n], \leq \rangle \models \phi'[\vec{a}_1 , \ldots, \vec{a}_m]$ if, and
only if, $\mathcal{A} \models \phi[b_1, \ldots, b_m]$, where for each $i \in
[m]$, $b_i \in [n^t]$ is such that $\vec{a}_i$ is the $b_i$th element of
$[n]^t$. It follows that if $\Phi'$ is an $\FP[\leq]$-interpretation then there
exists an equivalent $\FPC$-interpretation $\Phi$, with $\Phi$ defining the same
structure as $\Phi'$ over the number universe. Moreover, if $\Phi'$ is a
$t$-width interpretation then $\Phi$ evaluates to false for any assignment that
maps any of its free variables to values not in $[n^t]$. We say that $\Phi$ has
width $t$ if $\Phi'$ has width $t$.

Recall from Section~\ref{sec:symm-circ} that $T_{\rank} = \{\AND, \OR, \NAND,
\RANK, \MAJ\}$. It follows from the Immerman-Vardi theorem~\cite{Immerman198686,
  Vardi:1982}, the $\PT$-uniformity of $\mathcal{C}$, and
Lemma~\ref{lem:transparent-unique}, that there is an $\FPC$-interpretation $\Phi
:= (\phi_G, \phi_\Omega, (\phi_s)_{ s \in T_{\rank} \cup \rho \cup \{0,1\}},
\phi_{\rank}, (\phi_{\Lambda_R})_{R \in \rho}, \phi_L)$ that, when interpreted
in a structure $\mathcal{A}$ of size $n$, defines a symmetric rank-circuit with
unique labels equivalent to $C_n$. We abuse notation and also refer to to this
equivalent circuit as $C_n$, and let $t$ be the width of this interpretation. We
now define the formulas in the interpretation $\Phi$.

\begin{myitemize}
\item $\phi_G(\mu)$, when interpreted in $\mathcal{A}$, defines a subset of
  $[n^t]$, which we identify with $G$ (the set of gates in the circuit), and
  hence write $G \subseteq [n^t]$.
\item $\phi_{\Omega}(\nu_1, \ldots , \nu_q, \mu)$ is defined such that
  $\mathcal{A} \models \phi_\Omega[a_1, \ldots, a_q, g]$ if, and only if, $(a_1,
  \ldots, a_q) \in [n]^q$ and $\Omega(a_1, \ldots, a_q) = g$.
\item $\phi_s (\mu)$ is defined for $s \in T_{\rank} \uplus \rho \uplus \{0,1\}$
  such that $\mathcal{A} \models \phi_s [g]$ if, and only if, $g$ is an input
  gate and $\Sigma (g) = s$ or $g$ is an internal gate and $\Sigma$ maps $g$ to
  a function associated with the symbol $s$.
\item $\phi_\rank(\mu, \delta, \epsilon)$ is defined such that $\mathcal{A}
  \models \phi_{\rank}[g,r,p]$ if, and only if, there exists $a, b \in \nats$
  such that $\Sigma (g) = \RANK^r_p[a,b]$.
\item For $R \in \rho$ of arity $r$, $\phi_{\Lambda_R}(\nu_1, \ldots, \nu_r,
  \mu)$ is defined such that $\mathcal{A} \models \phi_{\Lambda_R} [a_1, \ldots,
  a_r, g]$ if, and only if, $(a_1, \ldots, a_r)\in [n]^r$ and $g$ is a
  relational gate such that $\Sigma (g) = R$ and $\Lambda_R (g) = (a_1, \ldots,
  a_r)$.
\item $\phi_L(\mu, \nu, \delta, \epsilon)$ is defined such that $\mathcal{A}
  \models \phi_L[g,h,i,j]$ if, and only if, $g$ and $h$ are gates and $L(g)(i,j)
  = h$.
\end{myitemize}

We should note that the interpretation $\Phi$ does not encode the circuit
exactly as one might expect. In particular, there is no formula in $\Phi$
corresponding to the function $\Sigma$ in $C_n$. Instead, we have included a
family of formulas $(\phi_s)_{s \in T_{\rank} \uplus \rho \uplus \{0,1\}\}}$,
where each $\phi_s$ defines the set of gates of type $s$, and the formula
$\phi_{\rank}$, which defines the the set of rank gates with a given
characteristic and threshold. It is easy to see that, along with $\phi_L$, these
formulas suffice to define $\Sigma$.
  
Let $n_0$ and $k$ be the constants in the statement of
Lemma~\ref{lem:row-column-supports}. Notice that for each $n \leq n_0$, there
are constantly many bijections from the universe of an $\rho$-structure
$\mathcal{A}$ of size $n$ to $[n]$. As such there exists an $\FPC$ formula that
evaluates $C_n$ for any $n \leq n_0$ by explicitly quantifying over all of these
constantly many bijections, and then evaluating the circuit with respect to each
bijection. As such we need only consider the case where $n > n_0$. In the rest
of this subsection we let $\mathcal{A}$ denote a $\rho$-structure of size $n$.

We should like to recursively construct $\EV_g$ for each gate $g$ in the
circuit. However, while we have from Lemma~\ref{lem:row-column-supports} that
the canonical support of $g$ has size at most $k$, it may not be exactly equal
to $k$. As such, if $\vert \consp(g) \vert = \ell$, we define

\begin{align*}
	\overline{\EV}_g = \{ (a_1, \ldots , a_k) \in [n]^k : (a_1 , \ldots , a_\ell ) \in \EV_g \text{ and } i \neq j \implies a_i \neq a_j \}. 
\end{align*}

In this subsection we use $\mu$ and $\nu$ to denote number variables that should
be assigned to gates and use $\epsilon$ and $\delta$ to denote variables that
should be assigned to elements of the universe of a gate. We use $\kappa$ and
$\pi$ to denote other number variables. We use $x, y, z, \ldots$ to denote
vertex variables and $U, V, \ldots$ to denote relational variables. We use $a,
b, c , \ldots$ to denote elements of the vertex sort. When a vector of values or
variables is used without reference to size, it is taken to be a $k$-tuple.

Let $s \subseteq [n]$, $X$ be a set, and $f \in X^{\underline{s}}$. Then, as
$[n]$, and hence $s$, is an ordered set, we let $\vec{s}$ denote the $\vert s
\vert$-tuple given by listing the elements of $s$ in order. We let $\vec{f}$
denote a $\vert s \vert$-tuple formed by applying $f$ to $s$ in order (i.e.\
$\vec{f} := f \circ \vec{s}$).

We seek to define the relation $V \subseteq [n^t] \times U^k$ in $\FPR$ such
that $V(g, \vec{a})$ if, and only if, $\vec{a} \in \overline {\EV}_g$. We first
construct an $\FPR$-formulas that recursively defines $\overline{\EV}_g$ for a
given $g$ in $C_n$. We recall that, for a given $g$ in $C_n$, the recursive
definition of $\overline{\EV}_g$ depends on the type of the gate $g$. As such,
we define for each possible type $s$ a formula $\theta_s(\mu, \vec{x})$ that
defines $\overline{\EV}_g$ for a given $s$-gate $g$. We then take a disjunction
over these formulas and, by an application of the fixed-point operator, define
the $\FPR$-formula $\theta(\mu, \vec{x})$. We show that $\theta$ defines $V$.


As mentioned above, $\theta_0$, $\theta_1$, $\theta_\AND$, $\theta_\OR$,
$\theta_\NAND$, $\theta_\MAJ$ and $(\theta_R)_{R \in \tau}$ been defined by
Anderson and Dawar~\cite{AndersonD17}. It remains for us to define an $\FPR$
formula $\theta_{\RANK}(\mu, \vec{x})$ such that, if $\mu$ denotes a
$\RANK$-gate $g$ and $\vec{x}$ denotes an assignment $\vec{a}$ to the support of
$g$, $\theta_{\RANK}(\mu, \vec{x})$ corresponds to the recursive construction of
$\EV_g$ presented in Section~\ref{sec:evaluating-circuits}. We first construct a
formula $\psi_M$ that defines the matrix $M$, as given in the previous
subsection, for a given gate and assignment to the support of that gate. We then
define $\theta_{\RANK}$, which applies the rank operator to the matrix defined
by $\psi_M$. In order to define $\psi_M$ and $\theta_\RANK$ succinctly we first
define a number of useful $\FPC$ formulas.

We define the closed number terms $\FA{m} := (\# x (x = x))*t$ and $\FA{s} := \#
x(x = x)$, denoting the bound on the domain of $\Phi$ and the size of the
structure over which we are working. We define two formulas that define the set
of elements of the first and second sort of the universe of a gate $g$ (i.e.\
the set of row and column indexes).
\begin{align*}
	\FA{row} (\mu, \delta) :=   & \exists \nu, \epsilon \leq \FA{m} \, (\Phi_L (\mu, \delta, \epsilon, \nu)) \\
	\FA{col} (\mu, \epsilon) := & \exists \nu, \delta \leq \FA{m} \, (\Phi_L (\mu, \nu, \delta , \epsilon))  
\end{align*}
From Lemma~\ref{lem:computing-support-orbit-index}, and invoking the
Immerman-Vardi theorem, there exists an $\FPC$-formula $\FA{orb}(\mu, \delta,
\epsilon)$ such that $\mathcal{A} \models \FA{orb}[g,i, i']$ if, and only if,
$i$ and $i'$ are in the universe of $g$ and $i \in \orb(i')$. The following
formula allows us to define the minimal element of this orbit
\begin{align*}
	\FA{min-orb} (\mu, \delta) := \forall \epsilon \leq \FA{m} \, (\FA{orb} (\mu, \delta, \epsilon) \implies \delta \leq \epsilon). 
\end{align*}

From~\cite{AndersonD17} there is an $\FPC$-formula $\FA{supp}$ such that
$\mathcal{A} \models \FA{supp} [g, u]$ if, and only if, $\mathcal{A} \models
\phi_G [g]$ and $u$ is in $\consp(g)$. In~\cite{AndersonD17} this formula is
used to inductively define a set of formulas $\FA{supp}_i$ for each $i \in
\nats$ such that $\mathcal{A} \models \FA{supp}_i[g, u]$ if, and only if, $u$ is
the $i$th element of $\consp(g)$.

Similarly, we can define $\FA{supp}^r$ such that $\mathcal{A} \models
\FA{supp}^r[g, a, u]$ if, and only if, $\mathcal{A} \models \phi_G [g] \land
\FA{row}[g, a]$ and $u$ is in $\consp(a)$, and $\FA{supp}^c$ such that
$\mathcal{A} \models \FA{supp}^c[g, a, u]$ if, and only if, $\mathcal{A} \models
\phi_G [g] \land \FA{col}[g, a]$ and $u$ is in $\consp(u)$. Similarly, can
define the formulas $\FA{supp}^r_i$ and $\FA{supp}^c_i$.

From~\cite{AndersonD17} we have an $\FPC$ formula $\FA{agree}(\mu, \nu, \vec{x},
\vec{y})$ such that $\mathcal{A} \models \FA{agree}[g, h, \vec{a}, \vec{b}]$ if,
and only if, $\vec{a}_{\vec{\consp}(g)}$ is compatible with
$\vec{b}_{\vec{\consp}(h)}$.

A similar argument allows us to define $\FA{agree}_L (\mu, \delta , \vec{x},
\vec{z})$ (where $\vec{z}$ is a tuple of size $2k$) such that $\mathcal{A}
\models \FA{agree}_L [g, i, \vec{a}, \vec{r}]$ if, and only if,
$\vec{a}_{\vec{\consp}(g)}$ and $\vec{r}_{\vec{\consp}(i)}$ are compatible.

It can be shown that given a number $n$, a gate $g$ in $C_n$ and two pairs $(i,
\vec{x}) \in I$ and $(j, \vec{y}) \in J$ the function $\vec{c}$ in
Lemma~\ref{lem:permutation-row-column} can be computed in time polynomial in
$n$. Thus, from the Immerman-Vardi Theorem, it can be defined by a number term.
Thus we assert that there exists an $\FPC$ formula $\FA{move} (\mu, \delta,
\epsilon, \vec{x}, \vec{y}, \vec{z}, \vec{w})$ (where $\vec{y}$ and $\vec{z}$
are $2k$-tuples) such that $\mathcal{A} \models \FA{move} [g, i, j, \vec{a},
\vec{b}, \vec{d}]$ if, and only if, $\mathcal{A} \models \phi_G [g] \land
\FA{row}_g[g,i] \land \FA{col}[g, j]$ and $\mathcal{A} \models \FA{agree}_L [g,
i, \vec{a}, \vec{b}] \land \FA{agree}_L [g, j, \vec{a}, \vec{d}]$ and
$\vec{d}_{\vec{\consp}(j)} = \vec{c}$, where $\vec{c}$ is the vector given in
Lemma~\ref{lem:permutation-row-column} defined from the pairs $(i,
\vec{a}_{\vec{\consp}(i)})$ and $(j, \vec{b}_{\vec{\consp}(j)})$.

It is possible to define a polynomial-time algorithm that takes in a gate $g$
and $j, j' \in \universe{g}$ and $\vec{c}$, a $2k$-tuple with values in $[n]$,
constructs a permutation $\sigma \in \spstab{g}$ such that $\sigma
(\vec{\consp}(j)) := \vec{c}$ and checks, using
Lemma~\ref{lem:compute-automorphisms-labels}, if $\sigma j = j'$.

It follows from the Immerman-Vardi theorem that there is an $\FPC$ formula
$\FA{map}(\mu, \delta_1, \delta_2, \vec{z})$ (where $\vec{z}$ is a $2k$-tuple)
such that $\mathcal{A} \models \FA{map}[g, j, j', \vec{c}]$ if, and only if,
$\mathcal{A} \models \phi_G [g] \land \FA{col}[g, j] \land \FA{col}[g, j']$ and
such that there exists $\sigma \in \spstab{g}$ where $\sigma (\vec{\consp}(j)) =
\vec{c}$ and $\sigma j = j'$.
				
We now define an $\FPC$-formula $\FA{merge}$ such that $\mathcal{A} \models
\FA{merge}[g, h, \vec{a}, \vec{b}, \vec{r}, \vec{c}]$ if, and only if, (i) $g$
and $h$ are gates such that $h \in H_g$, (ii) $\vec{a}$ and $\vec{b}$ are
$k$-tuples in $[n]$ and $\vec{r}$ and $\vec{c}$ are $2k$-tuples in $[n]$, and
(iii) $\vec{a}_{\vec{\consp}(h)}$, $\vec{b}_{\vec{\consp}(h)}$,
$\vec{r}_{\vec{\consp}(i)}$, and $\vec{c}_{\vec{\consp}(j)}$ (where $i := \row
(h)$ and $j := \column (j)$) are all compatible with one another. We note that
$\vec{w}_1$ and $\vec{w}_2$ are $2k$-tuples of variables. Let
\begin{align*}
	\FA{merge} (\mu, \nu, \vec{x}, \vec{y}, \vec{w}_1, \vec{w}_2) := & \exists \delta, \epsilon \leq \FA{m} \, ( \phi_L (\mu, \nu, \delta, \epsilon) \\ & \land \FA{agree}_L (\mu, \nu , \delta, \vec{x}, \vec{y}, \vec{w}_1) \land \\ & \FA{agree}_L (\mu, \nu , \epsilon, \vec{x}, \vec{y}, \vec{w}_2)).
\end{align*}
We are almost ready to define the matrix $M$ from the previous subsection. We
break this up into two formulas. The first is used to to check if an index is in
the domain of the matrix and the second defines $M$ for a given index. We note
that $\vec{y}$ and $\vec{z}$ are $2k$-tuples in the below formulas and let
$\FA{dom}_{\phi_L} (\mu, \delta, \epsilon) := \exists \nu \leq \FA{m} \, \phi_L
(\mu, \nu, \delta, \epsilon)$. Let
				
\begin{align*}
	\FA{check-dom}(\mu, \vec{x}, \delta, \vec{y}, \epsilon, \vec{z})  := & \FA{dom}_{\phi_L} (\mu, \delta, \epsilon) \land \FA{min-orb}(\mu, \delta)\land \FA{min-orb} (\mu, \epsilon) \\ & \land \FA{agree}_L (\mu, \delta , \vec{x}, \vec{y}) \land \FA{agree}_L (\mu, \epsilon , \vec{x}, \vec{z}), 
\end{align*}
and
\begin{align*}
	\FA{find-eval} (\mu, \vec{x}, \delta, \vec{y}, \epsilon, \vec{z}) := & \exists \vec{s} \leq \FA{s} \, (\FA{move}(\mu, \delta, \epsilon, \vec{x}, \vec{y}, \vec{z}, \vec{s}) \\ & \land (\exists \epsilon' \leq \FA{m} \, (\FA{map}^c (\mu, \epsilon, \epsilon', \vec{s}) \land (\exists \nu \leq \FA{m} \, (\phi_L (\mu, \delta, \epsilon', \nu) \\ & \land (\exists \vec{m} \, \leq \FA{s} \, (\FA{merge} (\mu, \nu, \vec{x}, \vec{m}, \vec{y}, \vec{z}) \land V(\nu, \vec{m})))))))).
\end{align*}
				
We are now ready to define the formula that defines the matrix $M$ for a given
gate and assignment to the support of $g$. Let
				
\begin{align*}
	\psi_M (\mu, \vec{x}, \delta, \vec{x}, \epsilon, \vec{y}) :=  \FA{check-dom}(\mu, \vec{x}, \delta, \vec{y}, \epsilon, \vec{z}) \land \FA{find-eval} (\mu, \vec{x}, \delta, \vec{y}, \epsilon, \vec{z}).
\end{align*}
				
We now define the formula that evaluates the $\RANK$-gate $g$. Let
				
\begin{align*}
	\theta_\RANK (\mu, \vec{x}) := & \bigwedge_{1 \leq i < j \leq k} x_i \neq x_j \land (\exists \pi, \kappa \leq \FA{m} \, (\phi_{\rank}(\mu, \pi, \kappa) \\ &\land [\rank (\vec{y}\delta\leq \FA{m}, \vec{z}\epsilon\leq \FA{m}, \pi \leq \FA{m}). \psi_M] \leq \kappa ))).
\end{align*}

It follows from Lemma~\ref{lem:rank-triple-equivilence} that, given an
appropriate assignment to the second-order variables implementing the recursion,
$\theta_{\RANK}[g, \vec{a}]$ if, and only if, $\vec{a} \in \overline{\EV}_g$. We
now define $\theta(\mu, \vec{x})$, the formula that defines the relation $V$.
Let

\begin{align*}
	\theta (\mu, \vec{x}) := [\ifp_{V,\nu \vec{y}} \bigvee_{s \in  S \uplus \tau \uplus \{0,1\}} (\phi_s(\mu) \land \theta_s (\nu, \vec{y} ))] (\mu, \vec{x}).
\end{align*}

The following $\FPR$ formula defines the $q$-ary query computed by the circuit
family $\mathcal{C}$~\cite{AndersonD17}. Let
				
\begin{align*}
	Q (z_1, \ldots z_q) := & \exists \vec{x} \exists \mu, \nu_1 , \ldots  \nu_q \eta_1 , \ldots , \nu_k \leq \FA{m} [\theta (\mu, \vec{x}) \land \phi_\Omega (\nu_1, \ldots \nu_q, \mu) \land \\
                         & \bigwedge_{1 \leq i \leq k} \FA{supp}_i (\mu, \eta_i) \wedge \forall \eta (\neg \FA{supp}_i (\mu, \eta)) \land \\
                         & \bigwedge_{1 \leq i \leq k} \bigwedge_{1 \leq j \leq q}(\FA{supp}_i (\mu, \eta_i) \land (x_i = z_j) \implies \nu_j = \eta_i) \land \\ &                                                                                                                                                        \bigwedge_{1 \leq j \leq q} \bigvee_{1 \leq i \leq k} (x_i = z_j \land \FA{supp}_i (\mu, \eta_i))].
\end{align*}
				
This completes the proof of our main theorem.

\end{document}