% !TEX root = ../main/thesis.tex
\documentclass[../main/thesis.tex]{subfiles}
\begin{document}
We have from the Immerman-Vardi~\cite{} theorem that over linearly ordered
structures $\FP$ captures polynomial time. But $\FP$ does not capture polynomial
time over unordered structures. This motivated the introduction $\FP^{\nats}$
which allows a formula to refer either to the domain of the structure (the
element sort) or to a separate ordered copy of the domain of the structure (the
number sort). This has lead to the study of logics that relate these two domains
in some way, for example $\FPC$ extends $\FP^{\nats}$ with an operator that
defines a number term from a formula, with the number term denoting the
cardinality of the set defined by the formula.

In this chapter we introduce \emph{generalised operators}. In the same way
Lindestr\"{o}m quantifiers generalise existential and universal quantifiers,
generalised operators generalise counting and rank operators. As such,
generalised operators act on a sequence of number terms and formulas and define
a number term. Generalised operators also allow for the inclusion of a
parameter, similar to the parameter given in the definition of a rank operator
by Gradel and Pakusa~\cite{}.

We know that each formula in $\FP$ is equivalent to a formula in
$\mathcal{L}^{\omega}$ and each formula in $\FPC$ is equivalent to a formula in
$\mathcal{C}^{\omega}$. These translations from fixed-point logics to infinitary
logics have been crucial for proving almost all key separation results
(see~\cite{}). In this chapter we define what we call \emph{substitution
  programs} and show that formulas of $\FP^{\nats}$ extended by a family of
operators may be translated to appropriate $\PT$-uniform families of
substitution programs. These $\PT$-uniform families of substitution programs may
in turn be translated into formula in an extension of $\mathcal{L}^{\omega}$ by
a corresponding family of \emph{extended quantifiers}.

We also show that these $\PT$-uniform substitution programs may be translated to
$\PT$-uniform families of transparent symmetric circuits. This result gives us a
complete translation from extension of $\FP^{\nats}$ by operators to
$\PT$-uniform families of symmetric circuits. In the final section of this
chapter we discuss some special cases.





% The translation from fixed-point logics to

% We show that each set of operators (or quantifiers) may be translated to a In
% the final section of this chapter we show that formulas of extensions of
% $\FP^{\nats}$ by generalised operators may be translated to a $\PT$-uniform
% family of transparent symmetric circuits over an appropriate basis.



% that the extension of $\FP^{\nats}$ by rank operators that take a prime as a
% parameter define a strictly more expressive logic than the extension by rank
% operators without such a parameter. As such,

% These general operators are intended to inspired by the generalised
% quantifiers of Lindestrom. However,





% As such, $\FP^{\nats}$ can

% We have from the Immerman-Vardi~\cite{} theorem that over linearly ordered
% structures $\FP$ captures polynomial time. But $\FP$ does not capture
% polynomial time over unordered structures as it can not, for example, express
% the $\parity$ query. The contrast between the expressive power over ordered
% and unordered structures motivates the introduction $\FP^{\nats}$.
% $\FP^{\nats}$ extends $\FP$ with a number-sort, allowing the logic to reason
% over the universe of the structure (the element sort) or over a separate
% linearly-ordered number domain (the number sort) with the same cardinality as
% the structure. As such $\FP^{\nats}$ can solve polynomial-time problems over
% it's number domain (and so can express $\parity$), but, importantly, the logic
% has no mechanism for relating the (unordered) domain of the structure with the
% ordered number domain. $\FPC$ extends $\FP^{\nats}$ with a counting operator
% that provide We can understand the many of the logics of interest, including
% $FPC$ and $\FPR$, as extending $\FP^{\nats}$ by operators that provide such a
% mechanism.


% by introducing operators that act on a sequence of formulas and number terms
% and evaluates to an element of the number sort. In the case of $\FPC$ this
% operator acts on a formula and evaluates to the cardinality of the set defined
% by that formula, and in the case of $\FPR$ this operator acts on a number-term
% and evaluates to the rank of the matrix over the appropriate field.

% Importantly, in the case of $\FPC$ we can associate this counting operator
% with

% In the same way Lindstr-om quantifiers provide a general mechanism for
% extending a logic by a quantifier, in this chapter we introduce the notion of
% a \emph{generalised operators} that provide a general mechanism for extending
% a logic by an operator. We show that each generalised operator may be
% assosiated with a family of quantifiers.

% In this chapter we provide a formal notion of a general operator and define
% what it means to extend a logic with such an operator. A general operator
% differs from a general quantifier in that (i) a general operator can evaluate
% to a number-term, (ii) a general operator


% that acts as a mechanism for

% as extenas introducing a mechanism for relating the unordered element domain
% with the ordered number domain. For example, $\FPC$ extends $\FP^{\nats}$ by
% introducing counting operators, operators that act on formulas and evaluate to
% the cardinality of the set define by the formula.

% In this chapter we will introduce a general framework


% an operator that

% that allows one to relate the element domain


% extends $\FP^{\nats}$ by introducing a counting




% `order on the side', allowi to refer to the structure or over a separate
% number domain


% , which extends $\FP$ with the ability to quantify over a second-order


% cases, where $\FP$ has the full power of a polynomial-time bounded Turing
% machine, and the unordered cases, where $\FP$ cannot express seemingly trivial
% queries, motivated the study of logics which introduce an `order on the side'.
% that allow for reasoning both of the structure and over a separate ordered
% number domain.



% In particular, $\FP^{\nats}$ can use the number domain to express
% polynomial-time queries for which membership depends just on the size of the
% structure. However, in $\FP^{\nats}$ there is no mechanism for relating the
% unordered domain of the structure and the associated ordered number-domain.

% Many of the logics that arise in the finite model theory extend $\FP^{\nats}$
% (or $\FO^{\nats}$) with operators that relate the element domain and the
% number domain. For example, $\FPC$ and extends $\FP^{\nats}$ with a counting
% operator that allows one to define defines the cardinality of a set (of either
% domain)





% the ordered and unordered domains. A natural extension of $\FP$ considered in
% the literature is $\FP^{\nats}$to extend the logic



% Instead of requiring an order on the structure $\FP$

% The logic $\FP^{\nats}$ is an extension of $\FP$ that allows references not
% just to the structure but also over a separate linearly ordered number domain.
% This `order on the side' allows us to express in $\FP^{\nats}$ any
% polynomial-time decidable query for which membership depends only on the size
% of the structure (e.g.\ parity).


% We cannot express all polynomial time decidable queries because in
% $\FP^{\nats}$ the domain of the structure and the number domain are treated as
% completely separate, and there is no mechanism for relating these two domains.


% In this extension, using the Immerman-Vardi~\cite{} theorem, allows us to
% express



% In the ordered domain allows us to express in $\FP^{\nats}$ polynomial-time
% decidable query such that membership depends only on the size of the structure
% (including parity).





% This has lead to the study of logics that allow for reasoning over


% This has lead to the study of logics such as $\FP^{\nats}$ which extend $\FP$
% allowing for reasoning over the domain of the structure and over a separate
% ordered number domain. If we could identify the domain of the structure with
% the number domain

% and extensions of $\FP^{\nats}$, e.g.\ $\FPC$ and $\FPR$. These extensions
% introduce operators that relate the domain of the structure with this ordered
% number-domain.


% The natural approach,


% In particular, we are interested in logics that extend $\FP^{\nats}$ with
% operators and quantifiers that allow us to relate the element-sort and
% number-sort

% In such a logic we have access not only to a structure but to an `order on the
% side', and t


% This has lead to study of extesnions logics that allow for reasoning both over
% the domain of the structure and a separate associated linearly-ordered
% number-sort (e.g.\ $\FP^{\nats}$).

% This lead to the study of extensions of $\FP$ with a number sort, allowing one
% to reason either over the domain of the structure (the element-sort) or over
% an ordered domain associated with the structure (the number-sort). In
% particular, we are interested in logics tha


% If we could define in the logic a bijection between these separate sorts, then
% we could order the domain and apply the Immerman-Vardi theorem. Research has
% thus focused on extensions of $\FP$

% Supposing we cou define a bijection between the element and number sort in
% such , then we could define linearly order the structure and the logic would
% capture polynomial-time.

% This has lead to the study of logics with counting (e.g.\ $\FOC$ and $\FPC$).
% These logics do not impose an order on the structure, but do argument the
% structure with an (ordered) number domain.

% In this chapter we introduce a general notion of an operator and show that a
% formula from a fixed-point logics extended with these operators

% \section{Substitution Programs}
% \begin{definition}
%   Let $L$ be a logic and $\tau$ be a vocabulary. For each $i > 0$ let $R_i$ be
%   a distinct relation symbol not in $\tau$ with arity $r_i$. We say a sequence
%   of formulas $\Phi := (\phi_1(\vec{y}_1), \ldots, \phi_k(\vec{y}_k))$ is a
%   \emph{substitution program} on $L$ if for each $i \in [k]$ we have that
%   $\phi_i$ is an $L[\tau_i]$-formula, where $\tau_i = \tau \uplus \{R_j : i <
%   j \leq k\}$, and each $\vec{y}_i$ has length $r_i$.
% \end{definition}


% %LaTeX with PSTricks extensions
%%Creator: inkscape 0.92.3
%%Please note this file requires PSTricks extensions
\psset{xunit=.5pt,yunit=.5pt,runit=.5pt}
\begin{pspicture}(793.7007874,1122.51968504)
{
\newrgbcolor{curcolor}{0 0 0}
\pscustom[linestyle=none,fillstyle=solid,fillcolor=curcolor]
{
\newpath
\moveto(77.28562084,957.06172621)
\lineto(76.43926666,957.06172621)
\lineto(76.43080312,959.60925231)
\curveto(75.83835519,959.62053703)(75.24590725,959.68824537)(74.65345932,959.81237731)
\curveto(74.06101139,959.94215162)(73.46574228,960.13399191)(72.86765199,960.38789816)
\lineto(72.86765199,961.9113357)
\curveto(73.44317284,961.55022458)(74.02433604,961.27657006)(74.61114161,961.09037214)
\curveto(75.20358954,960.90981658)(75.81296456,960.81671762)(76.43926666,960.81107526)
\lineto(76.43926666,964.67045035)
\curveto(75.19230482,964.87357535)(74.28388466,965.21775939)(73.71400618,965.70300246)
\curveto(73.14977005,966.18824552)(72.86765199,966.85404415)(72.86765199,967.70039834)
\curveto(72.86765199,968.62010322)(73.17516068,969.34514664)(73.79017805,969.8755286)
\curveto(74.40519543,970.40591056)(75.28822496,970.71059806)(76.43926666,970.78959112)
\lineto(76.43926666,972.77852346)
\lineto(77.28562084,972.77852346)
\lineto(77.28562084,970.81498175)
\curveto(77.81036044,970.7924123)(78.31817295,970.73598869)(78.80905838,970.64571091)
\curveto(79.29994381,970.56107549)(79.77954452,970.4425859)(80.2478605,970.29024215)
\lineto(80.2478605,968.80912232)
\curveto(79.77954452,969.0461015)(79.29712263,969.22947824)(78.80059484,969.35925254)
\curveto(78.30970941,969.48902685)(77.80471808,969.56519873)(77.28562084,969.58776818)
\lineto(77.28562084,965.9738358)
\curveto(78.56643685,965.77635315)(79.50871118,965.42370557)(80.11244383,964.91589306)
\curveto(80.71617648,964.40808055)(81.01804281,963.71407012)(81.01804281,962.83386176)
\curveto(81.01804281,961.88030271)(80.69642822,961.12704749)(80.05319904,960.57409608)
\curveto(79.41561222,960.02678704)(78.49308615,959.71081481)(77.28562084,959.62617939)
\closepath
\moveto(76.43926666,966.12617955)
\lineto(76.43926666,969.59623172)
\curveto(75.78475275,969.52288102)(75.28540378,969.3366831)(74.94121975,969.03763795)
\curveto(74.59703571,968.73859281)(74.42494369,968.34080634)(74.42494369,967.84427855)
\curveto(74.42494369,967.35903548)(74.58292981,966.98099728)(74.89890204,966.71016394)
\curveto(75.22051663,966.4393306)(75.7339715,966.24466914)(76.43926666,966.12617955)
\closepath
\moveto(77.28562084,964.50117951)
\lineto(77.28562084,960.83646588)
\curveto(78.00220072,960.93238602)(78.54104622,961.13551103)(78.90215734,961.4458409)
\curveto(79.26891082,961.75617076)(79.45228756,962.16524195)(79.45228756,962.67305447)
\curveto(79.45228756,963.16958226)(79.27737437,963.56454754)(78.92754797,963.85795033)
\curveto(78.58336393,964.15135311)(78.03605489,964.36576284)(77.28562084,964.50117951)
\closepath
}
}
{
\newrgbcolor{curcolor}{0 0 0}
\pscustom[linestyle=none,fillstyle=solid,fillcolor=curcolor]
{
\newpath
\moveto(83.90411082,972.24532032)
\lineto(88.30515259,958.00117935)
\lineto(86.86635047,958.00117935)
\lineto(82.4653087,972.24532032)
\closepath
}
}
{
\newrgbcolor{curcolor}{0 0 0}
\pscustom[linestyle=none,fillstyle=solid,fillcolor=curcolor]
{
\newpath
\moveto(91.4959083,971.77982552)
\lineto(91.4959083,969.0884192)
\lineto(94.70359067,969.0884192)
\lineto(94.70359067,967.87813272)
\lineto(91.4959083,967.87813272)
\lineto(91.4959083,962.73229926)
\curveto(91.4959083,961.95929577)(91.60029198,961.46276798)(91.80905935,961.24271589)
\curveto(92.02346907,961.0226638)(92.45510971,960.91263776)(93.10398125,960.91263776)
\lineto(94.70359067,960.91263776)
\lineto(94.70359067,959.60925231)
\lineto(93.10398125,959.60925231)
\curveto(91.90215831,959.60925231)(91.0727312,959.83212558)(90.61569994,960.27787212)
\curveto(90.15866868,960.72926102)(89.93015305,961.5474034)(89.93015305,962.73229926)
\lineto(89.93015305,967.87813272)
\lineto(88.7875749,967.87813272)
\lineto(88.7875749,969.0884192)
\lineto(89.93015305,969.0884192)
\lineto(89.93015305,971.77982552)
\closepath
}
}
{
\newrgbcolor{curcolor}{0 0 0}
\pscustom[linestyle=none,fillstyle=solid,fillcolor=curcolor]
{
\newpath
\moveto(104.63978857,965.33060661)
\lineto(104.63978857,959.60925231)
\lineto(103.08249687,959.60925231)
\lineto(103.08249687,965.27982536)
\curveto(103.08249687,966.1769608)(102.90758367,966.84840179)(102.55775727,967.29414833)
\curveto(102.20793088,967.73989487)(101.68319128,967.96276814)(100.98353848,967.96276814)
\curveto(100.14282666,967.96276814)(99.47984921,967.69475598)(98.99460614,967.15873166)
\curveto(98.50936308,966.62270734)(98.26674154,965.89202156)(98.26674154,964.96667431)
\lineto(98.26674154,959.60925231)
\lineto(96.7009863,959.60925231)
\lineto(96.7009863,972.77852346)
\lineto(98.26674154,972.77852346)
\lineto(98.26674154,967.61576292)
\curveto(98.63913739,968.18564141)(99.07642038,968.61163968)(99.57859053,968.89375774)
\curveto(100.08640305,969.1758758)(100.67038744,969.31693484)(101.3305437,969.31693484)
\curveto(102.41951942,969.31693484)(103.24330416,968.97839316)(103.80189793,968.30130981)
\curveto(104.36049169,967.62986882)(104.63978857,966.63963442)(104.63978857,965.33060661)
\closepath
}
}
{
\newrgbcolor{curcolor}{0 0 0}
\pscustom[linestyle=none,fillstyle=solid,fillcolor=curcolor]
{
\newpath
\moveto(115.8709095,964.73815868)
\lineto(115.8709095,963.97643992)
\lineto(108.71075308,963.97643992)
\curveto(108.77846142,962.90439128)(109.10007601,962.0862489)(109.67559685,961.52201277)
\curveto(110.25676006,960.96341901)(111.06361772,960.68412213)(112.09616983,960.68412213)
\curveto(112.69426012,960.68412213)(113.27260215,960.75747282)(113.83119591,960.90417422)
\curveto(114.39543204,961.05087561)(114.9540258,961.2709277)(115.5069772,961.56433048)
\lineto(115.5069772,960.0916742)
\curveto(114.94838344,959.85469502)(114.37568377,959.67413946)(113.7888782,959.55000752)
\curveto(113.20207263,959.42587557)(112.60680352,959.3638096)(112.00307087,959.3638096)
\curveto(110.49091805,959.3638096)(109.29191629,959.80391377)(108.40606557,960.68412213)
\curveto(107.52585722,961.56433048)(107.08575304,962.75486871)(107.08575304,964.2557368)
\curveto(107.08575304,965.80738614)(107.50328777,967.03742089)(108.33835724,967.94584105)
\curveto(109.17906906,968.85990357)(110.31036249,969.31693484)(111.73223753,969.31693484)
\curveto(113.00741117,969.31693484)(114.01457265,968.90504246)(114.75372198,968.08125772)
\curveto(115.49851366,967.26311534)(115.8709095,966.148749)(115.8709095,964.73815868)
\closepath
\moveto(114.3136178,965.19518994)
\curveto(114.30233308,966.04718649)(114.06253272,966.72709102)(113.59421674,967.23490354)
\curveto(113.13154312,967.74271605)(112.51652574,967.9966223)(111.74916461,967.9966223)
\curveto(110.88024098,967.9966223)(110.18340937,967.75117959)(109.65866977,967.26029416)
\curveto(109.13957254,966.76940873)(108.84052739,966.07821948)(108.76153433,965.1867264)
\closepath
}
}
{
\newrgbcolor{curcolor}{0 0 0}
\pscustom[linestyle=none,fillstyle=solid,fillcolor=curcolor]
{
\newpath
\moveto(119.96726338,971.77982552)
\lineto(119.96726338,969.0884192)
\lineto(123.17494575,969.0884192)
\lineto(123.17494575,967.87813272)
\lineto(119.96726338,967.87813272)
\lineto(119.96726338,962.73229926)
\curveto(119.96726338,961.95929577)(120.07164706,961.46276798)(120.28041443,961.24271589)
\curveto(120.49482415,961.0226638)(120.92646479,960.91263776)(121.57533633,960.91263776)
\lineto(123.17494575,960.91263776)
\lineto(123.17494575,959.60925231)
\lineto(121.57533633,959.60925231)
\curveto(120.37351339,959.60925231)(119.54408628,959.83212558)(119.08705502,960.27787212)
\curveto(118.63002376,960.72926102)(118.40150813,961.5474034)(118.40150813,962.73229926)
\lineto(118.40150813,967.87813272)
\lineto(117.25892998,967.87813272)
\lineto(117.25892998,969.0884192)
\lineto(118.40150813,969.0884192)
\lineto(118.40150813,971.77982552)
\closepath
}
}
{
\newrgbcolor{curcolor}{0 0 0}
\pscustom[linestyle=none,fillstyle=solid,fillcolor=curcolor]
{
\newpath
\moveto(129.53952898,964.37422638)
\curveto(128.28128242,964.37422638)(127.40953761,964.23034617)(126.92429454,963.94258575)
\curveto(126.43905148,963.65482532)(126.19642994,963.1639399)(126.19642994,962.46992946)
\curveto(126.19642994,961.91697806)(126.3769855,961.47687388)(126.73809662,961.14961693)
\curveto(127.1048501,960.82800234)(127.60137789,960.66719504)(128.22767999,960.66719504)
\curveto(129.09096126,960.66719504)(129.78215052,960.97188255)(130.30124775,961.58125757)
\curveto(130.82598735,962.19627494)(131.08835715,963.01159614)(131.08835715,964.02722117)
\lineto(131.08835715,964.37422638)
\closepath
\moveto(132.64564885,965.01745557)
\lineto(132.64564885,959.60925231)
\lineto(131.08835715,959.60925231)
\lineto(131.08835715,961.04805443)
\curveto(130.73288839,960.47253358)(130.28996303,960.04653531)(129.75958107,959.77005961)
\curveto(129.22919911,959.49922627)(128.58032757,959.3638096)(127.81296644,959.3638096)
\curveto(126.84248031,959.3638096)(126.06947682,959.63464294)(125.49395597,960.17630962)
\curveto(124.92407748,960.72361866)(124.63913824,961.45430444)(124.63913824,962.36836696)
\curveto(124.63913824,963.43477324)(124.994607,964.23880971)(125.70554452,964.78047639)
\curveto(126.42212439,965.32214307)(127.48853067,965.59297641)(128.90476334,965.59297641)
\lineto(131.08835715,965.59297641)
\lineto(131.08835715,965.74532017)
\curveto(131.08835715,966.46190004)(130.85137797,967.01485145)(130.37741963,967.40417437)
\curveto(129.90910364,967.79913966)(129.24894738,967.9966223)(128.39695083,967.9966223)
\curveto(127.85528415,967.9966223)(127.32772337,967.93173515)(126.8142685,967.80196084)
\curveto(126.30081363,967.67218653)(125.80710702,967.47752507)(125.33314867,967.21797645)
\lineto(125.33314867,968.65677857)
\curveto(125.90302716,968.87683066)(126.45597856,969.04045913)(126.99200288,969.147664)
\curveto(127.5280272,969.26051122)(128.04994561,969.31693484)(128.55775813,969.31693484)
\curveto(129.92885191,969.31693484)(130.95294048,968.96146608)(131.63002382,968.25052856)
\curveto(132.30710717,967.53959104)(132.64564885,966.46190004)(132.64564885,965.01745557)
\closepath
}
}
{
\newrgbcolor{curcolor}{0 0 0}
\pscustom[linestyle=none,fillstyle=solid,fillcolor=curcolor]
{
\newpath
\moveto(140.08510049,957.06172621)
\lineto(139.2387463,957.06172621)
\lineto(139.23028276,959.60925231)
\curveto(138.63783483,959.62053703)(138.0453869,959.68824537)(137.45293897,959.81237731)
\curveto(136.86049104,959.94215162)(136.26522193,960.13399191)(135.66713164,960.38789816)
\lineto(135.66713164,961.9113357)
\curveto(136.24265248,961.55022458)(136.82381569,961.27657006)(137.41062126,961.09037214)
\curveto(138.00306919,960.90981658)(138.61244421,960.81671762)(139.2387463,960.81107526)
\lineto(139.2387463,964.67045035)
\curveto(137.99178447,964.87357535)(137.08336431,965.21775939)(136.51348582,965.70300246)
\curveto(135.9492497,966.18824552)(135.66713164,966.85404415)(135.66713164,967.70039834)
\curveto(135.66713164,968.62010322)(135.97464032,969.34514664)(136.5896577,969.8755286)
\curveto(137.20467508,970.40591056)(138.08770461,970.71059806)(139.2387463,970.78959112)
\lineto(139.2387463,972.77852346)
\lineto(140.08510049,972.77852346)
\lineto(140.08510049,970.81498175)
\curveto(140.60984009,970.7924123)(141.1176526,970.73598869)(141.60853803,970.64571091)
\curveto(142.09942346,970.56107549)(142.57902416,970.4425859)(143.04734015,970.29024215)
\lineto(143.04734015,968.80912232)
\curveto(142.57902416,969.0461015)(142.09660228,969.22947824)(141.60007449,969.35925254)
\curveto(141.10918906,969.48902685)(140.60419773,969.56519873)(140.08510049,969.58776818)
\lineto(140.08510049,965.9738358)
\curveto(141.36591649,965.77635315)(142.30819082,965.42370557)(142.91192348,964.91589306)
\curveto(143.51565613,964.40808055)(143.81752246,963.71407012)(143.81752246,962.83386176)
\curveto(143.81752246,961.88030271)(143.49590787,961.12704749)(142.85267868,960.57409608)
\curveto(142.21509186,960.02678704)(141.2925658,959.71081481)(140.08510049,959.62617939)
\closepath
\moveto(139.2387463,966.12617955)
\lineto(139.2387463,969.59623172)
\curveto(138.5842324,969.52288102)(138.08488343,969.3366831)(137.74069939,969.03763795)
\curveto(137.39651536,968.73859281)(137.22442334,968.34080634)(137.22442334,967.84427855)
\curveto(137.22442334,967.35903548)(137.38240945,966.98099728)(137.69838168,966.71016394)
\curveto(138.01999628,966.4393306)(138.53345115,966.24466914)(139.2387463,966.12617955)
\closepath
\moveto(140.08510049,964.50117951)
\lineto(140.08510049,960.83646588)
\curveto(140.80168037,960.93238602)(141.34052587,961.13551103)(141.70163699,961.4458409)
\curveto(142.06839047,961.75617076)(142.25176721,962.16524195)(142.25176721,962.67305447)
\curveto(142.25176721,963.16958226)(142.07685401,963.56454754)(141.72702761,963.85795033)
\curveto(141.38284358,964.15135311)(140.83553454,964.36576284)(140.08510049,964.50117951)
\closepath
}
}
{
\newrgbcolor{curcolor}{0 0 0}
\pscustom[linestyle=none,fillstyle=solid,fillcolor=curcolor]
{
\newpath
\moveto(51.0385253,1044.62726759)
\lineto(50.33664692,1044.62726759)
\lineto(50.32962814,1046.73992153)
\curveto(49.83831327,1046.74927991)(49.3469984,1046.80543018)(48.85568353,1046.90837234)
\curveto(48.36436866,1047.01599369)(47.8707142,1047.17508613)(47.37472014,1047.38564964)
\lineto(47.37472014,1048.64903073)
\curveto(47.85199744,1048.34956262)(48.33395393,1048.12262194)(48.82058961,1047.9682087)
\curveto(49.31190448,1047.81847464)(49.81725692,1047.74126802)(50.33664692,1047.73658883)
\lineto(50.33664692,1050.93715426)
\curveto(49.3025461,1051.10560507)(48.54919664,1051.39103561)(48.07659852,1051.79344589)
\curveto(47.6086796,1052.19585616)(47.37472014,1052.74800049)(47.37472014,1053.44987887)
\curveto(47.37472014,1054.21258672)(47.62973595,1054.81386253)(48.13976758,1055.25370632)
\curveto(48.6497992,1055.6935501)(49.38209232,1055.94622632)(50.33664692,1056.01173497)
\lineto(50.33664692,1057.66114917)
\lineto(51.0385253,1057.66114917)
\lineto(51.0385253,1056.03279132)
\curveto(51.4736899,1056.01407457)(51.89481693,1055.96728267)(52.30190639,1055.89241565)
\curveto(52.70899586,1055.82222781)(53.10672694,1055.72396483)(53.49509964,1055.59762673)
\lineto(53.49509964,1054.36933955)
\curveto(53.10672694,1054.5658655)(52.70665626,1054.71793915)(52.29488761,1054.8255605)
\curveto(51.88779815,1054.93318186)(51.46901071,1054.99635091)(51.0385253,1055.01506767)
\lineto(51.0385253,1052.01804697)
\curveto(52.10070126,1051.85427535)(52.88212586,1051.56182602)(53.3827991,1051.14069899)
\curveto(53.88347235,1050.71957196)(54.13380897,1050.14403169)(54.13380897,1049.41407817)
\curveto(54.13380897,1048.62329519)(53.86709519,1047.99862343)(53.33366762,1047.54006288)
\curveto(52.80491923,1047.08618153)(52.0398718,1046.82414693)(51.0385253,1046.7539591)
\closepath
\moveto(50.33664692,1052.14438508)
\lineto(50.33664692,1055.02208645)
\curveto(49.79386097,1054.96125699)(49.37975272,1054.80684375)(49.09432218,1054.55884672)
\curveto(48.80889164,1054.31084969)(48.66617637,1053.98096685)(48.66617637,1053.5691982)
\curveto(48.66617637,1053.16678792)(48.79719367,1052.85328225)(49.05922826,1052.62868116)
\curveto(49.32594205,1052.40408008)(49.75174827,1052.24264805)(50.33664692,1052.14438508)
\closepath
\moveto(51.0385253,1050.79677858)
\lineto(51.0385253,1047.75764518)
\curveto(51.63278233,1047.8371914)(52.0796449,1048.00564221)(52.37911302,1048.26299762)
\curveto(52.68326031,1048.52035303)(52.83533396,1048.85959425)(52.83533396,1049.28072128)
\curveto(52.83533396,1049.69248993)(52.6902791,1050.02003317)(52.40016937,1050.26335101)
\curveto(52.11473882,1050.50666885)(51.66085747,1050.68447804)(51.0385253,1050.79677858)
\closepath
}
}
{
\newrgbcolor{curcolor}{0 0 0}
\pscustom[linestyle=none,fillstyle=solid,fillcolor=curcolor]
{
\newpath
\moveto(56.52721426,1057.21896579)
\lineto(60.17698185,1045.4063526)
\lineto(58.9837886,1045.4063526)
\lineto(55.33402101,1057.21896579)
\closepath
}
}
{
\newrgbcolor{curcolor}{0 0 0}
\pscustom[linestyle=none,fillstyle=solid,fillcolor=curcolor]
{
\newpath
\moveto(61.60179497,1057.21896579)
\lineto(67.6239115,1057.21896579)
\lineto(67.6239115,1056.02577254)
\lineto(63.01958931,1056.02577254)
\lineto(63.01958931,1052.93750765)
\lineto(67.17470934,1052.93750765)
\lineto(67.17470934,1051.7443144)
\lineto(63.01958931,1051.7443144)
\lineto(63.01958931,1046.73992153)
\lineto(61.60179497,1046.73992153)
\closepath
}
}
{
\newrgbcolor{curcolor}{0 0 0}
\pscustom[linestyle=none,fillstyle=solid,fillcolor=curcolor]
{
\newpath
\moveto(71.30175423,1056.05384767)
\lineto(71.30175423,1052.11630994)
\lineto(73.08452532,1052.11630994)
\curveto(73.744291,1052.11630994)(74.25432263,1052.28710035)(74.6146202,1052.62868116)
\curveto(74.97491777,1052.97026198)(75.15506656,1053.45689766)(75.15506656,1054.0885882)
\curveto(75.15506656,1054.71559956)(74.97491777,1055.19989564)(74.6146202,1055.54147645)
\curveto(74.25432263,1055.88305727)(73.744291,1056.05384767)(73.08452532,1056.05384767)
\closepath
\moveto(69.8839599,1057.21896579)
\lineto(73.08452532,1057.21896579)
\curveto(74.25900182,1057.21896579)(75.14570818,1056.95225201)(75.7446444,1056.41882443)
\curveto(76.34825981,1055.89007605)(76.65006751,1055.11333064)(76.65006751,1054.0885882)
\curveto(76.65006751,1053.05448738)(76.34825981,1052.27306278)(75.7446444,1051.7443144)
\curveto(75.14570818,1051.21556602)(74.25900182,1050.95119183)(73.08452532,1050.95119183)
\lineto(71.30175423,1050.95119183)
\lineto(71.30175423,1046.73992153)
\lineto(69.8839599,1046.73992153)
\closepath
}
}
{
\newrgbcolor{curcolor}{0 0 0}
\pscustom[linestyle=none,fillstyle=solid,fillcolor=curcolor]
{
\newpath
\moveto(83.85835851,1057.21896579)
\lineto(87.66955813,1053.3095032)
\lineto(86.25878258,1053.3095032)
\lineto(83.17051769,1056.08192281)
\lineto(80.08225281,1053.3095032)
\lineto(78.67147726,1053.3095032)
\lineto(82.48267688,1057.21896579)
\closepath
}
}
{
\newrgbcolor{curcolor}{0 0 0}
\pscustom[linestyle=none,fillstyle=solid,fillcolor=curcolor]
{
\newpath
\moveto(96.5413009,1045.4063526)
\lineto(96.5413009,1044.39564773)
\lineto(96.1061363,1044.39564773)
\curveto(94.94101818,1044.39564773)(94.15959358,1044.56877773)(93.7618625,1044.91503773)
\curveto(93.3688106,1045.26129773)(93.17228466,1045.95147814)(93.17228466,1046.98557896)
\lineto(93.17228466,1048.6630683)
\curveto(93.17228466,1049.36962587)(93.04594655,1049.85860114)(92.79327033,1050.12999412)
\curveto(92.54059411,1050.40138709)(92.08203357,1050.53708358)(91.4175887,1050.53708358)
\lineto(90.98944288,1050.53708358)
\lineto(90.98944288,1051.54076967)
\lineto(91.4175887,1051.54076967)
\curveto(92.08671276,1051.54076967)(92.5452733,1051.67412656)(92.79327033,1051.94084035)
\curveto(93.04594655,1052.21223332)(93.17228466,1052.69652941)(93.17228466,1053.3937286)
\lineto(93.17228466,1055.07823672)
\curveto(93.17228466,1056.11233754)(93.3688106,1056.80017836)(93.7618625,1057.14175917)
\curveto(94.15959358,1057.48801917)(94.94101818,1057.66114917)(96.1061363,1057.66114917)
\lineto(96.5413009,1057.66114917)
\lineto(96.5413009,1056.65746308)
\lineto(96.0640236,1056.65746308)
\curveto(95.40425792,1056.65746308)(94.97377251,1056.55452092)(94.77256737,1056.3486366)
\curveto(94.57136223,1056.14275227)(94.47075967,1055.70992727)(94.47075967,1055.05016159)
\lineto(94.47075967,1053.3095032)
\curveto(94.47075967,1052.57487049)(94.36313831,1052.04144292)(94.14789561,1051.70922048)
\curveto(93.93733209,1051.37699805)(93.57469493,1051.15239696)(93.05998411,1051.03541723)
\curveto(93.57937412,1050.90907912)(93.94435088,1050.67979885)(94.15491439,1050.34757642)
\curveto(94.36547791,1050.01535398)(94.47075967,1049.48426601)(94.47075967,1048.75431249)
\lineto(94.47075967,1047.0136541)
\curveto(94.47075967,1046.35388842)(94.57136223,1045.92106341)(94.77256737,1045.71517909)
\curveto(94.97377251,1045.50929476)(95.40425792,1045.4063526)(96.0640236,1045.4063526)
\closepath
}
}
{
\newrgbcolor{curcolor}{0 0 0}
\pscustom[linestyle=none,fillstyle=solid,fillcolor=curcolor]
{
\newpath
\moveto(99.53832159,1057.21896579)
\lineto(103.18808919,1045.4063526)
\lineto(101.99489594,1045.4063526)
\lineto(98.34512834,1057.21896579)
\closepath
}
}
{
\newrgbcolor{curcolor}{0 0 0}
\pscustom[linestyle=none,fillstyle=solid,fillcolor=curcolor]
{
\newpath
\moveto(111.09123978,1051.4846194)
\lineto(111.09123978,1046.73992153)
\lineto(109.79978356,1046.73992153)
\lineto(109.79978356,1051.4425067)
\curveto(109.79978356,1052.18649778)(109.65472869,1052.7433213)(109.36461896,1053.11297725)
\curveto(109.07450923,1053.4826332)(108.63934463,1053.66746117)(108.05912517,1053.66746117)
\curveto(107.36192597,1053.66746117)(106.81212124,1053.44519968)(106.40971097,1053.00067671)
\curveto(106.00730069,1052.55615373)(105.80609556,1051.95019873)(105.80609556,1051.18281169)
\lineto(105.80609556,1046.73992153)
\lineto(104.50762055,1046.73992153)
\lineto(104.50762055,1054.60095942)
\lineto(105.80609556,1054.60095942)
\lineto(105.80609556,1053.37969103)
\curveto(106.11492205,1053.85228915)(106.47755921,1054.20556793)(106.89400705,1054.43952739)
\curveto(107.31513408,1054.67348685)(107.79943017,1054.79046658)(108.3468953,1054.79046658)
\curveto(109.24997882,1054.79046658)(109.93314045,1054.50971523)(110.39638018,1053.94821252)
\curveto(110.85961992,1053.39138901)(111.09123978,1052.5701913)(111.09123978,1051.4846194)
\closepath
}
}
{
\newrgbcolor{curcolor}{0 0 0}
\pscustom[linestyle=none,fillstyle=solid,fillcolor=curcolor]
{
\newpath
\moveto(117.25373199,1050.69149683)
\curveto(116.21027279,1050.69149683)(115.48733806,1050.5721775)(115.08492778,1050.33353885)
\curveto(114.68251751,1050.0949002)(114.48131238,1049.68781074)(114.48131238,1049.11227046)
\curveto(114.48131238,1048.65370992)(114.63104643,1048.28873316)(114.93051454,1048.01734019)
\curveto(115.23466184,1047.7506264)(115.64643049,1047.61726951)(116.1658205,1047.61726951)
\curveto(116.88173645,1047.61726951)(117.45493713,1047.86994572)(117.88542253,1048.37529816)
\curveto(118.32058713,1048.88532979)(118.53816943,1049.56147263)(118.53816943,1050.40372669)
\lineto(118.53816943,1050.69149683)
\closepath
\moveto(119.82962566,1051.2249244)
\lineto(119.82962566,1046.73992153)
\lineto(118.53816943,1046.73992153)
\lineto(118.53816943,1047.93311478)
\curveto(118.24338051,1047.45583748)(117.87606416,1047.10255869)(117.43622037,1046.87327842)
\curveto(116.99637658,1046.64867734)(116.45826982,1046.5363768)(115.82190009,1046.5363768)
\curveto(115.01707954,1046.5363768)(114.37603062,1046.76097788)(113.89875332,1047.21018004)
\curveto(113.42615521,1047.6640614)(113.18985615,1048.2700164)(113.18985615,1049.02804506)
\curveto(113.18985615,1049.91241182)(113.48464507,1050.57919628)(114.07422291,1051.02839845)
\curveto(114.66847994,1051.47760062)(115.55284671,1051.7022017)(116.7273232,1051.7022017)
\lineto(118.53816943,1051.7022017)
\lineto(118.53816943,1051.82853981)
\curveto(118.53816943,1052.42279684)(118.34164348,1052.88135738)(117.94859159,1053.20422144)
\curveto(117.56021888,1053.53176468)(117.01275374,1053.69553631)(116.30619617,1053.69553631)
\curveto(115.85699401,1053.69553631)(115.41948981,1053.64172563)(114.9936836,1053.53410428)
\curveto(114.56787738,1053.42648293)(114.15844832,1053.2650509)(113.76539642,1053.04980819)
\lineto(113.76539642,1054.24300145)
\curveto(114.23799454,1054.42548983)(114.69655508,1054.56118631)(115.14107806,1054.65009091)
\curveto(115.58560103,1054.74367469)(116.01842603,1054.79046658)(116.43955306,1054.79046658)
\curveto(117.57659605,1054.79046658)(118.42586889,1054.49567766)(118.9873716,1053.90609982)
\curveto(119.5488743,1053.31652198)(119.82962566,1052.42279684)(119.82962566,1051.2249244)
\closepath
}
}
{
\newrgbcolor{curcolor}{0 0 0}
\pscustom[linestyle=none,fillstyle=solid,fillcolor=curcolor]
{
\newpath
\moveto(123.77418217,1056.83293268)
\lineto(123.77418217,1054.60095942)
\lineto(126.43430124,1054.60095942)
\lineto(126.43430124,1053.59727333)
\lineto(123.77418217,1053.59727333)
\lineto(123.77418217,1049.32985276)
\curveto(123.77418217,1048.68880384)(123.86074717,1048.27703519)(124.03387717,1048.09454681)
\curveto(124.21168636,1047.91205843)(124.56964434,1047.82081424)(125.1077511,1047.82081424)
\lineto(126.43430124,1047.82081424)
\lineto(126.43430124,1046.73992153)
\lineto(125.1077511,1046.73992153)
\curveto(124.1110838,1046.73992153)(123.42324298,1046.9247495)(123.04422865,1047.29440545)
\curveto(122.66521433,1047.66874059)(122.47570716,1048.34722303)(122.47570716,1049.32985276)
\lineto(122.47570716,1053.59727333)
\lineto(121.52817134,1053.59727333)
\lineto(121.52817134,1054.60095942)
\lineto(122.47570716,1054.60095942)
\lineto(122.47570716,1056.83293268)
\closepath
}
}
{
\newrgbcolor{curcolor}{0 0 0}
\pscustom[linestyle=none,fillstyle=solid,fillcolor=curcolor]
{
\newpath
\moveto(133.15127737,1054.36933955)
\lineto(133.15127737,1053.14807117)
\curveto(132.78630061,1053.33523874)(132.40728629,1053.47561441)(132.01423439,1053.5691982)
\curveto(131.6211825,1053.66278198)(131.21409303,1053.70957387)(130.792966,1053.70957387)
\curveto(130.15191708,1053.70957387)(129.66996059,1053.6113109)(129.34709653,1053.41478495)
\curveto(129.02891167,1053.21825901)(128.86981923,1052.92347008)(128.86981923,1052.53041819)
\curveto(128.86981923,1052.23095008)(128.98445937,1051.99465102)(129.21373964,1051.82152102)
\curveto(129.44301991,1051.65307021)(129.90392005,1051.49163818)(130.59644006,1051.33722494)
\lineto(131.03862344,1051.23896196)
\curveto(131.95574453,1051.04243602)(132.60615183,1050.76402426)(132.98984534,1050.40372669)
\curveto(133.37821805,1050.04810831)(133.5724044,1049.54977466)(133.5724044,1048.90872573)
\curveto(133.5724044,1048.17877221)(133.28229467,1047.60089234)(132.70207521,1047.17508613)
\curveto(132.12653493,1046.74927991)(131.33341236,1046.5363768)(130.32270749,1046.5363768)
\curveto(129.90158046,1046.5363768)(129.46173667,1046.5784895)(129.00317613,1046.66271491)
\curveto(128.54929477,1046.74226112)(128.06967788,1046.86392004)(127.56432544,1047.02769166)
\lineto(127.56432544,1048.36126059)
\curveto(128.04160274,1048.11326356)(128.51186126,1047.926096)(128.97510099,1047.79975789)
\curveto(129.43834072,1047.67809897)(129.89690127,1047.61726951)(130.35078262,1047.61726951)
\curveto(130.95907722,1047.61726951)(131.42699614,1047.72021167)(131.75453939,1047.926096)
\curveto(132.08208264,1048.13665951)(132.24585426,1048.43144843)(132.24585426,1048.81046276)
\curveto(132.24585426,1049.16140195)(132.12653493,1049.43045533)(131.88789628,1049.6176229)
\curveto(131.65393682,1049.80479047)(131.13688641,1049.98493925)(130.33674506,1050.15806925)
\lineto(129.88754289,1050.26335101)
\curveto(129.08740153,1050.43180182)(128.50952166,1050.68915723)(128.15390328,1051.03541723)
\curveto(127.7982849,1051.38635643)(127.62047571,1051.86597332)(127.62047571,1052.47426792)
\curveto(127.62047571,1053.21357982)(127.88251031,1053.7844409)(128.4065795,1054.18685117)
\curveto(128.93064869,1054.58926145)(129.67463978,1054.79046658)(130.63855276,1054.79046658)
\curveto(131.11583006,1054.79046658)(131.56503223,1054.75537267)(131.98615926,1054.68518483)
\curveto(132.40728629,1054.61499699)(132.79565899,1054.50971523)(133.15127737,1054.36933955)
\closepath
}
}
{
\newrgbcolor{curcolor}{0 0 0}
\pscustom[linestyle=none,fillstyle=solid,fillcolor=curcolor]
{
\newpath
\moveto(136.07811023,1045.4063526)
\lineto(136.5694251,1045.4063526)
\curveto(137.22451159,1045.4063526)(137.65031781,1045.50695517)(137.84684376,1045.7081603)
\curveto(138.04804889,1045.90936544)(138.14865146,1046.34453004)(138.14865146,1047.0136541)
\lineto(138.14865146,1048.75431249)
\curveto(138.14865146,1049.48426601)(138.25393322,1050.01535398)(138.46449673,1050.34757642)
\curveto(138.67506025,1050.67979885)(139.04003701,1050.90907912)(139.55942701,1051.03541723)
\curveto(139.04003701,1051.15239696)(138.67506025,1051.37699805)(138.46449673,1051.70922048)
\curveto(138.25393322,1052.04144292)(138.14865146,1052.57487049)(138.14865146,1053.3095032)
\lineto(138.14865146,1055.05016159)
\curveto(138.14865146,1055.71460646)(138.04804889,1056.14743146)(137.84684376,1056.3486366)
\curveto(137.65031781,1056.55452092)(137.22451159,1056.65746308)(136.5694251,1056.65746308)
\lineto(136.07811023,1056.65746308)
\lineto(136.07811023,1057.66114917)
\lineto(136.52029361,1057.66114917)
\curveto(137.68541173,1057.66114917)(138.46215714,1057.48801917)(138.85052985,1057.14175917)
\curveto(139.24358174,1056.80017836)(139.44010769,1056.11233754)(139.44010769,1055.07823672)
\lineto(139.44010769,1053.3937286)
\curveto(139.44010769,1052.69652941)(139.5664458,1052.21223332)(139.81912201,1051.94084035)
\curveto(140.07179823,1051.67412656)(140.53035878,1051.54076967)(141.19480365,1051.54076967)
\lineto(141.62996824,1051.54076967)
\lineto(141.62996824,1050.53708358)
\lineto(141.19480365,1050.53708358)
\curveto(140.53035878,1050.53708358)(140.07179823,1050.40138709)(139.81912201,1050.12999412)
\curveto(139.5664458,1049.85860114)(139.44010769,1049.36962587)(139.44010769,1048.6630683)
\lineto(139.44010769,1046.98557896)
\curveto(139.44010769,1045.95147814)(139.24358174,1045.26129773)(138.85052985,1044.91503773)
\curveto(138.46215714,1044.56877773)(137.68541173,1044.39564773)(136.52029361,1044.39564773)
\lineto(136.07811023,1044.39564773)
\closepath
}
}
{
\newrgbcolor{curcolor}{0 0 0}
\pscustom[linestyle=none,fillstyle=solid,fillcolor=curcolor]
{
\newpath
\moveto(147.89072342,1057.6471116)
\curveto(147.26371207,1056.57089808)(146.79813274,1055.50638254)(146.49398544,1054.45356496)
\curveto(146.18983814,1053.40074739)(146.03776449,1052.33389224)(146.03776449,1051.25299953)
\curveto(146.03776449,1050.17210682)(146.18983814,1049.10057249)(146.49398544,1048.03839654)
\curveto(146.80281193,1046.98089977)(147.26839126,1045.91638422)(147.89072342,1044.84484989)
\lineto(146.76771801,1044.84484989)
\curveto(146.06583963,1045.94445936)(145.53943084,1047.02535207)(145.18849165,1048.08752802)
\curveto(144.84223165,1049.14970398)(144.66910164,1050.20486115)(144.66910164,1051.25299953)
\curveto(144.66910164,1052.29645873)(144.84223165,1053.34693671)(145.18849165,1054.40443347)
\curveto(145.53475165,1055.46193024)(146.06116044,1056.54282295)(146.76771801,1057.6471116)
\closepath
}
}
{
\newrgbcolor{curcolor}{0 0 0}
\pscustom[linestyle=none,fillstyle=solid,fillcolor=curcolor]
{
\newpath
\moveto(150.24201601,1057.21896579)
\lineto(153.8917836,1045.4063526)
\lineto(152.69859035,1045.4063526)
\lineto(149.04882276,1057.21896579)
\closepath
}
}
{
\newrgbcolor{curcolor}{0 0 0}
\pscustom[linestyle=none,fillstyle=solid,fillcolor=curcolor]
{
\newpath
\moveto(159.56997972,1056.25739241)
\curveto(158.54055809,1056.25739241)(157.72169998,1055.87369889)(157.11340538,1055.10631186)
\curveto(156.50978997,1054.33892482)(156.20798227,1053.29312603)(156.20798227,1051.96891548)
\curveto(156.20798227,1050.64938412)(156.50978997,1049.60592493)(157.11340538,1048.83853789)
\curveto(157.72169998,1048.07115086)(158.54055809,1047.68745734)(159.56997972,1047.68745734)
\curveto(160.59940135,1047.68745734)(161.41358028,1048.07115086)(162.0125165,1048.83853789)
\curveto(162.61613191,1049.60592493)(162.91793961,1050.64938412)(162.91793961,1051.96891548)
\curveto(162.91793961,1053.29312603)(162.61613191,1054.33892482)(162.0125165,1055.10631186)
\curveto(161.41358028,1055.87369889)(160.59940135,1056.25739241)(159.56997972,1056.25739241)
\closepath
\moveto(159.56997972,1057.40847295)
\curveto(161.03924514,1057.40847295)(162.21372163,1056.91481849)(163.09340921,1055.92750957)
\curveto(163.97309678,1054.94487983)(164.41294057,1053.62534847)(164.41294057,1051.96891548)
\curveto(164.41294057,1050.31716169)(163.97309678,1048.99763033)(163.09340921,1048.0103214)
\curveto(162.21372163,1047.02769166)(161.03924514,1046.5363768)(159.56997972,1046.5363768)
\curveto(158.09603512,1046.5363768)(156.91687943,1047.02769166)(156.03251267,1048.0103214)
\curveto(155.1528251,1048.99295114)(154.71298131,1050.3124825)(154.71298131,1051.96891548)
\curveto(154.71298131,1053.62534847)(155.1528251,1054.94487983)(156.03251267,1055.92750957)
\curveto(156.91687943,1056.91481849)(158.09603512,1057.40847295)(159.56997972,1057.40847295)
\closepath
}
}
{
\newrgbcolor{curcolor}{0 0 0}
\pscustom[linestyle=none,fillstyle=solid,fillcolor=curcolor]
{
\newpath
\moveto(167.82406951,1047.91907721)
\lineto(167.82406951,1043.74991961)
\lineto(166.5255945,1043.74991961)
\lineto(166.5255945,1054.60095942)
\lineto(167.82406951,1054.60095942)
\lineto(167.82406951,1053.40776617)
\curveto(168.09546249,1053.87568509)(168.4370433,1054.22194509)(168.84881195,1054.44654618)
\curveto(169.26525979,1054.67582645)(169.76125385,1054.79046658)(170.33679412,1054.79046658)
\curveto(171.29134872,1054.79046658)(172.06575454,1054.41145226)(172.66001157,1053.6534236)
\curveto(173.25894779,1052.89539495)(173.5584159,1051.89872765)(173.5584159,1050.66342169)
\curveto(173.5584159,1049.42811574)(173.25894779,1048.43144843)(172.66001157,1047.67341978)
\curveto(172.06575454,1046.91539112)(171.29134872,1046.5363768)(170.33679412,1046.5363768)
\curveto(169.76125385,1046.5363768)(169.26525979,1046.64867734)(168.84881195,1046.87327842)
\curveto(168.4370433,1047.10255869)(168.09546249,1047.45115829)(167.82406951,1047.91907721)
\closepath
\moveto(172.21782819,1050.66342169)
\curveto(172.21782819,1051.6132971)(172.02130224,1052.35728819)(171.62825035,1052.89539495)
\curveto(171.23987764,1053.4381809)(170.70411048,1053.70957387)(170.02094885,1053.70957387)
\curveto(169.33778722,1053.70957387)(168.79968046,1053.4381809)(168.40662857,1052.89539495)
\curveto(168.01825586,1052.35728819)(167.82406951,1051.6132971)(167.82406951,1050.66342169)
\curveto(167.82406951,1049.71354628)(168.01825586,1048.9672156)(168.40662857,1048.42442965)
\curveto(168.79968046,1047.88632289)(169.33778722,1047.61726951)(170.02094885,1047.61726951)
\curveto(170.70411048,1047.61726951)(171.23987764,1047.88632289)(171.62825035,1048.42442965)
\curveto(172.02130224,1048.9672156)(172.21782819,1049.71354628)(172.21782819,1050.66342169)
\closepath
}
}
{
\newrgbcolor{curcolor}{0 0 0}
\pscustom[linestyle=none,fillstyle=solid,fillcolor=curcolor]
{
\newpath
\moveto(175.49560024,1057.6471116)
\lineto(176.61860565,1057.6471116)
\curveto(177.32048404,1056.54282295)(177.84455323,1055.46193024)(178.19081323,1054.40443347)
\curveto(178.54175243,1053.34693671)(178.71722202,1052.29645873)(178.71722202,1051.25299953)
\curveto(178.71722202,1050.20486115)(178.54175243,1049.14970398)(178.19081323,1048.08752802)
\curveto(177.84455323,1047.02535207)(177.32048404,1045.94445936)(176.61860565,1044.84484989)
\lineto(175.49560024,1044.84484989)
\curveto(176.11793241,1045.91638422)(176.58117214,1046.98089977)(176.88531944,1048.03839654)
\curveto(177.19414593,1049.10057249)(177.34855917,1050.17210682)(177.34855917,1051.25299953)
\curveto(177.34855917,1052.33389224)(177.19414593,1053.40074739)(176.88531944,1054.45356496)
\curveto(176.58117214,1055.50638254)(176.11793241,1056.57089808)(175.49560024,1057.6471116)
\closepath
}
}
{
\newrgbcolor{curcolor}{0 0 0}
\pscustom[linestyle=none,fillstyle=solid,fillcolor=curcolor]
{
\newpath
\moveto(184.81654517,1044.62726759)
\lineto(184.11466679,1044.62726759)
\lineto(184.10764801,1046.73992153)
\curveto(183.61633314,1046.74927991)(183.12501827,1046.80543018)(182.6337034,1046.90837234)
\curveto(182.14238853,1047.01599369)(181.64873407,1047.17508613)(181.15274001,1047.38564964)
\lineto(181.15274001,1048.64903073)
\curveto(181.63001731,1048.34956262)(182.1119738,1048.12262194)(182.59860948,1047.9682087)
\curveto(183.08992435,1047.81847464)(183.59527679,1047.74126802)(184.11466679,1047.73658883)
\lineto(184.11466679,1050.93715426)
\curveto(183.08056597,1051.10560507)(182.32721651,1051.39103561)(181.85461839,1051.79344589)
\curveto(181.38669947,1052.19585616)(181.15274001,1052.74800049)(181.15274001,1053.44987887)
\curveto(181.15274001,1054.21258672)(181.40775582,1054.81386253)(181.91778745,1055.25370632)
\curveto(182.42781907,1055.6935501)(183.16011219,1055.94622632)(184.11466679,1056.01173497)
\lineto(184.11466679,1057.66114917)
\lineto(184.81654517,1057.66114917)
\lineto(184.81654517,1056.03279132)
\curveto(185.25170977,1056.01407457)(185.6728368,1055.96728267)(186.07992626,1055.89241565)
\curveto(186.48701572,1055.82222781)(186.88474681,1055.72396483)(187.27311951,1055.59762673)
\lineto(187.27311951,1054.36933955)
\curveto(186.88474681,1054.5658655)(186.48467613,1054.71793915)(186.07290748,1054.8255605)
\curveto(185.66581802,1054.93318186)(185.24703058,1054.99635091)(184.81654517,1055.01506767)
\lineto(184.81654517,1052.01804697)
\curveto(185.87872113,1051.85427535)(186.66014573,1051.56182602)(187.16081897,1051.14069899)
\curveto(187.66149222,1050.71957196)(187.91182884,1050.14403169)(187.91182884,1049.41407817)
\curveto(187.91182884,1048.62329519)(187.64511506,1047.99862343)(187.11168749,1047.54006288)
\curveto(186.5829391,1047.08618153)(185.81789167,1046.82414693)(184.81654517,1046.7539591)
\closepath
\moveto(184.11466679,1052.14438508)
\lineto(184.11466679,1055.02208645)
\curveto(183.57188084,1054.96125699)(183.15777259,1054.80684375)(182.87234205,1054.55884672)
\curveto(182.58691151,1054.31084969)(182.44419624,1053.98096685)(182.44419624,1053.5691982)
\curveto(182.44419624,1053.16678792)(182.57521353,1052.85328225)(182.83724813,1052.62868116)
\curveto(183.10396192,1052.40408008)(183.52976814,1052.24264805)(184.11466679,1052.14438508)
\closepath
\moveto(184.81654517,1050.79677858)
\lineto(184.81654517,1047.75764518)
\curveto(185.4108022,1047.8371914)(185.85766477,1048.00564221)(186.15713288,1048.26299762)
\curveto(186.46128018,1048.52035303)(186.61335383,1048.85959425)(186.61335383,1049.28072128)
\curveto(186.61335383,1049.69248993)(186.46829897,1050.02003317)(186.17818924,1050.26335101)
\curveto(185.89275869,1050.50666885)(185.43887734,1050.68447804)(184.81654517,1050.79677858)
\closepath
}
}
{
\newrgbcolor{curcolor}{0 0 0}
\pscustom[linestyle=none,fillstyle=solid,fillcolor=curcolor]
{
\newpath
\moveto(436.28399413,1055.25480265)
\lineto(435.68961776,1055.25480265)
\lineto(435.683674,1057.04387551)
\curveto(435.26761054,1057.05180053)(434.85154709,1057.09935064)(434.43548363,1057.18652584)
\curveto(434.01942017,1057.27766355)(433.60137546,1057.41238886)(433.1813495,1057.59070177)
\lineto(433.1813495,1058.66057923)
\curveto(433.58552543,1058.40697864)(433.99366386,1058.21479695)(434.40576481,1058.08403415)
\curveto(434.82182827,1057.95723386)(435.24977925,1057.89185246)(435.68961776,1057.88788995)
\lineto(435.68961776,1060.59824618)
\curveto(434.81390325,1060.74089651)(434.17593928,1060.98260956)(433.77572586,1061.32338534)
\curveto(433.37947495,1061.66416113)(433.1813495,1062.1317372)(433.1813495,1062.72611357)
\curveto(433.1813495,1063.37200255)(433.39730624,1063.88118497)(433.82921974,1064.25366083)
\curveto(434.26113323,1064.62613668)(434.8812659,1064.84011217)(435.68961776,1064.8955873)
\lineto(435.68961776,1066.29237176)
\lineto(436.28399413,1066.29237176)
\lineto(436.28399413,1064.91341859)
\curveto(436.65250747,1064.89756856)(437.00913329,1064.85794347)(437.35387158,1064.79454332)
\curveto(437.69860988,1064.73510568)(438.03542315,1064.65189299)(438.36431141,1064.54490525)
\lineto(438.36431141,1063.50474661)
\curveto(438.03542315,1063.67117199)(437.69662862,1063.79995353)(437.34792782,1063.89109124)
\curveto(437.00318953,1063.98222895)(436.64854496,1064.03572283)(436.28399413,1064.05157286)
\lineto(436.28399413,1061.51358578)
\curveto(437.18348369,1061.37489796)(437.84522271,1061.12724114)(438.26921119,1060.77061532)
\curveto(438.69319966,1060.4139895)(438.9051939,1059.92660088)(438.9051939,1059.30844946)
\curveto(438.9051939,1058.63878543)(438.67933088,1058.10979046)(438.22760484,1057.72146457)
\curveto(437.77984131,1057.33710119)(437.13197107,1057.11520068)(436.28399413,1057.05576304)
\closepath
\moveto(435.68961776,1061.62057353)
\lineto(435.68961776,1064.05751663)
\curveto(435.2299667,1064.00600401)(434.87928465,1063.87524121)(434.63757159,1063.66522822)
\curveto(434.39585854,1063.45521524)(434.27500201,1063.17585835)(434.27500201,1062.82715755)
\curveto(434.27500201,1062.48638177)(434.38595227,1062.22089366)(434.60785278,1062.03069322)
\curveto(434.83371579,1061.84049278)(435.19430412,1061.70378622)(435.68961776,1061.62057353)
\closepath
\moveto(436.28399413,1060.47937091)
\lineto(436.28399413,1057.90572124)
\curveto(436.78723278,1057.9730839)(437.1656524,1058.11573422)(437.41925298,1058.33367223)
\curveto(437.67681608,1058.55161023)(437.80559762,1058.83889214)(437.80559762,1059.19551796)
\curveto(437.80559762,1059.54421876)(437.68275984,1059.82159439)(437.43708428,1060.02764487)
\curveto(437.19537122,1060.23369534)(436.81100784,1060.38427069)(436.28399413,1060.47937091)
\closepath
}
}
{
\newrgbcolor{curcolor}{0 0 0}
\pscustom[linestyle=none,fillstyle=solid,fillcolor=curcolor]
{
\newpath
\moveto(440.93201731,1065.91791465)
\lineto(444.02277441,1055.91456042)
\lineto(443.01233459,1055.91456042)
\lineto(439.92157748,1065.91791465)
\closepath
}
}
{
\newrgbcolor{curcolor}{0 0 0}
\pscustom[linestyle=none,fillstyle=solid,fillcolor=curcolor]
{
\newpath
\moveto(446.42999869,1064.93124988)
\lineto(446.42999869,1061.59679847)
\lineto(447.93971466,1061.59679847)
\curveto(448.49842844,1061.59679847)(448.93034193,1061.74143005)(449.23545513,1062.03069322)
\curveto(449.54056833,1062.31995638)(449.69312494,1062.73205733)(449.69312494,1063.26699606)
\curveto(449.69312494,1063.79797228)(449.54056833,1064.20809197)(449.23545513,1064.49735514)
\curveto(448.93034193,1064.7866183)(448.49842844,1064.93124988)(447.93971466,1064.93124988)
\closepath
\moveto(445.22935843,1065.91791465)
\lineto(447.93971466,1065.91791465)
\curveto(448.93430444,1065.91791465)(449.68519992,1065.69205163)(450.19240108,1065.24032559)
\curveto(450.70356476,1064.79256207)(450.95914659,1064.13478555)(450.95914659,1063.26699606)
\curveto(450.95914659,1062.39128155)(450.70356476,1061.72954253)(450.19240108,1061.281779)
\curveto(449.68519992,1060.83401547)(448.93430444,1060.61013371)(447.93971466,1060.61013371)
\lineto(446.42999869,1060.61013371)
\lineto(446.42999869,1057.04387551)
\lineto(445.22935843,1057.04387551)
\closepath
}
}
{
\newrgbcolor{curcolor}{0 0 0}
\pscustom[linestyle=none,fillstyle=solid,fillcolor=curcolor]
{
\newpath
\moveto(451.34549123,1065.91791465)
\lineto(458.85246473,1065.91791465)
\lineto(458.85246473,1064.90747483)
\lineto(455.70226999,1064.90747483)
\lineto(455.70226999,1057.04387551)
\lineto(454.49568597,1057.04387551)
\lineto(454.49568597,1064.90747483)
\lineto(451.34549123,1064.90747483)
\closepath
}
}
{
\newrgbcolor{curcolor}{0 0 0}
\pscustom[linestyle=none,fillstyle=solid,fillcolor=curcolor]
{
\newpath
\moveto(462.93583036,1055.25480265)
\lineto(462.341454,1055.25480265)
\lineto(462.33551023,1057.04387551)
\curveto(461.91944678,1057.05180053)(461.50338332,1057.09935064)(461.08731986,1057.18652584)
\curveto(460.67125641,1057.27766355)(460.2532117,1057.41238886)(459.83318573,1057.59070177)
\lineto(459.83318573,1058.66057923)
\curveto(460.23736166,1058.40697864)(460.6455001,1058.21479695)(461.05760105,1058.08403415)
\curveto(461.4736645,1057.95723386)(461.90161549,1057.89185246)(462.341454,1057.88788995)
\lineto(462.341454,1060.59824618)
\curveto(461.46573948,1060.74089651)(460.82777552,1060.98260956)(460.4275621,1061.32338534)
\curveto(460.03131119,1061.66416113)(459.83318573,1062.1317372)(459.83318573,1062.72611357)
\curveto(459.83318573,1063.37200255)(460.04914248,1063.88118497)(460.48105597,1064.25366083)
\curveto(460.91296946,1064.62613668)(461.53310214,1064.84011217)(462.341454,1064.8955873)
\lineto(462.341454,1066.29237176)
\lineto(462.93583036,1066.29237176)
\lineto(462.93583036,1064.91341859)
\curveto(463.30434371,1064.89756856)(463.66096953,1064.85794347)(464.00570782,1064.79454332)
\curveto(464.35044611,1064.73510568)(464.68725939,1064.65189299)(465.01614764,1064.54490525)
\lineto(465.01614764,1063.50474661)
\curveto(464.68725939,1063.67117199)(464.34846486,1063.79995353)(463.99976406,1063.89109124)
\curveto(463.65502576,1063.98222895)(463.3003812,1064.03572283)(462.93583036,1064.05157286)
\lineto(462.93583036,1061.51358578)
\curveto(463.83531993,1061.37489796)(464.49705895,1061.12724114)(464.92104742,1060.77061532)
\curveto(465.3450359,1060.4139895)(465.55703013,1059.92660088)(465.55703013,1059.30844946)
\curveto(465.55703013,1058.63878543)(465.33116712,1058.10979046)(464.87944108,1057.72146457)
\curveto(464.43167755,1057.33710119)(463.78380731,1057.11520068)(462.93583036,1057.05576304)
\closepath
\moveto(462.341454,1061.62057353)
\lineto(462.341454,1064.05751663)
\curveto(461.88180294,1064.00600401)(461.53112088,1063.87524121)(461.28940783,1063.66522822)
\curveto(461.04769477,1063.45521524)(460.92683825,1063.17585835)(460.92683825,1062.82715755)
\curveto(460.92683825,1062.48638177)(461.0377885,1062.22089366)(461.25968901,1062.03069322)
\curveto(461.48555203,1061.84049278)(461.84614036,1061.70378622)(462.341454,1061.62057353)
\closepath
\moveto(462.93583036,1060.47937091)
\lineto(462.93583036,1057.90572124)
\curveto(463.43906902,1057.9730839)(463.81748864,1058.11573422)(464.07108922,1058.33367223)
\curveto(464.32865231,1058.55161023)(464.45743386,1058.83889214)(464.45743386,1059.19551796)
\curveto(464.45743386,1059.54421876)(464.33459608,1059.82159439)(464.08892051,1060.02764487)
\curveto(463.84720746,1060.23369534)(463.46284407,1060.38427069)(462.93583036,1060.47937091)
\closepath
}
}
{
\newrgbcolor{curcolor}{0 0 0}
\pscustom[linestyle=none,fillstyle=solid,fillcolor=curcolor]
{
\newpath
\moveto(467.16779009,1060.86571554)
\lineto(470.3714787,1060.86571554)
\lineto(470.3714787,1059.8909383)
\lineto(467.16779009,1059.8909383)
\closepath
}
}
{
\newrgbcolor{curcolor}{0 0 0}
\pscustom[linestyle=none,fillstyle=solid,fillcolor=curcolor]
{
\newpath
\moveto(472.02978876,1065.91791465)
\lineto(473.23637278,1065.91791465)
\lineto(473.23637278,1060.52692101)
\curveto(473.23637278,1059.57591883)(473.40874192,1058.89040475)(473.75348022,1058.47037879)
\curveto(474.09821851,1058.05431533)(474.65693229,1057.84628361)(475.42962157,1057.84628361)
\curveto(476.19834833,1057.84628361)(476.75508086,1058.05431533)(477.09981916,1058.47037879)
\curveto(477.44455745,1058.89040475)(477.61692659,1059.57591883)(477.61692659,1060.52692101)
\lineto(477.61692659,1065.91791465)
\lineto(478.82351062,1065.91791465)
\lineto(478.82351062,1060.37832692)
\curveto(478.82351062,1059.22127426)(478.53622871,1058.34754101)(477.96166489,1057.75712715)
\curveto(477.39106357,1057.16671329)(476.54704914,1056.87150637)(475.42962157,1056.87150637)
\curveto(474.30823149,1056.87150637)(473.46025454,1057.16671329)(472.88569072,1057.75712715)
\curveto(472.31508941,1058.34754101)(472.02978876,1059.22127426)(472.02978876,1060.37832692)
\closepath
}
}
{
\newrgbcolor{curcolor}{0 0 0}
\pscustom[linestyle=none,fillstyle=solid,fillcolor=curcolor]
{
\newpath
\moveto(486.56823466,1061.06185974)
\lineto(486.56823466,1057.04387551)
\lineto(485.47458215,1057.04387551)
\lineto(485.47458215,1061.02619716)
\curveto(485.47458215,1061.65623611)(485.35174437,1062.12777469)(485.1060688,1062.44081291)
\curveto(484.86039324,1062.75385113)(484.49187989,1062.91037024)(484.00052876,1062.91037024)
\curveto(483.4101149,1062.91037024)(482.94452008,1062.72215106)(482.6037443,1062.34571269)
\curveto(482.26296852,1061.96927433)(482.09258063,1061.4561294)(482.09258063,1060.80627791)
\lineto(482.09258063,1057.04387551)
\lineto(480.99298435,1057.04387551)
\lineto(480.99298435,1063.70089081)
\lineto(482.09258063,1063.70089081)
\lineto(482.09258063,1062.66667593)
\curveto(482.35410623,1063.06688935)(482.66120068,1063.36605879)(483.01386399,1063.56418424)
\curveto(483.37048981,1063.7623097)(483.78060951,1063.86137243)(484.24422307,1063.86137243)
\curveto(485.00898733,1063.86137243)(485.58751366,1063.62362188)(485.97980206,1063.14812079)
\curveto(486.37209046,1062.6765822)(486.56823466,1061.98116186)(486.56823466,1061.06185974)
\closepath
}
}
{
\newrgbcolor{curcolor}{0 0 0}
\pscustom[linestyle=none,fillstyle=solid,fillcolor=curcolor]
{
\newpath
\moveto(488.76148345,1063.70089081)
\lineto(489.85513596,1063.70089081)
\lineto(489.85513596,1057.04387551)
\lineto(488.76148345,1057.04387551)
\closepath
\moveto(488.76148345,1066.29237176)
\lineto(489.85513596,1066.29237176)
\lineto(489.85513596,1064.90747483)
\lineto(488.76148345,1064.90747483)
\closepath
}
}
{
\newrgbcolor{curcolor}{0 0 0}
\pscustom[linestyle=none,fillstyle=solid,fillcolor=curcolor]
{
\newpath
\moveto(495.5076552,1066.29237176)
\lineto(495.5076552,1065.38297592)
\lineto(494.4615528,1065.38297592)
\curveto(494.06926439,1065.38297592)(493.79585127,1065.30372574)(493.64131341,1065.14522538)
\curveto(493.49073807,1064.98672501)(493.41545039,1064.70142436)(493.41545039,1064.28932341)
\lineto(493.41545039,1063.70089081)
\lineto(495.21641078,1063.70089081)
\lineto(495.21641078,1062.8509326)
\lineto(493.41545039,1062.8509326)
\lineto(493.41545039,1057.04387551)
\lineto(492.31585412,1057.04387551)
\lineto(492.31585412,1062.8509326)
\lineto(491.26975171,1062.8509326)
\lineto(491.26975171,1063.70089081)
\lineto(492.31585412,1063.70089081)
\lineto(492.31585412,1064.16450437)
\curveto(492.31585412,1064.90549357)(492.48822326,1065.44439481)(492.83296155,1065.78120809)
\curveto(493.17769985,1066.12198387)(493.7245261,1066.29237176)(494.47344032,1066.29237176)
\closepath
}
}
{
\newrgbcolor{curcolor}{0 0 0}
\pscustom[linestyle=none,fillstyle=solid,fillcolor=curcolor]
{
\newpath
\moveto(498.99664447,1062.9341453)
\curveto(498.41019312,1062.9341453)(497.94657955,1062.70431977)(497.60580377,1062.24466871)
\curveto(497.26502799,1061.78898016)(497.0946401,1061.16290373)(497.0946401,1060.3664394)
\curveto(497.0946401,1059.56997507)(497.26304673,1058.94191737)(497.59986001,1058.48226632)
\curveto(497.94063579,1058.02657777)(498.40623061,1057.7987335)(498.99664447,1057.7987335)
\curveto(499.5791333,1057.7987335)(500.04076561,1058.02855902)(500.3815414,1058.48821008)
\curveto(500.72231718,1058.94786114)(500.89270507,1059.57393757)(500.89270507,1060.3664394)
\curveto(500.89270507,1061.15497871)(500.72231718,1061.77907389)(500.3815414,1062.23872495)
\curveto(500.04076561,1062.70233851)(499.5791333,1062.9341453)(498.99664447,1062.9341453)
\closepath
\moveto(498.99664447,1063.86137243)
\curveto(499.94764665,1063.86137243)(500.69457962,1063.55229672)(501.23744336,1062.9341453)
\curveto(501.78030711,1062.31599387)(502.05173899,1061.46009191)(502.05173899,1060.3664394)
\curveto(502.05173899,1059.27674939)(501.78030711,1058.42084743)(501.23744336,1057.7987335)
\curveto(500.69457962,1057.18058208)(499.94764665,1056.87150637)(498.99664447,1056.87150637)
\curveto(498.04167977,1056.87150637)(497.29276555,1057.18058208)(496.7499018,1057.7987335)
\curveto(496.21100057,1058.42084743)(495.94154995,1059.27674939)(495.94154995,1060.3664394)
\curveto(495.94154995,1061.46009191)(496.21100057,1062.31599387)(496.7499018,1062.9341453)
\curveto(497.29276555,1063.55229672)(498.04167977,1063.86137243)(498.99664447,1063.86137243)
\closepath
}
}
{
\newrgbcolor{curcolor}{0 0 0}
\pscustom[linestyle=none,fillstyle=solid,fillcolor=curcolor]
{
\newpath
\moveto(507.71614575,1062.67856346)
\curveto(507.59330797,1062.74988862)(507.45858266,1062.80140124)(507.31196982,1062.83310131)
\curveto(507.16931949,1062.86876389)(507.01081913,1062.88659519)(506.83646873,1062.88659519)
\curveto(506.21831731,1062.88659519)(505.74281622,1062.68450722)(505.40996545,1062.28033129)
\curveto(505.0810772,1061.88011787)(504.91663307,1061.3035728)(504.91663307,1060.55069607)
\lineto(504.91663307,1057.04387551)
\lineto(503.81703679,1057.04387551)
\lineto(503.81703679,1063.70089081)
\lineto(504.91663307,1063.70089081)
\lineto(504.91663307,1062.66667593)
\curveto(505.1464586,1063.07085186)(505.44562803,1063.3700213)(505.81414138,1063.56418424)
\curveto(506.18265473,1063.7623097)(506.63041826,1063.86137243)(507.15743197,1063.86137243)
\curveto(507.23271964,1063.86137243)(507.31593233,1063.85542866)(507.40707004,1063.84354113)
\curveto(507.49820775,1063.83561612)(507.59925173,1063.82174733)(507.71020199,1063.80193479)
\closepath
}
}
{
\newrgbcolor{curcolor}{0 0 0}
\pscustom[linestyle=none,fillstyle=solid,fillcolor=curcolor]
{
\newpath
\moveto(513.84416608,1062.42298162)
\curveto(514.11757921,1062.91433275)(514.44448621,1063.27690233)(514.82488708,1063.51069037)
\curveto(515.20528796,1063.74447841)(515.65305149,1063.86137243)(516.16817767,1063.86137243)
\curveto(516.86161676,1063.86137243)(517.39655549,1063.61767812)(517.77299386,1063.1302895)
\curveto(518.14943222,1062.64686339)(518.3376514,1061.9573868)(518.3376514,1061.06185974)
\lineto(518.3376514,1057.04387551)
\lineto(517.23805513,1057.04387551)
\lineto(517.23805513,1061.02619716)
\curveto(517.23805513,1061.66416113)(517.12512362,1062.13768097)(516.8992606,1062.44675668)
\curveto(516.67339758,1062.75583239)(516.32865929,1062.91037024)(515.86504572,1062.91037024)
\curveto(515.29840692,1062.91037024)(514.85064339,1062.72215106)(514.52175514,1062.34571269)
\curveto(514.19286688,1061.96927433)(514.02842275,1061.4561294)(514.02842275,1060.80627791)
\lineto(514.02842275,1057.04387551)
\lineto(512.92882648,1057.04387551)
\lineto(512.92882648,1061.02619716)
\curveto(512.92882648,1061.66812364)(512.81589497,1062.14164347)(512.59003195,1062.44675668)
\curveto(512.36416893,1062.75583239)(512.01546813,1062.91037024)(511.54392954,1062.91037024)
\curveto(510.98521576,1062.91037024)(510.54141474,1062.7201698)(510.21252649,1062.33976893)
\curveto(509.88363823,1061.96333056)(509.7191941,1061.45216689)(509.7191941,1060.80627791)
\lineto(509.7191941,1057.04387551)
\lineto(508.61959783,1057.04387551)
\lineto(508.61959783,1063.70089081)
\lineto(509.7191941,1063.70089081)
\lineto(509.7191941,1062.66667593)
\curveto(509.96883218,1063.07481437)(510.26800161,1063.37596506)(510.61670241,1063.57012801)
\curveto(510.96540322,1063.76429095)(511.37948542,1063.86137243)(511.85894902,1063.86137243)
\curveto(512.34237513,1063.86137243)(512.75249482,1063.73853464)(513.0893081,1063.49285908)
\curveto(513.43008388,1063.24718351)(513.68170321,1062.8905577)(513.84416608,1062.42298162)
\closepath
}
}
{
\newrgbcolor{curcolor}{0 0 0}
\pscustom[linestyle=none,fillstyle=solid,fillcolor=curcolor]
{
\newpath
\moveto(396.40133999,1030.90914671)
\lineto(395.80696363,1030.90914671)
\lineto(395.80101986,1032.69821957)
\curveto(395.38495641,1032.70614459)(394.96889295,1032.7536947)(394.55282949,1032.8408699)
\curveto(394.13676604,1032.93200761)(393.71872133,1033.06673292)(393.29869536,1033.24504583)
\lineto(393.29869536,1034.31492329)
\curveto(393.70287129,1034.06132271)(394.11100973,1033.86914101)(394.52311068,1033.73837821)
\curveto(394.93917413,1033.61157792)(395.36712511,1033.54619652)(395.80696363,1033.54223401)
\lineto(395.80696363,1036.25259024)
\curveto(394.93124911,1036.39524057)(394.29328515,1036.63695362)(393.89307173,1036.97772941)
\curveto(393.49682082,1037.31850519)(393.29869536,1037.78608126)(393.29869536,1038.38045763)
\curveto(393.29869536,1039.02634661)(393.51465211,1039.53552903)(393.9465656,1039.90800489)
\curveto(394.37847909,1040.28048075)(394.99861177,1040.49445624)(395.80696363,1040.54993136)
\lineto(395.80696363,1041.94671582)
\lineto(396.40133999,1041.94671582)
\lineto(396.40133999,1040.56776266)
\curveto(396.76985334,1040.55191262)(397.12647916,1040.51228753)(397.47121745,1040.44888738)
\curveto(397.81595574,1040.38944975)(398.15276902,1040.30623705)(398.48165727,1040.19924931)
\lineto(398.48165727,1039.15909067)
\curveto(398.15276902,1039.32551605)(397.81397449,1039.4542976)(397.46527369,1039.54543531)
\curveto(397.12053539,1039.63657302)(396.76589083,1039.69006689)(396.40133999,1039.70591693)
\lineto(396.40133999,1037.16792984)
\curveto(397.30082956,1037.02924203)(397.96256858,1036.78158521)(398.38655705,1036.42495939)
\curveto(398.81054553,1036.06833357)(399.02253976,1035.58094495)(399.02253976,1034.96279353)
\curveto(399.02253976,1034.29312949)(398.79667674,1033.76413452)(398.34495071,1033.37580863)
\curveto(397.89718718,1032.99144525)(397.24931694,1032.76954474)(396.40133999,1032.7101071)
\closepath
\moveto(395.80696363,1037.27491759)
\lineto(395.80696363,1039.71186069)
\curveto(395.34731257,1039.66034807)(394.99663051,1039.52958527)(394.75491746,1039.31957229)
\curveto(394.5132044,1039.10955931)(394.39234788,1038.83020241)(394.39234788,1038.48150161)
\curveto(394.39234788,1038.14072583)(394.50329813,1037.87523772)(394.72519864,1037.68503728)
\curveto(394.95106166,1037.49483685)(395.31164999,1037.35813028)(395.80696363,1037.27491759)
\closepath
\moveto(396.40133999,1036.13371497)
\lineto(396.40133999,1033.5600653)
\curveto(396.90457865,1033.62742796)(397.28299827,1033.77007829)(397.53659885,1033.98801629)
\curveto(397.79416194,1034.20595429)(397.92294349,1034.4932362)(397.92294349,1034.84986202)
\curveto(397.92294349,1035.19856282)(397.80010571,1035.47593846)(397.55443014,1035.68198893)
\curveto(397.31271709,1035.8880394)(396.9283537,1036.03861475)(396.40133999,1036.13371497)
\closepath
}
}
{
\newrgbcolor{curcolor}{0 0 0}
\pscustom[linestyle=none,fillstyle=solid,fillcolor=curcolor]
{
\newpath
\moveto(401.04936317,1041.57225871)
\lineto(404.14012027,1031.56890448)
\lineto(403.12968045,1031.56890448)
\lineto(400.03892335,1041.57225871)
\closepath
}
}
{
\newrgbcolor{curcolor}{0 0 0}
\pscustom[linestyle=none,fillstyle=solid,fillcolor=curcolor]
{
\newpath
\moveto(405.34670429,1041.57225871)
\lineto(410.44645351,1041.57225871)
\lineto(410.44645351,1040.56181889)
\lineto(406.54734455,1040.56181889)
\lineto(406.54734455,1037.94656288)
\lineto(410.06605264,1037.94656288)
\lineto(410.06605264,1036.93612306)
\lineto(406.54734455,1036.93612306)
\lineto(406.54734455,1032.69821957)
\lineto(405.34670429,1032.69821957)
\closepath
}
}
{
\newrgbcolor{curcolor}{0 0 0}
\pscustom[linestyle=none,fillstyle=solid,fillcolor=curcolor]
{
\newpath
\moveto(415.96226618,1040.75796309)
\curveto(415.09051418,1040.75796309)(414.39707509,1040.43303735)(413.8819489,1039.78318585)
\curveto(413.37078523,1039.13333436)(413.11520339,1038.24771358)(413.11520339,1037.1263235)
\curveto(413.11520339,1036.00889593)(413.37078523,1035.1252564)(413.8819489,1034.47540491)
\curveto(414.39707509,1033.82555341)(415.09051418,1033.50062767)(415.96226618,1033.50062767)
\curveto(416.83401819,1033.50062767)(417.52349477,1033.82555341)(418.03069594,1034.47540491)
\curveto(418.54185961,1035.1252564)(418.79744145,1036.00889593)(418.79744145,1037.1263235)
\curveto(418.79744145,1038.24771358)(418.54185961,1039.13333436)(418.03069594,1039.78318585)
\curveto(417.52349477,1040.43303735)(416.83401819,1040.75796309)(415.96226618,1040.75796309)
\closepath
\moveto(415.96226618,1041.73274033)
\curveto(417.20649404,1041.73274033)(418.20108383,1041.31469562)(418.94603554,1040.4786062)
\curveto(419.69098725,1039.64647929)(420.06346311,1038.52905172)(420.06346311,1037.1263235)
\curveto(420.06346311,1035.72755778)(419.69098725,1034.61013022)(418.94603554,1033.7740408)
\curveto(418.20108383,1032.94191388)(417.20649404,1032.52585043)(415.96226618,1032.52585043)
\curveto(414.71407582,1032.52585043)(413.71552352,1032.94191388)(412.9666093,1033.7740408)
\curveto(412.22165759,1034.60616771)(411.84918173,1035.72359528)(411.84918173,1037.1263235)
\curveto(411.84918173,1038.52905172)(412.22165759,1039.64647929)(412.9666093,1040.4786062)
\curveto(413.71552352,1041.31469562)(414.71407582,1041.73274033)(415.96226618,1041.73274033)
\closepath
}
}
{
\newrgbcolor{curcolor}{0 0 0}
\pscustom[linestyle=none,fillstyle=solid,fillcolor=curcolor]
{
\newpath
\moveto(426.42923398,1041.57225871)
\lineto(429.65669765,1038.26158236)
\lineto(428.46200115,1038.26158236)
\lineto(425.84674515,1040.609369)
\lineto(423.23148914,1038.26158236)
\lineto(422.03679264,1038.26158236)
\lineto(425.26425631,1041.57225871)
\closepath
}
}
{
\newrgbcolor{curcolor}{0 0 0}
\pscustom[linestyle=none,fillstyle=solid,fillcolor=curcolor]
{
\newpath
\moveto(437.16961491,1031.56890448)
\lineto(437.16961491,1030.71300251)
\lineto(436.80110156,1030.71300251)
\curveto(435.8144368,1030.71300251)(435.15269778,1030.85961535)(434.8158845,1031.15284102)
\curveto(434.48303374,1031.4460667)(434.31660836,1032.03053679)(434.31660836,1032.9062513)
\lineto(434.31660836,1034.32681082)
\curveto(434.31660836,1034.92514969)(434.20962061,1035.33923189)(433.99564512,1035.56905742)
\curveto(433.78166963,1035.79888295)(433.39334373,1035.91379571)(432.83066744,1035.91379571)
\lineto(432.46809786,1035.91379571)
\lineto(432.46809786,1036.76375392)
\lineto(432.83066744,1036.76375392)
\curveto(433.39730624,1036.76375392)(433.78563214,1036.87668543)(433.99564512,1037.10254844)
\curveto(434.20962061,1037.33237397)(434.31660836,1037.74249366)(434.31660836,1038.33290752)
\lineto(434.31660836,1039.7594108)
\curveto(434.31660836,1040.63512531)(434.48303374,1041.21761415)(434.8158845,1041.50687731)
\curveto(435.15269778,1041.80010299)(435.8144368,1041.94671582)(436.80110156,1041.94671582)
\lineto(437.16961491,1041.94671582)
\lineto(437.16961491,1041.09675762)
\lineto(436.76543898,1041.09675762)
\curveto(436.2067252,1041.09675762)(435.84217436,1041.00958242)(435.67178647,1040.83523202)
\curveto(435.50139858,1040.66088162)(435.41620463,1040.29434953)(435.41620463,1039.73563574)
\lineto(435.41620463,1038.26158236)
\curveto(435.41620463,1037.63946843)(435.32506692,1037.18774239)(435.1427915,1036.90640424)
\curveto(434.96447859,1036.6250661)(434.65738414,1036.43486566)(434.22150814,1036.33580293)
\curveto(434.66134665,1036.22881519)(434.97042236,1036.03465224)(435.14873527,1035.75331409)
\curveto(435.32704818,1035.47197595)(435.41620463,1035.02223116)(435.41620463,1034.40407974)
\lineto(435.41620463,1032.93002636)
\curveto(435.41620463,1032.37131257)(435.50139858,1032.00478048)(435.67178647,1031.83043008)
\curveto(435.84217436,1031.65607968)(436.2067252,1031.56890448)(436.76543898,1031.56890448)
\closepath
}
}
{
\newrgbcolor{curcolor}{0 0 0}
\pscustom[linestyle=none,fillstyle=solid,fillcolor=curcolor]
{
\newpath
\moveto(439.70760199,1041.57225871)
\lineto(442.79835909,1031.56890448)
\lineto(441.78791927,1031.56890448)
\lineto(438.69716217,1041.57225871)
\closepath
}
}
{
\newrgbcolor{curcolor}{0 0 0}
\pscustom[linestyle=none,fillstyle=solid,fillcolor=curcolor]
{
\newpath
\moveto(449.49103697,1036.71620381)
\lineto(449.49103697,1032.69821957)
\lineto(448.39738446,1032.69821957)
\lineto(448.39738446,1036.68054122)
\curveto(448.39738446,1037.31058017)(448.27454668,1037.78211876)(448.02887111,1038.09515697)
\curveto(447.78319555,1038.40819519)(447.4146822,1038.5647143)(446.92333107,1038.5647143)
\curveto(446.33291721,1038.5647143)(445.8673224,1038.37649512)(445.52654661,1038.00005676)
\curveto(445.18577083,1037.62361839)(445.01538294,1037.11047346)(445.01538294,1036.46062197)
\lineto(445.01538294,1032.69821957)
\lineto(443.91578666,1032.69821957)
\lineto(443.91578666,1039.35523487)
\lineto(445.01538294,1039.35523487)
\lineto(445.01538294,1038.32101999)
\curveto(445.27690854,1038.72123341)(445.58400299,1039.02040285)(445.9366663,1039.21852831)
\curveto(446.29329212,1039.41665376)(446.70341182,1039.51571649)(447.16702538,1039.51571649)
\curveto(447.93178964,1039.51571649)(448.51031597,1039.27796594)(448.90260437,1038.80246485)
\curveto(449.29489277,1038.33092627)(449.49103697,1037.63550592)(449.49103697,1036.71620381)
\closepath
}
}
{
\newrgbcolor{curcolor}{0 0 0}
\pscustom[linestyle=none,fillstyle=solid,fillcolor=curcolor]
{
\newpath
\moveto(454.70966146,1036.04455851)
\curveto(453.82602193,1036.04455851)(453.21381427,1035.94351453)(452.87303849,1035.74142657)
\curveto(452.53226271,1035.5393386)(452.36187482,1035.19460031)(452.36187482,1034.70721169)
\curveto(452.36187482,1034.3188858)(452.48867511,1034.00981009)(452.74227569,1033.77998456)
\curveto(452.99983878,1033.55412154)(453.34853958,1033.44119003)(453.78837809,1033.44119003)
\curveto(454.39464199,1033.44119003)(454.88004935,1033.65516552)(455.24460019,1034.08311651)
\curveto(455.61311354,1034.51503)(455.79737021,1035.08761256)(455.79737021,1035.8008642)
\lineto(455.79737021,1036.04455851)
\closepath
\moveto(456.89102272,1036.49628455)
\lineto(456.89102272,1032.69821957)
\lineto(455.79737021,1032.69821957)
\lineto(455.79737021,1033.7086594)
\curveto(455.54773214,1033.30448347)(455.23667517,1033.00531403)(454.86419932,1032.81115108)
\curveto(454.49172346,1032.62095065)(454.03603491,1032.52585043)(453.49713368,1032.52585043)
\curveto(452.81558211,1032.52585043)(452.27271836,1032.71605087)(451.86854243,1033.09645174)
\curveto(451.46832901,1033.48081512)(451.2682223,1033.99396005)(451.2682223,1034.63588653)
\curveto(451.2682223,1035.38480075)(451.51786038,1035.94945829)(452.01713653,1036.32985917)
\curveto(452.52037518,1036.71026004)(453.2692894,1036.90046048)(454.26387919,1036.90046048)
\lineto(455.79737021,1036.90046048)
\lineto(455.79737021,1037.00744823)
\curveto(455.79737021,1037.51068688)(455.63094483,1037.89901277)(455.29809406,1038.1724259)
\curveto(454.96920581,1038.44980154)(454.50559224,1038.58848936)(453.90725337,1038.58848936)
\curveto(453.52685249,1038.58848936)(453.15635789,1038.5429205)(452.79576956,1038.45178279)
\curveto(452.43518124,1038.36064508)(452.08846169,1038.22393852)(451.75561092,1038.0416631)
\lineto(451.75561092,1039.05210292)
\curveto(452.15582434,1039.20664078)(452.54415024,1039.32155354)(452.9205886,1039.39684122)
\curveto(453.29702697,1039.4760914)(453.66355906,1039.51571649)(454.02018488,1039.51571649)
\curveto(454.98307459,1039.51571649)(455.70226999,1039.26607842)(456.17777108,1038.76680227)
\curveto(456.65327218,1038.26752612)(456.89102272,1037.51068688)(456.89102272,1036.49628455)
\closepath
}
}
{
\newrgbcolor{curcolor}{0 0 0}
\pscustom[linestyle=none,fillstyle=solid,fillcolor=curcolor]
{
\newpath
\moveto(460.2314179,1041.24535171)
\lineto(460.2314179,1039.35523487)
\lineto(462.48410432,1039.35523487)
\lineto(462.48410432,1038.50527667)
\lineto(460.2314179,1038.50527667)
\lineto(460.2314179,1034.89146836)
\curveto(460.2314179,1034.34860462)(460.30472432,1033.99990382)(460.45133715,1033.84536596)
\curveto(460.6019125,1033.69082811)(460.90504445,1033.61355918)(461.36073299,1033.61355918)
\lineto(462.48410432,1033.61355918)
\lineto(462.48410432,1032.69821957)
\lineto(461.36073299,1032.69821957)
\curveto(460.51671855,1032.69821957)(459.93422972,1032.85473868)(459.61326648,1033.1677769)
\curveto(459.29230324,1033.48477763)(459.13182162,1034.05934145)(459.13182162,1034.89146836)
\lineto(459.13182162,1038.50527667)
\lineto(458.32941353,1038.50527667)
\lineto(458.32941353,1039.35523487)
\lineto(459.13182162,1039.35523487)
\lineto(459.13182162,1041.24535171)
\closepath
}
}
{
\newrgbcolor{curcolor}{0 0 0}
\pscustom[linestyle=none,fillstyle=solid,fillcolor=curcolor]
{
\newpath
\moveto(468.17228614,1039.15909067)
\lineto(468.17228614,1038.12487579)
\curveto(467.86321043,1038.28337616)(467.5422472,1038.40225143)(467.20939643,1038.48150161)
\curveto(466.87654567,1038.56075179)(466.53180737,1038.60037689)(466.17518155,1038.60037689)
\curveto(465.63231781,1038.60037689)(465.22417937,1038.51716419)(464.95076624,1038.35073881)
\curveto(464.68131562,1038.18431343)(464.54659031,1037.93467536)(464.54659031,1037.60182459)
\curveto(464.54659031,1037.34822401)(464.64367179,1037.1481173)(464.83783473,1037.00150446)
\curveto(465.03199768,1036.85885413)(465.42230483,1036.72214757)(466.00875617,1036.59138477)
\lineto(466.38321328,1036.50817208)
\curveto(467.15986507,1036.3417467)(467.71065383,1036.1059774)(468.03557958,1035.8008642)
\curveto(468.36446783,1035.49971351)(468.52891196,1035.07770629)(468.52891196,1034.53484254)
\curveto(468.52891196,1033.91669112)(468.2832364,1033.42732125)(467.79188527,1033.06673292)
\curveto(467.30449665,1032.70614459)(466.63285136,1032.52585043)(465.77694939,1032.52585043)
\curveto(465.42032357,1032.52585043)(465.04784771,1032.56151301)(464.65952182,1032.63283817)
\curveto(464.27515844,1032.70020083)(463.86900126,1032.80322607)(463.44105027,1032.94191388)
\lineto(463.44105027,1034.07122898)
\curveto(463.8452262,1033.861216)(464.24345837,1033.70271563)(464.63574677,1033.59572789)
\curveto(465.02803517,1033.49270265)(465.41636106,1033.44119003)(465.80072444,1033.44119003)
\curveto(466.31585063,1033.44119003)(466.71210154,1033.52836523)(466.98947718,1033.70271563)
\curveto(467.26685281,1033.88102854)(467.40554063,1034.13066662)(467.40554063,1034.45162985)
\curveto(467.40554063,1034.74881804)(467.30449665,1034.97666231)(467.10240869,1035.13516267)
\curveto(466.90428323,1035.29366304)(466.46642597,1035.44621964)(465.78883692,1035.59283248)
\lineto(465.40843604,1035.68198893)
\curveto(464.73084699,1035.82463926)(464.24147711,1036.04257726)(463.94032642,1036.33580293)
\curveto(463.63917573,1036.63299112)(463.48860038,1037.0391483)(463.48860038,1037.55427448)
\curveto(463.48860038,1038.18035092)(463.71050089,1038.66377703)(464.15430191,1039.00455281)
\curveto(464.59810293,1039.3453286)(465.22814188,1039.51571649)(466.04441875,1039.51571649)
\curveto(466.44859468,1039.51571649)(466.82899556,1039.48599767)(467.18562138,1039.42656003)
\curveto(467.5422472,1039.3671224)(467.87113545,1039.27796594)(468.17228614,1039.15909067)
\closepath
}
}
{
\newrgbcolor{curcolor}{0 0 0}
\pscustom[linestyle=none,fillstyle=solid,fillcolor=curcolor]
{
\newpath
\moveto(470.65083559,1031.56890448)
\lineto(471.06689904,1031.56890448)
\curveto(471.62165032,1031.56890448)(471.98223865,1031.65409843)(472.14866403,1031.82448632)
\curveto(472.31905192,1031.99487421)(472.40424587,1032.36338756)(472.40424587,1032.93002636)
\lineto(472.40424587,1034.40407974)
\curveto(472.40424587,1035.02223116)(472.49340232,1035.47197595)(472.67171523,1035.75331409)
\curveto(472.85002814,1036.03465224)(473.15910385,1036.22881519)(473.59894236,1036.33580293)
\curveto(473.15910385,1036.43486566)(472.85002814,1036.6250661)(472.67171523,1036.90640424)
\curveto(472.49340232,1037.18774239)(472.40424587,1037.63946843)(472.40424587,1038.26158236)
\lineto(472.40424587,1039.73563574)
\curveto(472.40424587,1040.29831204)(472.31905192,1040.66484413)(472.14866403,1040.83523202)
\curveto(471.98223865,1041.00958242)(471.62165032,1041.09675762)(471.06689904,1041.09675762)
\lineto(470.65083559,1041.09675762)
\lineto(470.65083559,1041.94671582)
\lineto(471.0252927,1041.94671582)
\curveto(472.01195747,1041.94671582)(472.66973398,1041.80010299)(472.99862223,1041.50687731)
\curveto(473.331473,1041.21761415)(473.49789838,1040.63512531)(473.49789838,1039.7594108)
\lineto(473.49789838,1038.33290752)
\curveto(473.49789838,1037.74249366)(473.60488613,1037.33237397)(473.81886162,1037.10254844)
\curveto(474.03283711,1036.87668543)(474.421163,1036.76375392)(474.98383929,1036.76375392)
\lineto(475.35235264,1036.76375392)
\lineto(475.35235264,1035.91379571)
\lineto(474.98383929,1035.91379571)
\curveto(474.421163,1035.91379571)(474.03283711,1035.79888295)(473.81886162,1035.56905742)
\curveto(473.60488613,1035.33923189)(473.49789838,1034.92514969)(473.49789838,1034.32681082)
\lineto(473.49789838,1032.9062513)
\curveto(473.49789838,1032.03053679)(473.331473,1031.4460667)(472.99862223,1031.15284102)
\curveto(472.66973398,1030.85961535)(472.01195747,1030.71300251)(471.0252927,1030.71300251)
\lineto(470.65083559,1030.71300251)
\closepath
}
}
{
\newrgbcolor{curcolor}{0 0 0}
\pscustom[linestyle=none,fillstyle=solid,fillcolor=curcolor]
{
\newpath
\moveto(480.65418982,1041.9348283)
\curveto(480.1232136,1041.0234512)(479.72894395,1040.12198038)(479.47138085,1039.23041583)
\curveto(479.21381776,1038.33885128)(479.08503622,1037.43539921)(479.08503622,1036.52005961)
\curveto(479.08503622,1035.60472)(479.21381776,1034.69730542)(479.47138085,1033.79781585)
\curveto(479.73290646,1032.90228879)(480.12717611,1032.00081797)(480.65418982,1031.09340339)
\lineto(479.70318764,1031.09340339)
\curveto(479.10881127,1032.02459303)(478.663029,1032.93993263)(478.36584081,1033.8394222)
\curveto(478.07261514,1034.73891176)(477.9260023,1035.63245757)(477.9260023,1036.52005961)
\curveto(477.9260023,1037.40369914)(478.07261514,1038.29328243)(478.36584081,1039.18880949)
\curveto(478.65906649,1040.08433655)(479.10484876,1040.99967615)(479.70318764,1041.9348283)
\closepath
}
}
{
\newrgbcolor{curcolor}{0 0 0}
\pscustom[linestyle=none,fillstyle=solid,fillcolor=curcolor]
{
\newpath
\moveto(482.64535065,1041.57225871)
\lineto(485.73610775,1031.56890448)
\lineto(484.72566793,1031.56890448)
\lineto(481.63491083,1041.57225871)
\closepath
}
}
{
\newrgbcolor{curcolor}{0 0 0}
\pscustom[linestyle=none,fillstyle=solid,fillcolor=curcolor]
{
\newpath
\moveto(491.13898891,1039.15909067)
\lineto(491.13898891,1038.12487579)
\curveto(490.8299132,1038.28337616)(490.50894996,1038.40225143)(490.1760992,1038.48150161)
\curveto(489.84324843,1038.56075179)(489.49851014,1038.60037689)(489.14188432,1038.60037689)
\curveto(488.59902058,1038.60037689)(488.19088214,1038.51716419)(487.91746901,1038.35073881)
\curveto(487.64801839,1038.18431343)(487.51329308,1037.93467536)(487.51329308,1037.60182459)
\curveto(487.51329308,1037.34822401)(487.61037455,1037.1481173)(487.8045375,1037.00150446)
\curveto(487.99870045,1036.85885413)(488.38900759,1036.72214757)(488.97545894,1036.59138477)
\lineto(489.34991605,1036.50817208)
\curveto(490.12656784,1036.3417467)(490.6773566,1036.1059774)(491.00228235,1035.8008642)
\curveto(491.3311706,1035.49971351)(491.49561473,1035.07770629)(491.49561473,1034.53484254)
\curveto(491.49561473,1033.91669112)(491.24993917,1033.42732125)(490.75858804,1033.06673292)
\curveto(490.27119942,1032.70614459)(489.59955413,1032.52585043)(488.74365216,1032.52585043)
\curveto(488.38702634,1032.52585043)(488.01455048,1032.56151301)(487.62622459,1032.63283817)
\curveto(487.24186121,1032.70020083)(486.83570402,1032.80322607)(486.40775304,1032.94191388)
\lineto(486.40775304,1034.07122898)
\curveto(486.81192897,1033.861216)(487.21016114,1033.70271563)(487.60244954,1033.59572789)
\curveto(487.99473794,1033.49270265)(488.38306383,1033.44119003)(488.76742721,1033.44119003)
\curveto(489.2825534,1033.44119003)(489.67880431,1033.52836523)(489.95617994,1033.70271563)
\curveto(490.23355558,1033.88102854)(490.3722434,1034.13066662)(490.3722434,1034.45162985)
\curveto(490.3722434,1034.74881804)(490.27119942,1034.97666231)(490.06911145,1035.13516267)
\curveto(489.870986,1035.29366304)(489.43312874,1035.44621964)(488.75553969,1035.59283248)
\lineto(488.37513881,1035.68198893)
\curveto(487.69754975,1035.82463926)(487.20817988,1036.04257726)(486.90702919,1036.33580293)
\curveto(486.6058785,1036.63299112)(486.45530315,1037.0391483)(486.45530315,1037.55427448)
\curveto(486.45530315,1038.18035092)(486.67720366,1038.66377703)(487.12100468,1039.00455281)
\curveto(487.5648057,1039.3453286)(488.19484465,1039.51571649)(489.01112152,1039.51571649)
\curveto(489.41529745,1039.51571649)(489.79569833,1039.48599767)(490.15232415,1039.42656003)
\curveto(490.50894996,1039.3671224)(490.83783822,1039.27796594)(491.13898891,1039.15909067)
\closepath
}
}
{
\newrgbcolor{curcolor}{0 0 0}
\pscustom[linestyle=none,fillstyle=solid,fillcolor=curcolor]
{
\newpath
\moveto(498.93720683,1036.30014035)
\lineto(498.93720683,1035.76520162)
\lineto(493.90878278,1035.76520162)
\curveto(493.95633289,1035.01232489)(494.1821959,1034.43776107)(494.58637183,1034.04151016)
\curveto(494.99451027,1033.64922176)(495.56114907,1033.45307756)(496.28628824,1033.45307756)
\curveto(496.7063142,1033.45307756)(497.11247139,1033.50459018)(497.50475979,1033.60761541)
\curveto(497.9010107,1033.71064065)(498.2932991,1033.86517851)(498.68162499,1034.07122898)
\lineto(498.68162499,1033.0370141)
\curveto(498.28933659,1032.87058872)(497.88714192,1032.74378843)(497.47504097,1032.65661323)
\curveto(497.06294002,1032.56943803)(496.64489531,1032.52585043)(496.22090684,1032.52585043)
\curveto(495.1589544,1032.52585043)(494.31692121,1032.83492614)(493.69480728,1033.45307756)
\curveto(493.07665586,1034.07122898)(492.76758015,1034.9073184)(492.76758015,1035.96134582)
\curveto(492.76758015,1037.05103583)(493.06080583,1037.91486281)(493.64725718,1038.55282678)
\curveto(494.23767103,1039.19475325)(495.03215411,1039.51571649)(496.0307064,1039.51571649)
\curveto(496.92623346,1039.51571649)(497.63354133,1039.22645332)(498.15263003,1038.64792699)
\curveto(498.67568123,1038.07336317)(498.93720683,1037.29076763)(498.93720683,1036.30014035)
\closepath
\moveto(497.84355432,1036.62110359)
\curveto(497.8356293,1037.21944246)(497.66722266,1037.69692481)(497.33833441,1038.05355063)
\curveto(497.01340866,1038.41017645)(496.58149517,1038.58848936)(496.04259393,1038.58848936)
\curveto(495.43236753,1038.58848936)(494.94299765,1038.41612021)(494.57448431,1038.07138192)
\curveto(494.20993347,1037.72664363)(493.99992049,1037.24123626)(493.94444536,1036.61515982)
\closepath
}
}
{
\newrgbcolor{curcolor}{0 0 0}
\pscustom[linestyle=none,fillstyle=solid,fillcolor=curcolor]
{
\newpath
\moveto(501.81398844,1041.24535171)
\lineto(501.81398844,1039.35523487)
\lineto(504.06667486,1039.35523487)
\lineto(504.06667486,1038.50527667)
\lineto(501.81398844,1038.50527667)
\lineto(501.81398844,1034.89146836)
\curveto(501.81398844,1034.34860462)(501.88729486,1033.99990382)(502.03390769,1033.84536596)
\curveto(502.18448304,1033.69082811)(502.48761499,1033.61355918)(502.94330353,1033.61355918)
\lineto(504.06667486,1033.61355918)
\lineto(504.06667486,1032.69821957)
\lineto(502.94330353,1032.69821957)
\curveto(502.09928909,1032.69821957)(501.51680026,1032.85473868)(501.19583702,1033.1677769)
\curveto(500.87487378,1033.48477763)(500.71439216,1034.05934145)(500.71439216,1034.89146836)
\lineto(500.71439216,1038.50527667)
\lineto(499.91198407,1038.50527667)
\lineto(499.91198407,1039.35523487)
\lineto(500.71439216,1039.35523487)
\lineto(500.71439216,1041.24535171)
\closepath
}
}
{
\newrgbcolor{curcolor}{0 0 0}
\pscustom[linestyle=none,fillstyle=solid,fillcolor=curcolor]
{
\newpath
\moveto(508.09060286,1038.58848936)
\curveto(507.50415151,1038.58848936)(507.04053795,1038.35866383)(506.69976216,1037.89901277)
\curveto(506.35898638,1037.44332423)(506.18859849,1036.81724779)(506.18859849,1036.02078346)
\curveto(506.18859849,1035.22431913)(506.35700513,1034.59626144)(506.6938184,1034.13661038)
\curveto(507.03459418,1033.68092183)(507.500189,1033.45307756)(508.09060286,1033.45307756)
\curveto(508.6730917,1033.45307756)(509.13472401,1033.68290309)(509.47549979,1034.14255414)
\curveto(509.81627558,1034.6022052)(509.98666347,1035.22828164)(509.98666347,1036.02078346)
\curveto(509.98666347,1036.80932277)(509.81627558,1037.43341795)(509.47549979,1037.89306901)
\curveto(509.13472401,1038.35668258)(508.6730917,1038.58848936)(508.09060286,1038.58848936)
\closepath
\moveto(508.09060286,1039.51571649)
\curveto(509.04160505,1039.51571649)(509.78853801,1039.20664078)(510.33140176,1038.58848936)
\curveto(510.87426551,1037.97033794)(511.14569738,1037.11443597)(511.14569738,1036.02078346)
\curveto(511.14569738,1034.93109345)(510.87426551,1034.07519149)(510.33140176,1033.45307756)
\curveto(509.78853801,1032.83492614)(509.04160505,1032.52585043)(508.09060286,1032.52585043)
\curveto(507.13563817,1032.52585043)(506.38672395,1032.83492614)(505.8438602,1033.45307756)
\curveto(505.30495896,1034.07519149)(505.03550834,1034.93109345)(505.03550834,1036.02078346)
\curveto(505.03550834,1037.11443597)(505.30495896,1037.97033794)(505.8438602,1038.58848936)
\curveto(506.38672395,1039.20664078)(507.13563817,1039.51571649)(508.09060286,1039.51571649)
\closepath
}
}
{
\newrgbcolor{curcolor}{0 0 0}
\pscustom[linestyle=none,fillstyle=solid,fillcolor=curcolor]
{
\newpath
\moveto(514.01059146,1033.69677187)
\lineto(514.01059146,1030.16617626)
\lineto(512.91099519,1030.16617626)
\lineto(512.91099519,1039.35523487)
\lineto(514.01059146,1039.35523487)
\lineto(514.01059146,1038.34479505)
\curveto(514.24041699,1038.74104596)(514.52968015,1039.03427163)(514.87838096,1039.22447207)
\curveto(515.23104427,1039.41863502)(515.65107023,1039.51571649)(516.13845885,1039.51571649)
\curveto(516.94681071,1039.51571649)(517.60260597,1039.19475325)(518.10584462,1038.55282678)
\curveto(518.61304579,1037.9109003)(518.86664637,1037.06688586)(518.86664637,1036.02078346)
\curveto(518.86664637,1034.97468106)(518.61304579,1034.13066662)(518.10584462,1033.48874014)
\curveto(517.60260597,1032.84681367)(516.94681071,1032.52585043)(516.13845885,1032.52585043)
\curveto(515.65107023,1032.52585043)(515.23104427,1032.62095065)(514.87838096,1032.81115108)
\curveto(514.52968015,1033.00531403)(514.24041699,1033.30052096)(514.01059146,1033.69677187)
\closepath
\moveto(517.73138751,1036.02078346)
\curveto(517.73138751,1036.82517281)(517.56496213,1037.45521175)(517.23211136,1037.9109003)
\curveto(516.90322311,1038.37055136)(516.44951582,1038.60037689)(515.87098949,1038.60037689)
\curveto(515.29246316,1038.60037689)(514.83677461,1038.37055136)(514.50392385,1037.9109003)
\curveto(514.17503559,1037.45521175)(514.01059146,1036.82517281)(514.01059146,1036.02078346)
\curveto(514.01059146,1035.21639411)(514.17503559,1034.58437391)(514.50392385,1034.12472285)
\curveto(514.83677461,1033.66903431)(515.29246316,1033.44119003)(515.87098949,1033.44119003)
\curveto(516.44951582,1033.44119003)(516.90322311,1033.66903431)(517.23211136,1034.12472285)
\curveto(517.56496213,1034.58437391)(517.73138751,1035.21639411)(517.73138751,1036.02078346)
\closepath
}
}
{
\newrgbcolor{curcolor}{0 0 0}
\pscustom[linestyle=none,fillstyle=solid,fillcolor=curcolor]
{
\newpath
\moveto(520.50712514,1041.9348283)
\lineto(521.45812732,1041.9348283)
\curveto(522.05250369,1040.99967615)(522.49630471,1040.08433655)(522.78953038,1039.18880949)
\curveto(523.08671857,1038.29328243)(523.23531266,1037.40369914)(523.23531266,1036.52005961)
\curveto(523.23531266,1035.63245757)(523.08671857,1034.73891176)(522.78953038,1033.8394222)
\curveto(522.49630471,1032.93993263)(522.05250369,1032.02459303)(521.45812732,1031.09340339)
\lineto(520.50712514,1031.09340339)
\curveto(521.03413885,1032.00081797)(521.42642725,1032.90228879)(521.68399034,1033.79781585)
\curveto(521.94551594,1034.69730542)(522.07627874,1035.60472)(522.07627874,1036.52005961)
\curveto(522.07627874,1037.43539921)(521.94551594,1038.33885128)(521.68399034,1039.23041583)
\curveto(521.42642725,1040.12198038)(521.03413885,1041.0234512)(520.50712514,1041.9348283)
\closepath
}
}
{
\newrgbcolor{curcolor}{0 0 0}
\pscustom[linestyle=none,fillstyle=solid,fillcolor=curcolor]
{
\newpath
\moveto(528.40044327,1030.90914671)
\lineto(527.80606691,1030.90914671)
\lineto(527.80012315,1032.69821957)
\curveto(527.38405969,1032.70614459)(526.96799623,1032.7536947)(526.55193278,1032.8408699)
\curveto(526.13586932,1032.93200761)(525.71782461,1033.06673292)(525.29779865,1033.24504583)
\lineto(525.29779865,1034.31492329)
\curveto(525.70197457,1034.06132271)(526.11011301,1033.86914101)(526.52221396,1033.73837821)
\curveto(526.93827742,1033.61157792)(527.3662284,1033.54619652)(527.80606691,1033.54223401)
\lineto(527.80606691,1036.25259024)
\curveto(526.9303524,1036.39524057)(526.29238843,1036.63695362)(525.89217501,1036.97772941)
\curveto(525.4959241,1037.31850519)(525.29779865,1037.78608126)(525.29779865,1038.38045763)
\curveto(525.29779865,1039.02634661)(525.51375539,1039.53552903)(525.94566888,1039.90800489)
\curveto(526.37758238,1040.28048075)(526.99771505,1040.49445624)(527.80606691,1040.54993136)
\lineto(527.80606691,1041.94671582)
\lineto(528.40044327,1041.94671582)
\lineto(528.40044327,1040.56776266)
\curveto(528.76895662,1040.55191262)(529.12558244,1040.51228753)(529.47032073,1040.44888738)
\curveto(529.81505902,1040.38944975)(530.1518723,1040.30623705)(530.48076055,1040.19924931)
\lineto(530.48076055,1039.15909067)
\curveto(530.1518723,1039.32551605)(529.81307777,1039.4542976)(529.46437697,1039.54543531)
\curveto(529.11963868,1039.63657302)(528.76499411,1039.69006689)(528.40044327,1039.70591693)
\lineto(528.40044327,1037.16792984)
\curveto(529.29993284,1037.02924203)(529.96167186,1036.78158521)(530.38566034,1036.42495939)
\curveto(530.80964881,1036.06833357)(531.02164305,1035.58094495)(531.02164305,1034.96279353)
\curveto(531.02164305,1034.29312949)(530.79578003,1033.76413452)(530.34405399,1033.37580863)
\curveto(529.89629046,1032.99144525)(529.24842022,1032.76954474)(528.40044327,1032.7101071)
\closepath
\moveto(527.80606691,1037.27491759)
\lineto(527.80606691,1039.71186069)
\curveto(527.34641585,1039.66034807)(526.9957338,1039.52958527)(526.75402074,1039.31957229)
\curveto(526.51230769,1039.10955931)(526.39145116,1038.83020241)(526.39145116,1038.48150161)
\curveto(526.39145116,1038.14072583)(526.50240141,1037.87523772)(526.72430192,1037.68503728)
\curveto(526.95016494,1037.49483685)(527.31075327,1037.35813028)(527.80606691,1037.27491759)
\closepath
\moveto(528.40044327,1036.13371497)
\lineto(528.40044327,1033.5600653)
\curveto(528.90368193,1033.62742796)(529.28210155,1033.77007829)(529.53570213,1033.98801629)
\curveto(529.79326522,1034.20595429)(529.92204677,1034.4932362)(529.92204677,1034.84986202)
\curveto(529.92204677,1035.19856282)(529.79920899,1035.47593846)(529.55353342,1035.68198893)
\curveto(529.31182037,1035.8880394)(528.92745699,1036.03861475)(528.40044327,1036.13371497)
\closepath
}
}
{
\newrgbcolor{curcolor}{0 0 0}
\pscustom[linestyle=none,fillstyle=solid,fillcolor=curcolor]
{
\newpath
\moveto(536.5077369,1036.52005961)
\lineto(539.71142551,1036.52005961)
\lineto(539.71142551,1035.54528237)
\lineto(536.5077369,1035.54528237)
\closepath
}
}
{
\newrgbcolor{curcolor}{0 0 0}
\pscustom[linestyle=none,fillstyle=solid,fillcolor=curcolor]
{
\newpath
\moveto(550.70144451,1041.28101429)
\lineto(550.70144451,1040.11009285)
\curveto(550.24575597,1040.32803085)(549.81582373,1040.49049373)(549.4116478,1040.59748147)
\curveto(549.00747187,1040.70446922)(548.61716472,1040.75796309)(548.24072636,1040.75796309)
\curveto(547.58691236,1040.75796309)(547.08169245,1040.6311628)(546.72506663,1040.37756222)
\curveto(546.37240332,1040.12396164)(546.19607166,1039.76337331)(546.19607166,1039.29579723)
\curveto(546.19607166,1038.90350883)(546.31296568,1038.60632065)(546.54675372,1038.40423268)
\curveto(546.78450426,1038.20610723)(547.23226779,1038.04562561)(547.8900443,1037.92278783)
\lineto(548.61518347,1037.77419374)
\curveto(549.51071053,1037.60380585)(550.17046829,1037.30265515)(550.59445677,1036.87074166)
\curveto(551.02240775,1036.44279068)(551.23638324,1035.86822686)(551.23638324,1035.1470502)
\curveto(551.23638324,1034.28718573)(550.94712008,1033.63535298)(550.36859375,1033.19155196)
\curveto(549.79402993,1032.74775094)(548.95001549,1032.52585043)(547.83655043,1032.52585043)
\curveto(547.41652447,1032.52585043)(546.96876094,1032.57340054)(546.49325985,1032.66850076)
\curveto(546.02172126,1032.76360097)(545.53235139,1032.90427005)(545.02515022,1033.09050798)
\lineto(545.02515022,1034.32681082)
\curveto(545.51253884,1034.05339769)(545.99002119,1033.84734721)(546.45759726,1033.7086594)
\curveto(546.92517334,1033.56997158)(547.38482439,1033.50062767)(547.83655043,1033.50062767)
\curveto(548.52206451,1033.50062767)(549.05105947,1033.63535298)(549.42353533,1033.9048036)
\curveto(549.79601118,1034.17425422)(549.98224911,1034.5586176)(549.98224911,1035.05789375)
\curveto(549.98224911,1035.49376975)(549.8475238,1035.83454553)(549.57807318,1036.0802211)
\curveto(549.31258507,1036.32589666)(548.87472782,1036.51015333)(548.26450141,1036.63299112)
\lineto(547.53341848,1036.77564144)
\curveto(546.63789143,1036.95395435)(545.99002119,1037.23331124)(545.58980777,1037.61371212)
\curveto(545.18959435,1037.99411299)(544.98948764,1038.52310796)(544.98948764,1039.20069701)
\curveto(544.98948764,1039.98527382)(545.26488202,1040.60342524)(545.81567079,1041.05515128)
\curveto(546.37042206,1041.50687731)(547.13320507,1041.73274033)(548.1040198,1041.73274033)
\curveto(548.52008325,1041.73274033)(548.94407173,1041.6950965)(549.37598522,1041.61980882)
\curveto(549.80789871,1041.54452115)(550.24971848,1041.43158964)(550.70144451,1041.28101429)
\closepath
}
}
{
\newrgbcolor{curcolor}{0 0 0}
\pscustom[linestyle=none,fillstyle=solid,fillcolor=curcolor]
{
\newpath
\moveto(554.30930905,1040.58559395)
\lineto(554.30930905,1037.25114254)
\lineto(555.81902502,1037.25114254)
\curveto(556.37773881,1037.25114254)(556.8096523,1037.39577412)(557.1147655,1037.68503728)
\curveto(557.4198787,1037.97430045)(557.5724353,1038.38640139)(557.5724353,1038.92134012)
\curveto(557.5724353,1039.45231634)(557.4198787,1039.86243604)(557.1147655,1040.1516992)
\curveto(556.8096523,1040.44096236)(556.37773881,1040.58559395)(555.81902502,1040.58559395)
\closepath
\moveto(553.10866879,1041.57225871)
\lineto(555.81902502,1041.57225871)
\curveto(556.81361481,1041.57225871)(557.56451028,1041.34639569)(558.07171145,1040.89466966)
\curveto(558.58287512,1040.44690613)(558.83845696,1039.78912962)(558.83845696,1038.92134012)
\curveto(558.83845696,1038.04562561)(558.58287512,1037.38388659)(558.07171145,1036.93612306)
\curveto(557.56451028,1036.48835953)(556.81361481,1036.26447777)(555.81902502,1036.26447777)
\lineto(554.30930905,1036.26447777)
\lineto(554.30930905,1032.69821957)
\lineto(553.10866879,1032.69821957)
\closepath
}
}
{
\newrgbcolor{curcolor}{0 0 0}
\pscustom[linestyle=none,fillstyle=solid,fillcolor=curcolor]
{
\newpath
\moveto(133.9041662,832.92588114)
\lineto(133.30978983,832.92588114)
\lineto(133.30384607,834.714954)
\curveto(132.88778261,834.72287902)(132.47171916,834.77042913)(132.0556557,834.85760433)
\curveto(131.63959225,834.94874204)(131.22154754,835.08346735)(130.80152157,835.26178026)
\lineto(130.80152157,836.33165771)
\curveto(131.2056975,836.07805713)(131.61383594,835.88587544)(132.02593688,835.75511264)
\curveto(132.44200034,835.62831235)(132.86995132,835.56293095)(133.30978983,835.55896844)
\lineto(133.30978983,838.26932467)
\curveto(132.43407532,838.41197499)(131.79611136,838.65368805)(131.39589794,838.99446383)
\curveto(130.99964703,839.33523962)(130.80152157,839.80281569)(130.80152157,840.39719206)
\curveto(130.80152157,841.04308104)(131.01747832,841.55226346)(131.44939181,841.92473932)
\curveto(131.8813053,842.29721517)(132.50143798,842.51119066)(133.30978983,842.56666579)
\lineto(133.30978983,843.96345025)
\lineto(133.9041662,843.96345025)
\lineto(133.9041662,842.58449708)
\curveto(134.27267955,842.56864704)(134.62930537,842.52902195)(134.97404366,842.46562181)
\curveto(135.31878195,842.40618417)(135.65559522,842.32297148)(135.98448348,842.21598373)
\lineto(135.98448348,841.17582509)
\curveto(135.65559522,841.34225048)(135.3168007,841.47103202)(134.96809989,841.56216973)
\curveto(134.6233616,841.65330744)(134.26871704,841.70680131)(133.9041662,841.72265135)
\lineto(133.9041662,839.18466427)
\curveto(134.80365577,839.04597645)(135.46539479,838.79831963)(135.88938326,838.44169381)
\curveto(136.31337174,838.08506799)(136.52536597,837.59767937)(136.52536597,836.97952795)
\curveto(136.52536597,836.30986391)(136.29950295,835.78086895)(135.84777692,835.39254306)
\curveto(135.40001339,835.00817967)(134.75214315,834.78627916)(133.9041662,834.72684153)
\closepath
\moveto(133.30978983,839.29165202)
\lineto(133.30978983,841.72859511)
\curveto(132.85013878,841.6770825)(132.49945672,841.5463197)(132.25774367,841.33630671)
\curveto(132.01603061,841.12629373)(131.89517408,840.84693684)(131.89517408,840.49823604)
\curveto(131.89517408,840.15746025)(132.00612434,839.89197214)(132.22802485,839.70177171)
\curveto(132.45388787,839.51157127)(132.8144762,839.37486471)(133.30978983,839.29165202)
\closepath
\moveto(133.9041662,838.15044939)
\lineto(133.9041662,835.57679973)
\curveto(134.40740486,835.64416238)(134.78582448,835.78681271)(135.03942506,836.00475071)
\curveto(135.29698815,836.22268871)(135.4257697,836.50997062)(135.4257697,836.86659644)
\curveto(135.4257697,837.21529724)(135.30293191,837.49267288)(135.05725635,837.69872336)
\curveto(134.81554329,837.90477383)(134.43117991,838.05534917)(133.9041662,838.15044939)
\closepath
}
}
{
\newrgbcolor{curcolor}{0 0 0}
\pscustom[linestyle=none,fillstyle=solid,fillcolor=curcolor]
{
\newpath
\moveto(138.55218938,843.58899314)
\lineto(141.64294648,833.58563891)
\lineto(140.63250666,833.58563891)
\lineto(137.54174956,843.58899314)
\closepath
}
}
{
\newrgbcolor{curcolor}{0 0 0}
\pscustom[linestyle=none,fillstyle=solid,fillcolor=curcolor]
{
\newpath
\moveto(144.05017076,842.60232837)
\lineto(144.05017076,839.26787696)
\lineto(145.55988673,839.26787696)
\curveto(146.11860051,839.26787696)(146.55051401,839.41250854)(146.85562721,839.70177171)
\curveto(147.16074041,839.99103487)(147.31329701,840.40313582)(147.31329701,840.93807455)
\curveto(147.31329701,841.46905077)(147.16074041,841.87917046)(146.85562721,842.16843363)
\curveto(146.55051401,842.45769679)(146.11860051,842.60232837)(145.55988673,842.60232837)
\closepath
\moveto(142.8495305,843.58899314)
\lineto(145.55988673,843.58899314)
\curveto(146.55447652,843.58899314)(147.30537199,843.36313012)(147.81257316,842.91140408)
\curveto(148.32373683,842.46364055)(148.57931867,841.80586404)(148.57931867,840.93807455)
\curveto(148.57931867,840.06236004)(148.32373683,839.40062102)(147.81257316,838.95285749)
\curveto(147.30537199,838.50509396)(146.55447652,838.28121219)(145.55988673,838.28121219)
\lineto(144.05017076,838.28121219)
\lineto(144.05017076,834.714954)
\lineto(142.8495305,834.714954)
\closepath
}
}
{
\newrgbcolor{curcolor}{0 0 0}
\pscustom[linestyle=none,fillstyle=solid,fillcolor=curcolor]
{
\newpath
\moveto(148.96566331,843.58899314)
\lineto(156.4726368,843.58899314)
\lineto(156.4726368,842.57855332)
\lineto(153.32244207,842.57855332)
\lineto(153.32244207,834.714954)
\lineto(152.11585804,834.714954)
\lineto(152.11585804,842.57855332)
\lineto(148.96566331,842.57855332)
\closepath
}
}
{
\newrgbcolor{curcolor}{0 0 0}
\pscustom[linestyle=none,fillstyle=solid,fillcolor=curcolor]
{
\newpath
\moveto(160.55600244,832.92588114)
\lineto(159.96162607,832.92588114)
\lineto(159.95568231,834.714954)
\curveto(159.53961885,834.72287902)(159.12355539,834.77042913)(158.70749194,834.85760433)
\curveto(158.29142848,834.94874204)(157.87338377,835.08346735)(157.45335781,835.26178026)
\lineto(157.45335781,836.33165771)
\curveto(157.85753374,836.07805713)(158.26567217,835.88587544)(158.67777312,835.75511264)
\curveto(159.09383658,835.62831235)(159.52178756,835.56293095)(159.96162607,835.55896844)
\lineto(159.96162607,838.26932467)
\curveto(159.08591156,838.41197499)(158.44794759,838.65368805)(158.04773417,838.99446383)
\curveto(157.65148326,839.33523962)(157.45335781,839.80281569)(157.45335781,840.39719206)
\curveto(157.45335781,841.04308104)(157.66931455,841.55226346)(158.10122805,841.92473932)
\curveto(158.53314154,842.29721517)(159.15327421,842.51119066)(159.96162607,842.56666579)
\lineto(159.96162607,843.96345025)
\lineto(160.55600244,843.96345025)
\lineto(160.55600244,842.58449708)
\curveto(160.92451578,842.56864704)(161.2811416,842.52902195)(161.62587989,842.46562181)
\curveto(161.97061819,842.40618417)(162.30743146,842.32297148)(162.63631972,842.21598373)
\lineto(162.63631972,841.17582509)
\curveto(162.30743146,841.34225048)(161.96863693,841.47103202)(161.61993613,841.56216973)
\curveto(161.27519784,841.65330744)(160.92055327,841.70680131)(160.55600244,841.72265135)
\lineto(160.55600244,839.18466427)
\curveto(161.455492,839.04597645)(162.11723102,838.79831963)(162.5412195,838.44169381)
\curveto(162.96520797,838.08506799)(163.17720221,837.59767937)(163.17720221,836.97952795)
\curveto(163.17720221,836.30986391)(162.95133919,835.78086895)(162.49961315,835.39254306)
\curveto(162.05184962,835.00817967)(161.40397938,834.78627916)(160.55600244,834.72684153)
\closepath
\moveto(159.96162607,839.29165202)
\lineto(159.96162607,841.72859511)
\curveto(159.50197501,841.6770825)(159.15129296,841.5463197)(158.9095799,841.33630671)
\curveto(158.66786685,841.12629373)(158.54701032,840.84693684)(158.54701032,840.49823604)
\curveto(158.54701032,840.15746025)(158.65796057,839.89197214)(158.87986108,839.70177171)
\curveto(159.1057241,839.51157127)(159.46631243,839.37486471)(159.96162607,839.29165202)
\closepath
\moveto(160.55600244,838.15044939)
\lineto(160.55600244,835.57679973)
\curveto(161.05924109,835.64416238)(161.43766071,835.78681271)(161.69126129,836.00475071)
\curveto(161.94882439,836.22268871)(162.07760593,836.50997062)(162.07760593,836.86659644)
\curveto(162.07760593,837.21529724)(161.95476815,837.49267288)(161.70909259,837.69872336)
\curveto(161.46737953,837.90477383)(161.08301615,838.05534917)(160.55600244,838.15044939)
\closepath
}
}
{
\newrgbcolor{curcolor}{0 0 0}
\pscustom[linestyle=none,fillstyle=solid,fillcolor=curcolor]
{
\newpath
\moveto(164.78796216,838.53679403)
\lineto(167.99165077,838.53679403)
\lineto(167.99165077,837.56201679)
\lineto(164.78796216,837.56201679)
\closepath
}
}
{
\newrgbcolor{curcolor}{0 0 0}
\pscustom[linestyle=none,fillstyle=solid,fillcolor=curcolor]
{
\newpath
\moveto(169.64996083,843.58899314)
\lineto(170.85654485,843.58899314)
\lineto(170.85654485,838.1979995)
\curveto(170.85654485,837.24699732)(171.028914,836.56148324)(171.37365229,836.14145728)
\curveto(171.71839058,835.72539382)(172.27710437,835.51736209)(173.04979364,835.51736209)
\curveto(173.81852041,835.51736209)(174.37525294,835.72539382)(174.71999123,836.14145728)
\curveto(175.06472952,836.56148324)(175.23709867,837.24699732)(175.23709867,838.1979995)
\lineto(175.23709867,843.58899314)
\lineto(176.44368269,843.58899314)
\lineto(176.44368269,838.04940541)
\curveto(176.44368269,836.89235275)(176.15640078,836.0186195)(175.58183696,835.42820564)
\curveto(175.01123565,834.83779178)(174.16722121,834.54258485)(173.04979364,834.54258485)
\curveto(171.92840357,834.54258485)(171.08042662,834.83779178)(170.5058628,835.42820564)
\curveto(169.93526149,836.0186195)(169.64996083,836.89235275)(169.64996083,838.04940541)
\closepath
}
}
{
\newrgbcolor{curcolor}{0 0 0}
\pscustom[linestyle=none,fillstyle=solid,fillcolor=curcolor]
{
\newpath
\moveto(184.18840673,838.73293823)
\lineto(184.18840673,834.714954)
\lineto(183.09475422,834.714954)
\lineto(183.09475422,838.69727565)
\curveto(183.09475422,839.3273146)(182.97191644,839.79885318)(182.72624087,840.1118914)
\curveto(182.48056531,840.42492962)(182.11205196,840.58144873)(181.62070083,840.58144873)
\curveto(181.03028698,840.58144873)(180.56469216,840.39322955)(180.22391638,840.01679118)
\curveto(179.88314059,839.64035282)(179.7127527,839.12720789)(179.7127527,838.47735639)
\lineto(179.7127527,834.714954)
\lineto(178.61315642,834.714954)
\lineto(178.61315642,841.3719693)
\lineto(179.7127527,841.3719693)
\lineto(179.7127527,840.33775442)
\curveto(179.9742783,840.73796784)(180.28137276,841.03713728)(180.63403607,841.23526273)
\curveto(180.99066189,841.43338819)(181.40078158,841.53245091)(181.86439514,841.53245091)
\curveto(182.6291594,841.53245091)(183.20768573,841.29470037)(183.59997413,840.81919928)
\curveto(183.99226253,840.34766069)(184.18840673,839.65224034)(184.18840673,838.73293823)
\closepath
}
}
{
\newrgbcolor{curcolor}{0 0 0}
\pscustom[linestyle=none,fillstyle=solid,fillcolor=curcolor]
{
\newpath
\moveto(186.38165552,841.3719693)
\lineto(187.47530804,841.3719693)
\lineto(187.47530804,834.714954)
\lineto(186.38165552,834.714954)
\closepath
\moveto(186.38165552,843.96345025)
\lineto(187.47530804,843.96345025)
\lineto(187.47530804,842.57855332)
\lineto(186.38165552,842.57855332)
\closepath
}
}
{
\newrgbcolor{curcolor}{0 0 0}
\pscustom[linestyle=none,fillstyle=solid,fillcolor=curcolor]
{
\newpath
\moveto(193.12782727,843.96345025)
\lineto(193.12782727,843.05405441)
\lineto(192.08172487,843.05405441)
\curveto(191.68943647,843.05405441)(191.41602334,842.97480423)(191.26148549,842.81630386)
\curveto(191.11091014,842.6578035)(191.03562247,842.37250284)(191.03562247,841.9604019)
\lineto(191.03562247,841.3719693)
\lineto(192.83658285,841.3719693)
\lineto(192.83658285,840.52201109)
\lineto(191.03562247,840.52201109)
\lineto(191.03562247,834.714954)
\lineto(189.93602619,834.714954)
\lineto(189.93602619,840.52201109)
\lineto(188.88992379,840.52201109)
\lineto(188.88992379,841.3719693)
\lineto(189.93602619,841.3719693)
\lineto(189.93602619,841.83558286)
\curveto(189.93602619,842.57657206)(190.10839534,843.1154733)(190.45313363,843.45228657)
\curveto(190.79787192,843.79306236)(191.34469818,843.96345025)(192.0936124,843.96345025)
\closepath
}
}
{
\newrgbcolor{curcolor}{0 0 0}
\pscustom[linestyle=none,fillstyle=solid,fillcolor=curcolor]
{
\newpath
\moveto(196.61681654,840.60522378)
\curveto(196.03036519,840.60522378)(195.56675163,840.37539826)(195.22597584,839.9157472)
\curveto(194.88520006,839.46005865)(194.71481217,838.83398221)(194.71481217,838.03751788)
\curveto(194.71481217,837.24105355)(194.88321881,836.61299586)(195.22003208,836.1533448)
\curveto(195.56080786,835.69765626)(196.02640268,835.46981198)(196.61681654,835.46981198)
\curveto(197.19930538,835.46981198)(197.66093769,835.69963751)(198.00171347,836.15928857)
\curveto(198.34248925,836.61893962)(198.51287715,837.24501606)(198.51287715,838.03751788)
\curveto(198.51287715,838.8260572)(198.34248925,839.45015238)(198.00171347,839.90980344)
\curveto(197.66093769,840.373417)(197.19930538,840.60522378)(196.61681654,840.60522378)
\closepath
\moveto(196.61681654,841.53245091)
\curveto(197.56781872,841.53245091)(198.31475169,841.2233752)(198.85761544,840.60522378)
\curveto(199.40047919,839.98707236)(199.67191106,839.1311704)(199.67191106,838.03751788)
\curveto(199.67191106,836.94782788)(199.40047919,836.09192591)(198.85761544,835.46981198)
\curveto(198.31475169,834.85166056)(197.56781872,834.54258485)(196.61681654,834.54258485)
\curveto(195.66185185,834.54258485)(194.91293762,834.85166056)(194.37007388,835.46981198)
\curveto(193.83117264,836.09192591)(193.56172202,836.94782788)(193.56172202,838.03751788)
\curveto(193.56172202,839.1311704)(193.83117264,839.98707236)(194.37007388,840.60522378)
\curveto(194.91293762,841.2233752)(195.66185185,841.53245091)(196.61681654,841.53245091)
\closepath
}
}
{
\newrgbcolor{curcolor}{0 0 0}
\pscustom[linestyle=none,fillstyle=solid,fillcolor=curcolor]
{
\newpath
\moveto(205.33631782,840.34964195)
\curveto(205.21348004,840.42096711)(205.07875473,840.47247973)(204.9321419,840.5041798)
\curveto(204.78949157,840.53984238)(204.6309912,840.55767367)(204.4566408,840.55767367)
\curveto(203.83848938,840.55767367)(203.36298829,840.35558571)(203.03013753,839.95140978)
\curveto(202.70124927,839.55119636)(202.53680514,838.97465129)(202.53680514,838.22177456)
\lineto(202.53680514,834.714954)
\lineto(201.43720887,834.714954)
\lineto(201.43720887,841.3719693)
\lineto(202.53680514,841.3719693)
\lineto(202.53680514,840.33775442)
\curveto(202.76663067,840.74193035)(203.06580011,841.04109978)(203.43431345,841.23526273)
\curveto(203.8028268,841.43338819)(204.25059033,841.53245091)(204.77760404,841.53245091)
\curveto(204.85289171,841.53245091)(204.9361044,841.52650715)(205.02724211,841.51461962)
\curveto(205.11837982,841.5066946)(205.21942381,841.49282582)(205.33037406,841.47301328)
\closepath
}
}
{
\newrgbcolor{curcolor}{0 0 0}
\pscustom[linestyle=none,fillstyle=solid,fillcolor=curcolor]
{
\newpath
\moveto(211.46433815,840.09406011)
\curveto(211.73775128,840.58541124)(212.06465828,840.94798082)(212.44505916,841.18176886)
\curveto(212.82546003,841.4155569)(213.27322356,841.53245091)(213.78834974,841.53245091)
\curveto(214.48178884,841.53245091)(215.01672757,841.2887566)(215.39316593,840.80136798)
\curveto(215.7696043,840.31794187)(215.95782348,839.62846529)(215.95782348,838.73293823)
\lineto(215.95782348,834.714954)
\lineto(214.8582272,834.714954)
\lineto(214.8582272,838.69727565)
\curveto(214.8582272,839.33523962)(214.74529569,839.80875945)(214.51943267,840.11783516)
\curveto(214.29356965,840.42691087)(213.94883136,840.58144873)(213.4852178,840.58144873)
\curveto(212.918579,840.58144873)(212.47081547,840.39322955)(212.14192721,840.01679118)
\curveto(211.81303895,839.64035282)(211.64859483,839.12720789)(211.64859483,838.47735639)
\lineto(211.64859483,834.714954)
\lineto(210.54899855,834.714954)
\lineto(210.54899855,838.69727565)
\curveto(210.54899855,839.33920212)(210.43606704,839.81272196)(210.21020402,840.11783516)
\curveto(209.984341,840.42691087)(209.6356402,840.58144873)(209.16410162,840.58144873)
\curveto(208.60538783,840.58144873)(208.16158682,840.39124829)(207.83269856,840.01084742)
\curveto(207.5038103,839.63440905)(207.33936618,839.12324538)(207.33936618,838.47735639)
\lineto(207.33936618,834.714954)
\lineto(206.2397699,834.714954)
\lineto(206.2397699,841.3719693)
\lineto(207.33936618,841.3719693)
\lineto(207.33936618,840.33775442)
\curveto(207.58900425,840.74589286)(207.88817369,841.04704355)(208.23687449,841.24120649)
\curveto(208.58557529,841.43536944)(208.99965749,841.53245091)(209.47912109,841.53245091)
\curveto(209.9625472,841.53245091)(210.3726669,841.40961313)(210.70948017,841.16393757)
\curveto(211.05025595,840.918262)(211.30187528,840.56163618)(211.46433815,840.09406011)
\closepath
}
}
{
\newrgbcolor{curcolor}{0 0 0}
\pscustom[linestyle=none,fillstyle=solid,fillcolor=curcolor]
{
\newpath
\moveto(133.9041662,802.49381122)
\lineto(133.30978983,802.49381122)
\lineto(133.30384607,804.28288408)
\curveto(132.88778261,804.2908091)(132.47171916,804.33835921)(132.0556557,804.42553441)
\curveto(131.63959225,804.51667212)(131.22154754,804.65139743)(130.80152157,804.82971033)
\lineto(130.80152157,805.89958779)
\curveto(131.2056975,805.64598721)(131.61383594,805.45380552)(132.02593688,805.32304272)
\curveto(132.44200034,805.19624243)(132.86995132,805.13086103)(133.30978983,805.12689852)
\lineto(133.30978983,807.83725475)
\curveto(132.43407532,807.97990507)(131.79611136,808.22161813)(131.39589794,808.56239391)
\curveto(130.99964703,808.90316969)(130.80152157,809.37074577)(130.80152157,809.96512213)
\curveto(130.80152157,810.61101112)(131.01747832,811.12019354)(131.44939181,811.49266939)
\curveto(131.8813053,811.86514525)(132.50143798,812.07912074)(133.30978983,812.13459587)
\lineto(133.30978983,813.53138033)
\lineto(133.9041662,813.53138033)
\lineto(133.9041662,812.15242716)
\curveto(134.27267955,812.13657712)(134.62930537,812.09695203)(134.97404366,812.03355189)
\curveto(135.31878195,811.97411425)(135.65559522,811.89090156)(135.98448348,811.78391381)
\lineto(135.98448348,810.74375517)
\curveto(135.65559522,810.91018056)(135.3168007,811.0389621)(134.96809989,811.13009981)
\curveto(134.6233616,811.22123752)(134.26871704,811.27473139)(133.9041662,811.29058143)
\lineto(133.9041662,808.75259435)
\curveto(134.80365577,808.61390653)(135.46539479,808.36624971)(135.88938326,808.00962389)
\curveto(136.31337174,807.65299807)(136.52536597,807.16560945)(136.52536597,806.54745803)
\curveto(136.52536597,805.87779399)(136.29950295,805.34879903)(135.84777692,804.96047314)
\curveto(135.40001339,804.57610975)(134.75214315,804.35420924)(133.9041662,804.29477161)
\closepath
\moveto(133.30978983,808.85958209)
\lineto(133.30978983,811.29652519)
\curveto(132.85013878,811.24501257)(132.49945672,811.11424977)(132.25774367,810.90423679)
\curveto(132.01603061,810.69422381)(131.89517408,810.41486692)(131.89517408,810.06616612)
\curveto(131.89517408,809.72539033)(132.00612434,809.45990222)(132.22802485,809.26970179)
\curveto(132.45388787,809.07950135)(132.8144762,808.94279479)(133.30978983,808.85958209)
\closepath
\moveto(133.9041662,807.71837947)
\lineto(133.9041662,805.14472981)
\curveto(134.40740486,805.21209246)(134.78582448,805.35474279)(135.03942506,805.57268079)
\curveto(135.29698815,805.79061879)(135.4257697,806.0779007)(135.4257697,806.43452652)
\curveto(135.4257697,806.78322732)(135.30293191,807.06060296)(135.05725635,807.26665343)
\curveto(134.81554329,807.47270391)(134.43117991,807.62327925)(133.9041662,807.71837947)
\closepath
}
}
{
\newrgbcolor{curcolor}{0 0 0}
\pscustom[linestyle=none,fillstyle=solid,fillcolor=curcolor]
{
\newpath
\moveto(138.55218938,813.15692322)
\lineto(141.64294648,803.15356898)
\lineto(140.63250666,803.15356898)
\lineto(137.54174956,813.15692322)
\closepath
}
}
{
\newrgbcolor{curcolor}{0 0 0}
\pscustom[linestyle=none,fillstyle=solid,fillcolor=curcolor]
{
\newpath
\moveto(142.8495305,813.15692322)
\lineto(147.94927972,813.15692322)
\lineto(147.94927972,812.1464834)
\lineto(144.05017076,812.1464834)
\lineto(144.05017076,809.53122739)
\lineto(147.56887885,809.53122739)
\lineto(147.56887885,808.52078757)
\lineto(144.05017076,808.52078757)
\lineto(144.05017076,804.28288408)
\lineto(142.8495305,804.28288408)
\closepath
}
}
{
\newrgbcolor{curcolor}{0 0 0}
\pscustom[linestyle=none,fillstyle=solid,fillcolor=curcolor]
{
\newpath
\moveto(153.46509239,812.3426276)
\curveto(152.59334039,812.3426276)(151.8999013,812.01770185)(151.38477511,811.36785036)
\curveto(150.87361144,810.71799886)(150.6180296,809.83237808)(150.6180296,808.710988)
\curveto(150.6180296,807.59356044)(150.87361144,806.7099209)(151.38477511,806.06006941)
\curveto(151.8999013,805.41021792)(152.59334039,805.08529217)(153.46509239,805.08529217)
\curveto(154.3368444,805.08529217)(155.02632098,805.41021792)(155.53352215,806.06006941)
\curveto(156.04468582,806.7099209)(156.30026766,807.59356044)(156.30026766,808.710988)
\curveto(156.30026766,809.83237808)(156.04468582,810.71799886)(155.53352215,811.36785036)
\curveto(155.02632098,812.01770185)(154.3368444,812.3426276)(153.46509239,812.3426276)
\closepath
\moveto(153.46509239,813.31740484)
\curveto(154.70932025,813.31740484)(155.70391004,812.89936013)(156.44886175,812.0632707)
\curveto(157.19381346,811.23114379)(157.56628932,810.11371623)(157.56628932,808.710988)
\curveto(157.56628932,807.31222229)(157.19381346,806.19479472)(156.44886175,805.3587053)
\curveto(155.70391004,804.52657839)(154.70932025,804.11051493)(153.46509239,804.11051493)
\curveto(152.21690203,804.11051493)(151.21834973,804.52657839)(150.46943551,805.3587053)
\curveto(149.7244838,806.19083221)(149.35200794,807.30825978)(149.35200794,808.710988)
\curveto(149.35200794,810.11371623)(149.7244838,811.23114379)(150.46943551,812.0632707)
\curveto(151.21834973,812.89936013)(152.21690203,813.31740484)(153.46509239,813.31740484)
\closepath
}
}
{
\newrgbcolor{curcolor}{0 0 0}
\pscustom[linestyle=none,fillstyle=solid,fillcolor=curcolor]
{
\newpath
\moveto(162.02411206,813.5194928)
\curveto(161.49313584,812.60811571)(161.09886618,811.70664489)(160.84130309,810.81508034)
\curveto(160.58374,809.92351579)(160.45495845,809.02006371)(160.45495845,808.10472411)
\curveto(160.45495845,807.18938451)(160.58374,806.28196992)(160.84130309,805.38248035)
\curveto(161.10282869,804.4869533)(161.49709835,803.58548248)(162.02411206,802.67806789)
\lineto(161.07310987,802.67806789)
\curveto(160.47873351,803.60925753)(160.03295123,804.52459713)(159.73576305,805.4240867)
\curveto(159.44253738,806.32357627)(159.29592454,807.21712207)(159.29592454,808.10472411)
\curveto(159.29592454,808.98836364)(159.44253738,809.87794693)(159.73576305,810.77347399)
\curveto(160.02898872,811.66900105)(160.474771,812.58434065)(161.07310987,813.5194928)
\closepath
}
}
{
\newrgbcolor{curcolor}{0 0 0}
\pscustom[linestyle=none,fillstyle=solid,fillcolor=curcolor]
{
\newpath
\moveto(164.01527288,813.15692322)
\lineto(167.10602999,803.15356898)
\lineto(166.09559016,803.15356898)
\lineto(163.00483306,813.15692322)
\closepath
}
}
{
\newrgbcolor{curcolor}{0 0 0}
\pscustom[linestyle=none,fillstyle=solid,fillcolor=curcolor]
{
\newpath
\moveto(172.50891115,810.74375517)
\lineto(172.50891115,809.7095403)
\curveto(172.19983544,809.86804066)(171.8788722,809.98691593)(171.54602144,810.06616612)
\curveto(171.21317067,810.1454163)(170.86843238,810.18504139)(170.51180656,810.18504139)
\curveto(169.96894281,810.18504139)(169.56080438,810.1018287)(169.28739125,809.93540332)
\curveto(169.01794063,809.76897793)(168.88321532,809.51933986)(168.88321532,809.1864891)
\curveto(168.88321532,808.93288851)(168.98029679,808.7327818)(169.17445974,808.58616897)
\curveto(169.36862268,808.44351864)(169.75892983,808.30681207)(170.34538118,808.17604927)
\lineto(170.71983829,808.09283658)
\curveto(171.49649007,807.9264112)(172.04727884,807.69064191)(172.37220458,807.38552871)
\curveto(172.70109284,807.08437802)(172.86553697,806.6623708)(172.86553697,806.11950705)
\curveto(172.86553697,805.50135563)(172.6198614,805.01198575)(172.12851028,804.65139743)
\curveto(171.64112166,804.2908091)(170.96947636,804.11051493)(170.1135744,804.11051493)
\curveto(169.75694858,804.11051493)(169.38447272,804.14617751)(168.99614683,804.21750268)
\curveto(168.61178344,804.28486533)(168.20562626,804.38789057)(167.77767528,804.52657839)
\lineto(167.77767528,805.65589348)
\curveto(168.18185121,805.4458805)(168.58008337,805.28738014)(168.97237177,805.18039239)
\curveto(169.36466017,805.07736715)(169.75298607,805.02585454)(170.13734945,805.02585454)
\curveto(170.65247563,805.02585454)(171.04872654,805.11302974)(171.32610218,805.28738014)
\curveto(171.60347782,805.46569305)(171.74216564,805.71533112)(171.74216564,806.03629436)
\curveto(171.74216564,806.33348254)(171.64112166,806.56132681)(171.43903369,806.71982718)
\curveto(171.24090824,806.87832754)(170.80305098,807.03088414)(170.12546192,807.17749698)
\lineto(169.74506105,807.26665343)
\curveto(169.06747199,807.40930376)(168.57810212,807.62724176)(168.27695143,807.92046744)
\curveto(167.97580073,808.21765562)(167.82522539,808.6238128)(167.82522539,809.13893899)
\curveto(167.82522539,809.76501542)(168.0471259,810.24844154)(168.49092692,810.58921732)
\curveto(168.93472794,810.9299931)(169.56476688,811.10038099)(170.38104376,811.10038099)
\curveto(170.78521969,811.10038099)(171.16562056,811.07066217)(171.52224638,811.01122454)
\curveto(171.8788722,810.9517869)(172.20776046,810.86263045)(172.50891115,810.74375517)
\closepath
}
}
{
\newrgbcolor{curcolor}{0 0 0}
\pscustom[linestyle=none,fillstyle=solid,fillcolor=curcolor]
{
\newpath
\moveto(180.30712907,807.88480485)
\lineto(180.30712907,807.34986613)
\lineto(175.27870501,807.34986613)
\curveto(175.32625512,806.5969894)(175.55211814,806.02242558)(175.95629407,805.62617466)
\curveto(176.36443251,805.23388626)(176.93107131,805.03774206)(177.65621048,805.03774206)
\curveto(178.07623644,805.03774206)(178.48239362,805.08925468)(178.87468203,805.19227992)
\curveto(179.27093294,805.29530515)(179.66322134,805.44984301)(180.05154723,805.65589348)
\lineto(180.05154723,804.62167861)
\curveto(179.65925883,804.45525322)(179.25706415,804.32845293)(178.84496321,804.24127773)
\curveto(178.43286226,804.15410253)(178.01481755,804.11051493)(177.59082908,804.11051493)
\curveto(176.52887664,804.11051493)(175.68684345,804.41959064)(175.06472952,805.03774206)
\curveto(174.4465781,805.65589348)(174.13750239,806.4919829)(174.13750239,807.54601033)
\curveto(174.13750239,808.63570033)(174.43072806,809.49952731)(175.01717941,810.13749128)
\curveto(175.60759327,810.77941776)(176.40207634,811.10038099)(177.40062864,811.10038099)
\curveto(178.2961557,811.10038099)(179.00346357,810.81111783)(179.52255226,810.2325915)
\curveto(180.04560347,809.65802768)(180.30712907,808.87543213)(180.30712907,807.88480485)
\closepath
\moveto(179.21347655,808.20576809)
\curveto(179.20555154,808.80410697)(179.0371449,809.28158931)(178.70825664,809.63821513)
\curveto(178.3833309,809.99484095)(177.9514174,810.17315386)(177.41251617,810.17315386)
\curveto(176.80228976,810.17315386)(176.31291989,810.00078472)(175.94440654,809.65604642)
\curveto(175.5798557,809.31130813)(175.36984272,808.82590077)(175.31436759,808.19982433)
\closepath
}
}
{
\newrgbcolor{curcolor}{0 0 0}
\pscustom[linestyle=none,fillstyle=solid,fillcolor=curcolor]
{
\newpath
\moveto(183.18391068,812.83001622)
\lineto(183.18391068,810.93989937)
\lineto(185.4365971,810.93989937)
\lineto(185.4365971,810.08994117)
\lineto(183.18391068,810.08994117)
\lineto(183.18391068,806.47613287)
\curveto(183.18391068,805.93326912)(183.25721709,805.58456832)(183.40382993,805.43003046)
\curveto(183.55440528,805.27549261)(183.85753722,805.19822368)(184.31322577,805.19822368)
\lineto(185.4365971,805.19822368)
\lineto(185.4365971,804.28288408)
\lineto(184.31322577,804.28288408)
\curveto(183.46921133,804.28288408)(182.88672249,804.43940319)(182.56575926,804.75244141)
\curveto(182.24479602,805.06944214)(182.0843144,805.64400596)(182.0843144,806.47613287)
\lineto(182.0843144,810.08994117)
\lineto(181.28190631,810.08994117)
\lineto(181.28190631,810.93989937)
\lineto(182.0843144,810.93989937)
\lineto(182.0843144,812.83001622)
\closepath
}
}
{
\newrgbcolor{curcolor}{0 0 0}
\pscustom[linestyle=none,fillstyle=solid,fillcolor=curcolor]
{
\newpath
\moveto(187.53474567,807.60544796)
\curveto(187.53474567,806.80105861)(187.6991898,806.16903841)(188.02807806,805.70938736)
\curveto(188.36092882,805.25369881)(188.81661737,805.02585454)(189.3951437,805.02585454)
\curveto(189.97367003,805.02585454)(190.42935857,805.25369881)(190.76220934,805.70938736)
\curveto(191.0950601,806.16903841)(191.26148549,806.80105861)(191.26148549,807.60544796)
\curveto(191.26148549,808.40983731)(191.0950601,809.03987626)(190.76220934,809.49556481)
\curveto(190.42935857,809.95521586)(189.97367003,810.18504139)(189.3951437,810.18504139)
\curveto(188.81661737,810.18504139)(188.36092882,809.95521586)(188.02807806,809.49556481)
\curveto(187.6991898,809.03987626)(187.53474567,808.40983731)(187.53474567,807.60544796)
\closepath
\moveto(191.26148549,805.28143637)
\curveto(191.03165996,804.88518546)(190.74041554,804.58997853)(190.38775223,804.39581559)
\curveto(190.03905143,804.20561515)(189.61902546,804.11051493)(189.12767433,804.11051493)
\curveto(188.32328498,804.11051493)(187.66748973,804.43147817)(187.16028856,805.07340464)
\curveto(186.65704991,805.71533112)(186.40543058,806.55934556)(186.40543058,807.60544796)
\curveto(186.40543058,808.65155037)(186.65704991,809.49556481)(187.16028856,810.13749128)
\curveto(187.66748973,810.77941776)(188.32328498,811.10038099)(189.12767433,811.10038099)
\curveto(189.61902546,811.10038099)(190.03905143,811.00329952)(190.38775223,810.80913657)
\curveto(190.74041554,810.61893614)(191.03165996,810.32571046)(191.26148549,809.92945955)
\lineto(191.26148549,810.93989937)
\lineto(192.355138,810.93989937)
\lineto(192.355138,801.75084076)
\lineto(191.26148549,801.75084076)
\closepath
}
}
{
\newrgbcolor{curcolor}{0 0 0}
\pscustom[linestyle=none,fillstyle=solid,fillcolor=curcolor]
{
\newpath
\moveto(194.49489291,806.91002761)
\lineto(194.49489291,810.93989937)
\lineto(195.58854543,810.93989937)
\lineto(195.58854543,806.95163396)
\curveto(195.58854543,806.32159501)(195.71138321,805.84807517)(195.95705877,805.53107445)
\curveto(196.20273434,805.21803623)(196.57124769,805.06151712)(197.06259881,805.06151712)
\curveto(197.65301267,805.06151712)(198.11860749,805.2497363)(198.45938327,805.62617466)
\curveto(198.80412157,806.00261303)(198.97649071,806.51575796)(198.97649071,807.16560945)
\lineto(198.97649071,810.93989937)
\lineto(200.07014322,810.93989937)
\lineto(200.07014322,804.28288408)
\lineto(198.97649071,804.28288408)
\lineto(198.97649071,805.30521143)
\curveto(198.7110026,804.9010355)(198.40192689,804.59988481)(198.04926358,804.40175935)
\curveto(197.70056278,804.20759641)(197.2944056,804.11051493)(196.83079203,804.11051493)
\curveto(196.06602777,804.11051493)(195.48552019,804.34826548)(195.08926928,804.82376657)
\curveto(194.69301837,805.29926766)(194.49489291,805.99468801)(194.49489291,806.91002761)
\closepath
\moveto(197.24685549,811.10038099)
\closepath
}
}
{
\newrgbcolor{curcolor}{0 0 0}
\pscustom[linestyle=none,fillstyle=solid,fillcolor=curcolor]
{
\newpath
\moveto(205.36009288,807.62922302)
\curveto(204.47645335,807.62922302)(203.86424569,807.52817903)(203.52346991,807.32609107)
\curveto(203.18269413,807.12400311)(203.01230623,806.77926481)(203.01230623,806.29187619)
\curveto(203.01230623,805.9035503)(203.13910653,805.59447459)(203.39270711,805.36464906)
\curveto(203.6502702,805.13878605)(203.998971,805.02585454)(204.43880951,805.02585454)
\curveto(205.0450734,805.02585454)(205.53048077,805.23983003)(205.89503161,805.66778101)
\curveto(206.26354495,806.0996945)(206.44780163,806.67227707)(206.44780163,807.38552871)
\lineto(206.44780163,807.62922302)
\closepath
\moveto(207.54145414,808.08094906)
\lineto(207.54145414,804.28288408)
\lineto(206.44780163,804.28288408)
\lineto(206.44780163,805.2933239)
\curveto(206.19816355,804.88914797)(205.88710659,804.58997853)(205.51463073,804.39581559)
\curveto(205.14215488,804.20561515)(204.68646633,804.11051493)(204.14756509,804.11051493)
\curveto(203.46601353,804.11051493)(202.92314978,804.30071537)(202.51897385,804.68111624)
\curveto(202.11876043,805.06547963)(201.91865372,805.57862456)(201.91865372,806.22055103)
\curveto(201.91865372,806.96946525)(202.16829179,807.5341228)(202.66756794,807.91452367)
\curveto(203.1708066,808.29492455)(203.91972082,808.48512498)(204.9143106,808.48512498)
\lineto(206.44780163,808.48512498)
\lineto(206.44780163,808.59211273)
\curveto(206.44780163,809.09535139)(206.28137625,809.48367728)(205.94852548,809.75709041)
\curveto(205.61963722,810.03446604)(205.15602366,810.17315386)(204.55768478,810.17315386)
\curveto(204.17728391,810.17315386)(203.80678931,810.12758501)(203.44620098,810.0364473)
\curveto(203.08561265,809.94530959)(202.73889311,809.80860302)(202.40604234,809.62632761)
\lineto(202.40604234,810.63676743)
\curveto(202.80625576,810.79130528)(203.19458165,810.90621805)(203.57102002,810.98150572)
\curveto(203.94745838,811.0607559)(204.31399047,811.10038099)(204.67061629,811.10038099)
\curveto(205.63350601,811.10038099)(206.35270141,810.85074292)(206.8282025,810.35146677)
\curveto(207.30370359,809.85219062)(207.54145414,809.09535139)(207.54145414,808.08094906)
\closepath
}
}
{
\newrgbcolor{curcolor}{0 0 0}
\pscustom[linestyle=none,fillstyle=solid,fillcolor=curcolor]
{
\newpath
\moveto(215.33372829,808.30086831)
\lineto(215.33372829,804.28288408)
\lineto(214.24007578,804.28288408)
\lineto(214.24007578,808.26520573)
\curveto(214.24007578,808.89524468)(214.117238,809.36678326)(213.87156243,809.67982148)
\curveto(213.62588687,809.9928597)(213.25737352,810.14937881)(212.76602239,810.14937881)
\curveto(212.17560854,810.14937881)(211.71001372,809.96115963)(211.36923794,809.58472126)
\curveto(211.02846215,809.2082829)(210.85807426,808.69513797)(210.85807426,808.04528647)
\lineto(210.85807426,804.28288408)
\lineto(209.75847798,804.28288408)
\lineto(209.75847798,810.93989937)
\lineto(210.85807426,810.93989937)
\lineto(210.85807426,809.9056845)
\curveto(211.11959986,810.30589792)(211.42669432,810.60506735)(211.77935763,810.80319281)
\curveto(212.13598345,811.00131826)(212.54610314,811.10038099)(213.0097167,811.10038099)
\curveto(213.77448096,811.10038099)(214.35300729,810.86263045)(214.74529569,810.38712935)
\curveto(215.13758409,809.91559077)(215.33372829,809.22017042)(215.33372829,808.30086831)
\closepath
}
}
{
\newrgbcolor{curcolor}{0 0 0}
\pscustom[linestyle=none,fillstyle=solid,fillcolor=curcolor]
{
\newpath
\moveto(218.60874207,812.83001622)
\lineto(218.60874207,810.93989937)
\lineto(220.86142849,810.93989937)
\lineto(220.86142849,810.08994117)
\lineto(218.60874207,810.08994117)
\lineto(218.60874207,806.47613287)
\curveto(218.60874207,805.93326912)(218.68204849,805.58456832)(218.82866132,805.43003046)
\curveto(218.97923667,805.27549261)(219.28236862,805.19822368)(219.73805716,805.19822368)
\lineto(220.86142849,805.19822368)
\lineto(220.86142849,804.28288408)
\lineto(219.73805716,804.28288408)
\curveto(218.89404272,804.28288408)(218.31155389,804.43940319)(217.99059065,804.75244141)
\curveto(217.66962741,805.06944214)(217.50914579,805.64400596)(217.50914579,806.47613287)
\lineto(217.50914579,810.08994117)
\lineto(216.7067377,810.08994117)
\lineto(216.7067377,810.93989937)
\lineto(217.50914579,810.93989937)
\lineto(217.50914579,812.83001622)
\closepath
}
}
{
\newrgbcolor{curcolor}{0 0 0}
\pscustom[linestyle=none,fillstyle=solid,fillcolor=curcolor]
{
\newpath
\moveto(222.13339392,813.5194928)
\lineto(223.0843961,813.5194928)
\curveto(223.67877247,812.58434065)(224.12257349,811.66900105)(224.41579916,810.77347399)
\curveto(224.71298734,809.87794693)(224.86158144,808.98836364)(224.86158144,808.10472411)
\curveto(224.86158144,807.21712207)(224.71298734,806.32357627)(224.41579916,805.4240867)
\curveto(224.12257349,804.52459713)(223.67877247,803.60925753)(223.0843961,802.67806789)
\lineto(222.13339392,802.67806789)
\curveto(222.66040763,803.58548248)(223.05269603,804.4869533)(223.31025912,805.38248035)
\curveto(223.57178472,806.28196992)(223.70254752,807.18938451)(223.70254752,808.10472411)
\curveto(223.70254752,809.02006371)(223.57178472,809.92351579)(223.31025912,810.81508034)
\curveto(223.05269603,811.70664489)(222.66040763,812.60811571)(222.13339392,813.5194928)
\closepath
}
}
{
\newrgbcolor{curcolor}{0 0 0}
\pscustom[linestyle=none,fillstyle=solid,fillcolor=curcolor]
{
\newpath
\moveto(230.02671205,802.49381122)
\lineto(229.43233569,802.49381122)
\lineto(229.42639192,804.28288408)
\curveto(229.01032847,804.2908091)(228.59426501,804.33835921)(228.17820156,804.42553441)
\curveto(227.7621381,804.51667212)(227.34409339,804.65139743)(226.92406742,804.82971033)
\lineto(226.92406742,805.89958779)
\curveto(227.32824335,805.64598721)(227.73638179,805.45380552)(228.14848274,805.32304272)
\curveto(228.56454619,805.19624243)(228.99249718,805.13086103)(229.43233569,805.12689852)
\lineto(229.43233569,807.83725475)
\curveto(228.55662118,807.97990507)(227.91865721,808.22161813)(227.51844379,808.56239391)
\curveto(227.12219288,808.90316969)(226.92406742,809.37074577)(226.92406742,809.96512213)
\curveto(226.92406742,810.61101112)(227.14002417,811.12019354)(227.57193766,811.49266939)
\curveto(228.00385116,811.86514525)(228.62398383,812.07912074)(229.43233569,812.13459587)
\lineto(229.43233569,813.53138033)
\lineto(230.02671205,813.53138033)
\lineto(230.02671205,812.15242716)
\curveto(230.3952254,812.13657712)(230.75185122,812.09695203)(231.09658951,812.03355189)
\curveto(231.4413278,811.97411425)(231.77814108,811.89090156)(232.10702933,811.78391381)
\lineto(232.10702933,810.74375517)
\curveto(231.77814108,810.91018056)(231.43934655,811.0389621)(231.09064575,811.13009981)
\curveto(230.74590746,811.22123752)(230.39126289,811.27473139)(230.02671205,811.29058143)
\lineto(230.02671205,808.75259435)
\curveto(230.92620162,808.61390653)(231.58794064,808.36624971)(232.01192911,808.00962389)
\curveto(232.43591759,807.65299807)(232.64791183,807.16560945)(232.64791183,806.54745803)
\curveto(232.64791183,805.87779399)(232.42204881,805.34879903)(231.97032277,804.96047314)
\curveto(231.52255924,804.57610975)(230.874689,804.35420924)(230.02671205,804.29477161)
\closepath
\moveto(229.43233569,808.85958209)
\lineto(229.43233569,811.29652519)
\curveto(228.97268463,811.24501257)(228.62200258,811.11424977)(228.38028952,810.90423679)
\curveto(228.13857646,810.69422381)(228.01771994,810.41486692)(228.01771994,810.06616612)
\curveto(228.01771994,809.72539033)(228.12867019,809.45990222)(228.3505707,809.26970179)
\curveto(228.57643372,809.07950135)(228.93702205,808.94279479)(229.43233569,808.85958209)
\closepath
\moveto(230.02671205,807.71837947)
\lineto(230.02671205,805.14472981)
\curveto(230.52995071,805.21209246)(230.90837033,805.35474279)(231.16197091,805.57268079)
\curveto(231.419534,805.79061879)(231.54831555,806.0779007)(231.54831555,806.43452652)
\curveto(231.54831555,806.78322732)(231.42547777,807.06060296)(231.1798022,807.26665343)
\curveto(230.93808915,807.47270391)(230.55372576,807.62327925)(230.02671205,807.71837947)
\closepath
}
}
{
\newrgbcolor{curcolor}{0 0 0}
\pscustom[linestyle=none,fillstyle=solid,fillcolor=curcolor]
{
\newpath
\moveto(238.13400568,808.10472411)
\lineto(241.33769429,808.10472411)
\lineto(241.33769429,807.12994687)
\lineto(238.13400568,807.12994687)
\closepath
}
}
{
\newrgbcolor{curcolor}{0 0 0}
\pscustom[linestyle=none,fillstyle=solid,fillcolor=curcolor]
{
\newpath
\moveto(252.32771329,812.8656788)
\lineto(252.32771329,811.69475736)
\curveto(251.87202475,811.91269536)(251.44209251,812.07515823)(251.03791658,812.18214598)
\curveto(250.63374065,812.28913372)(250.2434335,812.3426276)(249.86699514,812.3426276)
\curveto(249.21318114,812.3426276)(248.70796123,812.21582731)(248.35133541,811.96222672)
\curveto(247.9986721,811.70862614)(247.82234044,811.34803781)(247.82234044,810.88046174)
\curveto(247.82234044,810.48817334)(247.93923446,810.19098515)(248.1730225,809.98889719)
\curveto(248.41077304,809.79077173)(248.85853657,809.63029011)(249.51631308,809.50745233)
\lineto(250.24145225,809.35885824)
\curveto(251.13697931,809.18847035)(251.79673707,808.88731966)(252.22072555,808.45540617)
\curveto(252.64867653,808.02745518)(252.86265202,807.45289136)(252.86265202,806.7317147)
\curveto(252.86265202,805.87185023)(252.57338886,805.22001748)(251.99486253,804.77621646)
\curveto(251.42029871,804.33241544)(250.57628427,804.11051493)(249.46281921,804.11051493)
\curveto(249.04279324,804.11051493)(248.59502972,804.15806504)(248.11952862,804.25316526)
\curveto(247.64799004,804.34826548)(247.15862017,804.48893455)(246.651419,804.67517248)
\lineto(246.651419,805.91147532)
\curveto(247.13880762,805.63806219)(247.61628997,805.43201172)(248.08386604,805.2933239)
\curveto(248.55144212,805.15463608)(249.01109317,805.08529217)(249.46281921,805.08529217)
\curveto(250.14833328,805.08529217)(250.67732825,805.22001748)(251.04980411,805.4894681)
\curveto(251.42227996,805.75891872)(251.60851789,806.1432821)(251.60851789,806.64255825)
\curveto(251.60851789,807.07843425)(251.47379258,807.41921003)(251.20434196,807.6648856)
\curveto(250.93885385,807.91056116)(250.5009966,808.09481784)(249.89077019,808.21765562)
\lineto(249.15968726,808.36030595)
\curveto(248.26416021,808.53861886)(247.61628997,808.81797575)(247.21607655,809.19837662)
\curveto(246.81586313,809.5787775)(246.61575642,810.10777246)(246.61575642,810.78536152)
\curveto(246.61575642,811.56993832)(246.8911508,812.18808974)(247.44193957,812.63981578)
\curveto(247.99669084,813.09154182)(248.75947384,813.31740484)(249.73028857,813.31740484)
\curveto(250.14635203,813.31740484)(250.5703405,813.279761)(251.002254,813.20447333)
\curveto(251.43416749,813.12918565)(251.87598725,813.01625414)(252.32771329,812.8656788)
\closepath
}
}
{
\newrgbcolor{curcolor}{0 0 0}
\pscustom[linestyle=none,fillstyle=solid,fillcolor=curcolor]
{
\newpath
\moveto(255.93557783,812.17025845)
\lineto(255.93557783,808.83580704)
\lineto(257.4452938,808.83580704)
\curveto(258.00400758,808.83580704)(258.43592108,808.98043862)(258.74103428,809.26970179)
\curveto(259.04614748,809.55896495)(259.19870408,809.9710659)(259.19870408,810.50600463)
\curveto(259.19870408,811.03698085)(259.04614748,811.44710054)(258.74103428,811.7363637)
\curveto(258.43592108,812.02562687)(258.00400758,812.17025845)(257.4452938,812.17025845)
\closepath
\moveto(254.73493757,813.15692322)
\lineto(257.4452938,813.15692322)
\curveto(258.43988359,813.15692322)(259.19077906,812.9310602)(259.69798023,812.47933416)
\curveto(260.2091439,812.03157063)(260.46472574,811.37379412)(260.46472574,810.50600463)
\curveto(260.46472574,809.63029011)(260.2091439,808.96855109)(259.69798023,808.52078757)
\curveto(259.19077906,808.07302404)(258.43988359,807.84914227)(257.4452938,807.84914227)
\lineto(255.93557783,807.84914227)
\lineto(255.93557783,804.28288408)
\lineto(254.73493757,804.28288408)
\closepath
}
}
{
\newrgbcolor{curcolor}{0 0 0}
\pscustom[linestyle=none,fillstyle=solid,fillcolor=curcolor]
{
\newpath
\moveto(131.1562807,757.42164579)
\lineto(132.7135724,757.42164579)
\lineto(132.7135724,744.25237464)
\lineto(131.1562807,744.25237464)
\closepath
}
}
{
\newrgbcolor{curcolor}{0 0 0}
\pscustom[linestyle=none,fillstyle=solid,fillcolor=curcolor]
{
\newpath
\moveto(135.96357262,753.73154154)
\lineto(137.52086432,753.73154154)
\lineto(137.52086432,744.25237464)
\lineto(135.96357262,744.25237464)
\closepath
\moveto(135.96357262,757.42164579)
\lineto(137.52086432,757.42164579)
\lineto(137.52086432,755.44964054)
\lineto(135.96357262,755.44964054)
\closepath
}
}
{
\newrgbcolor{curcolor}{0 0 0}
\pscustom[linestyle=none,fillstyle=solid,fillcolor=curcolor]
{
\newpath
\moveto(146.81383344,753.45224465)
\lineto(146.81383344,751.97958837)
\curveto(146.37372926,752.20528282)(145.916698,752.37455366)(145.44273966,752.48740088)
\curveto(144.96878131,752.60024811)(144.47789588,752.65667172)(143.97008337,752.65667172)
\curveto(143.19707988,752.65667172)(142.61591667,752.53818213)(142.22659375,752.30120296)
\curveto(141.84291318,752.06422379)(141.6510729,751.70875503)(141.6510729,751.23479668)
\curveto(141.6510729,750.87368556)(141.78931075,750.58874632)(142.06578645,750.37997895)
\curveto(142.34226215,750.17685395)(142.89803473,749.98219249)(143.7331042,749.79599457)
\lineto(144.26630734,749.67750498)
\curveto(145.37221014,749.44052581)(146.15649835,749.10480531)(146.61917198,748.6703435)
\curveto(147.08748796,748.24152404)(147.32164595,747.64061257)(147.32164595,746.86760908)
\curveto(147.32164595,745.98740072)(146.97181955,745.29056911)(146.27216676,744.77711424)
\curveto(145.57815633,744.26365936)(144.62177609,744.00693193)(143.40302607,744.00693193)
\curveto(142.89521355,744.00693193)(142.3648316,744.05771318)(141.81188019,744.15927568)
\curveto(141.26457115,744.25519582)(140.68622912,744.40189721)(140.07685411,744.59937986)
\lineto(140.07685411,746.20745281)
\curveto(140.65237496,745.90840767)(141.21943226,745.68271322)(141.77802603,745.53036946)
\curveto(142.33661979,745.38366807)(142.88957119,745.31031737)(143.43688023,745.31031737)
\curveto(144.17038719,745.31031737)(144.73462332,745.43444932)(145.12958861,745.68271322)
\curveto(145.52455389,745.93661947)(145.72203654,746.29208823)(145.72203654,746.74911949)
\curveto(145.72203654,747.17229659)(145.57815633,747.49673236)(145.2903959,747.72242681)
\curveto(145.00827784,747.94812126)(144.38479692,748.16535217)(143.41995315,748.37411953)
\lineto(142.87828647,748.50107266)
\curveto(141.9134427,748.70419766)(141.21661108,749.01452753)(140.78779163,749.43206227)
\curveto(140.35897217,749.85523936)(140.14456245,750.43358139)(140.14456245,751.16708835)
\curveto(140.14456245,752.05858143)(140.46053467,752.7469495)(141.09247913,753.23219257)
\curveto(141.72442359,753.71743563)(142.62155903,753.96005717)(143.78388545,753.96005717)
\curveto(144.3594063,753.96005717)(144.90107298,753.91773946)(145.40888549,753.83310404)
\curveto(145.916698,753.74846862)(146.38501398,753.62151549)(146.81383344,753.45224465)
\closepath
}
}
{
\newrgbcolor{curcolor}{0 0 0}
\pscustom[linestyle=none,fillstyle=solid,fillcolor=curcolor]
{
\newpath
\moveto(151.35029155,756.42294785)
\lineto(151.35029155,753.73154154)
\lineto(154.55797392,753.73154154)
\lineto(154.55797392,752.52125505)
\lineto(151.35029155,752.52125505)
\lineto(151.35029155,747.37542159)
\curveto(151.35029155,746.6024181)(151.45467523,746.10589031)(151.6634426,745.88583822)
\curveto(151.87785233,745.66578613)(152.30949296,745.55576009)(152.95836451,745.55576009)
\lineto(154.55797392,745.55576009)
\lineto(154.55797392,744.25237464)
\lineto(152.95836451,744.25237464)
\curveto(151.75654156,744.25237464)(150.92711446,744.47524791)(150.4700832,744.92099445)
\curveto(150.01305193,745.37238335)(149.7845363,746.19052573)(149.7845363,747.37542159)
\lineto(149.7845363,752.52125505)
\lineto(148.64195815,752.52125505)
\lineto(148.64195815,753.73154154)
\lineto(149.7845363,753.73154154)
\lineto(149.7845363,756.42294785)
\closepath
}
}
{
\newrgbcolor{curcolor}{0 0 0}
\pscustom[linestyle=none,fillstyle=solid,fillcolor=curcolor]
{
\newpath
\moveto(501.12818132,835.0121161)
\lineto(500.53380495,835.0121161)
\lineto(500.52786119,836.80118896)
\curveto(500.11179773,836.80911398)(499.69573428,836.85666409)(499.27967082,836.94383929)
\curveto(498.86360737,837.034977)(498.44556265,837.16970231)(498.02553669,837.34801522)
\lineto(498.02553669,838.41789267)
\curveto(498.42971262,838.16429209)(498.83785106,837.9721104)(499.249952,837.8413476)
\curveto(499.66601546,837.71454731)(500.09396644,837.64916591)(500.53380495,837.6452034)
\lineto(500.53380495,840.35555963)
\curveto(499.65809044,840.49820995)(499.02012647,840.73992301)(498.61991306,841.08069879)
\curveto(498.22366214,841.42147458)(498.02553669,841.88905065)(498.02553669,842.48342702)
\curveto(498.02553669,843.129316)(498.24149344,843.63849842)(498.67340693,844.01097428)
\curveto(499.10532042,844.38345013)(499.7254531,844.59742562)(500.53380495,844.65290075)
\lineto(500.53380495,846.04968521)
\lineto(501.12818132,846.04968521)
\lineto(501.12818132,844.67073204)
\curveto(501.49669466,844.65488201)(501.85332048,844.61525691)(502.19805878,844.55185677)
\curveto(502.54279707,844.49241913)(502.87961034,844.40920644)(503.2084986,844.30221869)
\lineto(503.2084986,843.26206006)
\curveto(502.87961034,843.42848544)(502.54081581,843.55726698)(502.19211501,843.64840469)
\curveto(501.84737672,843.7395424)(501.49273216,843.79303628)(501.12818132,843.80888631)
\lineto(501.12818132,841.27089923)
\curveto(502.02767088,841.13221141)(502.68940991,840.88455459)(503.11339838,840.52792877)
\curveto(503.53738685,840.17130295)(503.74938109,839.68391433)(503.74938109,839.06576291)
\curveto(503.74938109,838.39609887)(503.52351807,837.86710391)(503.07179203,837.47877802)
\curveto(502.62402851,837.09441463)(501.97615827,836.87251412)(501.12818132,836.81307649)
\closepath
\moveto(500.53380495,841.37788698)
\lineto(500.53380495,843.81483008)
\curveto(500.0741539,843.76331746)(499.72347184,843.63255466)(499.48175879,843.42254167)
\curveto(499.24004573,843.21252869)(499.1191892,842.9331718)(499.1191892,842.584471)
\curveto(499.1191892,842.24369522)(499.23013946,841.97820711)(499.45203997,841.78800667)
\curveto(499.67790299,841.59780623)(500.03849131,841.46109967)(500.53380495,841.37788698)
\closepath
\moveto(501.12818132,840.23668435)
\lineto(501.12818132,837.66303469)
\curveto(501.63141997,837.73039735)(502.00983959,837.87304767)(502.26344018,838.09098567)
\curveto(502.52100327,838.30892367)(502.64978481,838.59620558)(502.64978481,838.9528314)
\curveto(502.64978481,839.30153221)(502.52694703,839.57890784)(502.28127147,839.78495832)
\curveto(502.03955841,839.99100879)(501.65519503,840.14158414)(501.12818132,840.23668435)
\closepath
}
}
{
\newrgbcolor{curcolor}{0 0 0}
\pscustom[linestyle=none,fillstyle=solid,fillcolor=curcolor]
{
\newpath
\moveto(505.7762045,845.6752281)
\lineto(508.8669616,835.67187387)
\lineto(507.85652178,835.67187387)
\lineto(504.76576468,845.6752281)
\closepath
}
}
{
\newrgbcolor{curcolor}{0 0 0}
\pscustom[linestyle=none,fillstyle=solid,fillcolor=curcolor]
{
\newpath
\moveto(511.27418588,844.68856333)
\lineto(511.27418588,841.35411192)
\lineto(512.78390185,841.35411192)
\curveto(513.34261563,841.35411192)(513.77452912,841.4987435)(514.07964233,841.78800667)
\curveto(514.38475553,842.07726983)(514.53731213,842.48937078)(514.53731213,843.02430951)
\curveto(514.53731213,843.55528573)(514.38475553,843.96540542)(514.07964233,844.25466859)
\curveto(513.77452912,844.54393175)(513.34261563,844.68856333)(512.78390185,844.68856333)
\closepath
\moveto(510.07354562,845.6752281)
\lineto(512.78390185,845.6752281)
\curveto(513.77849163,845.6752281)(514.52938711,845.44936508)(515.03658827,844.99763904)
\curveto(515.54775195,844.54987551)(515.80333379,843.892099)(515.80333379,843.02430951)
\curveto(515.80333379,842.148595)(515.54775195,841.48685598)(515.03658827,841.03909245)
\curveto(514.52938711,840.59132892)(513.77849163,840.36744715)(512.78390185,840.36744715)
\lineto(511.27418588,840.36744715)
\lineto(511.27418588,836.80118896)
\lineto(510.07354562,836.80118896)
\closepath
}
}
{
\newrgbcolor{curcolor}{0 0 0}
\pscustom[linestyle=none,fillstyle=solid,fillcolor=curcolor]
{
\newpath
\moveto(516.18967842,845.6752281)
\lineto(523.69665192,845.6752281)
\lineto(523.69665192,844.66478828)
\lineto(520.54645718,844.66478828)
\lineto(520.54645718,836.80118896)
\lineto(519.33987316,836.80118896)
\lineto(519.33987316,844.66478828)
\lineto(516.18967842,844.66478828)
\closepath
}
}
{
\newrgbcolor{curcolor}{0 0 0}
\pscustom[linestyle=none,fillstyle=solid,fillcolor=curcolor]
{
\newpath
\moveto(527.78001755,835.0121161)
\lineto(527.18564119,835.0121161)
\lineto(527.17969742,836.80118896)
\curveto(526.76363397,836.80911398)(526.34757051,836.85666409)(525.93150706,836.94383929)
\curveto(525.5154436,837.034977)(525.09739889,837.16970231)(524.67737293,837.34801522)
\lineto(524.67737293,838.41789267)
\curveto(525.08154885,838.16429209)(525.48968729,837.9721104)(525.90178824,837.8413476)
\curveto(526.31785169,837.71454731)(526.74580268,837.64916591)(527.18564119,837.6452034)
\lineto(527.18564119,840.35555963)
\curveto(526.30992668,840.49820995)(525.67196271,840.73992301)(525.27174929,841.08069879)
\curveto(524.87549838,841.42147458)(524.67737293,841.88905065)(524.67737293,842.48342702)
\curveto(524.67737293,843.129316)(524.89332967,843.63849842)(525.32524316,844.01097428)
\curveto(525.75715666,844.38345013)(526.37728933,844.59742562)(527.18564119,844.65290075)
\lineto(527.18564119,846.04968521)
\lineto(527.78001755,846.04968521)
\lineto(527.78001755,844.67073204)
\curveto(528.1485309,844.65488201)(528.50515672,844.61525691)(528.84989501,844.55185677)
\curveto(529.1946333,844.49241913)(529.53144658,844.40920644)(529.86033483,844.30221869)
\lineto(529.86033483,843.26206006)
\curveto(529.53144658,843.42848544)(529.19265205,843.55726698)(528.84395125,843.64840469)
\curveto(528.49921296,843.7395424)(528.14456839,843.79303628)(527.78001755,843.80888631)
\lineto(527.78001755,841.27089923)
\curveto(528.67950712,841.13221141)(529.34124614,840.88455459)(529.76523462,840.52792877)
\curveto(530.18922309,840.17130295)(530.40121733,839.68391433)(530.40121733,839.06576291)
\curveto(530.40121733,838.39609887)(530.17535431,837.86710391)(529.72362827,837.47877802)
\curveto(529.27586474,837.09441463)(528.6279945,836.87251412)(527.78001755,836.81307649)
\closepath
\moveto(527.18564119,841.37788698)
\lineto(527.18564119,843.81483008)
\curveto(526.72599013,843.76331746)(526.37530808,843.63255466)(526.13359502,843.42254167)
\curveto(525.89188197,843.21252869)(525.77102544,842.9331718)(525.77102544,842.584471)
\curveto(525.77102544,842.24369522)(525.88197569,841.97820711)(526.1038762,841.78800667)
\curveto(526.32973922,841.59780623)(526.69032755,841.46109967)(527.18564119,841.37788698)
\closepath
\moveto(527.78001755,840.23668435)
\lineto(527.78001755,837.66303469)
\curveto(528.28325621,837.73039735)(528.66167583,837.87304767)(528.91527641,838.09098567)
\curveto(529.1728395,838.30892367)(529.30162105,838.59620558)(529.30162105,838.9528314)
\curveto(529.30162105,839.30153221)(529.17878327,839.57890784)(528.9331077,839.78495832)
\curveto(528.69139465,839.99100879)(528.30703126,840.14158414)(527.78001755,840.23668435)
\closepath
}
}
{
\newrgbcolor{curcolor}{0 0 0}
\pscustom[linestyle=none,fillstyle=solid,fillcolor=curcolor]
{
\newpath
\moveto(532.01197728,840.62302899)
\lineto(535.21566589,840.62302899)
\lineto(535.21566589,839.64825175)
\lineto(532.01197728,839.64825175)
\closepath
}
}
{
\newrgbcolor{curcolor}{0 0 0}
\pscustom[linestyle=none,fillstyle=solid,fillcolor=curcolor]
{
\newpath
\moveto(536.87397595,845.6752281)
\lineto(538.08055997,845.6752281)
\lineto(538.08055997,840.28423446)
\curveto(538.08055997,839.33323228)(538.25292912,838.6477182)(538.59766741,838.22769224)
\curveto(538.9424057,837.81162878)(539.50111948,837.60359705)(540.27380876,837.60359705)
\curveto(541.04253553,837.60359705)(541.59926806,837.81162878)(541.94400635,838.22769224)
\curveto(542.28874464,838.6477182)(542.46111379,839.33323228)(542.46111379,840.28423446)
\lineto(542.46111379,845.6752281)
\lineto(543.66769781,845.6752281)
\lineto(543.66769781,840.13564037)
\curveto(543.66769781,838.97858771)(543.3804159,838.10485446)(542.80585208,837.5144406)
\curveto(542.23525077,836.92402674)(541.39123633,836.62881981)(540.27380876,836.62881981)
\curveto(539.15241868,836.62881981)(538.30444174,836.92402674)(537.72987791,837.5144406)
\curveto(537.1592766,838.10485446)(536.87397595,838.97858771)(536.87397595,840.13564037)
\closepath
}
}
{
\newrgbcolor{curcolor}{0 0 0}
\pscustom[linestyle=none,fillstyle=solid,fillcolor=curcolor]
{
\newpath
\moveto(551.41242185,840.81917319)
\lineto(551.41242185,836.80118896)
\lineto(550.31876934,836.80118896)
\lineto(550.31876934,840.78351061)
\curveto(550.31876934,841.41354956)(550.19593156,841.88508814)(549.95025599,842.19812636)
\curveto(549.70458043,842.51116458)(549.33606708,842.66768369)(548.84471595,842.66768369)
\curveto(548.2543021,842.66768369)(547.78870728,842.47946451)(547.44793149,842.10302614)
\curveto(547.10715571,841.72658778)(546.93676782,841.21344285)(546.93676782,840.56359135)
\lineto(546.93676782,836.80118896)
\lineto(545.83717154,836.80118896)
\lineto(545.83717154,843.45820426)
\lineto(546.93676782,843.45820426)
\lineto(546.93676782,842.42398938)
\curveto(547.19829342,842.8242028)(547.50538788,843.12337224)(547.85805119,843.32149769)
\curveto(548.214677,843.51962315)(548.6247967,843.61868587)(549.08841026,843.61868587)
\curveto(549.85317452,843.61868587)(550.43170085,843.38093533)(550.82398925,842.90543424)
\curveto(551.21627765,842.43389565)(551.41242185,841.7384753)(551.41242185,840.81917319)
\closepath
}
}
{
\newrgbcolor{curcolor}{0 0 0}
\pscustom[linestyle=none,fillstyle=solid,fillcolor=curcolor]
{
\newpath
\moveto(553.60567064,843.45820426)
\lineto(554.69932315,843.45820426)
\lineto(554.69932315,836.80118896)
\lineto(553.60567064,836.80118896)
\closepath
\moveto(553.60567064,846.04968521)
\lineto(554.69932315,846.04968521)
\lineto(554.69932315,844.66478828)
\lineto(553.60567064,844.66478828)
\closepath
}
}
{
\newrgbcolor{curcolor}{0 0 0}
\pscustom[linestyle=none,fillstyle=solid,fillcolor=curcolor]
{
\newpath
\moveto(560.35184239,846.04968521)
\lineto(560.35184239,845.14028937)
\lineto(559.30573999,845.14028937)
\curveto(558.91345159,845.14028937)(558.64003846,845.06103919)(558.4855006,844.90253882)
\curveto(558.33492526,844.74403846)(558.25963758,844.4587378)(558.25963758,844.04663686)
\lineto(558.25963758,843.45820426)
\lineto(560.06059797,843.45820426)
\lineto(560.06059797,842.60824605)
\lineto(558.25963758,842.60824605)
\lineto(558.25963758,836.80118896)
\lineto(557.16004131,836.80118896)
\lineto(557.16004131,842.60824605)
\lineto(556.1139389,842.60824605)
\lineto(556.1139389,843.45820426)
\lineto(557.16004131,843.45820426)
\lineto(557.16004131,843.92181782)
\curveto(557.16004131,844.66280702)(557.33241045,845.20170826)(557.67714875,845.53852154)
\curveto(558.02188704,845.87929732)(558.56871329,846.04968521)(559.31762752,846.04968521)
\closepath
}
}
{
\newrgbcolor{curcolor}{0 0 0}
\pscustom[linestyle=none,fillstyle=solid,fillcolor=curcolor]
{
\newpath
\moveto(563.84083166,842.69145874)
\curveto(563.25438031,842.69145874)(562.79076675,842.46163322)(562.44999096,842.00198216)
\curveto(562.10921518,841.54629361)(561.93882729,840.92021717)(561.93882729,840.12375284)
\curveto(561.93882729,839.32728851)(562.10723392,838.69923082)(562.4440472,838.23957977)
\curveto(562.78482298,837.78389122)(563.2504178,837.55604694)(563.84083166,837.55604694)
\curveto(564.4233205,837.55604694)(564.88495281,837.78587247)(565.22572859,838.24552353)
\curveto(565.56650437,838.70517459)(565.73689226,839.33125102)(565.73689226,840.12375284)
\curveto(565.73689226,840.91229216)(565.56650437,841.53638734)(565.22572859,841.9960384)
\curveto(564.88495281,842.45965196)(564.4233205,842.69145874)(563.84083166,842.69145874)
\closepath
\moveto(563.84083166,843.61868587)
\curveto(564.79183384,843.61868587)(565.53876681,843.30961016)(566.08163056,842.69145874)
\curveto(566.6244943,842.07330732)(566.89592618,841.21740536)(566.89592618,840.12375284)
\curveto(566.89592618,839.03406284)(566.6244943,838.17816087)(566.08163056,837.55604694)
\curveto(565.53876681,836.93789552)(564.79183384,836.62881981)(563.84083166,836.62881981)
\curveto(562.88586696,836.62881981)(562.13695274,836.93789552)(561.594089,837.55604694)
\curveto(561.05518776,838.17816087)(560.78573714,839.03406284)(560.78573714,840.12375284)
\curveto(560.78573714,841.21740536)(561.05518776,842.07330732)(561.594089,842.69145874)
\curveto(562.13695274,843.30961016)(562.88586696,843.61868587)(563.84083166,843.61868587)
\closepath
}
}
{
\newrgbcolor{curcolor}{0 0 0}
\pscustom[linestyle=none,fillstyle=solid,fillcolor=curcolor]
{
\newpath
\moveto(572.56033294,842.43587691)
\curveto(572.43749516,842.50720207)(572.30276985,842.55871469)(572.15615701,842.59041476)
\curveto(572.01350669,842.62607734)(571.85500632,842.64390863)(571.68065592,842.64390863)
\curveto(571.0625045,842.64390863)(570.58700341,842.44182067)(570.25415264,842.03764474)
\curveto(569.92526439,841.63743132)(569.76082026,841.06088625)(569.76082026,840.30800952)
\lineto(569.76082026,836.80118896)
\lineto(568.66122398,836.80118896)
\lineto(568.66122398,843.45820426)
\lineto(569.76082026,843.45820426)
\lineto(569.76082026,842.42398938)
\curveto(569.99064579,842.82816531)(570.28981523,843.12733475)(570.65832857,843.32149769)
\curveto(571.02684192,843.51962315)(571.47460545,843.61868587)(572.00161916,843.61868587)
\curveto(572.07690683,843.61868587)(572.16011952,843.61274211)(572.25125723,843.60085458)
\curveto(572.34239494,843.59292957)(572.44343892,843.57906078)(572.55438918,843.55924824)
\closepath
}
}
{
\newrgbcolor{curcolor}{0 0 0}
\pscustom[linestyle=none,fillstyle=solid,fillcolor=curcolor]
{
\newpath
\moveto(578.68835327,842.18029507)
\curveto(578.9617664,842.6716462)(579.2886734,843.03421578)(579.66907428,843.26800382)
\curveto(580.04947515,843.50179186)(580.49723868,843.61868587)(581.01236486,843.61868587)
\curveto(581.70580395,843.61868587)(582.24074268,843.37499156)(582.61718105,842.88760294)
\curveto(582.99361941,842.40417683)(583.1818386,841.71470025)(583.1818386,840.81917319)
\lineto(583.1818386,836.80118896)
\lineto(582.08224232,836.80118896)
\lineto(582.08224232,840.78351061)
\curveto(582.08224232,841.42147458)(581.96931081,841.89499441)(581.74344779,842.20407012)
\curveto(581.51758477,842.51314583)(581.17284648,842.66768369)(580.70923292,842.66768369)
\curveto(580.14259411,842.66768369)(579.69483058,842.47946451)(579.36594233,842.10302614)
\curveto(579.03705407,841.72658778)(578.87260995,841.21344285)(578.87260995,840.56359135)
\lineto(578.87260995,836.80118896)
\lineto(577.77301367,836.80118896)
\lineto(577.77301367,840.78351061)
\curveto(577.77301367,841.42543709)(577.66008216,841.89895692)(577.43421914,842.20407012)
\curveto(577.20835612,842.51314583)(576.85965532,842.66768369)(576.38811674,842.66768369)
\curveto(575.82940295,842.66768369)(575.38560193,842.47748325)(575.05671368,842.09708238)
\curveto(574.72782542,841.72064401)(574.56338129,841.20948034)(574.56338129,840.56359135)
\lineto(574.56338129,836.80118896)
\lineto(573.46378502,836.80118896)
\lineto(573.46378502,843.45820426)
\lineto(574.56338129,843.45820426)
\lineto(574.56338129,842.42398938)
\curveto(574.81301937,842.83212782)(575.11218881,843.13327851)(575.46088961,843.32744146)
\curveto(575.80959041,843.5216044)(576.22367261,843.61868587)(576.70313621,843.61868587)
\curveto(577.18656232,843.61868587)(577.59668201,843.49584809)(577.93349529,843.25017253)
\curveto(578.27427107,843.00449696)(578.5258904,842.64787114)(578.68835327,842.18029507)
\closepath
}
}
{
\newrgbcolor{curcolor}{0 0 0}
\pscustom[linestyle=none,fillstyle=solid,fillcolor=curcolor]
{
\newpath
\moveto(501.12818132,804.58004618)
\lineto(500.53380495,804.58004618)
\lineto(500.52786119,806.36911904)
\curveto(500.11179773,806.37704406)(499.69573428,806.42459417)(499.27967082,806.51176937)
\curveto(498.86360737,806.60290708)(498.44556265,806.73763239)(498.02553669,806.9159453)
\lineto(498.02553669,807.98582275)
\curveto(498.42971262,807.73222217)(498.83785106,807.54004048)(499.249952,807.40927768)
\curveto(499.66601546,807.28247739)(500.09396644,807.21709599)(500.53380495,807.21313348)
\lineto(500.53380495,809.92348971)
\curveto(499.65809044,810.06614003)(499.02012647,810.30785309)(498.61991306,810.64862887)
\curveto(498.22366214,810.98940465)(498.02553669,811.45698073)(498.02553669,812.05135709)
\curveto(498.02553669,812.69724608)(498.24149344,813.2064285)(498.67340693,813.57890435)
\curveto(499.10532042,813.95138021)(499.7254531,814.1653557)(500.53380495,814.22083083)
\lineto(500.53380495,815.61761529)
\lineto(501.12818132,815.61761529)
\lineto(501.12818132,814.23866212)
\curveto(501.49669466,814.22281208)(501.85332048,814.18318699)(502.19805878,814.11978685)
\curveto(502.54279707,814.06034921)(502.87961034,813.97713652)(503.2084986,813.87014877)
\lineto(503.2084986,812.82999013)
\curveto(502.87961034,812.99641552)(502.54081581,813.12519706)(502.19211501,813.21633477)
\curveto(501.84737672,813.30747248)(501.49273216,813.36096635)(501.12818132,813.37681639)
\lineto(501.12818132,810.83882931)
\curveto(502.02767088,810.70014149)(502.68940991,810.45248467)(503.11339838,810.09585885)
\curveto(503.53738685,809.73923303)(503.74938109,809.25184441)(503.74938109,808.63369299)
\curveto(503.74938109,807.96402895)(503.52351807,807.43503399)(503.07179203,807.0467081)
\curveto(502.62402851,806.66234471)(501.97615827,806.4404442)(501.12818132,806.38100657)
\closepath
\moveto(500.53380495,810.94581705)
\lineto(500.53380495,813.38276015)
\curveto(500.0741539,813.33124754)(499.72347184,813.20048474)(499.48175879,812.99047175)
\curveto(499.24004573,812.78045877)(499.1191892,812.50110188)(499.1191892,812.15240108)
\curveto(499.1191892,811.81162529)(499.23013946,811.54613718)(499.45203997,811.35593675)
\curveto(499.67790299,811.16573631)(500.03849131,811.02902975)(500.53380495,810.94581705)
\closepath
\moveto(501.12818132,809.80461443)
\lineto(501.12818132,807.23096477)
\curveto(501.63141997,807.29832742)(502.00983959,807.44097775)(502.26344018,807.65891575)
\curveto(502.52100327,807.87685375)(502.64978481,808.16413566)(502.64978481,808.52076148)
\curveto(502.64978481,808.86946228)(502.52694703,809.14683792)(502.28127147,809.35288839)
\curveto(502.03955841,809.55893887)(501.65519503,809.70951421)(501.12818132,809.80461443)
\closepath
}
}
{
\newrgbcolor{curcolor}{0 0 0}
\pscustom[linestyle=none,fillstyle=solid,fillcolor=curcolor]
{
\newpath
\moveto(505.7762045,815.24315818)
\lineto(508.8669616,805.23980394)
\lineto(507.85652178,805.23980394)
\lineto(504.76576468,815.24315818)
\closepath
}
}
{
\newrgbcolor{curcolor}{0 0 0}
\pscustom[linestyle=none,fillstyle=solid,fillcolor=curcolor]
{
\newpath
\moveto(510.07354562,815.24315818)
\lineto(515.17329484,815.24315818)
\lineto(515.17329484,814.23271836)
\lineto(511.27418588,814.23271836)
\lineto(511.27418588,811.61746235)
\lineto(514.79289396,811.61746235)
\lineto(514.79289396,810.60702253)
\lineto(511.27418588,810.60702253)
\lineto(511.27418588,806.36911904)
\lineto(510.07354562,806.36911904)
\closepath
}
}
{
\newrgbcolor{curcolor}{0 0 0}
\pscustom[linestyle=none,fillstyle=solid,fillcolor=curcolor]
{
\newpath
\moveto(520.68910751,814.42886256)
\curveto(519.81735551,814.42886256)(519.12391642,814.10393681)(518.60879023,813.45408532)
\curveto(518.09762656,812.80423382)(517.84204472,811.91861304)(517.84204472,810.79722296)
\curveto(517.84204472,809.6797954)(518.09762656,808.79615587)(518.60879023,808.14630437)
\curveto(519.12391642,807.49645288)(519.81735551,807.17152713)(520.68910751,807.17152713)
\curveto(521.56085951,807.17152713)(522.2503361,807.49645288)(522.75753726,808.14630437)
\curveto(523.26870094,808.79615587)(523.52428278,809.6797954)(523.52428278,810.79722296)
\curveto(523.52428278,811.91861304)(523.26870094,812.80423382)(522.75753726,813.45408532)
\curveto(522.2503361,814.10393681)(521.56085951,814.42886256)(520.68910751,814.42886256)
\closepath
\moveto(520.68910751,815.4036398)
\curveto(521.93333537,815.4036398)(522.92792516,814.98559509)(523.67287687,814.14950567)
\curveto(524.41782858,813.31737875)(524.79030443,812.19995119)(524.79030443,810.79722296)
\curveto(524.79030443,809.39845725)(524.41782858,808.28102968)(523.67287687,807.44494026)
\curveto(522.92792516,806.61281335)(521.93333537,806.19674989)(520.68910751,806.19674989)
\curveto(519.44091714,806.19674989)(518.44236485,806.61281335)(517.69345063,807.44494026)
\curveto(516.94849892,808.27706717)(516.57602306,809.39449474)(516.57602306,810.79722296)
\curveto(516.57602306,812.19995119)(516.94849892,813.31737875)(517.69345063,814.14950567)
\curveto(518.44236485,814.98559509)(519.44091714,815.4036398)(520.68910751,815.4036398)
\closepath
}
}
{
\newrgbcolor{curcolor}{0 0 0}
\pscustom[linestyle=none,fillstyle=solid,fillcolor=curcolor]
{
\newpath
\moveto(529.24812718,815.60572776)
\curveto(528.71715096,814.69435067)(528.3228813,813.79287985)(528.06531821,812.9013153)
\curveto(527.80775512,812.00975075)(527.67897357,811.10629867)(527.67897357,810.19095907)
\curveto(527.67897357,809.27561947)(527.80775512,808.36820488)(528.06531821,807.46871532)
\curveto(528.32684381,806.57318826)(528.72111347,805.67171744)(529.24812718,804.76430285)
\lineto(528.29712499,804.76430285)
\curveto(527.70274863,805.69549249)(527.25696635,806.61083209)(526.95977817,807.51032166)
\curveto(526.6665525,808.40981123)(526.51993966,809.30335703)(526.51993966,810.19095907)
\curveto(526.51993966,811.0745986)(526.6665525,811.96418189)(526.95977817,812.85970895)
\curveto(527.25300384,813.75523601)(527.69878612,814.67057561)(528.29712499,815.60572776)
\closepath
}
}
{
\newrgbcolor{curcolor}{0 0 0}
\pscustom[linestyle=none,fillstyle=solid,fillcolor=curcolor]
{
\newpath
\moveto(531.239288,815.24315818)
\lineto(534.3300451,805.23980394)
\lineto(533.31960528,805.23980394)
\lineto(530.22884818,815.24315818)
\closepath
}
}
{
\newrgbcolor{curcolor}{0 0 0}
\pscustom[linestyle=none,fillstyle=solid,fillcolor=curcolor]
{
\newpath
\moveto(539.73292627,812.82999013)
\lineto(539.73292627,811.79577526)
\curveto(539.42385056,811.95427562)(539.10288732,812.07315089)(538.77003655,812.15240108)
\curveto(538.43718579,812.23165126)(538.0924475,812.27127635)(537.73582168,812.27127635)
\curveto(537.19295793,812.27127635)(536.78481949,812.18806366)(536.51140637,812.02163828)
\curveto(536.24195575,811.85521289)(536.10723044,811.60557482)(536.10723044,811.27272406)
\curveto(536.10723044,811.01912347)(536.20431191,810.81901676)(536.39847486,810.67240393)
\curveto(536.5926378,810.5297536)(536.98294495,810.39304703)(537.5693963,810.26228423)
\lineto(537.94385341,810.17907154)
\curveto(538.72050519,810.01264616)(539.27129396,809.77687687)(539.5962197,809.47176367)
\curveto(539.92510796,809.17061298)(540.08955209,808.74860576)(540.08955209,808.20574201)
\curveto(540.08955209,807.58759059)(539.84387652,807.09822071)(539.35252539,806.73763239)
\curveto(538.86513677,806.37704406)(538.19349148,806.19674989)(537.33758951,806.19674989)
\curveto(536.98096369,806.19674989)(536.60848784,806.23241248)(536.22016195,806.30373764)
\curveto(535.83579856,806.37110029)(535.42964138,806.47412553)(535.0016904,806.61281335)
\lineto(535.0016904,807.74212844)
\curveto(535.40586633,807.53211546)(535.80409849,807.3736151)(536.19638689,807.26662735)
\curveto(536.58867529,807.16360211)(536.97700119,807.1120895)(537.36136457,807.1120895)
\curveto(537.87649075,807.1120895)(538.27274166,807.1992647)(538.5501173,807.3736151)
\curveto(538.82749294,807.55192801)(538.96618076,807.80156608)(538.96618076,808.12252932)
\curveto(538.96618076,808.4197175)(538.86513677,808.64756177)(538.66304881,808.80606214)
\curveto(538.46492335,808.9645625)(538.0270661,809.1171191)(537.34947704,809.26373194)
\lineto(536.96907617,809.35288839)
\curveto(536.29148711,809.49553872)(535.80211724,809.71347672)(535.50096654,810.0067024)
\curveto(535.19981585,810.30389058)(535.04924051,810.71004776)(535.04924051,811.22517395)
\curveto(535.04924051,811.85125038)(535.27114102,812.3346765)(535.71494204,812.67545228)
\curveto(536.15874306,813.01622806)(536.788782,813.18661595)(537.60505888,813.18661595)
\curveto(538.00923481,813.18661595)(538.38963568,813.15689713)(538.7462615,813.0974595)
\curveto(539.10288732,813.03802186)(539.43177558,812.94886541)(539.73292627,812.82999013)
\closepath
}
}
{
\newrgbcolor{curcolor}{0 0 0}
\pscustom[linestyle=none,fillstyle=solid,fillcolor=curcolor]
{
\newpath
\moveto(547.53114418,809.97103981)
\lineto(547.53114418,809.43610109)
\lineto(542.50272013,809.43610109)
\curveto(542.55027024,808.68322436)(542.77613326,808.10866054)(543.18030919,807.71240963)
\curveto(543.58844763,807.32012122)(544.15508643,807.12397702)(544.88022559,807.12397702)
\curveto(545.30025156,807.12397702)(545.70640874,807.17548964)(546.09869714,807.27851488)
\curveto(546.49494805,807.38154012)(546.88723646,807.53607797)(547.27556235,807.74212844)
\lineto(547.27556235,806.70791357)
\curveto(546.88327395,806.54148819)(546.48107927,806.41468789)(546.06897833,806.32751269)
\curveto(545.65687738,806.24033749)(545.23883267,806.19674989)(544.81484419,806.19674989)
\curveto(543.75289175,806.19674989)(542.91085857,806.5058256)(542.28874464,807.12397702)
\curveto(541.67059322,807.74212844)(541.36151751,808.57821786)(541.36151751,809.63224529)
\curveto(541.36151751,810.72193529)(541.65474318,811.58576228)(542.24119453,812.22372624)
\curveto(542.83160839,812.86565272)(543.62609146,813.18661595)(544.62464376,813.18661595)
\curveto(545.52017081,813.18661595)(546.22747869,812.89735279)(546.74656738,812.31882646)
\curveto(547.26961858,811.74426264)(547.53114418,810.96166709)(547.53114418,809.97103981)
\closepath
\moveto(546.43749167,810.29200305)
\curveto(546.42956665,810.89034193)(546.26116002,811.36782427)(545.93227176,811.72445009)
\curveto(545.60734601,812.08107591)(545.17543252,812.25938882)(544.63653128,812.25938882)
\curveto(544.02630488,812.25938882)(543.53693501,812.08701968)(543.16842166,811.74228138)
\curveto(542.80387082,811.39754309)(542.59385784,810.91213573)(542.53838271,810.28605929)
\closepath
}
}
{
\newrgbcolor{curcolor}{0 0 0}
\pscustom[linestyle=none,fillstyle=solid,fillcolor=curcolor]
{
\newpath
\moveto(550.40792579,814.91625118)
\lineto(550.40792579,813.02613433)
\lineto(552.66061222,813.02613433)
\lineto(552.66061222,812.17617613)
\lineto(550.40792579,812.17617613)
\lineto(550.40792579,808.56236783)
\curveto(550.40792579,808.01950408)(550.48123221,807.67080328)(550.62784505,807.51626542)
\curveto(550.7784204,807.36172757)(551.08155234,807.28445864)(551.53724089,807.28445864)
\lineto(552.66061222,807.28445864)
\lineto(552.66061222,806.36911904)
\lineto(551.53724089,806.36911904)
\curveto(550.69322645,806.36911904)(550.11073761,806.52563815)(549.78977437,806.83867637)
\curveto(549.46881114,807.1556771)(549.30832952,807.73024092)(549.30832952,808.56236783)
\lineto(549.30832952,812.17617613)
\lineto(548.50592142,812.17617613)
\lineto(548.50592142,813.02613433)
\lineto(549.30832952,813.02613433)
\lineto(549.30832952,814.91625118)
\closepath
}
}
{
\newrgbcolor{curcolor}{0 0 0}
\pscustom[linestyle=none,fillstyle=solid,fillcolor=curcolor]
{
\newpath
\moveto(554.75876079,809.69168292)
\curveto(554.75876079,808.88729357)(554.92320492,808.25527337)(555.25209317,807.79562232)
\curveto(555.58494394,807.33993377)(556.04063249,807.1120895)(556.61915882,807.1120895)
\curveto(557.19768514,807.1120895)(557.65337369,807.33993377)(557.98622446,807.79562232)
\curveto(558.31907522,808.25527337)(558.4855006,808.88729357)(558.4855006,809.69168292)
\curveto(558.4855006,810.49607227)(558.31907522,811.12611122)(557.98622446,811.58179977)
\curveto(557.65337369,812.04145082)(557.19768514,812.27127635)(556.61915882,812.27127635)
\curveto(556.04063249,812.27127635)(555.58494394,812.04145082)(555.25209317,811.58179977)
\curveto(554.92320492,811.12611122)(554.75876079,810.49607227)(554.75876079,809.69168292)
\closepath
\moveto(558.4855006,807.36767133)
\curveto(558.25567508,806.97142042)(557.96443066,806.67621349)(557.61176735,806.48205055)
\curveto(557.26306654,806.29185011)(556.84304058,806.19674989)(556.35168945,806.19674989)
\curveto(555.5473001,806.19674989)(554.89150485,806.51771313)(554.38430368,807.15963961)
\curveto(553.88106502,807.80156608)(553.6294457,808.64558052)(553.6294457,809.69168292)
\curveto(553.6294457,810.73778533)(553.88106502,811.58179977)(554.38430368,812.22372624)
\curveto(554.89150485,812.86565272)(555.5473001,813.18661595)(556.35168945,813.18661595)
\curveto(556.84304058,813.18661595)(557.26306654,813.08953448)(557.61176735,812.89537153)
\curveto(557.96443066,812.7051711)(558.25567508,812.41194542)(558.4855006,812.01569451)
\lineto(558.4855006,813.02613433)
\lineto(559.57915312,813.02613433)
\lineto(559.57915312,803.83707572)
\lineto(558.4855006,803.83707572)
\closepath
}
}
{
\newrgbcolor{curcolor}{0 0 0}
\pscustom[linestyle=none,fillstyle=solid,fillcolor=curcolor]
{
\newpath
\moveto(561.71890803,808.99626258)
\lineto(561.71890803,813.02613433)
\lineto(562.81256055,813.02613433)
\lineto(562.81256055,809.03786892)
\curveto(562.81256055,808.40782997)(562.93539833,807.93431014)(563.18107389,807.61730941)
\curveto(563.42674946,807.30427119)(563.7952628,807.14775208)(564.28661393,807.14775208)
\curveto(564.87702779,807.14775208)(565.34262261,807.33597126)(565.68339839,807.71240963)
\curveto(566.02813668,808.08884799)(566.20050583,808.60199292)(566.20050583,809.25184441)
\lineto(566.20050583,813.02613433)
\lineto(567.29415834,813.02613433)
\lineto(567.29415834,806.36911904)
\lineto(566.20050583,806.36911904)
\lineto(566.20050583,807.39144639)
\curveto(565.93501772,806.98727046)(565.62594201,806.68611977)(565.2732787,806.48799431)
\curveto(564.9245779,806.29383137)(564.51842071,806.19674989)(564.05480715,806.19674989)
\curveto(563.29004289,806.19674989)(562.70953531,806.43450044)(562.3132844,806.91000153)
\curveto(561.91703349,807.38550262)(561.71890803,808.08092297)(561.71890803,808.99626258)
\closepath
\moveto(564.47087061,813.18661595)
\closepath
}
}
{
\newrgbcolor{curcolor}{0 0 0}
\pscustom[linestyle=none,fillstyle=solid,fillcolor=curcolor]
{
\newpath
\moveto(572.584108,809.71545798)
\curveto(571.70046847,809.71545798)(571.08826081,809.614414)(570.74748503,809.41232603)
\curveto(570.40670924,809.21023807)(570.23632135,808.86549977)(570.23632135,808.37811116)
\curveto(570.23632135,807.98978526)(570.36312164,807.68070955)(570.61672223,807.45088402)
\curveto(570.87428532,807.22502101)(571.22298612,807.1120895)(571.66282463,807.1120895)
\curveto(572.26908852,807.1120895)(572.75449589,807.32606499)(573.11904673,807.75401597)
\curveto(573.48756007,808.18592946)(573.67181675,808.75851203)(573.67181675,809.47176367)
\lineto(573.67181675,809.71545798)
\closepath
\moveto(574.76546926,810.16718402)
\lineto(574.76546926,806.36911904)
\lineto(573.67181675,806.36911904)
\lineto(573.67181675,807.37955886)
\curveto(573.42217867,806.97538293)(573.11112171,806.67621349)(572.73864585,806.48205055)
\curveto(572.36617,806.29185011)(571.91048145,806.19674989)(571.37158021,806.19674989)
\curveto(570.69002864,806.19674989)(570.1471649,806.38695033)(569.74298897,806.7673512)
\curveto(569.34277555,807.15171459)(569.14266884,807.66485952)(569.14266884,808.30678599)
\curveto(569.14266884,809.05570021)(569.39230691,809.62035776)(569.89158306,810.00075863)
\curveto(570.39482172,810.38115951)(571.14373594,810.57135994)(572.13832572,810.57135994)
\lineto(573.67181675,810.57135994)
\lineto(573.67181675,810.67834769)
\curveto(573.67181675,811.18158635)(573.50539136,811.56991224)(573.1725406,811.84332537)
\curveto(572.84365234,812.120701)(572.38003878,812.25938882)(571.7816999,812.25938882)
\curveto(571.40129903,812.25938882)(571.03080443,812.21381997)(570.6702161,812.12268226)
\curveto(570.30962777,812.03154455)(569.96290822,811.89483799)(569.63005746,811.71256257)
\lineto(569.63005746,812.72300239)
\curveto(570.03027088,812.87754024)(570.41859677,812.99245301)(570.79503514,813.06774068)
\curveto(571.1714735,813.14699086)(571.53800559,813.18661595)(571.89463141,813.18661595)
\curveto(572.85752112,813.18661595)(573.57671653,812.93697788)(574.05221762,812.43770173)
\curveto(574.52771871,811.93842559)(574.76546926,811.18158635)(574.76546926,810.16718402)
\closepath
}
}
{
\newrgbcolor{curcolor}{0 0 0}
\pscustom[linestyle=none,fillstyle=solid,fillcolor=curcolor]
{
\newpath
\moveto(582.55774341,810.38710327)
\lineto(582.55774341,806.36911904)
\lineto(581.4640909,806.36911904)
\lineto(581.4640909,810.35144069)
\curveto(581.4640909,810.98147964)(581.34125312,811.45301822)(581.09557755,811.76605644)
\curveto(580.84990199,812.07909466)(580.48138864,812.23561377)(579.99003751,812.23561377)
\curveto(579.39962366,812.23561377)(578.93402884,812.04739459)(578.59325305,811.67095622)
\curveto(578.25247727,811.29451786)(578.08208938,810.78137293)(578.08208938,810.13152143)
\lineto(578.08208938,806.36911904)
\lineto(576.9824931,806.36911904)
\lineto(576.9824931,813.02613433)
\lineto(578.08208938,813.02613433)
\lineto(578.08208938,811.99191946)
\curveto(578.34361498,812.39213288)(578.65070944,812.69130232)(579.00337275,812.88942777)
\curveto(579.35999857,813.08755323)(579.77011826,813.18661595)(580.23373182,813.18661595)
\curveto(580.99849608,813.18661595)(581.57702241,812.94886541)(581.96931081,812.47336431)
\curveto(582.36159921,812.00182573)(582.55774341,811.30640538)(582.55774341,810.38710327)
\closepath
}
}
{
\newrgbcolor{curcolor}{0 0 0}
\pscustom[linestyle=none,fillstyle=solid,fillcolor=curcolor]
{
\newpath
\moveto(585.83275719,814.91625118)
\lineto(585.83275719,813.02613433)
\lineto(588.08544361,813.02613433)
\lineto(588.08544361,812.17617613)
\lineto(585.83275719,812.17617613)
\lineto(585.83275719,808.56236783)
\curveto(585.83275719,808.01950408)(585.90606361,807.67080328)(586.05267644,807.51626542)
\curveto(586.20325179,807.36172757)(586.50638373,807.28445864)(586.96207228,807.28445864)
\lineto(588.08544361,807.28445864)
\lineto(588.08544361,806.36911904)
\lineto(586.96207228,806.36911904)
\curveto(586.11805784,806.36911904)(585.535569,806.52563815)(585.21460577,806.83867637)
\curveto(584.89364253,807.1556771)(584.73316091,807.73024092)(584.73316091,808.56236783)
\lineto(584.73316091,812.17617613)
\lineto(583.93075282,812.17617613)
\lineto(583.93075282,813.02613433)
\lineto(584.73316091,813.02613433)
\lineto(584.73316091,814.91625118)
\closepath
}
}
{
\newrgbcolor{curcolor}{0 0 0}
\pscustom[linestyle=none,fillstyle=solid,fillcolor=curcolor]
{
\newpath
\moveto(589.35740904,815.60572776)
\lineto(590.30841122,815.60572776)
\curveto(590.90278759,814.67057561)(591.34658861,813.75523601)(591.63981428,812.85970895)
\curveto(591.93700246,811.96418189)(592.08559655,811.0745986)(592.08559655,810.19095907)
\curveto(592.08559655,809.30335703)(591.93700246,808.40981123)(591.63981428,807.51032166)
\curveto(591.34658861,806.61083209)(590.90278759,805.69549249)(590.30841122,804.76430285)
\lineto(589.35740904,804.76430285)
\curveto(589.88442275,805.67171744)(590.27671115,806.57318826)(590.53427424,807.46871532)
\curveto(590.79579984,808.36820488)(590.92656264,809.27561947)(590.92656264,810.19095907)
\curveto(590.92656264,811.10629867)(590.79579984,812.00975075)(590.53427424,812.9013153)
\curveto(590.27671115,813.79287985)(589.88442275,814.69435067)(589.35740904,815.60572776)
\closepath
}
}
{
\newrgbcolor{curcolor}{0 0 0}
\pscustom[linestyle=none,fillstyle=solid,fillcolor=curcolor]
{
\newpath
\moveto(597.25072717,804.58004618)
\lineto(596.65635081,804.58004618)
\lineto(596.65040704,806.36911904)
\curveto(596.23434359,806.37704406)(595.81828013,806.42459417)(595.40221667,806.51176937)
\curveto(594.98615322,806.60290708)(594.56810851,806.73763239)(594.14808254,806.9159453)
\lineto(594.14808254,807.98582275)
\curveto(594.55225847,807.73222217)(594.96039691,807.54004048)(595.37249786,807.40927768)
\curveto(595.78856131,807.28247739)(596.21651229,807.21709599)(596.65635081,807.21313348)
\lineto(596.65635081,809.92348971)
\curveto(595.78063629,810.06614003)(595.14267233,810.30785309)(594.74245891,810.64862887)
\curveto(594.346208,810.98940465)(594.14808254,811.45698073)(594.14808254,812.05135709)
\curveto(594.14808254,812.69724608)(594.36403929,813.2064285)(594.79595278,813.57890435)
\curveto(595.22786627,813.95138021)(595.84799895,814.1653557)(596.65635081,814.22083083)
\lineto(596.65635081,815.61761529)
\lineto(597.25072717,815.61761529)
\lineto(597.25072717,814.23866212)
\curveto(597.61924052,814.22281208)(597.97586634,814.18318699)(598.32060463,814.11978685)
\curveto(598.66534292,814.06034921)(599.0021562,813.97713652)(599.33104445,813.87014877)
\lineto(599.33104445,812.82999013)
\curveto(599.0021562,812.99641552)(598.66336167,813.12519706)(598.31466087,813.21633477)
\curveto(597.96992257,813.30747248)(597.61527801,813.36096635)(597.25072717,813.37681639)
\lineto(597.25072717,810.83882931)
\curveto(598.15021674,810.70014149)(598.81195576,810.45248467)(599.23594423,810.09585885)
\curveto(599.65993271,809.73923303)(599.87192694,809.25184441)(599.87192694,808.63369299)
\curveto(599.87192694,807.96402895)(599.64606392,807.43503399)(599.19433789,807.0467081)
\curveto(598.74657436,806.66234471)(598.09870412,806.4404442)(597.25072717,806.38100657)
\closepath
\moveto(596.65635081,810.94581705)
\lineto(596.65635081,813.38276015)
\curveto(596.19669975,813.33124754)(595.84601769,813.20048474)(595.60430464,812.99047175)
\curveto(595.36259158,812.78045877)(595.24173506,812.50110188)(595.24173506,812.15240108)
\curveto(595.24173506,811.81162529)(595.35268531,811.54613718)(595.57458582,811.35593675)
\curveto(595.80044884,811.16573631)(596.16103717,811.02902975)(596.65635081,810.94581705)
\closepath
\moveto(597.25072717,809.80461443)
\lineto(597.25072717,807.23096477)
\curveto(597.75396583,807.29832742)(598.13238545,807.44097775)(598.38598603,807.65891575)
\curveto(598.64354912,807.87685375)(598.77233067,808.16413566)(598.77233067,808.52076148)
\curveto(598.77233067,808.86946228)(598.64949288,809.14683792)(598.40381732,809.35288839)
\curveto(598.16210427,809.55893887)(597.77774088,809.70951421)(597.25072717,809.80461443)
\closepath
}
}
{
\newrgbcolor{curcolor}{0 0 0}
\pscustom[linestyle=none,fillstyle=solid,fillcolor=curcolor]
{
\newpath
\moveto(605.3580208,810.19095907)
\lineto(608.56170941,810.19095907)
\lineto(608.56170941,809.21618183)
\lineto(605.3580208,809.21618183)
\closepath
}
}
{
\newrgbcolor{curcolor}{0 0 0}
\pscustom[linestyle=none,fillstyle=solid,fillcolor=curcolor]
{
\newpath
\moveto(619.55172841,814.95191376)
\lineto(619.55172841,813.78099232)
\curveto(619.09603986,813.99893032)(618.66610763,814.16139319)(618.2619317,814.26838094)
\curveto(617.85775577,814.37536868)(617.46744862,814.42886256)(617.09101026,814.42886256)
\curveto(616.43719625,814.42886256)(615.93197634,814.30206227)(615.57535052,814.04846168)
\curveto(615.22268721,813.7948611)(615.04635556,813.43427277)(615.04635556,812.9666967)
\curveto(615.04635556,812.5744083)(615.16324958,812.27722011)(615.39703761,812.07513215)
\curveto(615.63478816,811.87700669)(616.08255169,811.71652508)(616.7403282,811.59368729)
\lineto(617.46546737,811.4450932)
\curveto(618.36099442,811.27470531)(619.02075219,810.97355462)(619.44474066,810.54164113)
\curveto(619.87269165,810.11369014)(620.08666714,809.53912632)(620.08666714,808.81794967)
\curveto(620.08666714,807.95808519)(619.79740397,807.30625244)(619.21887765,806.86245142)
\curveto(618.64431383,806.4186504)(617.80029939,806.19674989)(616.68683433,806.19674989)
\curveto(616.26680836,806.19674989)(615.81904483,806.2443)(615.34354374,806.33940022)
\curveto(614.87200516,806.43450044)(614.38263528,806.57516951)(613.87543412,806.76140744)
\lineto(613.87543412,807.99771028)
\curveto(614.36282274,807.72429715)(614.84030509,807.51824668)(615.30788116,807.37955886)
\curveto(615.77545723,807.24087104)(616.23510829,807.17152713)(616.68683433,807.17152713)
\curveto(617.3723484,807.17152713)(617.90134337,807.30625244)(618.27381922,807.57570306)
\curveto(618.64629508,807.84515368)(618.83253301,808.22951706)(618.83253301,808.72879321)
\curveto(618.83253301,809.16466921)(618.6978077,809.505445)(618.42835708,809.75112056)
\curveto(618.16286897,809.99679612)(617.72501171,810.1810528)(617.11478531,810.30389058)
\lineto(616.38370238,810.44654091)
\curveto(615.48817532,810.62485382)(614.84030509,810.90421071)(614.44009167,811.28461158)
\curveto(614.03987825,811.66501246)(613.83977154,812.19400742)(613.83977154,812.87159648)
\curveto(613.83977154,813.65617328)(614.11516592,814.2743247)(614.66595468,814.72605074)
\curveto(615.22070596,815.17777678)(615.98348896,815.4036398)(616.95430369,815.4036398)
\curveto(617.37036715,815.4036398)(617.79435562,815.36599596)(618.22626911,815.29070829)
\curveto(618.65818261,815.21542061)(619.10000237,815.10248911)(619.55172841,814.95191376)
\closepath
}
}
{
\newrgbcolor{curcolor}{0 0 0}
\pscustom[linestyle=none,fillstyle=solid,fillcolor=curcolor]
{
\newpath
\moveto(623.15959295,814.25649341)
\lineto(623.15959295,810.922042)
\lineto(624.66930892,810.922042)
\curveto(625.2280227,810.922042)(625.65993619,811.06667358)(625.9650494,811.35593675)
\curveto(626.2701626,811.64519991)(626.4227192,812.05730086)(626.4227192,812.59223959)
\curveto(626.4227192,813.12321581)(626.2701626,813.5333355)(625.9650494,813.82259866)
\curveto(625.65993619,814.11186183)(625.2280227,814.25649341)(624.66930892,814.25649341)
\closepath
\moveto(621.95895269,815.24315818)
\lineto(624.66930892,815.24315818)
\curveto(625.6638987,815.24315818)(626.41479418,815.01729516)(626.92199534,814.56556912)
\curveto(627.43315902,814.11780559)(627.68874086,813.46002908)(627.68874086,812.59223959)
\curveto(627.68874086,811.71652508)(627.43315902,811.05478606)(626.92199534,810.60702253)
\curveto(626.41479418,810.159259)(625.6638987,809.93537723)(624.66930892,809.93537723)
\lineto(623.15959295,809.93537723)
\lineto(623.15959295,806.36911904)
\lineto(621.95895269,806.36911904)
\closepath
}
}
{
\newrgbcolor{curcolor}{0 0 0}
\pscustom[linestyle=none,fillstyle=solid,fillcolor=curcolor]
{
\newpath
\moveto(498.38029581,734.50788107)
\lineto(499.93758752,734.50788107)
\lineto(499.93758752,721.33860992)
\lineto(498.38029581,721.33860992)
\closepath
}
}
{
\newrgbcolor{curcolor}{0 0 0}
\pscustom[linestyle=none,fillstyle=solid,fillcolor=curcolor]
{
\newpath
\moveto(503.18758774,730.81777681)
\lineto(504.74487944,730.81777681)
\lineto(504.74487944,721.33860992)
\lineto(503.18758774,721.33860992)
\closepath
\moveto(503.18758774,734.50788107)
\lineto(504.74487944,734.50788107)
\lineto(504.74487944,732.53587581)
\lineto(503.18758774,732.53587581)
\closepath
}
}
{
\newrgbcolor{curcolor}{0 0 0}
\pscustom[linestyle=none,fillstyle=solid,fillcolor=curcolor]
{
\newpath
\moveto(514.03784856,730.53847993)
\lineto(514.03784856,729.06582364)
\curveto(513.59774438,729.29151809)(513.14071312,729.46078893)(512.66675477,729.57363616)
\curveto(512.19279643,729.68648338)(511.701911,729.74290699)(511.19409849,729.74290699)
\curveto(510.421095,729.74290699)(509.83993179,729.62441741)(509.45060886,729.38743823)
\curveto(509.0669283,729.15045906)(508.87508802,728.7949903)(508.87508802,728.32103196)
\curveto(508.87508802,727.95992084)(509.01332587,727.6749816)(509.28980157,727.46621423)
\curveto(509.56627727,727.26308923)(510.12204985,727.06842776)(510.95711932,726.88222984)
\lineto(511.49032245,726.76374026)
\curveto(512.59622526,726.52676108)(513.38051347,726.19104059)(513.84318709,725.75657877)
\curveto(514.31150308,725.32775932)(514.54566107,724.72684784)(514.54566107,723.95384435)
\curveto(514.54566107,723.073636)(514.19583467,722.37680439)(513.49618188,721.86334951)
\curveto(512.80217144,721.34989464)(511.84579121,721.0931672)(510.62704118,721.0931672)
\curveto(510.11922867,721.0931672)(509.58884671,721.14394845)(509.03589531,721.24551096)
\curveto(508.48858627,721.3414311)(507.91024424,721.48813249)(507.30086923,721.68561513)
\lineto(507.30086923,723.29368809)
\curveto(507.87639008,722.99464294)(508.44344738,722.76894849)(509.00204114,722.61660474)
\curveto(509.56063491,722.46990335)(510.11358631,722.39655265)(510.66089535,722.39655265)
\curveto(511.39440231,722.39655265)(511.95863844,722.5206846)(512.35360372,722.76894849)
\curveto(512.74856901,723.02285475)(512.94605166,723.37832351)(512.94605166,723.83535477)
\curveto(512.94605166,724.25853186)(512.80217144,724.58296763)(512.51441102,724.80866208)
\curveto(512.23229296,725.03435653)(511.60881204,725.25158744)(510.64396827,725.46035481)
\lineto(510.10230159,725.58730794)
\curveto(509.13745781,725.79043294)(508.4406262,726.10076281)(508.01180675,726.51829754)
\curveto(507.58298729,726.94147463)(507.36857756,727.51981666)(507.36857756,728.25332362)
\curveto(507.36857756,729.1448167)(507.68454979,729.83318477)(508.31649425,730.31842784)
\curveto(508.94843871,730.80367091)(509.84557415,731.04629244)(511.00790057,731.04629244)
\curveto(511.58342141,731.04629244)(512.12508809,731.00397473)(512.63290061,730.91933931)
\curveto(513.14071312,730.83470389)(513.6090291,730.70775077)(514.03784856,730.53847993)
\closepath
}
}
{
\newrgbcolor{curcolor}{0 0 0}
\pscustom[linestyle=none,fillstyle=solid,fillcolor=curcolor]
{
\newpath
\moveto(518.57430667,733.50918313)
\lineto(518.57430667,730.81777681)
\lineto(521.78198904,730.81777681)
\lineto(521.78198904,729.60749032)
\lineto(518.57430667,729.60749032)
\lineto(518.57430667,724.46165687)
\curveto(518.57430667,723.68865338)(518.67869035,723.19212559)(518.88745772,722.9720735)
\curveto(519.10186744,722.75202141)(519.53350808,722.64199536)(520.18237962,722.64199536)
\lineto(521.78198904,722.64199536)
\lineto(521.78198904,721.33860992)
\lineto(520.18237962,721.33860992)
\curveto(518.98055668,721.33860992)(518.15112957,721.56148319)(517.69409831,722.00722972)
\curveto(517.23706705,722.45861862)(517.00855142,723.276761)(517.00855142,724.46165687)
\lineto(517.00855142,729.60749032)
\lineto(515.86597327,729.60749032)
\lineto(515.86597327,730.81777681)
\lineto(517.00855142,730.81777681)
\lineto(517.00855142,733.50918313)
\closepath
}
}
{
\newrgbcolor{curcolor}{0 0 0}
\pscustom[linestyle=none,fillstyle=solid,fillcolor=curcolor]
{
\newpath
\moveto(398.96786154,1038.58805767)
\lineto(398.31681986,1038.58805767)
\lineto(398.31030944,1040.54769314)
\curveto(397.85458026,1040.5563737)(397.39885109,1040.60845703)(396.94312191,1040.70394314)
\curveto(396.48739273,1040.80376954)(396.02949341,1040.95133898)(395.56942395,1041.14665149)
\lineto(395.56942395,1042.31852652)
\curveto(396.0121323,1042.04074874)(396.45918092,1041.83024526)(396.91056982,1041.68701609)
\curveto(397.366299,1041.54812719)(397.83504901,1041.47651261)(398.31681986,1041.47217233)
\lineto(398.31681986,1044.44092241)
\curveto(397.35761845,1044.59717241)(396.65883371,1044.86192937)(396.22046564,1045.23519326)
\curveto(395.78643785,1045.60845716)(395.56942395,1046.12060995)(395.56942395,1046.77165164)
\curveto(395.56942395,1047.47911693)(395.8059691,1048.03684264)(396.27905939,1048.44482877)
\curveto(396.75214968,1048.85281489)(397.43140317,1049.08718989)(398.31681986,1049.14795378)
\lineto(398.31681986,1050.67790174)
\lineto(398.96786154,1050.67790174)
\lineto(398.96786154,1049.16748503)
\curveto(399.37150739,1049.15012392)(399.7621324,1049.10672114)(400.13973657,1049.0372767)
\curveto(400.51734075,1048.97217253)(400.88626437,1048.88102669)(401.24650744,1048.76383919)
\lineto(401.24650744,1047.62451624)
\curveto(400.88626437,1047.80680791)(400.51517061,1047.94786695)(400.13322616,1048.04769334)
\curveto(399.75562198,1048.14751973)(399.36716711,1048.20611348)(398.96786154,1048.22347459)
\lineto(398.96786154,1045.4435266)
\curveto(399.95310463,1045.29161688)(400.67793103,1045.02034951)(401.14234077,1044.6297245)
\curveto(401.6067505,1044.23909949)(401.83895537,1043.70524531)(401.83895537,1043.02816196)
\curveto(401.83895537,1042.29465499)(401.59155953,1041.71522789)(401.09676785,1041.28988066)
\curveto(400.60631645,1040.8688737)(399.89668101,1040.62581814)(398.96786154,1040.56071397)
\closepath
\moveto(398.31681986,1045.56071411)
\lineto(398.31681986,1048.22998501)
\curveto(397.81334762,1048.1735614)(397.42923303,1048.03033223)(397.16447608,1047.8002975)
\curveto(396.89971913,1047.57026277)(396.76734065,1047.26427318)(396.76734065,1046.88232872)
\curveto(396.76734065,1046.50906483)(396.88886843,1046.21826621)(397.131924,1046.00993287)
\curveto(397.37931984,1045.80159953)(397.77428512,1045.65185994)(398.31681986,1045.56071411)
\closepath
\moveto(398.96786154,1044.31071407)
\lineto(398.96786154,1041.49170358)
\curveto(399.51907684,1041.56548831)(399.93357337,1041.72173831)(400.21135116,1041.96045359)
\curveto(400.49346922,1042.19916888)(400.63452825,1042.51383903)(400.63452825,1042.90446404)
\curveto(400.63452825,1043.28640849)(400.49997964,1043.59022794)(400.23088241,1043.81592239)
\curveto(399.96612546,1044.04161684)(399.5451185,1044.2065474)(398.96786154,1044.31071407)
\closepath
}
}
{
\newrgbcolor{curcolor}{0 0 0}
\pscustom[linestyle=none,fillstyle=solid,fillcolor=curcolor]
{
\newpath
\moveto(404.05900768,1050.26774548)
\lineto(407.44442444,1039.31071394)
\lineto(406.33765358,1039.31071394)
\lineto(402.95223682,1050.26774548)
\closepath
}
}
{
\newrgbcolor{curcolor}{0 0 0}
\pscustom[linestyle=none,fillstyle=solid,fillcolor=curcolor]
{
\newpath
\moveto(409.89885131,1049.90967255)
\lineto(409.89885131,1047.83936)
\lineto(412.3662993,1047.83936)
\lineto(412.3662993,1046.90837039)
\lineto(409.89885131,1046.90837039)
\lineto(409.89885131,1042.95003695)
\curveto(409.89885131,1042.35541888)(409.97914646,1041.97347443)(410.13973674,1041.80420359)
\curveto(410.3046673,1041.63493275)(410.63669856,1041.55029733)(411.13583051,1041.55029733)
\lineto(412.3662993,1041.55029733)
\lineto(412.3662993,1040.54769314)
\lineto(411.13583051,1040.54769314)
\curveto(410.21135132,1040.54769314)(409.57333047,1040.71913412)(409.22176796,1041.06201607)
\curveto(408.87020545,1041.4092383)(408.6944242,1042.0385786)(408.6944242,1042.95003695)
\lineto(408.6944242,1046.90837039)
\lineto(407.81551793,1046.90837039)
\lineto(407.81551793,1047.83936)
\lineto(408.6944242,1047.83936)
\lineto(408.6944242,1049.90967255)
\closepath
}
}
{
\newrgbcolor{curcolor}{0 0 0}
\pscustom[linestyle=none,fillstyle=solid,fillcolor=curcolor]
{
\newpath
\moveto(420.00952929,1044.94873492)
\lineto(420.00952929,1040.54769314)
\lineto(418.8116126,1040.54769314)
\lineto(418.8116126,1044.90967242)
\curveto(418.8116126,1045.59977661)(418.67706398,1046.11626968)(418.40796675,1046.45915163)
\curveto(418.13886952,1046.80203358)(417.73522368,1046.97347456)(417.19702922,1046.97347456)
\curveto(416.55032782,1046.97347456)(416.04034516,1046.76731136)(415.66708126,1046.35498496)
\curveto(415.29381737,1045.94265856)(415.10718542,1045.38059257)(415.10718542,1044.668787)
\lineto(415.10718542,1040.54769314)
\lineto(413.9027583,1040.54769314)
\lineto(413.9027583,1050.67790174)
\lineto(415.10718542,1050.67790174)
\lineto(415.10718542,1046.70654747)
\curveto(415.39364376,1047.14491554)(415.73001529,1047.47260652)(416.11630003,1047.68962041)
\curveto(416.50692504,1047.90663431)(416.9561438,1048.01514125)(417.46395631,1048.01514125)
\curveto(418.30162994,1048.01514125)(418.93531052,1047.75472458)(419.36499803,1047.23389123)
\curveto(419.79468554,1046.71739816)(420.00952929,1045.95567939)(420.00952929,1044.94873492)
\closepath
}
}
{
\newrgbcolor{curcolor}{0 0 0}
\pscustom[linestyle=none,fillstyle=solid,fillcolor=curcolor]
{
\newpath
\moveto(428.64885256,1044.49300574)
\lineto(428.64885256,1043.90706823)
\lineto(423.14103991,1043.90706823)
\curveto(423.19312325,1043.08241543)(423.44051909,1042.45307514)(423.88322743,1042.01904735)
\curveto(424.33027606,1041.58935983)(424.95093579,1041.37451608)(425.74520665,1041.37451608)
\curveto(426.2052761,1041.37451608)(426.65015459,1041.43093969)(427.0798421,1041.54378692)
\curveto(427.51386989,1041.65663414)(427.9435574,1041.82590498)(428.36890463,1042.05159943)
\lineto(428.36890463,1040.9187869)
\curveto(427.93921712,1040.73649523)(427.49867892,1040.59760634)(427.04729002,1040.50212022)
\curveto(426.59590111,1040.40663411)(426.1380018,1040.35889105)(425.67359206,1040.35889105)
\curveto(424.51039759,1040.35889105)(423.58808854,1040.69743273)(422.90666491,1041.37451608)
\curveto(422.22958156,1042.05159943)(421.89103988,1042.96739807)(421.89103988,1044.12191198)
\curveto(421.89103988,1045.3154884)(422.21222044,1046.26166899)(422.85458157,1046.96045373)
\curveto(423.50128298,1047.66357874)(424.3715087,1048.01514125)(425.46525872,1048.01514125)
\curveto(426.44616153,1048.01514125)(427.22090113,1047.69830097)(427.78947754,1047.0646204)
\curveto(428.36239422,1046.4352801)(428.64885256,1045.57807522)(428.64885256,1044.49300574)
\closepath
\moveto(427.45093586,1044.84456825)
\curveto(427.4422553,1045.49995022)(427.25779349,1046.0229537)(426.89755043,1046.41357871)
\curveto(426.54164764,1046.80420372)(426.06855735,1046.99951623)(425.47827956,1046.99951623)
\curveto(424.80987676,1046.99951623)(424.27385244,1046.81071414)(423.8702066,1046.43310996)
\curveto(423.47090103,1046.05550579)(423.2408663,1045.52382174)(423.18010241,1044.83805784)
\closepath
}
}
{
\newrgbcolor{curcolor}{0 0 0}
\pscustom[linestyle=none,fillstyle=solid,fillcolor=curcolor]
{
\newpath
\moveto(431.79989389,1049.90967255)
\lineto(431.79989389,1047.83936)
\lineto(434.26734187,1047.83936)
\lineto(434.26734187,1046.90837039)
\lineto(431.79989389,1046.90837039)
\lineto(431.79989389,1042.95003695)
\curveto(431.79989389,1042.35541888)(431.88018903,1041.97347443)(432.04077931,1041.80420359)
\curveto(432.20570987,1041.63493275)(432.53774113,1041.55029733)(433.03687309,1041.55029733)
\lineto(434.26734187,1041.55029733)
\lineto(434.26734187,1040.54769314)
\lineto(433.03687309,1040.54769314)
\curveto(432.1123939,1040.54769314)(431.47437305,1040.71913412)(431.12281054,1041.06201607)
\curveto(430.77124803,1041.4092383)(430.59546678,1042.0385786)(430.59546678,1042.95003695)
\lineto(430.59546678,1046.90837039)
\lineto(429.7165605,1046.90837039)
\lineto(429.7165605,1047.83936)
\lineto(430.59546678,1047.83936)
\lineto(430.59546678,1049.90967255)
\closepath
}
}
{
\newrgbcolor{curcolor}{0 0 0}
\pscustom[linestyle=none,fillstyle=solid,fillcolor=curcolor]
{
\newpath
\moveto(439.16317716,1044.21305782)
\curveto(438.19529519,1044.21305782)(437.52472225,1044.10238073)(437.15145836,1043.88102656)
\curveto(436.77819446,1043.65967239)(436.59156251,1043.28206821)(436.59156251,1042.74821403)
\curveto(436.59156251,1042.3228668)(436.7304514,1041.98432512)(437.00822918,1041.73258901)
\curveto(437.29034725,1041.48519317)(437.6722917,1041.36149525)(438.15406255,1041.36149525)
\curveto(438.81812507,1041.36149525)(439.34980911,1041.59587025)(439.74911467,1042.06462026)
\curveto(440.15276052,1042.53771055)(440.35458344,1043.16488071)(440.35458344,1043.94613073)
\lineto(440.35458344,1044.21305782)
\closepath
\moveto(441.55250014,1044.7078495)
\lineto(441.55250014,1040.54769314)
\lineto(440.35458344,1040.54769314)
\lineto(440.35458344,1041.654464)
\curveto(440.08114593,1041.21175566)(439.74043412,1040.88406468)(439.332448,1040.67139106)
\curveto(438.92446187,1040.46305772)(438.42532992,1040.35889105)(437.83505212,1040.35889105)
\curveto(437.08852433,1040.35889105)(436.49390625,1040.56722439)(436.05119791,1040.98389107)
\curveto(435.61282984,1041.40489802)(435.39364581,1041.96696401)(435.39364581,1042.67008903)
\curveto(435.39364581,1043.49040155)(435.66708332,1044.10889115)(436.21395833,1044.52555783)
\curveto(436.76517362,1044.94222451)(437.58548614,1045.15055785)(438.6748959,1045.15055785)
\lineto(440.35458344,1045.15055785)
\lineto(440.35458344,1045.26774535)
\curveto(440.35458344,1045.81896064)(440.17229177,1046.24430787)(439.80770843,1046.54378705)
\curveto(439.44746536,1046.8476065)(438.93965285,1046.99951623)(438.28427088,1046.99951623)
\curveto(437.86760421,1046.99951623)(437.46178822,1046.94960303)(437.06682294,1046.84977664)
\curveto(436.67185765,1046.74995025)(436.29208333,1046.60021066)(435.92749999,1046.40055788)
\lineto(435.92749999,1047.50732874)
\curveto(436.36586806,1047.67659958)(436.79121529,1047.80246764)(437.20354169,1047.88493292)
\curveto(437.61586809,1047.97173847)(438.01734379,1048.01514125)(438.4079688,1048.01514125)
\curveto(439.46265633,1048.01514125)(440.25041677,1047.74170375)(440.77125012,1047.19482873)
\curveto(441.29208346,1046.64795372)(441.55250014,1045.81896064)(441.55250014,1044.7078495)
\closepath
}
}
{
\newrgbcolor{curcolor}{0 0 0}
\pscustom[linestyle=none,fillstyle=solid,fillcolor=curcolor]
{
\newpath
\moveto(449.5668222,1038.33415142)
\lineto(449.5668222,1037.40316181)
\lineto(442.63973868,1037.40316181)
\lineto(442.63973868,1038.33415142)
\closepath
}
}
{
\newrgbcolor{curcolor}{0 0 0}
\pscustom[linestyle=none,fillstyle=solid,fillcolor=curcolor]
{
\newpath
\moveto(456.75432179,1044.94873492)
\lineto(456.75432179,1040.54769314)
\lineto(455.55640509,1040.54769314)
\lineto(455.55640509,1044.90967242)
\curveto(455.55640509,1045.59977661)(455.42185648,1046.11626968)(455.15275925,1046.45915163)
\curveto(454.88366202,1046.80203358)(454.48001618,1046.97347456)(453.94182172,1046.97347456)
\curveto(453.29512031,1046.97347456)(452.78513766,1046.76731136)(452.41187376,1046.35498496)
\curveto(452.03860986,1045.94265856)(451.85197791,1045.38059257)(451.85197791,1044.668787)
\lineto(451.85197791,1040.54769314)
\lineto(450.6475508,1040.54769314)
\lineto(450.6475508,1047.83936)
\lineto(451.85197791,1047.83936)
\lineto(451.85197791,1046.70654747)
\curveto(452.13843625,1047.14491554)(452.47480779,1047.47260652)(452.86109252,1047.68962041)
\curveto(453.25171753,1047.90663431)(453.70093629,1048.01514125)(454.20874881,1048.01514125)
\curveto(455.04642244,1048.01514125)(455.68010301,1047.75472458)(456.10979052,1047.23389123)
\curveto(456.53947804,1046.71739816)(456.75432179,1045.95567939)(456.75432179,1044.94873492)
\closepath
}
}
{
\newrgbcolor{curcolor}{0 0 0}
\pscustom[linestyle=none,fillstyle=solid,fillcolor=curcolor]
{
\newpath
\moveto(463.55770968,1046.6023808)
\lineto(471.90406407,1046.6023808)
\lineto(471.90406407,1045.50863077)
\lineto(463.55770968,1045.50863077)
\closepath
\moveto(463.55770968,1043.94613073)
\lineto(471.90406407,1043.94613073)
\lineto(471.90406407,1042.83935987)
\lineto(463.55770968,1042.83935987)
\closepath
}
}
{
\newrgbcolor{curcolor}{0 0 0}
\pscustom[linestyle=none,fillstyle=solid,fillcolor=curcolor]
{
\newpath
\moveto(478.66838071,1050.26774548)
\lineto(482.05379747,1039.31071394)
\lineto(480.9470266,1039.31071394)
\lineto(477.56160985,1050.26774548)
\closepath
}
}
{
\newrgbcolor{curcolor}{0 0 0}
\pscustom[linestyle=none,fillstyle=solid,fillcolor=curcolor]
{
\newpath
\moveto(488.55770898,1044.18701615)
\curveto(488.55770898,1045.06809257)(488.37541731,1045.75819675)(488.01083397,1046.25732871)
\curveto(487.6505909,1046.76080094)(487.15362908,1047.01253706)(486.51994851,1047.01253706)
\curveto(485.88626794,1047.01253706)(485.38713598,1046.76080094)(485.02255264,1046.25732871)
\curveto(484.66230958,1045.75819675)(484.48218804,1045.06809257)(484.48218804,1044.18701615)
\curveto(484.48218804,1043.30593974)(484.66230958,1042.61366542)(485.02255264,1042.11019318)
\curveto(485.38713598,1041.61106122)(485.88626794,1041.36149525)(486.51994851,1041.36149525)
\curveto(487.15362908,1041.36149525)(487.6505909,1041.61106122)(488.01083397,1042.11019318)
\curveto(488.37541731,1042.61366542)(488.55770898,1043.30593974)(488.55770898,1044.18701615)
\closepath
\moveto(484.48218804,1046.73258914)
\curveto(484.73392416,1047.16661693)(485.05076445,1047.48779749)(485.4327089,1047.69613083)
\curveto(485.81899363,1047.90880445)(486.27906309,1048.01514125)(486.81291727,1048.01514125)
\curveto(487.69833396,1048.01514125)(488.41664995,1047.66357874)(488.96786524,1046.96045373)
\curveto(489.52342081,1046.25732871)(489.8011986,1045.33284952)(489.8011986,1044.18701615)
\curveto(489.8011986,1043.04118279)(489.52342081,1042.1167036)(488.96786524,1041.41357858)
\curveto(488.41664995,1040.71045356)(487.69833396,1040.35889105)(486.81291727,1040.35889105)
\curveto(486.27906309,1040.35889105)(485.81899363,1040.46305772)(485.4327089,1040.67139106)
\curveto(485.05076445,1040.88406468)(484.73392416,1041.20741538)(484.48218804,1041.64144317)
\lineto(484.48218804,1040.54769314)
\lineto(483.27776093,1040.54769314)
\lineto(483.27776093,1050.67790174)
\lineto(484.48218804,1050.67790174)
\closepath
}
}
{
\newrgbcolor{curcolor}{0 0 0}
\pscustom[linestyle=none,fillstyle=solid,fillcolor=curcolor]
{
\newpath
\moveto(498.02385399,1044.49300574)
\lineto(498.02385399,1043.90706823)
\lineto(492.51604135,1043.90706823)
\curveto(492.56812468,1043.08241543)(492.81552052,1042.45307514)(493.25822887,1042.01904735)
\curveto(493.70527749,1041.58935983)(494.32593723,1041.37451608)(495.12020808,1041.37451608)
\curveto(495.58027754,1041.37451608)(496.02515602,1041.43093969)(496.45484353,1041.54378692)
\curveto(496.88887132,1041.65663414)(497.31855883,1041.82590498)(497.74390607,1042.05159943)
\lineto(497.74390607,1040.9187869)
\curveto(497.31421856,1040.73649523)(496.87368035,1040.59760634)(496.42229145,1040.50212022)
\curveto(495.97090255,1040.40663411)(495.51300323,1040.35889105)(495.0485935,1040.35889105)
\curveto(493.88539902,1040.35889105)(492.96308997,1040.69743273)(492.28166634,1041.37451608)
\curveto(491.60458299,1042.05159943)(491.26604131,1042.96739807)(491.26604131,1044.12191198)
\curveto(491.26604131,1045.3154884)(491.58722188,1046.26166899)(492.22958301,1046.96045373)
\curveto(492.87628441,1047.66357874)(493.74651013,1048.01514125)(494.84026016,1048.01514125)
\curveto(495.82116296,1048.01514125)(496.59590257,1047.69830097)(497.16447897,1047.0646204)
\curveto(497.73739565,1046.4352801)(498.02385399,1045.57807522)(498.02385399,1044.49300574)
\closepath
\moveto(496.82593729,1044.84456825)
\curveto(496.81725674,1045.49995022)(496.63279493,1046.0229537)(496.27255186,1046.41357871)
\curveto(495.91664907,1046.80420372)(495.44355878,1046.99951623)(494.85328099,1046.99951623)
\curveto(494.1848782,1046.99951623)(493.64885388,1046.81071414)(493.24520803,1046.43310996)
\curveto(492.84590247,1046.05550579)(492.61586774,1045.52382174)(492.55510385,1044.83805784)
\closepath
}
}
{
\newrgbcolor{curcolor}{0 0 0}
\pscustom[linestyle=none,fillstyle=solid,fillcolor=curcolor]
{
\newpath
\moveto(504.78817428,1044.27816199)
\curveto(504.78817428,1045.14621757)(504.60805275,1045.81896064)(504.24780969,1046.29639121)
\curveto(503.8919069,1046.77382178)(503.3906048,1047.01253706)(502.7439034,1047.01253706)
\curveto(502.10154227,1047.01253706)(501.60024017,1046.77382178)(501.23999711,1046.29639121)
\curveto(500.88409432,1045.81896064)(500.70614293,1045.14621757)(500.70614293,1044.27816199)
\curveto(500.70614293,1043.41444669)(500.88409432,1042.74387375)(501.23999711,1042.26644319)
\curveto(501.60024017,1041.78901262)(502.10154227,1041.55029733)(502.7439034,1041.55029733)
\curveto(503.3906048,1041.55029733)(503.8919069,1041.78901262)(504.24780969,1042.26644319)
\curveto(504.60805275,1042.74387375)(504.78817428,1043.41444669)(504.78817428,1044.27816199)
\closepath
\moveto(505.98609098,1041.45264108)
\curveto(505.98609098,1040.2113216)(505.71048334,1039.28901255)(505.15926804,1038.68571392)
\curveto(504.60805275,1038.07807502)(503.7638687,1037.77425557)(502.62671589,1037.77425557)
\curveto(502.20570894,1037.77425557)(501.80857351,1037.80680765)(501.43530961,1037.87191182)
\curveto(501.06204571,1037.93267571)(500.69963251,1038.02816182)(500.34807,1038.15837016)
\lineto(500.34807,1039.32373478)
\curveto(500.69963251,1039.13276255)(501.04685474,1038.99170352)(501.38973669,1038.90055768)
\curveto(501.73261865,1038.80941184)(502.08201102,1038.76383893)(502.43791381,1038.76383893)
\curveto(503.2235041,1038.76383893)(503.81161176,1038.97000213)(504.20223677,1039.38232853)
\curveto(504.59286178,1039.79031465)(504.78817428,1040.40880425)(504.78817428,1041.23779733)
\lineto(504.78817428,1041.83024526)
\curveto(504.54077844,1041.40055775)(504.22393816,1041.07937718)(503.83765343,1040.86670357)
\curveto(503.45136869,1040.65402995)(502.9891291,1040.54769314)(502.45093464,1040.54769314)
\curveto(501.55683739,1040.54769314)(500.83635126,1040.88840496)(500.28947625,1041.56982858)
\curveto(499.74260123,1042.25125221)(499.46916373,1043.15403001)(499.46916373,1044.27816199)
\curveto(499.46916373,1045.40663424)(499.74260123,1046.31158218)(500.28947625,1046.99300581)
\curveto(500.83635126,1047.67442944)(501.55683739,1048.01514125)(502.45093464,1048.01514125)
\curveto(502.9891291,1048.01514125)(503.45136869,1047.90880445)(503.83765343,1047.69613083)
\curveto(504.22393816,1047.48345721)(504.54077844,1047.16227665)(504.78817428,1046.73258914)
\lineto(504.78817428,1047.83936)
\lineto(505.98609098,1047.83936)
\closepath
}
}
{
\newrgbcolor{curcolor}{0 0 0}
\pscustom[linestyle=none,fillstyle=solid,fillcolor=curcolor]
{
\newpath
\moveto(508.45353789,1047.83936)
\lineto(509.65145459,1047.83936)
\lineto(509.65145459,1040.54769314)
\lineto(508.45353789,1040.54769314)
\closepath
\moveto(508.45353789,1050.67790174)
\lineto(509.65145459,1050.67790174)
\lineto(509.65145459,1049.16097462)
\lineto(508.45353789,1049.16097462)
\closepath
}
}
{
\newrgbcolor{curcolor}{0 0 0}
\pscustom[linestyle=none,fillstyle=solid,fillcolor=curcolor]
{
\newpath
\moveto(518.21265437,1044.94873492)
\lineto(518.21265437,1040.54769314)
\lineto(517.01473768,1040.54769314)
\lineto(517.01473768,1044.90967242)
\curveto(517.01473768,1045.59977661)(516.88018906,1046.11626968)(516.61109183,1046.45915163)
\curveto(516.3419946,1046.80203358)(515.93834876,1046.97347456)(515.4001543,1046.97347456)
\curveto(514.75345289,1046.97347456)(514.24347024,1046.76731136)(513.87020634,1046.35498496)
\curveto(513.49694244,1045.94265856)(513.3103105,1045.38059257)(513.3103105,1044.668787)
\lineto(513.3103105,1040.54769314)
\lineto(512.10588338,1040.54769314)
\lineto(512.10588338,1047.83936)
\lineto(513.3103105,1047.83936)
\lineto(513.3103105,1046.70654747)
\curveto(513.59676884,1047.14491554)(513.93314037,1047.47260652)(514.31942511,1047.68962041)
\curveto(514.71005012,1047.90663431)(515.15926888,1048.01514125)(515.66708139,1048.01514125)
\curveto(516.50475502,1048.01514125)(517.1384356,1047.75472458)(517.56812311,1047.23389123)
\curveto(517.99781062,1046.71739816)(518.21265437,1045.95567939)(518.21265437,1044.94873492)
\closepath
}
}
{
\newrgbcolor{curcolor}{0 0 0}
\pscustom[linestyle=none,fillstyle=solid,fillcolor=curcolor]
{
\newpath
\moveto(526.17489309,1039.31071394)
\lineto(526.17489309,1038.37321392)
\lineto(525.77124725,1038.37321392)
\curveto(524.69051805,1038.37321392)(523.96569165,1038.5338042)(523.59676803,1038.85498476)
\curveto(523.23218468,1039.17616533)(523.04989301,1039.81635632)(523.04989301,1040.77555773)
\lineto(523.04989301,1042.33154735)
\curveto(523.04989301,1042.98692932)(522.93270551,1043.44048836)(522.6983305,1043.69222447)
\curveto(522.4639555,1043.94396059)(522.03860826,1044.06982865)(521.4222888,1044.06982865)
\lineto(521.02515337,1044.06982865)
\lineto(521.02515337,1045.00081826)
\lineto(521.4222888,1045.00081826)
\curveto(522.04294854,1045.00081826)(522.46829577,1045.12451618)(522.6983305,1045.37191202)
\curveto(522.93270551,1045.62364814)(523.04989301,1046.0728669)(523.04989301,1046.7195683)
\lineto(523.04989301,1048.28206834)
\curveto(523.04989301,1049.24126976)(523.23218468,1049.87929061)(523.59676803,1050.19613089)
\curveto(523.96569165,1050.51731146)(524.69051805,1050.67790174)(525.77124725,1050.67790174)
\lineto(526.17489309,1050.67790174)
\lineto(526.17489309,1049.74691213)
\lineto(525.73218475,1049.74691213)
\curveto(525.12020557,1049.74691213)(524.7209,1049.65142602)(524.53426805,1049.46045379)
\curveto(524.3476361,1049.26948156)(524.25432013,1048.86800586)(524.25432013,1048.25602668)
\lineto(524.25432013,1046.6414433)
\curveto(524.25432013,1045.96001967)(524.15449373,1045.46522799)(523.95484095,1045.15706826)
\curveto(523.75952845,1044.84890853)(523.42315691,1044.64057519)(522.94572634,1044.53206825)
\curveto(523.42749719,1044.41488074)(523.76603886,1044.20220713)(523.96135137,1043.8940474)
\curveto(524.15666387,1043.58588766)(524.25432013,1043.09326612)(524.25432013,1042.41618277)
\lineto(524.25432013,1040.8015994)
\curveto(524.25432013,1040.18962021)(524.3476361,1039.78814451)(524.53426805,1039.59717228)
\curveto(524.7209,1039.40620005)(525.12020557,1039.31071394)(525.73218475,1039.31071394)
\closepath
}
}
{
\newrgbcolor{curcolor}{0 0 0}
\pscustom[linestyle=none,fillstyle=solid,fillcolor=curcolor]
{
\newpath
\moveto(534.35198337,1047.55941208)
\lineto(534.35198337,1046.43962038)
\curveto(534.0134417,1046.62625233)(533.67272988,1046.76514122)(533.32984793,1046.85628706)
\curveto(532.99130625,1046.95177317)(532.6484243,1046.99951623)(532.30120207,1046.99951623)
\curveto(531.52429233,1046.99951623)(530.9209937,1046.75212039)(530.49130619,1046.25732871)
\curveto(530.06161868,1045.76687731)(529.84677492,1045.07677312)(529.84677492,1044.18701615)
\curveto(529.84677492,1043.29725919)(530.06161868,1042.60498486)(530.49130619,1042.11019318)
\curveto(530.9209937,1041.61974178)(531.52429233,1041.37451608)(532.30120207,1041.37451608)
\curveto(532.6484243,1041.37451608)(532.99130625,1041.420089)(533.32984793,1041.51123483)
\curveto(533.67272988,1041.60672095)(534.0134417,1041.74777998)(534.35198337,1041.93441193)
\lineto(534.35198337,1040.82764106)
\curveto(534.01778198,1040.67139106)(533.67055974,1040.55420356)(533.31031668,1040.47607856)
\curveto(532.95441389,1040.39795355)(532.57463958,1040.35889105)(532.17099373,1040.35889105)
\curveto(531.07290343,1040.35889105)(530.20050757,1040.70394314)(529.55380616,1041.39404733)
\curveto(528.90710476,1042.08415151)(528.58375406,1043.01514112)(528.58375406,1044.18701615)
\curveto(528.58375406,1045.3762523)(528.9092749,1046.31158218)(529.56031658,1046.99300581)
\curveto(530.21569854,1047.67442944)(531.11196593,1048.01514125)(532.24911874,1048.01514125)
\curveto(532.61804236,1048.01514125)(532.97828542,1047.97607875)(533.32984793,1047.89795375)
\curveto(533.68141044,1047.82416903)(534.02212225,1047.7113218)(534.35198337,1047.55941208)
\closepath
}
}
{
\newrgbcolor{curcolor}{0 0 0}
\pscustom[linestyle=none,fillstyle=solid,fillcolor=curcolor]
{
\newpath
\moveto(539.76213698,1044.21305782)
\curveto(538.79425501,1044.21305782)(538.12368208,1044.10238073)(537.75041818,1043.88102656)
\curveto(537.37715428,1043.65967239)(537.19052233,1043.28206821)(537.19052233,1042.74821403)
\curveto(537.19052233,1042.3228668)(537.32941123,1041.98432512)(537.60718901,1041.73258901)
\curveto(537.88930707,1041.48519317)(538.27125153,1041.36149525)(538.75302237,1041.36149525)
\curveto(539.41708489,1041.36149525)(539.94876893,1041.59587025)(540.3480745,1042.06462026)
\curveto(540.75172034,1042.53771055)(540.95354327,1043.16488071)(540.95354327,1043.94613073)
\lineto(540.95354327,1044.21305782)
\closepath
\moveto(542.15145996,1044.7078495)
\lineto(542.15145996,1040.54769314)
\lineto(540.95354327,1040.54769314)
\lineto(540.95354327,1041.654464)
\curveto(540.68010576,1041.21175566)(540.33939394,1040.88406468)(539.93140782,1040.67139106)
\curveto(539.5234217,1040.46305772)(539.02428974,1040.35889105)(538.43401195,1040.35889105)
\curveto(537.68748415,1040.35889105)(537.09286608,1040.56722439)(536.65015774,1040.98389107)
\curveto(536.21178967,1041.40489802)(535.99260564,1041.96696401)(535.99260564,1042.67008903)
\curveto(535.99260564,1043.49040155)(536.26604314,1044.10889115)(536.81291816,1044.52555783)
\curveto(537.36413345,1044.94222451)(538.18444597,1045.15055785)(539.27385572,1045.15055785)
\lineto(540.95354327,1045.15055785)
\lineto(540.95354327,1045.26774535)
\curveto(540.95354327,1045.81896064)(540.77125159,1046.24430787)(540.40666825,1046.54378705)
\curveto(540.04642519,1046.8476065)(539.53861267,1046.99951623)(538.88323071,1046.99951623)
\curveto(538.46656403,1046.99951623)(538.06074805,1046.94960303)(537.66578276,1046.84977664)
\curveto(537.27081747,1046.74995025)(536.89104316,1046.60021066)(536.52645982,1046.40055788)
\lineto(536.52645982,1047.50732874)
\curveto(536.96482788,1047.67659958)(537.39017512,1047.80246764)(537.80250152,1047.88493292)
\curveto(538.21482792,1047.97173847)(538.61630362,1048.01514125)(539.00692863,1048.01514125)
\curveto(540.06161616,1048.01514125)(540.8493766,1047.74170375)(541.37020994,1047.19482873)
\curveto(541.89104329,1046.64795372)(542.15145996,1045.81896064)(542.15145996,1044.7078495)
\closepath
}
}
{
\newrgbcolor{curcolor}{0 0 0}
\pscustom[linestyle=none,fillstyle=solid,fillcolor=curcolor]
{
\newpath
\moveto(549.27385492,1047.62451624)
\lineto(549.27385492,1046.49170371)
\curveto(548.93531324,1046.66531483)(548.58375073,1046.79552317)(548.21916739,1046.88232872)
\curveto(547.85458405,1046.96913428)(547.47697987,1047.01253706)(547.08635486,1047.01253706)
\curveto(546.49173679,1047.01253706)(546.04468817,1046.92139122)(545.74520899,1046.73909955)
\curveto(545.4500701,1046.55680788)(545.30250065,1046.28337037)(545.30250065,1045.91878703)
\curveto(545.30250065,1045.64100925)(545.40883746,1045.42182521)(545.62151107,1045.26123493)
\curveto(545.83418469,1045.10498493)(546.26170206,1044.95524534)(546.90406319,1044.81201617)
\lineto(547.31421945,1044.72087033)
\curveto(548.16491392,1044.53857866)(548.76821254,1044.28033213)(549.12411533,1043.94613073)
\curveto(549.4843584,1043.61626961)(549.66447993,1043.15403001)(549.66447993,1042.55941194)
\curveto(549.66447993,1041.88232859)(549.3953827,1041.34630427)(548.85718824,1040.95133898)
\curveto(548.32333406,1040.5563737)(547.58765696,1040.35889105)(546.65015693,1040.35889105)
\curveto(546.25953192,1040.35889105)(545.8515458,1040.39795355)(545.42619857,1040.47607856)
\curveto(545.00519161,1040.54986328)(544.56031313,1040.6627105)(544.09156312,1040.81462023)
\lineto(544.09156312,1042.05159943)
\curveto(544.53427146,1041.8215647)(544.97046939,1041.64795359)(545.4001569,1041.53076608)
\curveto(545.82984441,1041.41791886)(546.25519164,1041.36149525)(546.6761986,1041.36149525)
\curveto(547.24043473,1041.36149525)(547.67446251,1041.45698136)(547.97828197,1041.64795359)
\curveto(548.28210142,1041.84326609)(548.43401115,1042.1167036)(548.43401115,1042.46826611)
\curveto(548.43401115,1042.79378695)(548.32333406,1043.04335293)(548.10197989,1043.21696404)
\curveto(547.88496599,1043.39057516)(547.40536529,1043.55767586)(546.66317777,1043.71826614)
\lineto(546.24651109,1043.81592239)
\curveto(545.50432357,1043.9721724)(544.96829925,1044.21088768)(544.63843813,1044.53206825)
\curveto(544.30857701,1044.85758909)(544.14364645,1045.30246757)(544.14364645,1045.8667037)
\curveto(544.14364645,1046.5524676)(544.38670201,1047.08198151)(544.87281314,1047.45524541)
\curveto(545.35892426,1047.8285093)(546.04902844,1048.01514125)(546.94312569,1048.01514125)
\curveto(547.38583404,1048.01514125)(547.80250071,1047.98258917)(548.19312572,1047.917485)
\curveto(548.58375073,1047.85238083)(548.9439938,1047.75472458)(549.27385492,1047.62451624)
\closepath
}
}
{
\newrgbcolor{curcolor}{0 0 0}
\pscustom[linestyle=none,fillstyle=solid,fillcolor=curcolor]
{
\newpath
\moveto(557.81552598,1044.49300574)
\lineto(557.81552598,1043.90706823)
\lineto(552.30771333,1043.90706823)
\curveto(552.35979667,1043.08241543)(552.60719251,1042.45307514)(553.04990085,1042.01904735)
\curveto(553.49694947,1041.58935983)(554.11760921,1041.37451608)(554.91188007,1041.37451608)
\curveto(555.37194952,1041.37451608)(555.81682801,1041.43093969)(556.24651552,1041.54378692)
\curveto(556.68054331,1041.65663414)(557.11023082,1041.82590498)(557.53557805,1042.05159943)
\lineto(557.53557805,1040.9187869)
\curveto(557.10589054,1040.73649523)(556.66535234,1040.59760634)(556.21396343,1040.50212022)
\curveto(555.76257453,1040.40663411)(555.30467522,1040.35889105)(554.84026548,1040.35889105)
\curveto(553.67707101,1040.35889105)(552.75476196,1040.69743273)(552.07333833,1041.37451608)
\curveto(551.39625497,1042.05159943)(551.0577133,1042.96739807)(551.0577133,1044.12191198)
\curveto(551.0577133,1045.3154884)(551.37889386,1046.26166899)(552.02125499,1046.96045373)
\curveto(552.6679564,1047.66357874)(553.53818211,1048.01514125)(554.63193214,1048.01514125)
\curveto(555.61283495,1048.01514125)(556.38757455,1047.69830097)(556.95615095,1047.0646204)
\curveto(557.52906764,1046.4352801)(557.81552598,1045.57807522)(557.81552598,1044.49300574)
\closepath
\moveto(556.61760928,1044.84456825)
\curveto(556.60892872,1045.49995022)(556.42446691,1046.0229537)(556.06422385,1046.41357871)
\curveto(555.70832106,1046.80420372)(555.23523077,1046.99951623)(554.64495298,1046.99951623)
\curveto(553.97655018,1046.99951623)(553.44052586,1046.81071414)(553.03688002,1046.43310996)
\curveto(552.63757445,1046.05550579)(552.40753972,1045.52382174)(552.34677583,1044.83805784)
\closepath
}
}
{
\newrgbcolor{curcolor}{0 0 0}
\pscustom[linestyle=none,fillstyle=solid,fillcolor=curcolor]
{
\newpath
\moveto(564.43010668,1047.62451624)
\lineto(564.43010668,1046.49170371)
\curveto(564.09156501,1046.66531483)(563.7400025,1046.79552317)(563.37541915,1046.88232872)
\curveto(563.01083581,1046.96913428)(562.63323163,1047.01253706)(562.24260662,1047.01253706)
\curveto(561.64798855,1047.01253706)(561.20093993,1046.92139122)(560.90146076,1046.73909955)
\curveto(560.60632186,1046.55680788)(560.45875241,1046.28337037)(560.45875241,1045.91878703)
\curveto(560.45875241,1045.64100925)(560.56508922,1045.42182521)(560.77776284,1045.26123493)
\curveto(560.99043645,1045.10498493)(561.41795382,1044.95524534)(562.06031495,1044.81201617)
\lineto(562.47047121,1044.72087033)
\curveto(563.32116568,1044.53857866)(563.92446431,1044.28033213)(564.28036709,1043.94613073)
\curveto(564.64061016,1043.61626961)(564.82073169,1043.15403001)(564.82073169,1042.55941194)
\curveto(564.82073169,1041.88232859)(564.55163446,1041.34630427)(564.01344,1040.95133898)
\curveto(563.47958582,1040.5563737)(562.74390872,1040.35889105)(561.8064087,1040.35889105)
\curveto(561.41578369,1040.35889105)(561.00779756,1040.39795355)(560.58245033,1040.47607856)
\curveto(560.16144338,1040.54986328)(559.71656489,1040.6627105)(559.24781488,1040.81462023)
\lineto(559.24781488,1042.05159943)
\curveto(559.69052322,1041.8215647)(560.12672115,1041.64795359)(560.55640866,1041.53076608)
\curveto(560.98609617,1041.41791886)(561.41144341,1041.36149525)(561.83245036,1041.36149525)
\curveto(562.39668649,1041.36149525)(562.83071428,1041.45698136)(563.13453373,1041.64795359)
\curveto(563.43835318,1041.84326609)(563.59026291,1042.1167036)(563.59026291,1042.46826611)
\curveto(563.59026291,1042.79378695)(563.47958582,1043.04335293)(563.25823165,1043.21696404)
\curveto(563.04121776,1043.39057516)(562.56161705,1043.55767586)(561.81942953,1043.71826614)
\lineto(561.40276285,1043.81592239)
\curveto(560.66057533,1043.9721724)(560.12455101,1044.21088768)(559.79468989,1044.53206825)
\curveto(559.46482877,1044.85758909)(559.29989821,1045.30246757)(559.29989821,1045.8667037)
\curveto(559.29989821,1046.5524676)(559.54295378,1047.08198151)(560.0290649,1047.45524541)
\curveto(560.51517602,1047.8285093)(561.20528021,1048.01514125)(562.09937745,1048.01514125)
\curveto(562.5420858,1048.01514125)(562.95875248,1047.98258917)(563.34937749,1047.917485)
\curveto(563.7400025,1047.85238083)(564.10024556,1047.75472458)(564.43010668,1047.62451624)
\closepath
}
}
{
\newrgbcolor{curcolor}{0 0 0}
\pscustom[linestyle=none,fillstyle=solid,fillcolor=curcolor]
{
\newpath
\moveto(567.14495467,1039.31071394)
\lineto(567.60068385,1039.31071394)
\curveto(568.20832275,1039.31071394)(568.60328804,1039.40402992)(568.78557971,1039.59066187)
\curveto(568.97221166,1039.77729381)(569.06552764,1040.18093966)(569.06552764,1040.8015994)
\lineto(569.06552764,1042.41618277)
\curveto(569.06552764,1043.09326612)(569.16318389,1043.58588766)(569.3584964,1043.8940474)
\curveto(569.5538089,1044.20220713)(569.89235058,1044.41488074)(570.37412142,1044.53206825)
\curveto(569.89235058,1044.64057519)(569.5538089,1044.84890853)(569.3584964,1045.15706826)
\curveto(569.16318389,1045.46522799)(569.06552764,1045.96001967)(569.06552764,1046.6414433)
\lineto(569.06552764,1048.25602668)
\curveto(569.06552764,1048.87234614)(568.97221166,1049.27382184)(568.78557971,1049.46045379)
\curveto(568.60328804,1049.65142602)(568.20832275,1049.74691213)(567.60068385,1049.74691213)
\lineto(567.14495467,1049.74691213)
\lineto(567.14495467,1050.67790174)
\lineto(567.55511093,1050.67790174)
\curveto(568.63584013,1050.67790174)(569.35632626,1050.51731146)(569.71656932,1050.19613089)
\curveto(570.08115266,1049.87929061)(570.26344434,1049.24126976)(570.26344434,1048.28206834)
\lineto(570.26344434,1046.7195683)
\curveto(570.26344434,1046.0728669)(570.38063184,1045.62364814)(570.61500685,1045.37191202)
\curveto(570.84938185,1045.12451618)(571.27472909,1045.00081826)(571.89104855,1045.00081826)
\lineto(572.29469439,1045.00081826)
\lineto(572.29469439,1044.06982865)
\lineto(571.89104855,1044.06982865)
\curveto(571.27472909,1044.06982865)(570.84938185,1043.94396059)(570.61500685,1043.69222447)
\curveto(570.38063184,1043.44048836)(570.26344434,1042.98692932)(570.26344434,1042.33154735)
\lineto(570.26344434,1040.77555773)
\curveto(570.26344434,1039.81635632)(570.08115266,1039.17616533)(569.71656932,1038.85498476)
\curveto(569.35632626,1038.5338042)(568.63584013,1038.37321392)(567.55511093,1038.37321392)
\lineto(567.14495467,1038.37321392)
\closepath
}
}
{
\newrgbcolor{curcolor}{0 0 0}
\pscustom[linestyle=none,fillstyle=solid,fillcolor=curcolor]
{
\newpath
\moveto(395.56942395,1022.54526575)
\lineto(398.95484071,1011.58823421)
\lineto(397.84806985,1011.58823421)
\lineto(394.46265309,1022.54526575)
\closepath
}
}
{
\newrgbcolor{curcolor}{0 0 0}
\pscustom[linestyle=none,fillstyle=solid,fillcolor=curcolor]
{
\newpath
\moveto(401.40926788,1022.18719282)
\lineto(401.40926788,1020.11688027)
\lineto(403.87671587,1020.11688027)
\lineto(403.87671587,1019.18589066)
\lineto(401.40926788,1019.18589066)
\lineto(401.40926788,1015.22755722)
\curveto(401.40926788,1014.63293915)(401.48956302,1014.25099469)(401.65015331,1014.08172386)
\curveto(401.81508387,1013.91245302)(402.14711513,1013.8278176)(402.64624708,1013.8278176)
\lineto(403.87671587,1013.8278176)
\lineto(403.87671587,1012.82521341)
\lineto(402.64624708,1012.82521341)
\curveto(401.72176789,1012.82521341)(401.08374704,1012.99665438)(400.73218453,1013.33953634)
\curveto(400.38062202,1013.68675857)(400.20484077,1014.31609886)(400.20484077,1015.22755722)
\lineto(400.20484077,1019.18589066)
\lineto(399.3259345,1019.18589066)
\lineto(399.3259345,1020.11688027)
\lineto(400.20484077,1020.11688027)
\lineto(400.20484077,1022.18719282)
\closepath
}
}
{
\newrgbcolor{curcolor}{0 0 0}
\pscustom[linestyle=none,fillstyle=solid,fillcolor=curcolor]
{
\newpath
\moveto(411.51994497,1017.22625519)
\lineto(411.51994497,1012.82521341)
\lineto(410.32202827,1012.82521341)
\lineto(410.32202827,1017.18719269)
\curveto(410.32202827,1017.87729687)(410.18747966,1018.39378994)(409.91838243,1018.7366719)
\curveto(409.6492852,1019.07955385)(409.24563935,1019.25099483)(408.7074449,1019.25099483)
\curveto(408.06074349,1019.25099483)(407.55076084,1019.04483163)(407.17749694,1018.63250523)
\curveto(406.80423304,1018.22017883)(406.61760109,1017.65811284)(406.61760109,1016.94630727)
\lineto(406.61760109,1012.82521341)
\lineto(405.41317398,1012.82521341)
\lineto(405.41317398,1022.95542201)
\lineto(406.61760109,1022.95542201)
\lineto(406.61760109,1018.98406774)
\curveto(406.90405943,1019.4224358)(407.24043097,1019.75012678)(407.6267157,1019.96714068)
\curveto(408.01734071,1020.18415457)(408.46655947,1020.29266152)(408.97437199,1020.29266152)
\curveto(409.81204562,1020.29266152)(410.44572619,1020.03224485)(410.8754137,1019.5114115)
\curveto(411.30510121,1018.99491843)(411.51994497,1018.23319966)(411.51994497,1017.22625519)
\closepath
}
}
{
\newrgbcolor{curcolor}{0 0 0}
\pscustom[linestyle=none,fillstyle=solid,fillcolor=curcolor]
{
\newpath
\moveto(420.15926823,1016.77052601)
\lineto(420.15926823,1016.1845885)
\lineto(414.65145559,1016.1845885)
\curveto(414.70353892,1015.3599357)(414.95093476,1014.7305954)(415.39364311,1014.29656761)
\curveto(415.84069173,1013.8668801)(416.46135147,1013.65203635)(417.25562232,1013.65203635)
\curveto(417.71569178,1013.65203635)(418.16057026,1013.70845996)(418.59025777,1013.82130718)
\curveto(419.02428556,1013.93415441)(419.45397307,1014.10342525)(419.87932031,1014.3291197)
\lineto(419.87932031,1013.19630717)
\curveto(419.4496328,1013.0140155)(419.00909459,1012.8751266)(418.55770569,1012.77964049)
\curveto(418.10631679,1012.68415438)(417.64841747,1012.63641132)(417.18400774,1012.63641132)
\curveto(416.02081326,1012.63641132)(415.09850421,1012.97495299)(414.41708058,1013.65203635)
\curveto(413.73999723,1014.3291197)(413.40145555,1015.24491833)(413.40145555,1016.39943225)
\curveto(413.40145555,1017.59300867)(413.72263612,1018.53918925)(414.36499725,1019.23797399)
\curveto(415.01169865,1019.94109901)(415.88192437,1020.29266152)(416.9756744,1020.29266152)
\curveto(417.9565772,1020.29266152)(418.73131681,1019.97582123)(419.29989321,1019.34214066)
\curveto(419.87280989,1018.71280037)(420.15926823,1017.85559548)(420.15926823,1016.77052601)
\closepath
\moveto(418.96135153,1017.12208852)
\curveto(418.95267098,1017.77747048)(418.76820917,1018.30047397)(418.4079661,1018.69109898)
\curveto(418.05206332,1019.08172399)(417.57897303,1019.27703649)(416.98869523,1019.27703649)
\curveto(416.32029244,1019.27703649)(415.78426812,1019.0882344)(415.38062227,1018.71063023)
\curveto(414.98131671,1018.33302605)(414.75128198,1017.80134201)(414.69051809,1017.1155781)
\closepath
}
}
{
\newrgbcolor{curcolor}{0 0 0}
\pscustom[linestyle=none,fillstyle=solid,fillcolor=curcolor]
{
\newpath
\moveto(423.31031076,1022.18719282)
\lineto(423.31031076,1020.11688027)
\lineto(425.77775874,1020.11688027)
\lineto(425.77775874,1019.18589066)
\lineto(423.31031076,1019.18589066)
\lineto(423.31031076,1015.22755722)
\curveto(423.31031076,1014.63293915)(423.3906059,1014.25099469)(423.55119618,1014.08172386)
\curveto(423.71612674,1013.91245302)(424.048158,1013.8278176)(424.54728996,1013.8278176)
\lineto(425.77775874,1013.8278176)
\lineto(425.77775874,1012.82521341)
\lineto(424.54728996,1012.82521341)
\curveto(423.62281077,1012.82521341)(422.98478992,1012.99665438)(422.63322741,1013.33953634)
\curveto(422.2816649,1013.68675857)(422.10588364,1014.31609886)(422.10588364,1015.22755722)
\lineto(422.10588364,1019.18589066)
\lineto(421.22697737,1019.18589066)
\lineto(421.22697737,1020.11688027)
\lineto(422.10588364,1020.11688027)
\lineto(422.10588364,1022.18719282)
\closepath
}
}
{
\newrgbcolor{curcolor}{0 0 0}
\pscustom[linestyle=none,fillstyle=solid,fillcolor=curcolor]
{
\newpath
\moveto(430.67359164,1016.49057809)
\curveto(429.70570967,1016.49057809)(429.03513674,1016.379901)(428.66187284,1016.15854683)
\curveto(428.28860894,1015.93719266)(428.10197699,1015.55958848)(428.10197699,1015.0257343)
\curveto(428.10197699,1014.60038706)(428.24086588,1014.26184539)(428.51864367,1014.01010927)
\curveto(428.80076173,1013.76271343)(429.18270618,1013.63901551)(429.66447703,1013.63901551)
\curveto(430.32853955,1013.63901551)(430.86022359,1013.87339052)(431.25952916,1014.34214053)
\curveto(431.663175,1014.81523082)(431.86499792,1015.44240098)(431.86499792,1016.223651)
\lineto(431.86499792,1016.49057809)
\closepath
\moveto(433.06291462,1016.98536977)
\lineto(433.06291462,1012.82521341)
\lineto(431.86499792,1012.82521341)
\lineto(431.86499792,1013.93198427)
\curveto(431.59156041,1013.48927592)(431.2508486,1013.16158494)(430.84286248,1012.94891133)
\curveto(430.43487636,1012.74057799)(429.9357444,1012.63641132)(429.3454666,1012.63641132)
\curveto(428.59893881,1012.63641132)(428.00432074,1012.84474466)(427.56161239,1013.26141133)
\curveto(427.12324432,1013.68241829)(426.90406029,1014.24448428)(426.90406029,1014.9476093)
\curveto(426.90406029,1015.76792182)(427.1774978,1016.38641142)(427.72437281,1016.80307809)
\curveto(428.2755881,1017.21974477)(429.09590063,1017.42807811)(430.18531038,1017.42807811)
\lineto(431.86499792,1017.42807811)
\lineto(431.86499792,1017.54526561)
\curveto(431.86499792,1018.09648091)(431.68270625,1018.52182814)(431.31812291,1018.82130731)
\curveto(430.95787984,1019.12512677)(430.45006733,1019.27703649)(429.79468537,1019.27703649)
\curveto(429.37801869,1019.27703649)(428.97220271,1019.2271233)(428.57723742,1019.12729691)
\curveto(428.18227213,1019.02747051)(427.80249781,1018.87773093)(427.43791447,1018.67807814)
\lineto(427.43791447,1019.78484901)
\curveto(427.87628254,1019.95411984)(428.30162977,1020.0799879)(428.71395617,1020.16245318)
\curveto(429.12628257,1020.24925874)(429.52775828,1020.29266152)(429.91838329,1020.29266152)
\curveto(430.97307081,1020.29266152)(431.76083125,1020.01922401)(432.2816646,1019.472349)
\curveto(432.80249795,1018.92547398)(433.06291462,1018.09648091)(433.06291462,1016.98536977)
\closepath
}
}
{
\newrgbcolor{curcolor}{0 0 0}
\pscustom[linestyle=none,fillstyle=solid,fillcolor=curcolor]
{
\newpath
\moveto(441.07723907,1010.61167168)
\lineto(441.07723907,1009.68068207)
\lineto(434.15015555,1009.68068207)
\lineto(434.15015555,1010.61167168)
\closepath
}
}
{
\newrgbcolor{curcolor}{0 0 0}
\pscustom[linestyle=none,fillstyle=solid,fillcolor=curcolor]
{
\newpath
\moveto(447.76343418,1011.58823421)
\lineto(447.76343418,1010.65073418)
\lineto(447.35978833,1010.65073418)
\curveto(446.27905914,1010.65073418)(445.55423273,1010.81132446)(445.18530911,1011.13250503)
\curveto(444.82072577,1011.45368559)(444.63843409,1012.09387658)(444.63843409,1013.053078)
\lineto(444.63843409,1014.60906762)
\curveto(444.63843409,1015.26444958)(444.52124659,1015.71800862)(444.28687158,1015.96974474)
\curveto(444.05249658,1016.22148086)(443.62714935,1016.34734892)(443.01082988,1016.34734892)
\lineto(442.61369446,1016.34734892)
\lineto(442.61369446,1017.27833852)
\lineto(443.01082988,1017.27833852)
\curveto(443.63148962,1017.27833852)(444.05683686,1017.40203644)(444.28687158,1017.64943228)
\curveto(444.52124659,1017.9011684)(444.63843409,1018.35038716)(444.63843409,1018.99708857)
\lineto(444.63843409,1020.55958861)
\curveto(444.63843409,1021.51879002)(444.82072577,1022.15681087)(445.18530911,1022.47365116)
\curveto(445.55423273,1022.79483172)(446.27905914,1022.95542201)(447.35978833,1022.95542201)
\lineto(447.76343418,1022.95542201)
\lineto(447.76343418,1022.0244324)
\lineto(447.32072583,1022.0244324)
\curveto(446.70874665,1022.0244324)(446.30944108,1021.92894629)(446.12280913,1021.73797406)
\curveto(445.93617718,1021.54700183)(445.84286121,1021.14552613)(445.84286121,1020.53354694)
\lineto(445.84286121,1018.91896357)
\curveto(445.84286121,1018.23753994)(445.74303482,1017.74274826)(445.54338203,1017.43458853)
\curveto(445.34806953,1017.1264288)(445.01169799,1016.91809546)(444.53426742,1016.80958851)
\curveto(445.01603827,1016.69240101)(445.35457995,1016.47972739)(445.54989245,1016.17156766)
\curveto(445.74520496,1015.86340793)(445.84286121,1015.37078639)(445.84286121,1014.69370304)
\lineto(445.84286121,1013.07911966)
\curveto(445.84286121,1012.46714048)(445.93617718,1012.06566478)(446.12280913,1011.87469255)
\curveto(446.30944108,1011.68372032)(446.70874665,1011.58823421)(447.32072583,1011.58823421)
\closepath
}
}
{
\newrgbcolor{curcolor}{0 0 0}
\pscustom[linestyle=none,fillstyle=solid,fillcolor=curcolor]
{
\newpath
\moveto(456.75432179,1017.22625519)
\lineto(456.75432179,1012.82521341)
\lineto(455.55640509,1012.82521341)
\lineto(455.55640509,1017.18719269)
\curveto(455.55640509,1017.87729687)(455.42185648,1018.39378994)(455.15275925,1018.7366719)
\curveto(454.88366202,1019.07955385)(454.48001618,1019.25099483)(453.94182172,1019.25099483)
\curveto(453.29512031,1019.25099483)(452.78513766,1019.04483163)(452.41187376,1018.63250523)
\curveto(452.03860986,1018.22017883)(451.85197791,1017.65811284)(451.85197791,1016.94630727)
\lineto(451.85197791,1012.82521341)
\lineto(450.6475508,1012.82521341)
\lineto(450.6475508,1020.11688027)
\lineto(451.85197791,1020.11688027)
\lineto(451.85197791,1018.98406774)
\curveto(452.13843625,1019.4224358)(452.47480779,1019.75012678)(452.86109252,1019.96714068)
\curveto(453.25171753,1020.18415457)(453.70093629,1020.29266152)(454.20874881,1020.29266152)
\curveto(455.04642244,1020.29266152)(455.68010301,1020.03224485)(456.10979052,1019.5114115)
\curveto(456.53947804,1018.99491843)(456.75432179,1018.23319966)(456.75432179,1017.22625519)
\closepath
}
}
{
\newrgbcolor{curcolor}{0 0 0}
\pscustom[linestyle=none,fillstyle=solid,fillcolor=curcolor]
{
\newpath
\moveto(459.46265412,1019.71974484)
\lineto(460.83635207,1019.71974484)
\lineto(460.83635207,1018.06609896)
\lineto(459.46265412,1018.06609896)
\closepath
\moveto(459.46265412,1014.47885928)
\lineto(460.83635207,1014.47885928)
\lineto(460.83635207,1013.35906759)
\lineto(459.76864371,1011.2757342)
\lineto(458.92879994,1011.2757342)
\lineto(459.46265412,1013.35906759)
\closepath
}
}
{
\newrgbcolor{curcolor}{0 0 0}
\pscustom[linestyle=none,fillstyle=solid,fillcolor=curcolor]
{
\newpath
\moveto(468.30380388,1013.93198427)
\lineto(470.45224143,1013.93198427)
\lineto(470.45224143,1021.34734905)
\lineto(468.11500179,1020.87859904)
\lineto(468.11500179,1022.07651573)
\lineto(470.4392206,1022.54526575)
\lineto(471.7543248,1022.54526575)
\lineto(471.7543248,1013.93198427)
\lineto(473.90276236,1013.93198427)
\lineto(473.90276236,1012.82521341)
\lineto(468.30380388,1012.82521341)
\closepath
}
}
{
\newrgbcolor{curcolor}{0 0 0}
\pscustom[linestyle=none,fillstyle=solid,fillcolor=curcolor]
{
\newpath
\moveto(476.80641023,1011.58823421)
\lineto(477.26213941,1011.58823421)
\curveto(477.86977831,1011.58823421)(478.2647436,1011.68155018)(478.44703527,1011.86818213)
\curveto(478.63366722,1012.05481408)(478.7269832,1012.45845992)(478.7269832,1013.07911966)
\lineto(478.7269832,1014.69370304)
\curveto(478.7269832,1015.37078639)(478.82463945,1015.86340793)(479.01995195,1016.17156766)
\curveto(479.21526446,1016.47972739)(479.55380613,1016.69240101)(480.03557698,1016.80958851)
\curveto(479.55380613,1016.91809546)(479.21526446,1017.1264288)(479.01995195,1017.43458853)
\curveto(478.82463945,1017.74274826)(478.7269832,1018.23753994)(478.7269832,1018.91896357)
\lineto(478.7269832,1020.53354694)
\curveto(478.7269832,1021.1498664)(478.63366722,1021.55134211)(478.44703527,1021.73797406)
\curveto(478.2647436,1021.92894629)(477.86977831,1022.0244324)(477.26213941,1022.0244324)
\lineto(476.80641023,1022.0244324)
\lineto(476.80641023,1022.95542201)
\lineto(477.21656649,1022.95542201)
\curveto(478.29729569,1022.95542201)(479.01778182,1022.79483172)(479.37802488,1022.47365116)
\curveto(479.74260822,1022.15681087)(479.92489989,1021.51879002)(479.92489989,1020.55958861)
\lineto(479.92489989,1018.99708857)
\curveto(479.92489989,1018.35038716)(480.0420874,1017.9011684)(480.2764624,1017.64943228)
\curveto(480.51083741,1017.40203644)(480.93618464,1017.27833852)(481.5525041,1017.27833852)
\lineto(481.95614995,1017.27833852)
\lineto(481.95614995,1016.34734892)
\lineto(481.5525041,1016.34734892)
\curveto(480.93618464,1016.34734892)(480.51083741,1016.22148086)(480.2764624,1015.96974474)
\curveto(480.0420874,1015.71800862)(479.92489989,1015.26444958)(479.92489989,1014.60906762)
\lineto(479.92489989,1013.053078)
\curveto(479.92489989,1012.09387658)(479.74260822,1011.45368559)(479.37802488,1011.13250503)
\curveto(479.01778182,1010.81132446)(478.29729569,1010.65073418)(477.21656649,1010.65073418)
\lineto(476.80641023,1010.65073418)
\closepath
}
}
{
\newrgbcolor{curcolor}{0 0 0}
\pscustom[linestyle=none,fillstyle=solid,fillcolor=curcolor]
{
\newpath
\moveto(488.98089027,1022.54526575)
\lineto(492.36630703,1011.58823421)
\lineto(491.25953617,1011.58823421)
\lineto(487.87411941,1022.54526575)
\closepath
}
}
{
\newrgbcolor{curcolor}{0 0 0}
\pscustom[linestyle=none,fillstyle=solid,fillcolor=curcolor]
{
\newpath
\moveto(493.48609428,1022.54526575)
\lineto(496.87151103,1011.58823421)
\lineto(495.76474017,1011.58823421)
\lineto(492.37932342,1022.54526575)
\closepath
}
}
{
\newrgbcolor{curcolor}{0 0 0}
\pscustom[linestyle=none,fillstyle=solid,fillcolor=curcolor]
{
\newpath
\moveto(395.56942395,994.82278601)
\lineto(398.95484071,983.86575447)
\lineto(397.84806985,983.86575447)
\lineto(394.46265309,994.82278601)
\closepath
}
}
{
\newrgbcolor{curcolor}{0 0 0}
\pscustom[linestyle=none,fillstyle=solid,fillcolor=curcolor]
{
\newpath
\moveto(401.40926788,994.46471309)
\lineto(401.40926788,992.39440053)
\lineto(403.87671587,992.39440053)
\lineto(403.87671587,991.46341092)
\lineto(401.40926788,991.46341092)
\lineto(401.40926788,987.50507749)
\curveto(401.40926788,986.91045941)(401.48956302,986.52851496)(401.65015331,986.35924412)
\curveto(401.81508387,986.18997328)(402.14711513,986.10533787)(402.64624708,986.10533787)
\lineto(403.87671587,986.10533787)
\lineto(403.87671587,985.10273367)
\lineto(402.64624708,985.10273367)
\curveto(401.72176789,985.10273367)(401.08374704,985.27417465)(400.73218453,985.6170566)
\curveto(400.38062202,985.96427883)(400.20484077,986.59361913)(400.20484077,987.50507749)
\lineto(400.20484077,991.46341092)
\lineto(399.3259345,991.46341092)
\lineto(399.3259345,992.39440053)
\lineto(400.20484077,992.39440053)
\lineto(400.20484077,994.46471309)
\closepath
}
}
{
\newrgbcolor{curcolor}{0 0 0}
\pscustom[linestyle=none,fillstyle=solid,fillcolor=curcolor]
{
\newpath
\moveto(411.51994497,989.50377546)
\lineto(411.51994497,985.10273367)
\lineto(410.32202827,985.10273367)
\lineto(410.32202827,989.46471295)
\curveto(410.32202827,990.15481714)(410.18747966,990.67131021)(409.91838243,991.01419216)
\curveto(409.6492852,991.35707412)(409.24563935,991.52851509)(408.7074449,991.52851509)
\curveto(408.06074349,991.52851509)(407.55076084,991.32235189)(407.17749694,990.91002549)
\curveto(406.80423304,990.49769909)(406.61760109,989.93563311)(406.61760109,989.22382753)
\lineto(406.61760109,985.10273367)
\lineto(405.41317398,985.10273367)
\lineto(405.41317398,995.23294227)
\lineto(406.61760109,995.23294227)
\lineto(406.61760109,991.261588)
\curveto(406.90405943,991.69995607)(407.24043097,992.02764705)(407.6267157,992.24466094)
\curveto(408.01734071,992.46167484)(408.46655947,992.57018179)(408.97437199,992.57018179)
\curveto(409.81204562,992.57018179)(410.44572619,992.30976511)(410.8754137,991.78893177)
\curveto(411.30510121,991.2724387)(411.51994497,990.51071993)(411.51994497,989.50377546)
\closepath
}
}
{
\newrgbcolor{curcolor}{0 0 0}
\pscustom[linestyle=none,fillstyle=solid,fillcolor=curcolor]
{
\newpath
\moveto(420.15926823,989.04804628)
\lineto(420.15926823,988.46210876)
\lineto(414.65145559,988.46210876)
\curveto(414.70353892,987.63745596)(414.95093476,987.00811567)(415.39364311,986.57408788)
\curveto(415.84069173,986.14440037)(416.46135147,985.92955661)(417.25562232,985.92955661)
\curveto(417.71569178,985.92955661)(418.16057026,985.98598022)(418.59025777,986.09882745)
\curveto(419.02428556,986.21167467)(419.45397307,986.38094551)(419.87932031,986.60663996)
\lineto(419.87932031,985.47382743)
\curveto(419.4496328,985.29153576)(419.00909459,985.15264687)(418.55770569,985.05716075)
\curveto(418.10631679,984.96167464)(417.64841747,984.91393158)(417.18400774,984.91393158)
\curveto(416.02081326,984.91393158)(415.09850421,985.25247326)(414.41708058,985.92955661)
\curveto(413.73999723,986.60663996)(413.40145555,987.5224386)(413.40145555,988.67695252)
\curveto(413.40145555,989.87052894)(413.72263612,990.81670952)(414.36499725,991.51549426)
\curveto(415.01169865,992.21861928)(415.88192437,992.57018179)(416.9756744,992.57018179)
\curveto(417.9565772,992.57018179)(418.73131681,992.2533415)(419.29989321,991.61966093)
\curveto(419.87280989,990.99032063)(420.15926823,990.13311575)(420.15926823,989.04804628)
\closepath
\moveto(418.96135153,989.39960879)
\curveto(418.95267098,990.05499075)(418.76820917,990.57799423)(418.4079661,990.96861924)
\curveto(418.05206332,991.35924425)(417.57897303,991.55455676)(416.98869523,991.55455676)
\curveto(416.32029244,991.55455676)(415.78426812,991.36575467)(415.38062227,990.98815049)
\curveto(414.98131671,990.61054632)(414.75128198,990.07886228)(414.69051809,989.39309837)
\closepath
}
}
{
\newrgbcolor{curcolor}{0 0 0}
\pscustom[linestyle=none,fillstyle=solid,fillcolor=curcolor]
{
\newpath
\moveto(423.31031076,994.46471309)
\lineto(423.31031076,992.39440053)
\lineto(425.77775874,992.39440053)
\lineto(425.77775874,991.46341092)
\lineto(423.31031076,991.46341092)
\lineto(423.31031076,987.50507749)
\curveto(423.31031076,986.91045941)(423.3906059,986.52851496)(423.55119618,986.35924412)
\curveto(423.71612674,986.18997328)(424.048158,986.10533787)(424.54728996,986.10533787)
\lineto(425.77775874,986.10533787)
\lineto(425.77775874,985.10273367)
\lineto(424.54728996,985.10273367)
\curveto(423.62281077,985.10273367)(422.98478992,985.27417465)(422.63322741,985.6170566)
\curveto(422.2816649,985.96427883)(422.10588364,986.59361913)(422.10588364,987.50507749)
\lineto(422.10588364,991.46341092)
\lineto(421.22697737,991.46341092)
\lineto(421.22697737,992.39440053)
\lineto(422.10588364,992.39440053)
\lineto(422.10588364,994.46471309)
\closepath
}
}
{
\newrgbcolor{curcolor}{0 0 0}
\pscustom[linestyle=none,fillstyle=solid,fillcolor=curcolor]
{
\newpath
\moveto(430.67359164,988.76809835)
\curveto(429.70570967,988.76809835)(429.03513674,988.65742127)(428.66187284,988.43606709)
\curveto(428.28860894,988.21471292)(428.10197699,987.83710874)(428.10197699,987.30325456)
\curveto(428.10197699,986.87790733)(428.24086588,986.53936566)(428.51864367,986.28762954)
\curveto(428.80076173,986.0402337)(429.18270618,985.91653578)(429.66447703,985.91653578)
\curveto(430.32853955,985.91653578)(430.86022359,986.15091078)(431.25952916,986.6196608)
\curveto(431.663175,987.09275109)(431.86499792,987.71992124)(431.86499792,988.50117126)
\lineto(431.86499792,988.76809835)
\closepath
\moveto(433.06291462,989.26289003)
\lineto(433.06291462,985.10273367)
\lineto(431.86499792,985.10273367)
\lineto(431.86499792,986.20950454)
\curveto(431.59156041,985.76679619)(431.2508486,985.43910521)(430.84286248,985.22643159)
\curveto(430.43487636,985.01809825)(429.9357444,984.91393158)(429.3454666,984.91393158)
\curveto(428.59893881,984.91393158)(428.00432074,985.12226492)(427.56161239,985.5389316)
\curveto(427.12324432,985.95993856)(426.90406029,986.52200454)(426.90406029,987.22512956)
\curveto(426.90406029,988.04544208)(427.1774978,988.66393168)(427.72437281,989.08059836)
\curveto(428.2755881,989.49726504)(429.09590063,989.70559838)(430.18531038,989.70559838)
\lineto(431.86499792,989.70559838)
\lineto(431.86499792,989.82278588)
\curveto(431.86499792,990.37400117)(431.68270625,990.79934841)(431.31812291,991.09882758)
\curveto(430.95787984,991.40264703)(430.45006733,991.55455676)(429.79468537,991.55455676)
\curveto(429.37801869,991.55455676)(428.97220271,991.50464356)(428.57723742,991.40481717)
\curveto(428.18227213,991.30499078)(427.80249781,991.15525119)(427.43791447,990.95559841)
\lineto(427.43791447,992.06236927)
\curveto(427.87628254,992.23164011)(428.30162977,992.35750817)(428.71395617,992.43997345)
\curveto(429.12628257,992.52677901)(429.52775828,992.57018179)(429.91838329,992.57018179)
\curveto(430.97307081,992.57018179)(431.76083125,992.29674428)(432.2816646,991.74986926)
\curveto(432.80249795,991.20299425)(433.06291462,990.37400117)(433.06291462,989.26289003)
\closepath
}
}
{
\newrgbcolor{curcolor}{0 0 0}
\pscustom[linestyle=none,fillstyle=solid,fillcolor=curcolor]
{
\newpath
\moveto(441.07723907,982.88919195)
\lineto(441.07723907,981.95820234)
\lineto(434.15015555,981.95820234)
\lineto(434.15015555,982.88919195)
\closepath
}
}
{
\newrgbcolor{curcolor}{0 0 0}
\pscustom[linestyle=none,fillstyle=solid,fillcolor=curcolor]
{
\newpath
\moveto(447.76343418,983.86575447)
\lineto(447.76343418,982.92825445)
\lineto(447.35978833,982.92825445)
\curveto(446.27905914,982.92825445)(445.55423273,983.08884473)(445.18530911,983.41002529)
\curveto(444.82072577,983.73120586)(444.63843409,984.37139685)(444.63843409,985.33059826)
\lineto(444.63843409,986.88658789)
\curveto(444.63843409,987.54196985)(444.52124659,987.99552889)(444.28687158,988.24726501)
\curveto(444.05249658,988.49900112)(443.62714935,988.62486918)(443.01082988,988.62486918)
\lineto(442.61369446,988.62486918)
\lineto(442.61369446,989.55585879)
\lineto(443.01082988,989.55585879)
\curveto(443.63148962,989.55585879)(444.05683686,989.67955671)(444.28687158,989.92695255)
\curveto(444.52124659,990.17868867)(444.63843409,990.62790743)(444.63843409,991.27460884)
\lineto(444.63843409,992.83710888)
\curveto(444.63843409,993.79631029)(444.82072577,994.43433114)(445.18530911,994.75117143)
\curveto(445.55423273,995.07235199)(446.27905914,995.23294227)(447.35978833,995.23294227)
\lineto(447.76343418,995.23294227)
\lineto(447.76343418,994.30195266)
\lineto(447.32072583,994.30195266)
\curveto(446.70874665,994.30195266)(446.30944108,994.20646655)(446.12280913,994.01549432)
\curveto(445.93617718,993.8245221)(445.84286121,993.42304639)(445.84286121,992.81106721)
\lineto(445.84286121,991.19648383)
\curveto(445.84286121,990.5150602)(445.74303482,990.02026852)(445.54338203,989.71210879)
\curveto(445.34806953,989.40394906)(445.01169799,989.19561572)(444.53426742,989.08710878)
\curveto(445.01603827,988.96992127)(445.35457995,988.75724766)(445.54989245,988.44908793)
\curveto(445.74520496,988.1409282)(445.84286121,987.64830666)(445.84286121,986.97122331)
\lineto(445.84286121,985.35663993)
\curveto(445.84286121,984.74466075)(445.93617718,984.34318504)(446.12280913,984.15221281)
\curveto(446.30944108,983.96124059)(446.70874665,983.86575447)(447.32072583,983.86575447)
\closepath
}
}
{
\newrgbcolor{curcolor}{0 0 0}
\pscustom[linestyle=none,fillstyle=solid,fillcolor=curcolor]
{
\newpath
\moveto(456.75432179,989.50377546)
\lineto(456.75432179,985.10273367)
\lineto(455.55640509,985.10273367)
\lineto(455.55640509,989.46471295)
\curveto(455.55640509,990.15481714)(455.42185648,990.67131021)(455.15275925,991.01419216)
\curveto(454.88366202,991.35707412)(454.48001618,991.52851509)(453.94182172,991.52851509)
\curveto(453.29512031,991.52851509)(452.78513766,991.32235189)(452.41187376,990.91002549)
\curveto(452.03860986,990.49769909)(451.85197791,989.93563311)(451.85197791,989.22382753)
\lineto(451.85197791,985.10273367)
\lineto(450.6475508,985.10273367)
\lineto(450.6475508,992.39440053)
\lineto(451.85197791,992.39440053)
\lineto(451.85197791,991.261588)
\curveto(452.13843625,991.69995607)(452.47480779,992.02764705)(452.86109252,992.24466094)
\curveto(453.25171753,992.46167484)(453.70093629,992.57018179)(454.20874881,992.57018179)
\curveto(455.04642244,992.57018179)(455.68010301,992.30976511)(456.10979052,991.78893177)
\curveto(456.53947804,991.2724387)(456.75432179,990.51071993)(456.75432179,989.50377546)
\closepath
}
}
{
\newrgbcolor{curcolor}{0 0 0}
\pscustom[linestyle=none,fillstyle=solid,fillcolor=curcolor]
{
\newpath
\moveto(459.46265412,991.9972651)
\lineto(460.83635207,991.9972651)
\lineto(460.83635207,990.34361923)
\lineto(459.46265412,990.34361923)
\closepath
\moveto(459.46265412,986.75637955)
\lineto(460.83635207,986.75637955)
\lineto(460.83635207,985.63658785)
\lineto(459.76864371,983.55325447)
\lineto(458.92879994,983.55325447)
\lineto(459.46265412,985.63658785)
\closepath
}
}
{
\newrgbcolor{curcolor}{0 0 0}
\pscustom[linestyle=none,fillstyle=solid,fillcolor=curcolor]
{
\newpath
\moveto(469.20875182,986.20950454)
\lineto(473.79859569,986.20950454)
\lineto(473.79859569,985.10273367)
\lineto(467.62672053,985.10273367)
\lineto(467.62672053,986.20950454)
\curveto(468.12585248,986.7259976)(468.80510597,987.41827193)(469.664481,988.28632751)
\curveto(470.5281963,989.15872336)(471.07073103,989.72078935)(471.29208521,989.97252547)
\curveto(471.71309216,990.44561576)(472.00606092,990.84492132)(472.17099148,991.17044217)
\curveto(472.34026232,991.50030329)(472.42489774,991.82365399)(472.42489774,992.14049427)
\curveto(472.42489774,992.65698734)(472.24260606,993.0779943)(471.87802272,993.40351514)
\curveto(471.51777966,993.72903598)(471.04685951,993.8917964)(470.46526227,993.8917964)
\curveto(470.05293587,993.8917964)(469.61673794,993.82018182)(469.15666848,993.67695265)
\curveto(468.7009393,993.53372348)(468.21265804,993.31670958)(467.69182469,993.02591096)
\lineto(467.69182469,994.354036)
\curveto(468.2213386,994.56670962)(468.71613028,994.7272999)(469.17619973,994.83580685)
\curveto(469.63626919,994.94431379)(470.05727615,994.99856727)(470.4392206,994.99856727)
\curveto(471.44616507,994.99856727)(472.24911648,994.74683115)(472.84807483,994.24335891)
\curveto(473.44703318,993.73988668)(473.74651235,993.0671436)(473.74651235,992.22512969)
\curveto(473.74651235,991.82582413)(473.67055749,991.44604981)(473.51864776,991.08580675)
\curveto(473.37107832,990.72990396)(473.09981095,990.308897)(472.70484566,989.82278588)
\curveto(472.59633871,989.69691782)(472.25128662,989.33233448)(471.66968938,988.72903585)
\curveto(471.08809215,988.1300775)(470.26777962,987.29023373)(469.20875182,986.20950454)
\closepath
}
}
{
\newrgbcolor{curcolor}{0 0 0}
\pscustom[linestyle=none,fillstyle=solid,fillcolor=curcolor]
{
\newpath
\moveto(476.80641023,983.86575447)
\lineto(477.26213941,983.86575447)
\curveto(477.86977831,983.86575447)(478.2647436,983.95907045)(478.44703527,984.1457024)
\curveto(478.63366722,984.33233435)(478.7269832,984.73598019)(478.7269832,985.35663993)
\lineto(478.7269832,986.97122331)
\curveto(478.7269832,987.64830666)(478.82463945,988.1409282)(479.01995195,988.44908793)
\curveto(479.21526446,988.75724766)(479.55380613,988.96992127)(480.03557698,989.08710878)
\curveto(479.55380613,989.19561572)(479.21526446,989.40394906)(479.01995195,989.71210879)
\curveto(478.82463945,990.02026852)(478.7269832,990.5150602)(478.7269832,991.19648383)
\lineto(478.7269832,992.81106721)
\curveto(478.7269832,993.42738667)(478.63366722,993.82886237)(478.44703527,994.01549432)
\curveto(478.2647436,994.20646655)(477.86977831,994.30195266)(477.26213941,994.30195266)
\lineto(476.80641023,994.30195266)
\lineto(476.80641023,995.23294227)
\lineto(477.21656649,995.23294227)
\curveto(478.29729569,995.23294227)(479.01778182,995.07235199)(479.37802488,994.75117143)
\curveto(479.74260822,994.43433114)(479.92489989,993.79631029)(479.92489989,992.83710888)
\lineto(479.92489989,991.27460884)
\curveto(479.92489989,990.62790743)(480.0420874,990.17868867)(480.2764624,989.92695255)
\curveto(480.51083741,989.67955671)(480.93618464,989.55585879)(481.5525041,989.55585879)
\lineto(481.95614995,989.55585879)
\lineto(481.95614995,988.62486918)
\lineto(481.5525041,988.62486918)
\curveto(480.93618464,988.62486918)(480.51083741,988.49900112)(480.2764624,988.24726501)
\curveto(480.0420874,987.99552889)(479.92489989,987.54196985)(479.92489989,986.88658789)
\lineto(479.92489989,985.33059826)
\curveto(479.92489989,984.37139685)(479.74260822,983.73120586)(479.37802488,983.41002529)
\curveto(479.01778182,983.08884473)(478.29729569,982.92825445)(477.21656649,982.92825445)
\lineto(476.80641023,982.92825445)
\closepath
}
}
{
\newrgbcolor{curcolor}{0 0 0}
\pscustom[linestyle=none,fillstyle=solid,fillcolor=curcolor]
{
\newpath
\moveto(488.98089027,994.82278601)
\lineto(492.36630703,983.86575447)
\lineto(491.25953617,983.86575447)
\lineto(487.87411941,994.82278601)
\closepath
}
}
{
\newrgbcolor{curcolor}{0 0 0}
\pscustom[linestyle=none,fillstyle=solid,fillcolor=curcolor]
{
\newpath
\moveto(493.48609428,994.82278601)
\lineto(496.87151103,983.86575447)
\lineto(495.76474017,983.86575447)
\lineto(492.37932342,994.82278601)
\closepath
}
}
{
\newrgbcolor{curcolor}{0 0 0}
\pscustom[linestyle=none,fillstyle=solid,fillcolor=curcolor]
{
\newpath
\moveto(395.56942395,967.10030628)
\lineto(398.95484071,956.14327474)
\lineto(397.84806985,956.14327474)
\lineto(394.46265309,967.10030628)
\closepath
}
}
{
\newrgbcolor{curcolor}{0 0 0}
\pscustom[linestyle=none,fillstyle=solid,fillcolor=curcolor]
{
\newpath
\moveto(400.22437202,967.51046254)
\lineto(401.42228872,967.51046254)
\lineto(401.42228872,957.38025394)
\lineto(400.22437202,957.38025394)
\closepath
}
}
{
\newrgbcolor{curcolor}{0 0 0}
\pscustom[linestyle=none,fillstyle=solid,fillcolor=curcolor]
{
\newpath
\moveto(408.72046555,963.56514994)
\lineto(408.72046555,967.51046254)
\lineto(409.91838225,967.51046254)
\lineto(409.91838225,957.38025394)
\lineto(408.72046555,957.38025394)
\lineto(408.72046555,958.47400397)
\curveto(408.46872943,958.03997618)(408.14971901,957.71662548)(407.76343427,957.50395186)
\curveto(407.38148982,957.29561852)(406.92142036,957.19145185)(406.38322591,957.19145185)
\curveto(405.50214949,957.19145185)(404.7838335,957.54301436)(404.22827793,958.24613938)
\curveto(403.67706264,958.9492644)(403.40145499,959.87374359)(403.40145499,961.01957695)
\curveto(403.40145499,962.16541031)(403.67706264,963.08988951)(404.22827793,963.79301452)
\curveto(404.7838335,964.49613954)(405.50214949,964.84770205)(406.38322591,964.84770205)
\curveto(406.92142036,964.84770205)(407.38148982,964.74136524)(407.76343427,964.52869163)
\curveto(408.14971901,964.32035829)(408.46872943,963.99917772)(408.72046555,963.56514994)
\closepath
\moveto(404.63843419,961.01957695)
\curveto(404.63843419,960.13850054)(404.81855572,959.44622622)(405.17879879,958.94275398)
\curveto(405.54338213,958.44362202)(406.04251409,958.19405604)(406.67619466,958.19405604)
\curveto(407.30987523,958.19405604)(407.80900719,958.44362202)(408.17359054,958.94275398)
\curveto(408.53817388,959.44622622)(408.72046555,960.13850054)(408.72046555,961.01957695)
\curveto(408.72046555,961.90065336)(408.53817388,962.59075755)(408.17359054,963.08988951)
\curveto(407.80900719,963.59336174)(407.30987523,963.84509786)(406.67619466,963.84509786)
\curveto(406.04251409,963.84509786)(405.54338213,963.59336174)(405.17879879,963.08988951)
\curveto(404.81855572,962.59075755)(404.63843419,961.90065336)(404.63843419,961.01957695)
\closepath
}
}
{
\newrgbcolor{curcolor}{0 0 0}
\pscustom[linestyle=none,fillstyle=solid,fillcolor=curcolor]
{
\newpath
\moveto(415.21135126,963.83207703)
\curveto(414.56899013,963.83207703)(414.06117761,963.58034091)(413.68791372,963.07686867)
\curveto(413.31464982,962.57773671)(413.12801787,961.89197281)(413.12801787,961.01957695)
\curveto(413.12801787,960.1471811)(413.31247968,959.45924705)(413.6814033,958.95577481)
\curveto(414.0546672,958.45664286)(414.56464985,958.20707688)(415.21135126,958.20707688)
\curveto(415.84937211,958.20707688)(416.35501448,958.458813)(416.72827838,958.96228523)
\curveto(417.10154228,959.46575747)(417.28817423,960.15152137)(417.28817423,961.01957695)
\curveto(417.28817423,961.88329225)(417.10154228,962.56688602)(416.72827838,963.07035826)
\curveto(416.35501448,963.57817077)(415.84937211,963.83207703)(415.21135126,963.83207703)
\closepath
\moveto(415.21135126,964.84770205)
\curveto(416.25301795,964.84770205)(417.07116033,964.50916038)(417.6657784,963.83207703)
\curveto(418.26039648,963.15499367)(418.55770551,962.21749365)(418.55770551,961.01957695)
\curveto(418.55770551,959.82600053)(418.26039648,958.88850051)(417.6657784,958.20707688)
\curveto(417.07116033,957.52999353)(416.25301795,957.19145185)(415.21135126,957.19145185)
\curveto(414.16534428,957.19145185)(413.34503176,957.52999353)(412.75041369,958.20707688)
\curveto(412.1601359,958.88850051)(411.864997,959.82600053)(411.864997,961.01957695)
\curveto(411.864997,962.21749365)(412.1601359,963.15499367)(412.75041369,963.83207703)
\curveto(413.34503176,964.50916038)(414.16534428,964.84770205)(415.21135126,964.84770205)
\closepath
}
}
{
\newrgbcolor{curcolor}{0 0 0}
\pscustom[linestyle=none,fillstyle=solid,fillcolor=curcolor]
{
\newpath
\moveto(421.72176757,966.74223335)
\lineto(421.72176757,964.6719208)
\lineto(424.18921555,964.6719208)
\lineto(424.18921555,963.74093119)
\lineto(421.72176757,963.74093119)
\lineto(421.72176757,959.78259775)
\curveto(421.72176757,959.18797968)(421.80206271,958.80603523)(421.96265299,958.63676439)
\curveto(422.12758355,958.46749355)(422.45961481,958.38285813)(422.95874677,958.38285813)
\lineto(424.18921555,958.38285813)
\lineto(424.18921555,957.38025394)
\lineto(422.95874677,957.38025394)
\curveto(422.03426758,957.38025394)(421.39624673,957.55169492)(421.04468422,957.89457687)
\curveto(420.69312171,958.2417991)(420.51734045,958.87113939)(420.51734045,959.78259775)
\lineto(420.51734045,963.74093119)
\lineto(419.63843418,963.74093119)
\lineto(419.63843418,964.6719208)
\lineto(420.51734045,964.6719208)
\lineto(420.51734045,966.74223335)
\closepath
}
}
{
\newrgbcolor{curcolor}{0 0 0}
\pscustom[linestyle=none,fillstyle=solid,fillcolor=curcolor]
{
\newpath
\moveto(430.41968629,964.45707704)
\lineto(430.41968629,963.32426451)
\curveto(430.08114461,963.49787563)(429.7295821,963.62808396)(429.36499876,963.71488952)
\curveto(429.00041542,963.80169508)(428.62281124,963.84509786)(428.23218623,963.84509786)
\curveto(427.63756816,963.84509786)(427.19051954,963.75395202)(426.89104036,963.57166035)
\curveto(426.59590147,963.38936868)(426.44833202,963.11593117)(426.44833202,962.75134783)
\curveto(426.44833202,962.47357005)(426.55466883,962.25438601)(426.76734244,962.09379573)
\curveto(426.98001606,961.93754573)(427.40753343,961.78780614)(428.04989456,961.64457697)
\lineto(428.46005082,961.55343113)
\curveto(429.31074529,961.37113946)(429.91404392,961.11289293)(430.2699467,960.77869153)
\curveto(430.63018977,960.44883041)(430.8103113,959.98659081)(430.8103113,959.39197274)
\curveto(430.8103113,958.71488939)(430.54121407,958.17886507)(430.00301961,957.78389978)
\curveto(429.46916543,957.38893449)(428.73348833,957.19145185)(427.7959883,957.19145185)
\curveto(427.40536329,957.19145185)(426.99737717,957.23051435)(426.57202994,957.30863935)
\curveto(426.15102298,957.38242408)(425.7061445,957.4952713)(425.23739449,957.64718103)
\lineto(425.23739449,958.88416023)
\curveto(425.68010283,958.6541255)(426.11630076,958.48051438)(426.54598827,958.36332688)
\curveto(426.97567578,958.25047966)(427.40102302,958.19405604)(427.82202997,958.19405604)
\curveto(428.3862661,958.19405604)(428.82029389,958.28954216)(429.12411334,958.48051438)
\curveto(429.42793279,958.67582689)(429.57984252,958.9492644)(429.57984252,959.30082691)
\curveto(429.57984252,959.62634775)(429.46916543,959.87591373)(429.24781126,960.04952484)
\curveto(429.03079736,960.22313596)(428.55119666,960.39023666)(427.80900914,960.55082694)
\lineto(427.39234246,960.64848319)
\curveto(426.65015494,960.8047332)(426.11413062,961.04344848)(425.7842695,961.36462904)
\curveto(425.45440838,961.69014989)(425.28947782,962.13502837)(425.28947782,962.6992645)
\curveto(425.28947782,963.3850284)(425.53253338,963.91454231)(426.01864451,964.2878062)
\curveto(426.50475563,964.6610701)(427.19485982,964.84770205)(428.08895706,964.84770205)
\curveto(428.53166541,964.84770205)(428.94833208,964.81514997)(429.33895709,964.7500458)
\curveto(429.7295821,964.68494163)(430.08982517,964.58728538)(430.41968629,964.45707704)
\closepath
}
}
{
\newrgbcolor{curcolor}{0 0 0}
\pscustom[linestyle=none,fillstyle=solid,fillcolor=curcolor]
{
\newpath
\moveto(436.81942403,967.10030628)
\lineto(440.20484079,956.14327474)
\lineto(439.09806993,956.14327474)
\lineto(435.71265317,967.10030628)
\closepath
}
}
{
\newrgbcolor{curcolor}{0 0 0}
\pscustom[linestyle=none,fillstyle=solid,fillcolor=curcolor]
{
\newpath
\moveto(441.32463281,967.10030628)
\lineto(444.71004956,956.14327474)
\lineto(443.6032787,956.14327474)
\lineto(440.21786195,967.10030628)
\closepath
}
}
{
\newrgbcolor{curcolor}{0 0 0}
\pscustom[linestyle=none,fillstyle=solid,fillcolor=curcolor]
{
\newpath
\moveto(395.56942395,939.37782654)
\lineto(398.95484071,928.42079501)
\lineto(397.84806985,928.42079501)
\lineto(394.46265309,939.37782654)
\closepath
}
}
{
\newrgbcolor{curcolor}{0 0 0}
\pscustom[linestyle=none,fillstyle=solid,fillcolor=curcolor]
{
\newpath
\moveto(401.40926788,939.01975362)
\lineto(401.40926788,936.94944106)
\lineto(403.87671587,936.94944106)
\lineto(403.87671587,936.01845146)
\lineto(401.40926788,936.01845146)
\lineto(401.40926788,932.06011802)
\curveto(401.40926788,931.46549995)(401.48956302,931.08355549)(401.65015331,930.91428465)
\curveto(401.81508387,930.74501382)(402.14711513,930.6603784)(402.64624708,930.6603784)
\lineto(403.87671587,930.6603784)
\lineto(403.87671587,929.65777421)
\lineto(402.64624708,929.65777421)
\curveto(401.72176789,929.65777421)(401.08374704,929.82921518)(400.73218453,930.17209714)
\curveto(400.38062202,930.51931937)(400.20484077,931.14865966)(400.20484077,932.06011802)
\lineto(400.20484077,936.01845146)
\lineto(399.3259345,936.01845146)
\lineto(399.3259345,936.94944106)
\lineto(400.20484077,936.94944106)
\lineto(400.20484077,939.01975362)
\closepath
}
}
{
\newrgbcolor{curcolor}{0 0 0}
\pscustom[linestyle=none,fillstyle=solid,fillcolor=curcolor]
{
\newpath
\moveto(411.51994497,934.05881599)
\lineto(411.51994497,929.65777421)
\lineto(410.32202827,929.65777421)
\lineto(410.32202827,934.01975349)
\curveto(410.32202827,934.70985767)(410.18747966,935.22635074)(409.91838243,935.56923269)
\curveto(409.6492852,935.91211465)(409.24563935,936.08355562)(408.7074449,936.08355562)
\curveto(408.06074349,936.08355562)(407.55076084,935.87739242)(407.17749694,935.46506602)
\curveto(406.80423304,935.05273962)(406.61760109,934.49067364)(406.61760109,933.77886806)
\lineto(406.61760109,929.65777421)
\lineto(405.41317398,929.65777421)
\lineto(405.41317398,939.7879828)
\lineto(406.61760109,939.7879828)
\lineto(406.61760109,935.81662853)
\curveto(406.90405943,936.2549966)(407.24043097,936.58268758)(407.6267157,936.79970148)
\curveto(408.01734071,937.01671537)(408.46655947,937.12522232)(408.97437199,937.12522232)
\curveto(409.81204562,937.12522232)(410.44572619,936.86480564)(410.8754137,936.3439723)
\curveto(411.30510121,935.82747923)(411.51994497,935.06576046)(411.51994497,934.05881599)
\closepath
}
}
{
\newrgbcolor{curcolor}{0 0 0}
\pscustom[linestyle=none,fillstyle=solid,fillcolor=curcolor]
{
\newpath
\moveto(420.15926823,933.60308681)
\lineto(420.15926823,933.01714929)
\lineto(414.65145559,933.01714929)
\curveto(414.70353892,932.19249649)(414.95093476,931.5631562)(415.39364311,931.12912841)
\curveto(415.84069173,930.6994409)(416.46135147,930.48459714)(417.25562232,930.48459714)
\curveto(417.71569178,930.48459714)(418.16057026,930.54102076)(418.59025777,930.65386798)
\curveto(419.02428556,930.76671521)(419.45397307,930.93598604)(419.87932031,931.16168049)
\lineto(419.87932031,930.02886796)
\curveto(419.4496328,929.84657629)(419.00909459,929.7076874)(418.55770569,929.61220129)
\curveto(418.10631679,929.51671517)(417.64841747,929.46897212)(417.18400774,929.46897212)
\curveto(416.02081326,929.46897212)(415.09850421,929.80751379)(414.41708058,930.48459714)
\curveto(413.73999723,931.16168049)(413.40145555,932.07747913)(413.40145555,933.23199305)
\curveto(413.40145555,934.42556947)(413.72263612,935.37175005)(414.36499725,936.07053479)
\curveto(415.01169865,936.77365981)(415.88192437,937.12522232)(416.9756744,937.12522232)
\curveto(417.9565772,937.12522232)(418.73131681,936.80838203)(419.29989321,936.17470146)
\curveto(419.87280989,935.54536117)(420.15926823,934.68815628)(420.15926823,933.60308681)
\closepath
\moveto(418.96135153,933.95464932)
\curveto(418.95267098,934.61003128)(418.76820917,935.13303477)(418.4079661,935.52365978)
\curveto(418.05206332,935.91428479)(417.57897303,936.10959729)(416.98869523,936.10959729)
\curveto(416.32029244,936.10959729)(415.78426812,935.9207952)(415.38062227,935.54319103)
\curveto(414.98131671,935.16558685)(414.75128198,934.63390281)(414.69051809,933.9481389)
\closepath
}
}
{
\newrgbcolor{curcolor}{0 0 0}
\pscustom[linestyle=none,fillstyle=solid,fillcolor=curcolor]
{
\newpath
\moveto(423.31031076,939.01975362)
\lineto(423.31031076,936.94944106)
\lineto(425.77775874,936.94944106)
\lineto(425.77775874,936.01845146)
\lineto(423.31031076,936.01845146)
\lineto(423.31031076,932.06011802)
\curveto(423.31031076,931.46549995)(423.3906059,931.08355549)(423.55119618,930.91428465)
\curveto(423.71612674,930.74501382)(424.048158,930.6603784)(424.54728996,930.6603784)
\lineto(425.77775874,930.6603784)
\lineto(425.77775874,929.65777421)
\lineto(424.54728996,929.65777421)
\curveto(423.62281077,929.65777421)(422.98478992,929.82921518)(422.63322741,930.17209714)
\curveto(422.2816649,930.51931937)(422.10588364,931.14865966)(422.10588364,932.06011802)
\lineto(422.10588364,936.01845146)
\lineto(421.22697737,936.01845146)
\lineto(421.22697737,936.94944106)
\lineto(422.10588364,936.94944106)
\lineto(422.10588364,939.01975362)
\closepath
}
}
{
\newrgbcolor{curcolor}{0 0 0}
\pscustom[linestyle=none,fillstyle=solid,fillcolor=curcolor]
{
\newpath
\moveto(430.67359164,933.32313888)
\curveto(429.70570967,933.32313888)(429.03513674,933.2124618)(428.66187284,932.99110763)
\curveto(428.28860894,932.76975345)(428.10197699,932.39214928)(428.10197699,931.8582951)
\curveto(428.10197699,931.43294786)(428.24086588,931.09440619)(428.51864367,930.84267007)
\curveto(428.80076173,930.59527423)(429.18270618,930.47157631)(429.66447703,930.47157631)
\curveto(430.32853955,930.47157631)(430.86022359,930.70595132)(431.25952916,931.17470133)
\curveto(431.663175,931.64779162)(431.86499792,932.27496177)(431.86499792,933.05621179)
\lineto(431.86499792,933.32313888)
\closepath
\moveto(433.06291462,933.81793056)
\lineto(433.06291462,929.65777421)
\lineto(431.86499792,929.65777421)
\lineto(431.86499792,930.76454507)
\curveto(431.59156041,930.32183672)(431.2508486,929.99414574)(430.84286248,929.78147213)
\curveto(430.43487636,929.57313879)(429.9357444,929.46897212)(429.3454666,929.46897212)
\curveto(428.59893881,929.46897212)(428.00432074,929.67730546)(427.56161239,930.09397213)
\curveto(427.12324432,930.51497909)(426.90406029,931.07704508)(426.90406029,931.78017009)
\curveto(426.90406029,932.60048262)(427.1774978,933.21897222)(427.72437281,933.63563889)
\curveto(428.2755881,934.05230557)(429.09590063,934.26063891)(430.18531038,934.26063891)
\lineto(431.86499792,934.26063891)
\lineto(431.86499792,934.37782641)
\curveto(431.86499792,934.9290417)(431.68270625,935.35438894)(431.31812291,935.65386811)
\curveto(430.95787984,935.95768757)(430.45006733,936.10959729)(429.79468537,936.10959729)
\curveto(429.37801869,936.10959729)(428.97220271,936.0596841)(428.57723742,935.9598577)
\curveto(428.18227213,935.86003131)(427.80249781,935.71029173)(427.43791447,935.51063894)
\lineto(427.43791447,936.6174098)
\curveto(427.87628254,936.78668064)(428.30162977,936.9125487)(428.71395617,936.99501398)
\curveto(429.12628257,937.08181954)(429.52775828,937.12522232)(429.91838329,937.12522232)
\curveto(430.97307081,937.12522232)(431.76083125,936.85178481)(432.2816646,936.3049098)
\curveto(432.80249795,935.75803478)(433.06291462,934.9290417)(433.06291462,933.81793056)
\closepath
}
}
{
\newrgbcolor{curcolor}{0 0 0}
\pscustom[linestyle=none,fillstyle=solid,fillcolor=curcolor]
{
\newpath
\moveto(441.07723907,927.44423248)
\lineto(441.07723907,926.51324287)
\lineto(434.15015555,926.51324287)
\lineto(434.15015555,927.44423248)
\closepath
}
}
{
\newrgbcolor{curcolor}{0 0 0}
\pscustom[linestyle=none,fillstyle=solid,fillcolor=curcolor]
{
\newpath
\moveto(447.76343418,928.42079501)
\lineto(447.76343418,927.48329498)
\lineto(447.35978833,927.48329498)
\curveto(446.27905914,927.48329498)(445.55423273,927.64388526)(445.18530911,927.96506583)
\curveto(444.82072577,928.28624639)(444.63843409,928.92643738)(444.63843409,929.88563879)
\lineto(444.63843409,931.44162842)
\curveto(444.63843409,932.09701038)(444.52124659,932.55056942)(444.28687158,932.80230554)
\curveto(444.05249658,933.05404166)(443.62714935,933.17990971)(443.01082988,933.17990971)
\lineto(442.61369446,933.17990971)
\lineto(442.61369446,934.11089932)
\lineto(443.01082988,934.11089932)
\curveto(443.63148962,934.11089932)(444.05683686,934.23459724)(444.28687158,934.48199308)
\curveto(444.52124659,934.7337292)(444.63843409,935.18294796)(444.63843409,935.82964937)
\lineto(444.63843409,937.39214941)
\curveto(444.63843409,938.35135082)(444.82072577,938.98937167)(445.18530911,939.30621196)
\curveto(445.55423273,939.62739252)(446.27905914,939.7879828)(447.35978833,939.7879828)
\lineto(447.76343418,939.7879828)
\lineto(447.76343418,938.8569932)
\lineto(447.32072583,938.8569932)
\curveto(446.70874665,938.8569932)(446.30944108,938.76150708)(446.12280913,938.57053486)
\curveto(445.93617718,938.37956263)(445.84286121,937.97808692)(445.84286121,937.36610774)
\lineto(445.84286121,935.75152437)
\curveto(445.84286121,935.07010074)(445.74303482,934.57530906)(445.54338203,934.26714933)
\curveto(445.34806953,933.9589896)(445.01169799,933.75065626)(444.53426742,933.64214931)
\curveto(445.01603827,933.52496181)(445.35457995,933.31228819)(445.54989245,933.00412846)
\curveto(445.74520496,932.69596873)(445.84286121,932.20334719)(445.84286121,931.52626384)
\lineto(445.84286121,929.91168046)
\curveto(445.84286121,929.29970128)(445.93617718,928.89822557)(446.12280913,928.70725335)
\curveto(446.30944108,928.51628112)(446.70874665,928.42079501)(447.32072583,928.42079501)
\closepath
}
}
{
\newrgbcolor{curcolor}{0 0 0}
\pscustom[linestyle=none,fillstyle=solid,fillcolor=curcolor]
{
\newpath
\moveto(456.75432179,934.05881599)
\lineto(456.75432179,929.65777421)
\lineto(455.55640509,929.65777421)
\lineto(455.55640509,934.01975349)
\curveto(455.55640509,934.70985767)(455.42185648,935.22635074)(455.15275925,935.56923269)
\curveto(454.88366202,935.91211465)(454.48001618,936.08355562)(453.94182172,936.08355562)
\curveto(453.29512031,936.08355562)(452.78513766,935.87739242)(452.41187376,935.46506602)
\curveto(452.03860986,935.05273962)(451.85197791,934.49067364)(451.85197791,933.77886806)
\lineto(451.85197791,929.65777421)
\lineto(450.6475508,929.65777421)
\lineto(450.6475508,936.94944106)
\lineto(451.85197791,936.94944106)
\lineto(451.85197791,935.81662853)
\curveto(452.13843625,936.2549966)(452.47480779,936.58268758)(452.86109252,936.79970148)
\curveto(453.25171753,937.01671537)(453.70093629,937.12522232)(454.20874881,937.12522232)
\curveto(455.04642244,937.12522232)(455.68010301,936.86480564)(456.10979052,936.3439723)
\curveto(456.53947804,935.82747923)(456.75432179,935.06576046)(456.75432179,934.05881599)
\closepath
}
}
{
\newrgbcolor{curcolor}{0 0 0}
\pscustom[linestyle=none,fillstyle=solid,fillcolor=curcolor]
{
\newpath
\moveto(459.46265412,936.55230564)
\lineto(460.83635207,936.55230564)
\lineto(460.83635207,934.89865976)
\lineto(459.46265412,934.89865976)
\closepath
\moveto(459.46265412,931.31142008)
\lineto(460.83635207,931.31142008)
\lineto(460.83635207,930.19162839)
\lineto(459.76864371,928.108295)
\lineto(458.92879994,928.108295)
\lineto(459.46265412,930.19162839)
\closepath
}
}
{
\newrgbcolor{curcolor}{0 0 0}
\pscustom[linestyle=none,fillstyle=solid,fillcolor=curcolor]
{
\newpath
\moveto(473.96786653,934.05881599)
\lineto(473.96786653,929.65777421)
\lineto(472.76994983,929.65777421)
\lineto(472.76994983,934.01975349)
\curveto(472.76994983,934.70985767)(472.63540121,935.22635074)(472.36630398,935.56923269)
\curveto(472.09720676,935.91211465)(471.69356091,936.08355562)(471.15536645,936.08355562)
\curveto(470.50866505,936.08355562)(469.99868239,935.87739242)(469.6254185,935.46506602)
\curveto(469.2521546,935.05273962)(469.06552265,934.49067364)(469.06552265,933.77886806)
\lineto(469.06552265,929.65777421)
\lineto(467.86109553,929.65777421)
\lineto(467.86109553,936.94944106)
\lineto(469.06552265,936.94944106)
\lineto(469.06552265,935.81662853)
\curveto(469.35198099,936.2549966)(469.68835253,936.58268758)(470.07463726,936.79970148)
\curveto(470.46526227,937.01671537)(470.91448103,937.12522232)(471.42229354,937.12522232)
\curveto(472.25996718,937.12522232)(472.89364775,936.86480564)(473.32333526,936.3439723)
\curveto(473.75302277,935.82747923)(473.96786653,935.06576046)(473.96786653,934.05881599)
\closepath
}
}
{
\newrgbcolor{curcolor}{0 0 0}
\pscustom[linestyle=none,fillstyle=solid,fillcolor=curcolor]
{
\newpath
\moveto(481.33765254,939.37782654)
\lineto(484.87280889,935.75152437)
\lineto(483.5642151,935.75152437)
\lineto(480.69963169,938.32313902)
\lineto(477.83504828,935.75152437)
\lineto(476.5264545,935.75152437)
\lineto(480.06161084,939.37782654)
\closepath
}
}
{
\newrgbcolor{curcolor}{0 0 0}
\pscustom[linestyle=none,fillstyle=solid,fillcolor=curcolor]
{
\newpath
\moveto(487.4965066,939.7879828)
\lineto(488.70093371,939.7879828)
\lineto(488.70093371,933.80490973)
\lineto(492.27515256,936.94944106)
\lineto(493.80510051,936.94944106)
\lineto(489.93791291,933.53798264)
\lineto(493.96786094,929.65777421)
\lineto(492.40536089,929.65777421)
\lineto(488.70093371,933.21897222)
\lineto(488.70093371,929.65777421)
\lineto(487.4965066,929.65777421)
\closepath
}
}
{
\newrgbcolor{curcolor}{0 0 0}
\pscustom[linestyle=none,fillstyle=solid,fillcolor=curcolor]
{
\newpath
\moveto(495.68661079,928.42079501)
\lineto(496.14233997,928.42079501)
\curveto(496.74997888,928.42079501)(497.14494416,928.51411098)(497.32723583,928.70074293)
\curveto(497.51386778,928.88737488)(497.60718376,929.29102072)(497.60718376,929.91168046)
\lineto(497.60718376,931.52626384)
\curveto(497.60718376,932.20334719)(497.70484001,932.69596873)(497.90015252,933.00412846)
\curveto(498.09546502,933.31228819)(498.4340067,933.52496181)(498.91577754,933.64214931)
\curveto(498.4340067,933.75065626)(498.09546502,933.9589896)(497.90015252,934.26714933)
\curveto(497.70484001,934.57530906)(497.60718376,935.07010074)(497.60718376,935.75152437)
\lineto(497.60718376,937.36610774)
\curveto(497.60718376,937.9824272)(497.51386778,938.38390291)(497.32723583,938.57053486)
\curveto(497.14494416,938.76150708)(496.74997888,938.8569932)(496.14233997,938.8569932)
\lineto(495.68661079,938.8569932)
\lineto(495.68661079,939.7879828)
\lineto(496.09676705,939.7879828)
\curveto(497.17749625,939.7879828)(497.89798238,939.62739252)(498.25822544,939.30621196)
\curveto(498.62280879,938.98937167)(498.80510046,938.35135082)(498.80510046,937.39214941)
\lineto(498.80510046,935.82964937)
\curveto(498.80510046,935.18294796)(498.92228796,934.7337292)(499.15666297,934.48199308)
\curveto(499.39103797,934.23459724)(499.81638521,934.11089932)(500.43270467,934.11089932)
\lineto(500.83635051,934.11089932)
\lineto(500.83635051,933.17990971)
\lineto(500.43270467,933.17990971)
\curveto(499.81638521,933.17990971)(499.39103797,933.05404166)(499.15666297,932.80230554)
\curveto(498.92228796,932.55056942)(498.80510046,932.09701038)(498.80510046,931.44162842)
\lineto(498.80510046,929.88563879)
\curveto(498.80510046,928.92643738)(498.62280879,928.28624639)(498.25822544,927.96506583)
\curveto(497.89798238,927.64388526)(497.17749625,927.48329498)(496.09676705,927.48329498)
\lineto(495.68661079,927.48329498)
\closepath
}
}
{
\newrgbcolor{curcolor}{0 0 0}
\pscustom[linestyle=none,fillstyle=solid,fillcolor=curcolor]
{
\newpath
\moveto(395.56942395,911.65534681)
\lineto(398.95484071,900.69831527)
\lineto(397.84806985,900.69831527)
\lineto(394.46265309,911.65534681)
\closepath
}
}
{
\newrgbcolor{curcolor}{0 0 0}
\pscustom[linestyle=none,fillstyle=solid,fillcolor=curcolor]
{
\newpath
\moveto(406.46135135,905.88060708)
\lineto(406.46135135,905.29466956)
\lineto(400.9535387,905.29466956)
\curveto(401.00562204,904.47001676)(401.25301788,903.84067647)(401.69572622,903.40664868)
\curveto(402.14277485,902.97696117)(402.76343459,902.76211741)(403.55770544,902.76211741)
\curveto(404.0177749,902.76211741)(404.46265338,902.81854102)(404.89234089,902.93138825)
\curveto(405.32636868,903.04423547)(405.75605619,903.21350631)(406.18140343,903.43920076)
\lineto(406.18140343,902.30638823)
\curveto(405.75171591,902.12409656)(405.31117771,901.98520767)(404.85978881,901.88972155)
\curveto(404.40839991,901.79423544)(403.95050059,901.74649238)(403.48609085,901.74649238)
\curveto(402.32289638,901.74649238)(401.40058733,902.08503406)(400.7191637,902.76211741)
\curveto(400.04208035,903.43920076)(399.70353867,904.3549994)(399.70353867,905.50951332)
\curveto(399.70353867,906.70308974)(400.02471924,907.64927032)(400.66708036,908.34805506)
\curveto(401.31378177,909.05118008)(402.18400749,909.40274258)(403.27775752,909.40274258)
\curveto(404.25866032,909.40274258)(405.03339992,909.0859023)(405.60197633,908.45222173)
\curveto(406.17489301,907.82288143)(406.46135135,906.96567655)(406.46135135,905.88060708)
\closepath
\moveto(405.26343465,906.23216958)
\curveto(405.2547541,906.88755155)(405.07029229,907.41055503)(404.71004922,907.80118004)
\curveto(404.35414643,908.19180505)(403.88105614,908.38711756)(403.29077835,908.38711756)
\curveto(402.62237555,908.38711756)(402.08635123,908.19831547)(401.68270539,907.82071129)
\curveto(401.28339982,907.44310712)(401.0533651,906.91142307)(400.99260121,906.22565917)
\closepath
}
}
{
\newrgbcolor{curcolor}{0 0 0}
\pscustom[linestyle=none,fillstyle=solid,fillcolor=curcolor]
{
\newpath
\moveto(414.48869519,906.33633625)
\lineto(414.48869519,901.93529447)
\lineto(413.29077849,901.93529447)
\lineto(413.29077849,906.29727375)
\curveto(413.29077849,906.98737794)(413.15622988,907.50387101)(412.88713265,907.84675296)
\curveto(412.61803542,908.18963491)(412.21438958,908.36107589)(411.67619512,908.36107589)
\curveto(411.02949371,908.36107589)(410.51951106,908.15491269)(410.14624716,907.74258629)
\curveto(409.77298326,907.33025989)(409.58635131,906.7681939)(409.58635131,906.05638833)
\lineto(409.58635131,901.93529447)
\lineto(408.3819242,901.93529447)
\lineto(408.3819242,909.22696133)
\lineto(409.58635131,909.22696133)
\lineto(409.58635131,908.0941488)
\curveto(409.87280965,908.53251687)(410.20918119,908.86020785)(410.59546592,909.07722174)
\curveto(410.98609093,909.29423564)(411.4353097,909.40274258)(411.94312221,909.40274258)
\curveto(412.78079584,909.40274258)(413.41447641,909.14232591)(413.84416393,908.62149256)
\curveto(414.27385144,908.10499949)(414.48869519,907.34328072)(414.48869519,906.33633625)
\closepath
}
}
{
\newrgbcolor{curcolor}{0 0 0}
\pscustom[linestyle=none,fillstyle=solid,fillcolor=curcolor]
{
\newpath
\moveto(421.68921753,908.12019047)
\lineto(421.68921753,912.06550307)
\lineto(422.88713422,912.06550307)
\lineto(422.88713422,901.93529447)
\lineto(421.68921753,901.93529447)
\lineto(421.68921753,903.0290445)
\curveto(421.43748141,902.59501671)(421.11847098,902.27166601)(420.73218625,902.05899239)
\curveto(420.3502418,901.85065905)(419.89017234,901.74649238)(419.35197788,901.74649238)
\curveto(418.47090147,901.74649238)(417.75258548,902.09805489)(417.19702991,902.80117991)
\curveto(416.64581462,903.50430493)(416.37020697,904.42878412)(416.37020697,905.57461748)
\curveto(416.37020697,906.72045085)(416.64581462,907.64493004)(417.19702991,908.34805506)
\curveto(417.75258548,909.05118008)(418.47090147,909.40274258)(419.35197788,909.40274258)
\curveto(419.89017234,909.40274258)(420.3502418,909.29640578)(420.73218625,909.08373216)
\curveto(421.11847098,908.87539882)(421.43748141,908.55421826)(421.68921753,908.12019047)
\closepath
\moveto(417.60718617,905.57461748)
\curveto(417.60718617,904.69354107)(417.7873077,904.00126675)(418.14755077,903.49779451)
\curveto(418.51213411,902.99866255)(419.01126607,902.74909658)(419.64494664,902.74909658)
\curveto(420.27862721,902.74909658)(420.77775917,902.99866255)(421.14234251,903.49779451)
\curveto(421.50692585,904.00126675)(421.68921753,904.69354107)(421.68921753,905.57461748)
\curveto(421.68921753,906.4556939)(421.50692585,907.14579808)(421.14234251,907.64493004)
\curveto(420.77775917,908.14840227)(420.27862721,908.40013839)(419.64494664,908.40013839)
\curveto(419.01126607,908.40013839)(418.51213411,908.14840227)(418.14755077,907.64493004)
\curveto(417.7873077,907.14579808)(417.60718617,906.4556939)(417.60718617,905.57461748)
\closepath
}
}
{
\newrgbcolor{curcolor}{0 0 0}
\pscustom[linestyle=none,fillstyle=solid,fillcolor=curcolor]
{
\newpath
\moveto(430.91447711,900.69831527)
\lineto(430.91447711,899.76081525)
\lineto(430.51083127,899.76081525)
\curveto(429.43010207,899.76081525)(428.70527566,899.92140553)(428.33635204,900.24258609)
\curveto(427.9717687,900.56376666)(427.78947703,901.20395765)(427.78947703,902.16315906)
\lineto(427.78947703,903.71914868)
\curveto(427.78947703,904.37453065)(427.67228953,904.82808969)(427.43791452,905.0798258)
\curveto(427.20353951,905.33156192)(426.77819228,905.45742998)(426.16187282,905.45742998)
\lineto(425.76473739,905.45742998)
\lineto(425.76473739,906.38841959)
\lineto(426.16187282,906.38841959)
\curveto(426.78253256,906.38841959)(427.20787979,906.51211751)(427.43791452,906.75951335)
\curveto(427.67228953,907.01124947)(427.78947703,907.46046823)(427.78947703,908.10716963)
\lineto(427.78947703,909.66966967)
\curveto(427.78947703,910.62887109)(427.9717687,911.26689194)(428.33635204,911.58373223)
\curveto(428.70527566,911.90491279)(429.43010207,912.06550307)(430.51083127,912.06550307)
\lineto(430.91447711,912.06550307)
\lineto(430.91447711,911.13451346)
\lineto(430.47176877,911.13451346)
\curveto(429.85978958,911.13451346)(429.46048402,911.03902735)(429.27385207,910.84805512)
\curveto(429.08722012,910.6570829)(428.99390414,910.25560719)(428.99390414,909.64362801)
\lineto(428.99390414,908.02904463)
\curveto(428.99390414,907.347621)(428.89407775,906.85282932)(428.69442497,906.54466959)
\curveto(428.49911246,906.23650986)(428.16274093,906.02817652)(427.68531036,905.91966958)
\curveto(428.16708121,905.80248207)(428.50562288,905.58980846)(428.70093539,905.28164873)
\curveto(428.89624789,904.973489)(428.99390414,904.48086745)(428.99390414,903.8037841)
\lineto(428.99390414,902.18920073)
\curveto(428.99390414,901.57722155)(429.08722012,901.17574584)(429.27385207,900.98477361)
\curveto(429.46048402,900.79380139)(429.85978958,900.69831527)(430.47176877,900.69831527)
\closepath
}
}
{
\newrgbcolor{curcolor}{0 0 0}
\pscustom[linestyle=none,fillstyle=solid,fillcolor=curcolor]
{
\newpath
\moveto(439.09156024,908.94701341)
\lineto(439.09156024,907.82722171)
\curveto(438.75301856,908.01385366)(438.41230675,908.15274255)(438.06942479,908.24388839)
\curveto(437.73088312,908.3393745)(437.38800116,908.38711756)(437.04077893,908.38711756)
\curveto(436.26386919,908.38711756)(435.66057056,908.13972172)(435.23088305,907.64493004)
\curveto(434.80119554,907.15447864)(434.58635178,906.46437445)(434.58635178,905.57461748)
\curveto(434.58635178,904.68486052)(434.80119554,903.99258619)(435.23088305,903.49779451)
\curveto(435.66057056,903.00734311)(436.26386919,902.76211741)(437.04077893,902.76211741)
\curveto(437.38800116,902.76211741)(437.73088312,902.80769033)(438.06942479,902.89883616)
\curveto(438.41230675,902.99432228)(438.75301856,903.13538131)(439.09156024,903.32201326)
\lineto(439.09156024,902.2152424)
\curveto(438.75735884,902.05899239)(438.41013661,901.94180489)(438.04989354,901.86367989)
\curveto(437.69399076,901.78555488)(437.31421644,901.74649238)(436.9105706,901.74649238)
\curveto(435.81248029,901.74649238)(434.94008443,902.09154448)(434.29338303,902.78164866)
\curveto(433.64668162,903.47175285)(433.32333092,904.40274245)(433.32333092,905.57461748)
\curveto(433.32333092,906.76385363)(433.64885176,907.69918351)(434.29989344,908.38060714)
\curveto(434.95527541,909.06203077)(435.85154279,909.40274258)(436.9886956,909.40274258)
\curveto(437.35761922,909.40274258)(437.71786228,909.36368008)(438.06942479,909.28555508)
\curveto(438.4209873,909.21177036)(438.76169912,909.09892313)(439.09156024,908.94701341)
\closepath
}
}
{
\newrgbcolor{curcolor}{0 0 0}
\pscustom[linestyle=none,fillstyle=solid,fillcolor=curcolor]
{
\newpath
\moveto(444.50171623,905.60065915)
\curveto(443.53383426,905.60065915)(442.86326133,905.48998206)(442.48999743,905.26862789)
\curveto(442.11673353,905.04727372)(441.93010158,904.66966954)(441.93010158,904.13581536)
\curveto(441.93010158,903.71046813)(442.06899047,903.37192645)(442.34676826,903.12019034)
\curveto(442.62888632,902.8727945)(443.01083078,902.74909658)(443.49260162,902.74909658)
\curveto(444.15666414,902.74909658)(444.68834818,902.98347158)(445.08765375,903.45222159)
\curveto(445.49129959,903.92531188)(445.69312251,904.55248204)(445.69312251,905.33373206)
\lineto(445.69312251,905.60065915)
\closepath
\moveto(446.89103921,906.09545083)
\lineto(446.89103921,901.93529447)
\lineto(445.69312251,901.93529447)
\lineto(445.69312251,903.04206533)
\curveto(445.41968501,902.59935699)(445.07897319,902.27166601)(444.67098707,902.05899239)
\curveto(444.26300095,901.85065905)(443.76386899,901.74649238)(443.1735912,901.74649238)
\curveto(442.4270634,901.74649238)(441.83244533,901.95482572)(441.38973698,902.3714924)
\curveto(440.95136892,902.79249935)(440.73218488,903.35456534)(440.73218488,904.05769036)
\curveto(440.73218488,904.87800288)(441.00562239,905.49649248)(441.55249741,905.91315916)
\curveto(442.1037127,906.32982584)(442.92402522,906.53815918)(444.01343497,906.53815918)
\lineto(445.69312251,906.53815918)
\lineto(445.69312251,906.65534668)
\curveto(445.69312251,907.20656197)(445.51083084,907.6319092)(445.1462475,907.93138838)
\curveto(444.78600443,908.23520783)(444.27819192,908.38711756)(443.62280996,908.38711756)
\curveto(443.20614328,908.38711756)(442.8003273,908.33720436)(442.40536201,908.23737797)
\curveto(442.01039672,908.13755158)(441.63062241,907.98781199)(441.26603906,907.78815921)
\lineto(441.26603906,908.89493007)
\curveto(441.70440713,909.06420091)(442.12975436,909.19006897)(442.54208076,909.27253425)
\curveto(442.95440716,909.35933981)(443.35588287,909.40274258)(443.74650788,909.40274258)
\curveto(444.80119541,909.40274258)(445.58895584,909.12930508)(446.10978919,908.58243006)
\curveto(446.63062254,908.03555505)(446.89103921,907.20656197)(446.89103921,906.09545083)
\closepath
}
}
{
\newrgbcolor{curcolor}{0 0 0}
\pscustom[linestyle=none,fillstyle=solid,fillcolor=curcolor]
{
\newpath
\moveto(454.01343894,909.01211757)
\lineto(454.01343894,907.87930504)
\curveto(453.67489726,908.05291616)(453.32333475,908.1831245)(452.95875141,908.26993005)
\curveto(452.59416807,908.35673561)(452.21656389,908.40013839)(451.82593888,908.40013839)
\curveto(451.23132081,908.40013839)(450.78427219,908.30899256)(450.48479301,908.12670088)
\curveto(450.18965411,907.94440921)(450.04208467,907.67097171)(450.04208467,907.30638836)
\curveto(450.04208467,907.02861058)(450.14842147,906.80942654)(450.36109509,906.64883626)
\curveto(450.57376871,906.49258626)(451.00128608,906.34284667)(451.64364721,906.1996175)
\lineto(452.05380347,906.10847166)
\curveto(452.90449794,905.92617999)(453.50779656,905.66793346)(453.86369935,905.33373206)
\curveto(454.22394241,905.00387094)(454.40406395,904.54163135)(454.40406395,903.94701327)
\curveto(454.40406395,903.26992992)(454.13496672,902.7339056)(453.59677226,902.33894032)
\curveto(453.06291808,901.94397503)(452.32724098,901.74649238)(451.38974095,901.74649238)
\curveto(450.99911594,901.74649238)(450.59112982,901.78555488)(450.16578259,901.86367989)
\curveto(449.74477563,901.93746461)(449.29989715,902.05031184)(448.83114713,902.20222156)
\lineto(448.83114713,903.43920076)
\curveto(449.27385548,903.20916603)(449.71005341,903.03555492)(450.13974092,902.91836741)
\curveto(450.56942843,902.80552019)(450.99477566,902.74909658)(451.41578262,902.74909658)
\curveto(451.98001874,902.74909658)(452.41404653,902.84458269)(452.71786599,903.03555492)
\curveto(453.02168544,903.23086742)(453.17359516,903.50430493)(453.17359516,903.85586744)
\curveto(453.17359516,904.18138828)(453.06291808,904.43095426)(452.84156391,904.60456537)
\curveto(452.62455001,904.77817649)(452.1449493,904.94527719)(451.40276178,905.10586747)
\lineto(450.98609511,905.20352372)
\curveto(450.24390759,905.35977373)(449.70788327,905.59848901)(449.37802215,905.91966958)
\curveto(449.04816103,906.24519042)(448.88323047,906.6900689)(448.88323047,907.25430503)
\curveto(448.88323047,907.94006893)(449.12628603,908.46958284)(449.61239715,908.84284674)
\curveto(450.09850828,909.21611064)(450.78861246,909.40274258)(451.68270971,909.40274258)
\curveto(452.12541805,909.40274258)(452.54208473,909.3701905)(452.93270974,909.30508633)
\curveto(453.32333475,909.23998216)(453.68357782,909.14232591)(454.01343894,909.01211757)
\closepath
}
}
{
\newrgbcolor{curcolor}{0 0 0}
\pscustom[linestyle=none,fillstyle=solid,fillcolor=curcolor]
{
\newpath
\moveto(462.55510522,905.88060708)
\lineto(462.55510522,905.29466956)
\lineto(457.04729258,905.29466956)
\curveto(457.09937591,904.47001676)(457.34677175,903.84067647)(457.7894801,903.40664868)
\curveto(458.23652872,902.97696117)(458.85718846,902.76211741)(459.65145932,902.76211741)
\curveto(460.11152877,902.76211741)(460.55640726,902.81854102)(460.98609477,902.93138825)
\curveto(461.42012256,903.04423547)(461.84981007,903.21350631)(462.2751573,903.43920076)
\lineto(462.2751573,902.30638823)
\curveto(461.84546979,902.12409656)(461.40493158,901.98520767)(460.95354268,901.88972155)
\curveto(460.50215378,901.79423544)(460.04425446,901.74649238)(459.57984473,901.74649238)
\curveto(458.41665025,901.74649238)(457.4943412,902.08503406)(456.81291757,902.76211741)
\curveto(456.13583422,903.43920076)(455.79729255,904.3549994)(455.79729255,905.50951332)
\curveto(455.79729255,906.70308974)(456.11847311,907.64927032)(456.76083424,908.34805506)
\curveto(457.40753565,909.05118008)(458.27776136,909.40274258)(459.37151139,909.40274258)
\curveto(460.35241419,909.40274258)(461.1271538,909.0859023)(461.6957302,908.45222173)
\curveto(462.26864688,907.82288143)(462.55510522,906.96567655)(462.55510522,905.88060708)
\closepath
\moveto(461.35718853,906.23216958)
\curveto(461.34850797,906.88755155)(461.16404616,907.41055503)(460.8038031,907.80118004)
\curveto(460.44790031,908.19180505)(459.97481002,908.38711756)(459.38453222,908.38711756)
\curveto(458.71612943,908.38711756)(458.18010511,908.19831547)(457.77645927,907.82071129)
\curveto(457.3771537,907.44310712)(457.14711897,906.91142307)(457.08635508,906.22565917)
\closepath
}
}
{
\newrgbcolor{curcolor}{0 0 0}
\pscustom[linestyle=none,fillstyle=solid,fillcolor=curcolor]
{
\newpath
\moveto(469.16968593,909.01211757)
\lineto(469.16968593,907.87930504)
\curveto(468.83114425,908.05291616)(468.47958174,908.1831245)(468.1149984,908.26993005)
\curveto(467.75041506,908.35673561)(467.37281088,908.40013839)(466.98218587,908.40013839)
\curveto(466.3875678,908.40013839)(465.94051918,908.30899256)(465.64104,908.12670088)
\curveto(465.34590111,907.94440921)(465.19833166,907.67097171)(465.19833166,907.30638836)
\curveto(465.19833166,907.02861058)(465.30466847,906.80942654)(465.51734208,906.64883626)
\curveto(465.7300157,906.49258626)(466.15753307,906.34284667)(466.7998942,906.1996175)
\lineto(467.21005046,906.10847166)
\curveto(468.06074493,905.92617999)(468.66404356,905.66793346)(469.01994634,905.33373206)
\curveto(469.38018941,905.00387094)(469.56031094,904.54163135)(469.56031094,903.94701327)
\curveto(469.56031094,903.26992992)(469.29121371,902.7339056)(468.75301925,902.33894032)
\curveto(468.21916507,901.94397503)(467.48348797,901.74649238)(466.54598794,901.74649238)
\curveto(466.15536293,901.74649238)(465.74737681,901.78555488)(465.32202958,901.86367989)
\curveto(464.90102262,901.93746461)(464.45614414,902.05031184)(463.98739413,902.20222156)
\lineto(463.98739413,903.43920076)
\curveto(464.43010247,903.20916603)(464.8663004,903.03555492)(465.29598791,902.91836741)
\curveto(465.72567542,902.80552019)(466.15102266,902.74909658)(466.57202961,902.74909658)
\curveto(467.13626574,902.74909658)(467.57029353,902.84458269)(467.87411298,903.03555492)
\curveto(468.17793243,903.23086742)(468.32984216,903.50430493)(468.32984216,903.85586744)
\curveto(468.32984216,904.18138828)(468.21916507,904.43095426)(467.9978109,904.60456537)
\curveto(467.780797,904.77817649)(467.3011963,904.94527719)(466.55900878,905.10586747)
\lineto(466.1423421,905.20352372)
\curveto(465.40015458,905.35977373)(464.86413026,905.59848901)(464.53426914,905.91966958)
\curveto(464.20440802,906.24519042)(464.03947746,906.6900689)(464.03947746,907.25430503)
\curveto(464.03947746,907.94006893)(464.28253302,908.46958284)(464.76864415,908.84284674)
\curveto(465.25475527,909.21611064)(465.94485946,909.40274258)(466.8389567,909.40274258)
\curveto(467.28166505,909.40274258)(467.69833172,909.3701905)(468.08895673,909.30508633)
\curveto(468.47958174,909.23998216)(468.83982481,909.14232591)(469.16968593,909.01211757)
\closepath
}
}
{
\newrgbcolor{curcolor}{0 0 0}
\pscustom[linestyle=none,fillstyle=solid,fillcolor=curcolor]
{
\newpath
\moveto(471.88452915,900.69831527)
\lineto(472.34025833,900.69831527)
\curveto(472.94789723,900.69831527)(473.34286252,900.79163125)(473.52515419,900.9782632)
\curveto(473.71178614,901.16489515)(473.80510212,901.56854099)(473.80510212,902.18920073)
\lineto(473.80510212,903.8037841)
\curveto(473.80510212,904.48086745)(473.90275837,904.973489)(474.09807087,905.28164873)
\curveto(474.29338338,905.58980846)(474.63192505,905.80248207)(475.1136959,905.91966958)
\curveto(474.63192505,906.02817652)(474.29338338,906.23650986)(474.09807087,906.54466959)
\curveto(473.90275837,906.85282932)(473.80510212,907.347621)(473.80510212,908.02904463)
\lineto(473.80510212,909.64362801)
\curveto(473.80510212,910.25994747)(473.71178614,910.66142317)(473.52515419,910.84805512)
\curveto(473.34286252,911.03902735)(472.94789723,911.13451346)(472.34025833,911.13451346)
\lineto(471.88452915,911.13451346)
\lineto(471.88452915,912.06550307)
\lineto(472.29468541,912.06550307)
\curveto(473.3754146,912.06550307)(474.09590073,911.90491279)(474.4561438,911.58373223)
\curveto(474.82072714,911.26689194)(475.00301881,910.62887109)(475.00301881,909.66966967)
\lineto(475.00301881,908.10716963)
\curveto(475.00301881,907.46046823)(475.12020632,907.01124947)(475.35458132,906.75951335)
\curveto(475.58895633,906.51211751)(476.01430356,906.38841959)(476.63062302,906.38841959)
\lineto(477.03426887,906.38841959)
\lineto(477.03426887,905.45742998)
\lineto(476.63062302,905.45742998)
\curveto(476.01430356,905.45742998)(475.58895633,905.33156192)(475.35458132,905.0798258)
\curveto(475.12020632,904.82808969)(475.00301881,904.37453065)(475.00301881,903.71914868)
\lineto(475.00301881,902.16315906)
\curveto(475.00301881,901.20395765)(474.82072714,900.56376666)(474.4561438,900.24258609)
\curveto(474.09590073,899.92140553)(473.3754146,899.76081525)(472.29468541,899.76081525)
\lineto(471.88452915,899.76081525)
\closepath
}
}
{
\newrgbcolor{curcolor}{0 0 0}
\pscustom[linestyle=none,fillstyle=solid,fillcolor=curcolor]
{
\newpath
\moveto(483.21266118,899.975659)
\lineto(482.56161949,899.975659)
\lineto(482.55510908,901.93529447)
\curveto(482.0993799,901.94397503)(481.64365072,901.99605836)(481.18792154,902.09154448)
\curveto(480.73219236,902.19137087)(480.27429305,902.33894032)(479.81422359,902.53425282)
\lineto(479.81422359,903.70612785)
\curveto(480.25693193,903.42835007)(480.70398056,903.21784659)(481.15536946,903.07461742)
\curveto(481.61109864,902.93572853)(482.07984865,902.86411394)(482.56161949,902.85977366)
\lineto(482.56161949,905.82852374)
\curveto(481.60241808,905.98477374)(480.90363334,906.2495307)(480.46526527,906.62279459)
\curveto(480.03123748,906.99605849)(479.81422359,907.50821128)(479.81422359,908.15925297)
\curveto(479.81422359,908.86671826)(480.05076873,909.42444397)(480.52385902,909.8324301)
\curveto(480.99694931,910.24041622)(481.6762028,910.47479122)(482.56161949,910.53555511)
\lineto(482.56161949,912.06550307)
\lineto(483.21266118,912.06550307)
\lineto(483.21266118,910.55508636)
\curveto(483.61630702,910.53772525)(484.00693203,910.49432247)(484.38453621,910.42487803)
\curveto(484.76214039,910.35977386)(485.13106401,910.26862802)(485.49130707,910.15144052)
\lineto(485.49130707,909.01211757)
\curveto(485.13106401,909.19440925)(484.75997025,909.33546828)(484.37802579,909.43529467)
\curveto(484.00042162,909.53512106)(483.61196674,909.59371481)(483.21266118,909.61107592)
\lineto(483.21266118,906.83112793)
\curveto(484.19790426,906.67921821)(484.92273067,906.40795084)(485.3871404,906.01732583)
\curveto(485.85155014,905.62670082)(486.083755,905.09284664)(486.083755,904.41576329)
\curveto(486.083755,903.68225632)(485.83635916,903.10282922)(485.34156748,902.67748199)
\curveto(484.85111608,902.25647504)(484.14148065,902.01341947)(483.21266118,901.9483153)
\closepath
\moveto(482.56161949,906.94831544)
\lineto(482.56161949,909.61758634)
\curveto(482.05814726,909.56116273)(481.67403267,909.41793356)(481.40927571,909.18789883)
\curveto(481.14451876,908.9578641)(481.01214029,908.65187451)(481.01214029,908.26993005)
\curveto(481.01214029,907.89666616)(481.13366807,907.60586754)(481.37672363,907.3975342)
\curveto(481.62411947,907.18920086)(482.01908476,907.03946127)(482.56161949,906.94831544)
\closepath
\moveto(483.21266118,905.6983154)
\lineto(483.21266118,902.87930491)
\curveto(483.76387647,902.95308964)(484.17837301,903.10933964)(484.45615079,903.34805493)
\curveto(484.73826886,903.58677021)(484.87932789,903.90144036)(484.87932789,904.29206537)
\curveto(484.87932789,904.67400982)(484.74477927,904.97782927)(484.47568204,905.20352372)
\curveto(484.21092509,905.42921817)(483.78991814,905.59414873)(483.21266118,905.6983154)
\closepath
}
}
{
\newrgbcolor{curcolor}{0 0 0}
\pscustom[linewidth=1.00157475,linecolor=curcolor]
{
\newpath
\moveto(213.57143055,992.85714898)
\lineto(350.00000126,992.85714898)
}
}
{
\newrgbcolor{curcolor}{0 0 0}
\pscustom[linestyle=none,fillstyle=solid,fillcolor=curcolor]
{
\newpath
\moveto(339.98425377,992.85714898)
\lineto(335.97795477,988.85084998)
\lineto(350.00000126,992.85714898)
\lineto(335.97795477,996.86344797)
\closepath
}
}
{
\newrgbcolor{curcolor}{0 0 0}
\pscustom[linewidth=1.06834643,linecolor=curcolor]
{
\newpath
\moveto(339.98425377,992.85714898)
\lineto(335.97795477,988.85084998)
\lineto(350.00000126,992.85714898)
\lineto(335.97795477,996.86344797)
\closepath
}
}
{
\newrgbcolor{curcolor}{0 0 0}
\pscustom[linewidth=1.00157475,linecolor=curcolor]
{
\newpath
\moveto(308.67065197,782.14285984)
\lineto(445.0992,782.14285984)
}
}
{
\newrgbcolor{curcolor}{0 0 0}
\pscustom[linestyle=none,fillstyle=solid,fillcolor=curcolor]
{
\newpath
\moveto(435.08345251,782.14285984)
\lineto(431.07715351,778.13656085)
\lineto(445.0992,782.14285984)
\lineto(431.07715351,786.14915884)
\closepath
}
}
{
\newrgbcolor{curcolor}{0 0 0}
\pscustom[linewidth=1.06834643,linecolor=curcolor]
{
\newpath
\moveto(435.08345251,782.14285984)
\lineto(431.07715351,778.13656085)
\lineto(445.0992,782.14285984)
\lineto(431.07715351,786.14915884)
\closepath
}
}
{
\newrgbcolor{curcolor}{0 0 0}
\pscustom[linewidth=0.99999995,linecolor=curcolor]
{
\newpath
\moveto(677.14284094,948.57142677)
\lineto(677.14284094,900.71428913)
\lineto(40.71428409,900.71428913)
\lineto(40.71428409,783.57143055)
\lineto(82.14285732,783.57143055)
}
}
{
\newrgbcolor{curcolor}{0 0 0}
\pscustom[linestyle=none,fillstyle=solid,fillcolor=curcolor]
{
\newpath
\moveto(72.14285785,783.57143055)
\lineto(68.14285806,779.57143076)
\lineto(82.14285732,783.57143055)
\lineto(68.14285806,787.57143034)
\closepath
}
}
{
\newrgbcolor{curcolor}{0 0 0}
\pscustom[linewidth=1.06666664,linecolor=curcolor]
{
\newpath
\moveto(72.14285785,783.57143055)
\lineto(68.14285806,779.57143076)
\lineto(82.14285732,783.57143055)
\lineto(68.14285806,787.57143034)
\closepath
}
}
{
\newrgbcolor{curcolor}{0 0 0}
\pscustom[linestyle=none,fillstyle=solid,fillcolor=curcolor]
{
\newpath
\moveto(221.15801428,1015.78103325)
\lineto(222.47962882,1015.78103325)
\lineto(222.47962882,1009.87608555)
\curveto(222.47962882,1008.83441892)(222.66843089,1008.08355089)(223.04603505,1007.62348146)
\curveto(223.4236392,1007.16775231)(224.03561834,1006.93988774)(224.88197248,1006.93988774)
\curveto(225.72398634,1006.93988774)(226.33379534,1007.16775231)(226.7113995,1007.62348146)
\curveto(227.08900365,1008.08355089)(227.27780573,1008.83441892)(227.27780573,1009.87608555)
\lineto(227.27780573,1015.78103325)
\lineto(228.59942026,1015.78103325)
\lineto(228.59942026,1009.71332513)
\curveto(228.59942026,1008.44596407)(228.28475013,1007.48893285)(227.65540988,1006.84223149)
\curveto(227.0304099,1006.19553012)(226.10593077,1005.87217944)(224.88197248,1005.87217944)
\curveto(223.65367391,1005.87217944)(222.7248545,1006.19553012)(222.09551425,1006.84223149)
\curveto(221.47051427,1007.48893285)(221.15801428,1008.44596407)(221.15801428,1009.71332513)
\closepath
}
}
{
\newrgbcolor{curcolor}{0 0 0}
\pscustom[linestyle=none,fillstyle=solid,fillcolor=curcolor]
{
\newpath
\moveto(237.0824931,1010.46202302)
\lineto(237.0824931,1006.06098152)
\lineto(235.88457648,1006.06098152)
\lineto(235.88457648,1010.42296053)
\curveto(235.88457648,1011.11306467)(235.75002787,1011.6295577)(235.48093066,1011.97243964)
\curveto(235.21183345,1012.31532157)(234.80818763,1012.48676253)(234.2699932,1012.48676253)
\curveto(233.62329184,1012.48676253)(233.11330922,1012.28059935)(232.74004534,1011.86827297)
\curveto(232.36678147,1011.4559466)(232.18014953,1010.89388065)(232.18014953,1010.18207512)
\lineto(232.18014953,1006.06098152)
\lineto(230.97572249,1006.06098152)
\lineto(230.97572249,1013.35264792)
\lineto(232.18014953,1013.35264792)
\lineto(232.18014953,1012.21983546)
\curveto(232.46660785,1012.6582035)(232.80297937,1012.98589446)(233.18926408,1013.20290834)
\curveto(233.57988906,1013.41992222)(234.0291078,1013.52842916)(234.53692028,1013.52842916)
\curveto(235.37459386,1013.52842916)(236.00827439,1013.2680125)(236.43796188,1012.74717919)
\curveto(236.86764936,1012.23068615)(237.0824931,1011.46896743)(237.0824931,1010.46202302)
\closepath
}
}
{
\newrgbcolor{curcolor}{0 0 0}
\pscustom[linestyle=none,fillstyle=solid,fillcolor=curcolor]
{
\newpath
\moveto(243.71009659,1012.23285629)
\curveto(243.57554799,1012.31098129)(243.42797855,1012.3674049)(243.26738828,1012.40212712)
\curveto(243.11113828,1012.44118962)(242.93752718,1012.46072087)(242.74655496,1012.46072087)
\curveto(242.06947165,1012.46072087)(241.54863834,1012.23936671)(241.18405502,1011.79665839)
\curveto(240.82381198,1011.35829035)(240.64369046,1010.72677996)(240.64369046,1009.90212721)
\lineto(240.64369046,1006.06098152)
\lineto(239.43926342,1006.06098152)
\lineto(239.43926342,1013.35264792)
\lineto(240.64369046,1013.35264792)
\lineto(240.64369046,1012.21983546)
\curveto(240.89542656,1012.66254378)(241.22311752,1012.99023474)(241.62676334,1013.20290834)
\curveto(242.03040915,1013.41992222)(242.52086053,1013.52842916)(243.09811745,1013.52842916)
\curveto(243.18058272,1013.52842916)(243.27172855,1013.52191875)(243.37155494,1013.50889791)
\curveto(243.47138132,1013.50021736)(243.5820584,1013.48502639)(243.70358618,1013.463325)
\closepath
}
}
{
\newrgbcolor{curcolor}{0 0 0}
\pscustom[linestyle=none,fillstyle=solid,fillcolor=curcolor]
{
\newpath
\moveto(247.51869096,1012.5128042)
\curveto(246.87632988,1012.5128042)(246.3685174,1012.2610681)(245.99525352,1011.75759589)
\curveto(245.62198964,1011.25846397)(245.43535771,1010.5727001)(245.43535771,1009.7003043)
\curveto(245.43535771,1008.8279085)(245.61981951,1008.1399745)(245.9887431,1007.63650229)
\curveto(246.36200698,1007.13737037)(246.8719896,1006.8878044)(247.51869096,1006.8878044)
\curveto(248.15671177,1006.8878044)(248.66235412,1007.13954051)(249.03561799,1007.64301271)
\curveto(249.40888187,1008.14648491)(249.59551381,1008.83224878)(249.59551381,1009.7003043)
\curveto(249.59551381,1010.56401955)(249.40888187,1011.24761327)(249.03561799,1011.75108548)
\curveto(248.66235412,1012.25889796)(248.15671177,1012.5128042)(247.51869096,1012.5128042)
\closepath
\moveto(247.51869096,1013.52842916)
\curveto(248.56035759,1013.52842916)(249.37849992,1013.18988751)(249.97311796,1012.5128042)
\curveto(250.56773599,1011.83572089)(250.86504501,1010.89822092)(250.86504501,1009.7003043)
\curveto(250.86504501,1008.50672796)(250.56773599,1007.56922799)(249.97311796,1006.8878044)
\curveto(249.37849992,1006.2107211)(248.56035759,1005.87217944)(247.51869096,1005.87217944)
\curveto(246.47268406,1005.87217944)(245.65237159,1006.2107211)(245.05775355,1006.8878044)
\curveto(244.4674758,1007.56922799)(244.17233692,1008.50672796)(244.17233692,1009.7003043)
\curveto(244.17233692,1010.89822092)(244.4674758,1011.83572089)(245.05775355,1012.5128042)
\curveto(245.65237159,1013.18988751)(246.47268406,1013.52842916)(247.51869096,1013.52842916)
\closepath
}
}
{
\newrgbcolor{curcolor}{0 0 0}
\pscustom[linestyle=none,fillstyle=solid,fillcolor=curcolor]
{
\newpath
\moveto(252.84420926,1016.19118948)
\lineto(254.04212588,1016.19118948)
\lineto(254.04212588,1006.06098152)
\lineto(252.84420926,1006.06098152)
\closepath
}
}
{
\newrgbcolor{curcolor}{0 0 0}
\pscustom[linestyle=none,fillstyle=solid,fillcolor=curcolor]
{
\newpath
\moveto(256.54212715,1016.19118948)
\lineto(257.74004377,1016.19118948)
\lineto(257.74004377,1006.06098152)
\lineto(256.54212715,1006.06098152)
\closepath
}
}
{
\newrgbcolor{curcolor}{0 0 0}
\pscustom[linestyle=none,fillstyle=solid,fillcolor=curcolor]
{
\newpath
\moveto(264.53691744,1015.78103325)
\lineto(270.12285474,1015.78103325)
\lineto(270.12285474,1014.67426245)
\lineto(265.85202156,1014.67426245)
\lineto(265.85202156,1011.80967922)
\lineto(269.70618808,1011.80967922)
\lineto(269.70618808,1010.70290843)
\lineto(265.85202156,1010.70290843)
\lineto(265.85202156,1006.06098152)
\lineto(264.53691744,1006.06098152)
\closepath
}
}
{
\newrgbcolor{curcolor}{0 0 0}
\pscustom[linestyle=none,fillstyle=solid,fillcolor=curcolor]
{
\newpath
\moveto(273.5343144,1014.70030412)
\lineto(273.5343144,1011.0479605)
\lineto(275.18796017,1011.0479605)
\curveto(275.79993932,1011.0479605)(276.27302958,1011.20638064)(276.60723095,1011.5232209)
\curveto(276.94143233,1011.84006117)(277.10853302,1012.29145004)(277.10853302,1012.87738752)
\curveto(277.10853302,1013.45898472)(276.94143233,1013.90820345)(276.60723095,1014.22504372)
\curveto(276.27302958,1014.54188399)(275.79993932,1014.70030412)(275.18796017,1014.70030412)
\closepath
\moveto(272.21921028,1015.78103325)
\lineto(275.18796017,1015.78103325)
\curveto(276.27736986,1015.78103325)(277.09985246,1015.53363742)(277.655408,1015.03884577)
\curveto(278.21530381,1014.5483944)(278.49525172,1013.82790832)(278.49525172,1012.87738752)
\curveto(278.49525172,1011.91818617)(278.21530381,1011.1933598)(277.655408,1010.70290843)
\curveto(277.09985246,1010.21245706)(276.27736986,1009.96723138)(275.18796017,1009.96723138)
\lineto(273.5343144,1009.96723138)
\lineto(273.5343144,1006.06098152)
\lineto(272.21921028,1006.06098152)
\closepath
}
}
{
\newrgbcolor{curcolor}{0 0 0}
\pscustom[linestyle=none,fillstyle=solid,fillcolor=curcolor]
{
\newpath
\moveto(288.45618661,1014.8891062)
\curveto(287.50132553,1014.8891062)(286.74177695,1014.53320343)(286.17754086,1013.8213979)
\curveto(285.61764504,1013.10959237)(285.33769714,1012.13954032)(285.33769714,1010.91124176)
\curveto(285.33769714,1009.68728347)(285.61764504,1008.71940156)(286.17754086,1008.00759603)
\curveto(286.74177695,1007.2957905)(287.50132553,1006.93988774)(288.45618661,1006.93988774)
\curveto(289.41104768,1006.93988774)(290.16625599,1007.2957905)(290.72181152,1008.00759603)
\curveto(291.28170734,1008.71940156)(291.56165524,1009.68728347)(291.56165524,1010.91124176)
\curveto(291.56165524,1012.13954032)(291.28170734,1013.10959237)(290.72181152,1013.8213979)
\curveto(290.16625599,1014.53320343)(289.41104768,1014.8891062)(288.45618661,1014.8891062)
\closepath
\moveto(288.45618661,1015.95681449)
\curveto(289.81903378,1015.95681449)(290.90844346,1015.4989152)(291.72441565,1014.58311662)
\curveto(292.54038785,1013.67165832)(292.94837394,1012.44770003)(292.94837394,1010.91124176)
\curveto(292.94837394,1009.37912376)(292.54038785,1008.15516547)(291.72441565,1007.23936689)
\curveto(290.90844346,1006.32790859)(289.81903378,1005.87217944)(288.45618661,1005.87217944)
\curveto(287.08899916,1005.87217944)(285.9952492,1006.32790859)(285.17493673,1007.23936689)
\curveto(284.35896453,1008.15082519)(283.95097844,1009.37478348)(283.95097844,1010.91124176)
\curveto(283.95097844,1012.44770003)(284.35896453,1013.67165832)(285.17493673,1014.58311662)
\curveto(285.9952492,1015.4989152)(287.08899916,1015.95681449)(288.45618661,1015.95681449)
\closepath
}
}
{
\newrgbcolor{curcolor}{0 0 0}
\pscustom[linestyle=none,fillstyle=solid,fillcolor=curcolor]
{
\newpath
\moveto(296.11243785,1007.15473148)
\lineto(296.11243785,1003.28754412)
\lineto(294.90801081,1003.28754412)
\lineto(294.90801081,1013.35264792)
\lineto(296.11243785,1013.35264792)
\lineto(296.11243785,1012.24587713)
\curveto(296.36417395,1012.67990489)(296.68101422,1013.00108543)(297.06295865,1013.20941876)
\curveto(297.44924336,1013.42209236)(297.90931278,1013.52842916)(298.44316693,1013.52842916)
\curveto(299.32858357,1013.52842916)(300.04689951,1013.17686667)(300.59811477,1012.4737417)
\curveto(301.1536703,1011.77061673)(301.43144807,1010.84613759)(301.43144807,1009.7003043)
\curveto(301.43144807,1008.55447101)(301.1536703,1007.62999188)(300.59811477,1006.9268669)
\curveto(300.04689951,1006.22374193)(299.32858357,1005.87217944)(298.44316693,1005.87217944)
\curveto(297.90931278,1005.87217944)(297.44924336,1005.9763461)(297.06295865,1006.18467943)
\curveto(296.68101422,1006.39735303)(296.36417395,1006.72070372)(296.11243785,1007.15473148)
\closepath
\moveto(300.18795853,1009.7003043)
\curveto(300.18795853,1010.58138066)(300.00566687,1011.2714848)(299.64108355,1011.77061673)
\curveto(299.28084051,1012.27408893)(298.78387872,1012.52582503)(298.15019819,1012.52582503)
\curveto(297.51651766,1012.52582503)(297.01738573,1012.27408893)(296.65280241,1011.77061673)
\curveto(296.29255937,1011.2714848)(296.11243785,1010.58138066)(296.11243785,1009.7003043)
\curveto(296.11243785,1008.81922794)(296.29255937,1008.12695366)(296.65280241,1007.62348146)
\curveto(297.01738573,1007.12434953)(297.51651766,1006.87478357)(298.15019819,1006.87478357)
\curveto(298.78387872,1006.87478357)(299.28084051,1007.12434953)(299.64108355,1007.62348146)
\curveto(300.00566687,1008.12695366)(300.18795853,1008.81922794)(300.18795853,1009.7003043)
\closepath
}
}
{
\newrgbcolor{curcolor}{0 0 0}
\pscustom[linestyle=none,fillstyle=solid,fillcolor=curcolor]
{
\newpath
\moveto(309.65410168,1010.00629387)
\lineto(309.65410168,1009.4203564)
\lineto(304.14628938,1009.4203564)
\curveto(304.19837271,1008.59570365)(304.44576854,1007.96636339)(304.88847685,1007.53233563)
\curveto(305.33552545,1007.10264815)(305.95618515,1006.8878044)(306.75045595,1006.8878044)
\curveto(307.21052538,1006.8878044)(307.65540384,1006.94422801)(308.08509132,1007.05707523)
\curveto(308.51911908,1007.16992245)(308.94880657,1007.33919328)(309.37415377,1007.56488771)
\lineto(309.37415377,1006.43207525)
\curveto(308.94446629,1006.24978359)(308.50392811,1006.11089471)(308.05253924,1006.0154086)
\curveto(307.60115037,1005.91992249)(307.14325108,1005.87217944)(306.67884137,1005.87217944)
\curveto(305.51564697,1005.87217944)(304.59333798,1006.2107211)(303.91191439,1006.8878044)
\curveto(303.23483108,1007.56488771)(302.89628943,1008.48068629)(302.89628943,1009.63520014)
\curveto(302.89628943,1010.82877648)(303.21746997,1011.774957)(303.85983106,1012.4737417)
\curveto(304.50653242,1013.17686667)(305.37675809,1013.52842916)(306.47050805,1013.52842916)
\curveto(307.45141079,1013.52842916)(308.22615034,1013.2115889)(308.79472671,1012.57790836)
\curveto(309.36764336,1011.94856811)(309.65410168,1011.09136328)(309.65410168,1010.00629387)
\closepath
\moveto(308.45618506,1010.35785636)
\curveto(308.4475045,1011.01323828)(308.2630427,1011.53624173)(307.90279966,1011.92686672)
\curveto(307.5468969,1012.31749171)(307.07380664,1012.5128042)(306.48352888,1012.5128042)
\curveto(305.81512613,1012.5128042)(305.27910184,1012.32400212)(304.87545602,1011.94639797)
\curveto(304.47615048,1011.56879382)(304.24611577,1011.03710981)(304.18535188,1010.35134594)
\closepath
}
}
{
\newrgbcolor{curcolor}{0 0 0}
\pscustom[linestyle=none,fillstyle=solid,fillcolor=curcolor]
{
\newpath
\moveto(315.8455079,1012.23285629)
\curveto(315.71095929,1012.31098129)(315.56338986,1012.3674049)(315.40279958,1012.40212712)
\curveto(315.24654959,1012.44118962)(315.07293848,1012.46072087)(314.88196627,1012.46072087)
\curveto(314.20488296,1012.46072087)(313.68404965,1012.23936671)(313.31946633,1011.79665839)
\curveto(312.95922328,1011.35829035)(312.77910176,1010.72677996)(312.77910176,1009.90212721)
\lineto(312.77910176,1006.06098152)
\lineto(311.57467472,1006.06098152)
\lineto(311.57467472,1013.35264792)
\lineto(312.77910176,1013.35264792)
\lineto(312.77910176,1012.21983546)
\curveto(313.03083786,1012.66254378)(313.35852882,1012.99023474)(313.76217464,1013.20290834)
\curveto(314.16582046,1013.41992222)(314.65627183,1013.52842916)(315.23352876,1013.52842916)
\curveto(315.31599403,1013.52842916)(315.40713986,1013.52191875)(315.50696625,1013.50889791)
\curveto(315.60679263,1013.50021736)(315.71746971,1013.48502639)(315.83899748,1013.463325)
\closepath
}
}
{
\newrgbcolor{curcolor}{0 0 0}
\pscustom[linestyle=none,fillstyle=solid,fillcolor=curcolor]
{
\newpath
\moveto(320.42884628,1009.72634597)
\curveto(319.46096438,1009.72634597)(318.79039148,1009.61566889)(318.41712761,1009.39431473)
\curveto(318.04386373,1009.17296057)(317.85723179,1008.79535642)(317.85723179,1008.26150227)
\curveto(317.85723179,1007.83615506)(317.99612068,1007.49761341)(318.27389845,1007.24587731)
\curveto(318.55601649,1006.99848148)(318.93796092,1006.87478357)(319.41973174,1006.87478357)
\curveto(320.08379421,1006.87478357)(320.61547822,1007.10915856)(321.01478376,1007.57790855)
\curveto(321.41842958,1008.05099881)(321.62025249,1008.67816892)(321.62025249,1009.45941889)
\lineto(321.62025249,1009.72634597)
\closepath
\moveto(322.81816911,1010.22113762)
\lineto(322.81816911,1006.06098152)
\lineto(321.62025249,1006.06098152)
\lineto(321.62025249,1007.16775231)
\curveto(321.346815,1006.72504399)(321.00610321,1006.39735303)(320.59811711,1006.18467943)
\curveto(320.19013102,1005.9763461)(319.69099909,1005.87217944)(319.10072133,1005.87217944)
\curveto(318.35419358,1005.87217944)(317.75957555,1006.08051277)(317.31686723,1006.49717942)
\curveto(316.87849919,1006.91818635)(316.65931517,1007.4802523)(316.65931517,1008.18337727)
\curveto(316.65931517,1009.00368974)(316.93275266,1009.6221793)(317.47962764,1010.03884596)
\curveto(318.0308429,1010.45551261)(318.85115537,1010.66384593)(319.94056505,1010.66384593)
\lineto(321.62025249,1010.66384593)
\lineto(321.62025249,1010.78103343)
\curveto(321.62025249,1011.33224869)(321.43796083,1011.75759589)(321.07337751,1012.05707505)
\curveto(320.71313447,1012.36089448)(320.20532199,1012.5128042)(319.54994007,1012.5128042)
\curveto(319.13327341,1012.5128042)(318.72745746,1012.46289101)(318.33249219,1012.36306462)
\curveto(317.93752693,1012.26323824)(317.55775264,1012.11349866)(317.19316932,1011.91384589)
\lineto(317.19316932,1013.02061668)
\curveto(317.63153736,1013.18988751)(318.05688457,1013.31575556)(318.46921094,1013.39822083)
\curveto(318.88153731,1013.48502639)(319.28301299,1013.52842916)(319.67363798,1013.52842916)
\curveto(320.72832544,1013.52842916)(321.51608583,1013.25499167)(322.03691914,1012.70811669)
\curveto(322.55775246,1012.16124171)(322.81816911,1011.33224869)(322.81816911,1010.22113762)
\closepath
}
}
{
\newrgbcolor{curcolor}{0 0 0}
\pscustom[linestyle=none,fillstyle=solid,fillcolor=curcolor]
{
\newpath
\moveto(326.47701826,1015.42296034)
\lineto(326.47701826,1013.35264792)
\lineto(328.94446608,1013.35264792)
\lineto(328.94446608,1012.42165837)
\lineto(326.47701826,1012.42165837)
\lineto(326.47701826,1008.46332518)
\curveto(326.47701826,1007.86870715)(326.55731339,1007.48676272)(326.71790366,1007.31749189)
\curveto(326.88283421,1007.14822106)(327.21486545,1007.06358565)(327.71399738,1007.06358565)
\lineto(328.94446608,1007.06358565)
\lineto(328.94446608,1006.06098152)
\lineto(327.71399738,1006.06098152)
\curveto(326.78951824,1006.06098152)(326.15149743,1006.23242248)(325.79993495,1006.57530442)
\curveto(325.44837246,1006.92252662)(325.27259122,1007.55186688)(325.27259122,1008.46332518)
\lineto(325.27259122,1012.42165837)
\lineto(324.393685,1012.42165837)
\lineto(324.393685,1013.35264792)
\lineto(325.27259122,1013.35264792)
\lineto(325.27259122,1015.42296034)
\closepath
}
}
{
\newrgbcolor{curcolor}{0 0 0}
\pscustom[linestyle=none,fillstyle=solid,fillcolor=curcolor]
{
\newpath
\moveto(333.35201813,1012.5128042)
\curveto(332.70965704,1012.5128042)(332.20184456,1012.2610681)(331.82858069,1011.75759589)
\curveto(331.45531681,1011.25846397)(331.26868487,1010.5727001)(331.26868487,1009.7003043)
\curveto(331.26868487,1008.8279085)(331.45314667,1008.1399745)(331.82207027,1007.63650229)
\curveto(332.19533414,1007.13737037)(332.70531676,1006.8878044)(333.35201813,1006.8878044)
\curveto(333.99003894,1006.8878044)(334.49568128,1007.13954051)(334.86894516,1007.64301271)
\curveto(335.24220903,1008.14648491)(335.42884097,1008.83224878)(335.42884097,1009.7003043)
\curveto(335.42884097,1010.56401955)(335.24220903,1011.24761327)(334.86894516,1011.75108548)
\curveto(334.49568128,1012.25889796)(333.99003894,1012.5128042)(333.35201813,1012.5128042)
\closepath
\moveto(333.35201813,1013.52842916)
\curveto(334.39368476,1013.52842916)(335.21182709,1013.18988751)(335.80644512,1012.5128042)
\curveto(336.40106316,1011.83572089)(336.69837217,1010.89822092)(336.69837217,1009.7003043)
\curveto(336.69837217,1008.50672796)(336.40106316,1007.56922799)(335.80644512,1006.8878044)
\curveto(335.21182709,1006.2107211)(334.39368476,1005.87217944)(333.35201813,1005.87217944)
\curveto(332.30601122,1005.87217944)(331.48569875,1006.2107211)(330.89108072,1006.8878044)
\curveto(330.30080296,1007.56922799)(330.00566409,1008.50672796)(330.00566409,1009.7003043)
\curveto(330.00566409,1010.89822092)(330.30080296,1011.83572089)(330.89108072,1012.5128042)
\curveto(331.48569875,1013.18988751)(332.30601122,1013.52842916)(333.35201813,1013.52842916)
\closepath
}
}
{
\newrgbcolor{curcolor}{0 0 0}
\pscustom[linestyle=none,fillstyle=solid,fillcolor=curcolor]
{
\newpath
\moveto(342.90279668,1012.23285629)
\curveto(342.76824808,1012.31098129)(342.62067864,1012.3674049)(342.46008837,1012.40212712)
\curveto(342.30383837,1012.44118962)(342.13022727,1012.46072087)(341.93925505,1012.46072087)
\curveto(341.26217174,1012.46072087)(340.74133843,1012.23936671)(340.37675511,1011.79665839)
\curveto(340.01651207,1011.35829035)(339.83639055,1010.72677996)(339.83639055,1009.90212721)
\lineto(339.83639055,1006.06098152)
\lineto(338.63196351,1006.06098152)
\lineto(338.63196351,1013.35264792)
\lineto(339.83639055,1013.35264792)
\lineto(339.83639055,1012.21983546)
\curveto(340.08812665,1012.66254378)(340.41581761,1012.99023474)(340.81946343,1013.20290834)
\curveto(341.22310924,1013.41992222)(341.71356062,1013.52842916)(342.29081754,1013.52842916)
\curveto(342.37328281,1013.52842916)(342.46442864,1013.52191875)(342.56425503,1013.50889791)
\curveto(342.66408141,1013.50021736)(342.77475849,1013.48502639)(342.89628627,1013.463325)
\closepath
}
}
{
\newrgbcolor{curcolor}{0 0 0}
\pscustom[linestyle=none,fillstyle=solid,fillcolor=curcolor]
{
\newpath
\moveto(229.27734878,981.51785477)
\lineto(230.5924529,981.51785477)
\lineto(230.5924529,972.90457384)
\lineto(235.32552565,972.90457384)
\lineto(235.32552565,971.79780305)
\lineto(229.27734878,971.79780305)
\closepath
}
}
{
\newrgbcolor{curcolor}{0 0 0}
\pscustom[linestyle=none,fillstyle=solid,fillcolor=curcolor]
{
\newpath
\moveto(242.6497443,975.7431154)
\lineto(242.6497443,975.15717792)
\lineto(237.141932,975.15717792)
\curveto(237.19401533,974.33252518)(237.44141116,973.70318492)(237.88411947,973.26915716)
\curveto(238.33116807,972.83946967)(238.95182777,972.62462593)(239.74609857,972.62462593)
\curveto(240.206168,972.62462593)(240.65104646,972.68104954)(241.08073394,972.79389676)
\curveto(241.5147617,972.90674398)(241.94444919,973.0760148)(242.36979639,973.30170924)
\lineto(242.36979639,972.16889678)
\curveto(241.94010891,971.98660512)(241.49957073,971.84771624)(241.04818186,971.75223013)
\curveto(240.59679299,971.65674402)(240.1388937,971.60900097)(239.67448399,971.60900097)
\curveto(238.51128959,971.60900097)(237.5889806,971.94754262)(236.90755701,972.62462593)
\curveto(236.2304737,973.30170924)(235.89193205,974.21750782)(235.89193205,975.37202167)
\curveto(235.89193205,976.56559801)(236.21311259,977.51177853)(236.85547368,978.21056323)
\curveto(237.50217504,978.9136882)(238.37240071,979.26525069)(239.46615067,979.26525069)
\curveto(240.44705341,979.26525069)(241.22179296,978.94841042)(241.79036933,978.31472989)
\curveto(242.36328598,977.68538964)(242.6497443,976.82818481)(242.6497443,975.7431154)
\closepath
\moveto(241.45182768,976.09467789)
\curveto(241.44314712,976.75005981)(241.25868532,977.27306326)(240.89844228,977.66368825)
\curveto(240.54253952,978.05431323)(240.06944926,978.24962573)(239.4791715,978.24962573)
\curveto(238.81076875,978.24962573)(238.27474446,978.06082365)(237.87109864,977.6832195)
\curveto(237.4717931,977.30561534)(237.24175839,976.77393134)(237.1809945,976.08816747)
\closepath
}
}
{
\newrgbcolor{curcolor}{0 0 0}
\pscustom[linestyle=none,fillstyle=solid,fillcolor=curcolor]
{
\newpath
\moveto(250.29297338,977.68972991)
\curveto(250.59245254,978.22792434)(250.95052544,978.62505974)(251.3671921,978.88113612)
\curveto(251.78385875,979.1372125)(252.27431012,979.26525069)(252.83854621,979.26525069)
\curveto(253.59809479,979.26525069)(254.18403227,978.99832362)(254.59635864,978.46446947)
\curveto(255.00868502,977.9349556)(255.21484821,977.17974729)(255.21484821,976.19884455)
\lineto(255.21484821,971.79780305)
\lineto(254.01042117,971.79780305)
\lineto(254.01042117,976.15978205)
\curveto(254.01042117,976.85856675)(253.88672325,977.37722993)(253.63932743,977.71577158)
\curveto(253.39193161,978.05431323)(253.01432745,978.22358406)(252.50651497,978.22358406)
\curveto(251.88585527,978.22358406)(251.3954039,978.01742087)(251.03516086,977.6050945)
\curveto(250.67491782,977.19276813)(250.49479629,976.63070217)(250.49479629,975.91889665)
\lineto(250.49479629,971.79780305)
\lineto(249.29036925,971.79780305)
\lineto(249.29036925,976.15978205)
\curveto(249.29036925,976.86290703)(249.16667134,977.3815702)(248.91927552,977.71577158)
\curveto(248.67187969,978.05431323)(248.28993526,978.22358406)(247.77344223,978.22358406)
\curveto(247.16146308,978.22358406)(246.67535199,978.01525074)(246.31510895,977.59858408)
\curveto(245.9548659,977.18625771)(245.77474438,976.6263619)(245.77474438,975.91889665)
\lineto(245.77474438,971.79780305)
\lineto(244.57031734,971.79780305)
\lineto(244.57031734,979.08946945)
\lineto(245.77474438,979.08946945)
\lineto(245.77474438,977.95665699)
\curveto(246.04818187,978.40370558)(246.37587283,978.73356668)(246.75781726,978.94624028)
\curveto(247.13976169,979.15891389)(247.59332071,979.26525069)(248.1184943,979.26525069)
\curveto(248.64800817,979.26525069)(249.0972269,979.13070208)(249.4661505,978.86160487)
\curveto(249.83941437,978.59250766)(250.115022,978.20188267)(250.29297338,977.68972991)
\closepath
}
}
{
\newrgbcolor{curcolor}{0 0 0}
\pscustom[linestyle=none,fillstyle=solid,fillcolor=curcolor]
{
\newpath
\moveto(263.28776425,977.68972991)
\curveto(263.58724341,978.22792434)(263.94531631,978.62505974)(264.36198296,978.88113612)
\curveto(264.77864961,979.1372125)(265.26910099,979.26525069)(265.83333708,979.26525069)
\curveto(266.59288566,979.26525069)(267.17882314,978.99832362)(267.59114951,978.46446947)
\curveto(268.00347589,977.9349556)(268.20963907,977.17974729)(268.20963907,976.19884455)
\lineto(268.20963907,971.79780305)
\lineto(267.00521203,971.79780305)
\lineto(267.00521203,976.15978205)
\curveto(267.00521203,976.85856675)(266.88151412,977.37722993)(266.6341183,977.71577158)
\curveto(266.38672247,978.05431323)(266.00911832,978.22358406)(265.50130584,978.22358406)
\curveto(264.88064614,978.22358406)(264.39019477,978.01742087)(264.02995172,977.6050945)
\curveto(263.66970868,977.19276813)(263.48958716,976.63070217)(263.48958716,975.91889665)
\lineto(263.48958716,971.79780305)
\lineto(262.28516012,971.79780305)
\lineto(262.28516012,976.15978205)
\curveto(262.28516012,976.86290703)(262.16146221,977.3815702)(261.91406639,977.71577158)
\curveto(261.66667056,978.05431323)(261.28472613,978.22358406)(260.76823309,978.22358406)
\curveto(260.15625395,978.22358406)(259.67014286,978.01525074)(259.30989981,977.59858408)
\curveto(258.94965677,977.18625771)(258.76953525,976.6263619)(258.76953525,975.91889665)
\lineto(258.76953525,971.79780305)
\lineto(257.56510821,971.79780305)
\lineto(257.56510821,979.08946945)
\lineto(258.76953525,979.08946945)
\lineto(258.76953525,977.95665699)
\curveto(259.04297274,978.40370558)(259.3706637,978.73356668)(259.75260813,978.94624028)
\curveto(260.13455256,979.15891389)(260.58811157,979.26525069)(261.11328516,979.26525069)
\curveto(261.64279903,979.26525069)(262.09201777,979.13070208)(262.46094137,978.86160487)
\curveto(262.83420524,978.59250766)(263.10981287,978.20188267)(263.28776425,977.68972991)
\closepath
}
}
{
\newrgbcolor{curcolor}{0 0 0}
\pscustom[linestyle=none,fillstyle=solid,fillcolor=curcolor]
{
\newpath
\moveto(273.91927396,975.4631675)
\curveto(272.95139205,975.4631675)(272.28081915,975.35249042)(271.90755528,975.13113626)
\curveto(271.5342914,974.9097821)(271.34765947,974.53217795)(271.34765947,973.9983238)
\curveto(271.34765947,973.57297659)(271.48654835,973.23443494)(271.76432612,972.98269884)
\curveto(272.04644416,972.73530301)(272.42838859,972.6116051)(272.91015941,972.6116051)
\curveto(273.57422188,972.6116051)(274.10590589,972.84598009)(274.50521143,973.31473007)
\curveto(274.90885725,973.78782033)(275.11068016,974.41499045)(275.11068016,975.19624042)
\lineto(275.11068016,975.4631675)
\closepath
\moveto(276.30859678,975.95795914)
\lineto(276.30859678,971.79780305)
\lineto(275.11068016,971.79780305)
\lineto(275.11068016,972.90457384)
\curveto(274.83724267,972.46186552)(274.49653088,972.13417456)(274.08854478,971.92150096)
\curveto(273.68055869,971.71316763)(273.18142676,971.60900097)(272.591149,971.60900097)
\curveto(271.84462125,971.60900097)(271.25000322,971.81733429)(270.8072949,972.23400095)
\curveto(270.36892686,972.65500788)(270.14974284,973.21707383)(270.14974284,973.9201988)
\curveto(270.14974284,974.74051127)(270.42318033,975.35900083)(270.97005531,975.77566748)
\curveto(271.52127057,976.19233414)(272.34158304,976.40066746)(273.43099272,976.40066746)
\lineto(275.11068016,976.40066746)
\lineto(275.11068016,976.51785496)
\curveto(275.11068016,977.06907021)(274.9283885,977.49441742)(274.56380518,977.79389658)
\curveto(274.20356214,978.09771601)(273.69574966,978.24962573)(273.04036774,978.24962573)
\curveto(272.62370109,978.24962573)(272.21788513,978.19971253)(271.82291987,978.09988615)
\curveto(271.4279546,978.00005976)(271.04818031,977.85032019)(270.68359699,977.65066742)
\lineto(270.68359699,978.75743821)
\curveto(271.12196503,978.92670904)(271.54731224,979.05257709)(271.95963861,979.13504236)
\curveto(272.37196498,979.22184791)(272.77344066,979.26525069)(273.16406565,979.26525069)
\curveto(274.21875311,979.26525069)(275.0065135,978.9918132)(275.52734681,978.44493822)
\curveto(276.04818013,977.89806324)(276.30859678,977.06907021)(276.30859678,975.95795914)
\closepath
}
}
{
\newrgbcolor{curcolor}{0 0 0}
\pscustom[linestyle=none,fillstyle=solid,fillcolor=curcolor]
{
\newpath
\moveto(287.28515781,977.11681327)
\lineto(287.28515781,975.95795914)
\curveto(286.82942866,975.61507721)(286.40625159,975.36985153)(286.0156266,975.22228209)
\curveto(285.6293419,975.07471265)(285.22569608,975.00092793)(284.80468915,975.00092793)
\curveto(284.32725861,975.00092793)(283.77170308,975.12896612)(283.13802254,975.3850425)
\curveto(283.09027949,975.40240361)(283.05555727,975.41542444)(283.03385588,975.424105)
\curveto(283.00347394,975.43712583)(282.95573088,975.45448694)(282.89062672,975.47618833)
\curveto(282.21788369,975.74528554)(281.67751912,975.87983415)(281.26953303,975.87983415)
\curveto(280.8875886,975.87983415)(280.50998444,975.79519873)(280.13672057,975.62592791)
\curveto(279.76345669,975.46099736)(279.36415115,975.19841056)(278.93880395,974.83816752)
\lineto(278.93880395,975.99702164)
\curveto(279.3945331,976.33990357)(279.81554002,976.58512926)(280.20182473,976.7326987)
\curveto(280.59244972,976.88460842)(280.99826568,976.96056327)(281.41927261,976.96056327)
\curveto(281.89670314,976.96056327)(282.45442882,976.83035495)(283.09244963,976.56993829)
\lineto(283.19010587,976.53087579)
\curveto(283.2248281,976.51785496)(283.27257115,976.50049385)(283.33333504,976.47879246)
\curveto(284.00607807,976.20969525)(284.54644263,976.07514664)(284.95442873,976.07514664)
\curveto(285.3276926,976.07514664)(285.6966162,976.15761191)(286.06119952,976.32254246)
\curveto(286.43012312,976.48747301)(286.83810921,976.75222995)(287.28515781,977.11681327)
\closepath
}
}
{
\newrgbcolor{curcolor}{0 0 0}
\pscustom[linestyle=none,fillstyle=solid,fillcolor=curcolor]
{
\newpath
\moveto(289.80468923,981.51785477)
\lineto(293.19010578,970.56082392)
\lineto(292.08333498,970.56082392)
\lineto(288.69791844,981.51785477)
\closepath
}
}
{
\newrgbcolor{curcolor}{0 0 0}
\pscustom[linestyle=none,fillstyle=solid,fillcolor=curcolor]
{
\newpath
\moveto(298.68489989,977.96967782)
\curveto(298.55035129,978.04780282)(298.40278185,978.10422643)(298.24219158,978.13894865)
\curveto(298.08594158,978.17801115)(297.91233048,978.1975424)(297.72135826,978.1975424)
\curveto(297.04427495,978.1975424)(296.52344164,977.97618824)(296.15885832,977.53347992)
\curveto(295.79861528,977.09511188)(295.61849375,976.46360149)(295.61849375,975.63894874)
\lineto(295.61849375,971.79780305)
\lineto(294.41406672,971.79780305)
\lineto(294.41406672,979.08946945)
\lineto(295.61849375,979.08946945)
\lineto(295.61849375,977.95665699)
\curveto(295.87022986,978.3993653)(296.19792082,978.72705626)(296.60156664,978.93972987)
\curveto(297.00521245,979.15674375)(297.49566383,979.26525069)(298.07292075,979.26525069)
\curveto(298.15538602,979.26525069)(298.24653185,979.25874027)(298.34635824,979.24571944)
\curveto(298.44618462,979.23703889)(298.5568617,979.22184791)(298.67838948,979.20014653)
\closepath
}
}
{
\newrgbcolor{curcolor}{0 0 0}
\pscustom[linestyle=none,fillstyle=solid,fillcolor=curcolor]
{
\newpath
\moveto(305.90494675,975.7431154)
\lineto(305.90494675,975.15717792)
\lineto(300.39713445,975.15717792)
\curveto(300.44921778,974.33252518)(300.69661361,973.70318492)(301.13932192,973.26915716)
\curveto(301.58637052,972.83946967)(302.20703022,972.62462593)(303.00130102,972.62462593)
\curveto(303.46137045,972.62462593)(303.90624891,972.68104954)(304.33593639,972.79389676)
\curveto(304.76996415,972.90674398)(305.19965164,973.0760148)(305.62499884,973.30170924)
\lineto(305.62499884,972.16889678)
\curveto(305.19531136,971.98660512)(304.75477318,971.84771624)(304.30338431,971.75223013)
\curveto(303.85199544,971.65674402)(303.39409615,971.60900097)(302.92968644,971.60900097)
\curveto(301.76649204,971.60900097)(300.84418305,971.94754262)(300.16275946,972.62462593)
\curveto(299.48567615,973.30170924)(299.1471345,974.21750782)(299.1471345,975.37202167)
\curveto(299.1471345,976.56559801)(299.46831504,977.51177853)(300.11067613,978.21056323)
\curveto(300.75737749,978.9136882)(301.62760316,979.26525069)(302.72135312,979.26525069)
\curveto(303.70225586,979.26525069)(304.47699541,978.94841042)(305.04557178,978.31472989)
\curveto(305.61848843,977.68538964)(305.90494675,976.82818481)(305.90494675,975.7431154)
\closepath
\moveto(304.70703013,976.09467789)
\curveto(304.69834957,976.75005981)(304.51388777,977.27306326)(304.15364473,977.66368825)
\curveto(303.79774197,978.05431323)(303.32465171,978.24962573)(302.73437395,978.24962573)
\curveto(302.0659712,978.24962573)(301.52994691,978.06082365)(301.12630109,977.6832195)
\curveto(300.72699555,977.30561534)(300.49696084,976.77393134)(300.43619695,976.08816747)
\closepath
}
}
{
\newrgbcolor{curcolor}{0 0 0}
\pscustom[linestyle=none,fillstyle=solid,fillcolor=curcolor]
{
\newpath
\moveto(311.56249882,981.92801101)
\lineto(311.56249882,980.9319173)
\lineto(310.41666553,980.9319173)
\curveto(309.98697805,980.9319173)(309.68749889,980.84511174)(309.51822807,980.67150064)
\curveto(309.35329752,980.49788953)(309.27083224,980.18538955)(309.27083224,979.73400067)
\lineto(309.27083224,979.08946945)
\lineto(311.24348842,979.08946945)
\lineto(311.24348842,978.1584799)
\lineto(309.27083224,978.1584799)
\lineto(309.27083224,971.79780305)
\lineto(308.0664052,971.79780305)
\lineto(308.0664052,978.1584799)
\lineto(306.92057191,978.1584799)
\lineto(306.92057191,979.08946945)
\lineto(308.0664052,979.08946945)
\lineto(308.0664052,979.59728193)
\curveto(308.0664052,980.40891384)(308.25520728,980.9991916)(308.63281143,981.3681152)
\curveto(309.01041558,981.74137907)(309.6093739,981.92801101)(310.42968637,981.92801101)
\closepath
}
}
{
\newrgbcolor{curcolor}{0 0 0}
\pscustom[linestyle=none,fillstyle=solid,fillcolor=curcolor]
{
\newpath
\moveto(318.11848828,970.56082392)
\lineto(318.11848828,969.62332396)
\lineto(317.71484246,969.62332396)
\curveto(316.63411333,969.62332396)(315.90928697,969.78391423)(315.54036337,970.10509477)
\curveto(315.17578005,970.42627532)(314.99348839,971.06646627)(314.99348839,972.02566762)
\lineto(314.99348839,973.58165715)
\curveto(314.99348839,974.23703907)(314.8763009,974.69059808)(314.64192591,974.94233418)
\curveto(314.40755092,975.19407028)(313.98220371,975.31993833)(313.36588429,975.31993833)
\lineto(312.96874888,975.31993833)
\lineto(312.96874888,976.25092788)
\lineto(313.36588429,976.25092788)
\curveto(313.98654399,976.25092788)(314.41189119,976.3746258)(314.64192591,976.62202162)
\curveto(314.8763009,976.87375772)(314.99348839,977.32297646)(314.99348839,977.96967782)
\lineto(314.99348839,979.53217776)
\curveto(314.99348839,980.49137912)(315.17578005,981.12939993)(315.54036337,981.44624019)
\curveto(315.90928697,981.76742074)(316.63411333,981.92801101)(317.71484246,981.92801101)
\lineto(318.11848828,981.92801101)
\lineto(318.11848828,980.99702146)
\lineto(317.67577996,980.99702146)
\curveto(317.06380082,980.99702146)(316.66449528,980.90153535)(316.47786334,980.71056314)
\curveto(316.2912314,980.51959092)(316.19791543,980.11811524)(316.19791543,979.5061361)
\lineto(316.19791543,977.89155282)
\curveto(316.19791543,977.21012924)(316.09808905,976.71533759)(315.89843628,976.40717788)
\curveto(315.70312378,976.09901817)(315.36675227,975.89068484)(314.88932173,975.7821779)
\curveto(315.37109255,975.6649904)(315.7096342,975.4523168)(315.90494669,975.14415709)
\curveto(316.10025919,974.83599738)(316.19791543,974.34337587)(316.19791543,973.66629256)
\lineto(316.19791543,972.05170929)
\curveto(316.19791543,971.43973014)(316.2912314,971.03825446)(316.47786334,970.84728225)
\curveto(316.66449528,970.65631003)(317.06380082,970.56082392)(317.67577996,970.56082392)
\closepath
}
}
{
\newrgbcolor{curcolor}{0 0 0}
\pscustom[linestyle=none,fillstyle=solid,fillcolor=curcolor]
{
\newpath
\moveto(321.45833465,970.56082392)
\lineto(321.9140638,970.56082392)
\curveto(322.52170267,970.56082392)(322.91666793,970.65413989)(323.09895959,970.84077183)
\curveto(323.28559153,971.02740377)(323.3789075,971.43104959)(323.3789075,972.05170929)
\lineto(323.3789075,973.66629256)
\curveto(323.3789075,974.34337587)(323.47656375,974.83599738)(323.67187624,975.14415709)
\curveto(323.86718873,975.4523168)(324.20573039,975.6649904)(324.6875012,975.7821779)
\curveto(324.20573039,975.89068484)(323.86718873,976.09901817)(323.67187624,976.40717788)
\curveto(323.47656375,976.71533759)(323.3789075,977.21012924)(323.3789075,977.89155282)
\lineto(323.3789075,979.5061361)
\curveto(323.3789075,980.12245552)(323.28559153,980.5239312)(323.09895959,980.71056314)
\curveto(322.91666793,980.90153535)(322.52170267,980.99702146)(321.9140638,980.99702146)
\lineto(321.45833465,980.99702146)
\lineto(321.45833465,981.92801101)
\lineto(321.86849089,981.92801101)
\curveto(322.94922002,981.92801101)(323.6697061,981.76742074)(324.02994914,981.44624019)
\curveto(324.39453246,981.12939993)(324.57682412,980.49137912)(324.57682412,979.53217776)
\lineto(324.57682412,977.96967782)
\curveto(324.57682412,977.32297646)(324.69401162,976.87375772)(324.92838661,976.62202162)
\curveto(325.1627616,976.3746258)(325.58810881,976.25092788)(326.20442823,976.25092788)
\lineto(326.60807405,976.25092788)
\lineto(326.60807405,975.31993833)
\lineto(326.20442823,975.31993833)
\curveto(325.58810881,975.31993833)(325.1627616,975.19407028)(324.92838661,974.94233418)
\curveto(324.69401162,974.69059808)(324.57682412,974.23703907)(324.57682412,973.58165715)
\lineto(324.57682412,972.02566762)
\curveto(324.57682412,971.06646627)(324.39453246,970.42627532)(324.02994914,970.10509477)
\curveto(323.6697061,969.78391423)(322.94922002,969.62332396)(321.86849089,969.62332396)
\lineto(321.45833465,969.62332396)
\closepath
}
}
\end{pspicture}

% \begin{tikzpicture}
[node distance = 1cm, auto,font=\footnotesize,
% STYLES
every node/.style={node distance=3cm},
% The comment style is used to describe the characteristics of each force
comment/.style={rectangle, inner sep= 5pt, text width=4cm, node distance=0.25cm, font=\scriptsize\sffamily},
% The force style is used to draw the forces' name
force/.style={rectangle, draw, fill=black!10, inner sep=5pt, text width=4cm, text badly centered, minimum height=1.2cm, font=\bfseries\footnotesize\sffamily}] 

% Draw forces
\node [force] (FP) {FP + Operator};
\node [force, right=1cm of FP] (FON) {P-uniform FON SB};
% \node [force, text width=3cm, dashed, left=1cm of substitutes] (state) {Public policies};
% \node [force, left=1cm of rivalry] (suppliers) {Bargaining power of suppliers};
% \node [force, right=1cm of rivalry] (users) {Bargaining power of users};
% \node [force, below of=rivalry] (entrants) {Threat of new entrants};

%%%%%%%%%%%%%%%
% Change data from here

% RIVALRY
% \node [comment, below=0.25 of rivalry] (comment-rivalry) {(+) A war against Microsoft\\
% (+) Limiting sunk costs\\
% (+) Coopetition};

% % SUPPLIERS
% \node [comment, below=0.25cm of suppliers] {(+) Efficiency\\
% (+) Attracting other developers\\
% (+) Creating a Chrome community};

% % SUBSTITUTES
% \node [comment, right=0.25 of substitutes] {(+) Portability};

% % USERS
% \node [comment, below=0.25 of users] {(+) Increasing the user information\\
% (+) Reducing the switching costs};

% % NEW ENTRANTS
% \node [comment, right=0.25 of entrants] {(+) EC vs. Microsoft};

% % PUBLIC POLICIES
% \node [comment, text width=3cm, below=0.25 of state] {(+) Positively framed\\
% (+) Transparency\\
% (--) A new monopoly?};

%%%%%%%%%%%%%%%%

% Draw the links between forces
\path[->,thick] 
(FP) edge (FON);

% \path[->,thick] 
% (substitutes) edge (rivalry)
% (suppliers) edge (rivalry)
% (users) edge (rivalry)
% (entrants) edge (comment-rivalry);

\end{tikzpicture} 

\section{Structures with Number-Valued Functions}
In order to define a Lindestrom quantifier we begin with a class of structures
$\mathcal{C}$ over some relational vocabulary $\rho$. An application of a
Lindestrom quantifier acts on a sequence of formulas which together define a
$\rho$-interpretation, and hence determine a $\rho$-structure. A formula with
the quantifier at its head evaluates to true if, and only if, the defined
$\rho$-structure is in $\mathcal{C}$ (for details see~\ref{}). In contrast with
Lindestrom quantifiers, generalised operators act on a sequence of number terms
(as rank operators do) and formulas (as counting operators do). As such, we
define a generalised operator from a class of (many-sorted) structures
consisting of both relations and number-valued functions. In this section we
generalise the notion of a vocabulary so as to allow for such structures and
correspondingly generalise the notion of an interpretation.

\begin{definition}
  A \emph{many-sorted vocabulary with number-valued functions} is a tuple $\tau
  := (R, F, S, \zeta)$ where $R$ is a sequence of relation symbols, $F$ is a
  sequence of function symbols, $S$ is a sequence of sort symbols, and $\zeta$
  is a function that assigns each $H \in R \cup F$ of arity $h$ to a tuple of
  sort symbols $(s_1, \ldots, s_h)$.

  We say that $\mathcal{A}$ is a $\tau$-structure if $\mathcal{A} = (\uplus_{s
    \in S} A_s; (R^{\mathcal{A}}_i)_{R_i \in R} ; (F^{\mathcal{A}}_i)_{F_i \in
    F})$ where
  \begin{myitemize}
  \item for each $s \in S$, $A_s$ is a non-empty set,
  \item for each $R_i \in R$, $R^{\mathcal{A}}_i \subseteq A_{\zeta(R_i)(1)}
    \times \ldots \times A_{\zeta(R_i)(\ar(R_i))}$, and
  \item for each $F_i \in R$, $F^{\mathcal{A}}_i : A_{\zeta(R_i)(1)} \times
    \ldots \times A_{\zeta(R_i)(\arity(R_i))} \ra \nats$.
  \end{myitemize}

  Let $\mathcal{A}$ and $\mathcal{B}$ be $\tau$-structures. We call $h : A \ra
  B$ a \emph{homomorphism} if (i) $h$ preserves sorts, (ii) for all $R_i \in R$
  and $a \in R^{\mathcal{A}}_i$ we have $h(a) \in R^{\mathcal{B}}_i$, and (iii)
  for all $F_i \in F$ and $a \in \dom (F^{\mathcal{A}}_i)$ we have
  $F^{\mathcal{A}}_i(a) = F^{\mathcal{B}}_i(h(a))$. If $h$ is a bijection we
  call $h$ an \emph{isomorphism}.
\end{definition}


We now define a many-sorted interpretation with number-valued functions.

\begin{definition}
  Let $L$ be a logic with a number sort. Let $\rho$ be a vocabulary. Let $\tau =
  (R, F, S, \zeta)$ be a many-sorted vocabulary with number-valued functions,
  where $R = \{R_1, \ldots, R_{r}\}$, $F = \{ F_1, \ldots, F_{f} \}$, and $S =
  \{s_1, \ldots, s_{q}\}$. For each $i \in [r]$ and $j \in [f]$ let $r_i$ denote
  the arity of $R_i$ and $f_j$ be the arity of $F_j$.

  Let $k \in \nats$ and let $\ar : S \times [2] \ra \nats$. A \emph{many-sorted
    $L [\rho, \tau]$-interpretation} with \emph{width} $\ar$ and \emph{parameter
    width} $k$ is a sequence $\mathcal{I} := (\vec{\phi}^D,
  \vec{\phi}^{\approx}, (\phi_i)_{i \in [r]}, (\eta_j)_{j \in [f]})$, with
  mixed-sort $k$-sequence of variables $\vec{w}$ such that

  \begin{myenum}
  \item $\vec{\phi}^D = (\phi^D_1, \ldots, \phi^D_q)$, and for each $i \in [q]$,
    $\phi^D_i(\vec{x}, \vec{\mu}; \vec{w})$ where $\vert \vec{x} \vert =
    \ar(s_i, 1)$ and $\vert \vec{\mu} \vert = \ar(s_i, 2)$,
  \item $\vec{\phi}^{\approx} = (\phi^{\approx}_1, \ldots, \phi^{\approx}_q)$,
    and for each $i \in [q]$ $\phi^{\approx}_i(\vec{x}_1\vec{\mu}_1, \vec{x}_2
    \vec{\mu}_2 ; \vec{w})$ where $\vert \vec{x}_1 \vert = \vert \vec{x}_2 \vert
    = \ar(s_i, 1)$ and $\vert \vec{\mu}_1 \vert = \vert \vec{\mu}_2 \vert =
    \ar(s_i, 2)$,
  \item for each $i \in [r]$, $\phi_i (\vec{x}^i_1 \vec{\mu}^i_1, \ldots,
    \vec{x}^{i}_{r_i} \vec{\mu}^{i}_{r_i} ; \vec{w})$ where for each $j \in
    [r_i]$, $\vert \vec{x}^i_j\vert = \ar(\zeta(R_i)(j), 1)$ and $\vert
    \vec{\mu}^i_j \vert = \ar(\zeta(R_i)(j), 2)$, and
  \item for each $i \in [f]$, $\nu_i (\vec{x}^i_1 \vec{\mu}^i_1, \ldots,
    \vec{x}^{i}_{f_i} \vec{\mu}^{i}_{f_i} ; \vec{w})$ where for each $j \in
    [f_i]$, $\vert \vec{x}^i_j \vert = \ar(\zeta(R_i)(j), 1)$ and $\vert
    \vec{\mu}^i_j \vert = \ar(\zeta(R_i)(j), 2)$.
  \end{myenum}

  Let $\mathcal{A} \in \fin[\rho]$ and let $n := \vert \mathcal{A} \vert$. Let
  $\beta$ be an assignment to $\vec{w}$ in $\mathcal{A}$. For each $i \in [q]$
  let $(\phi^D_i)^{(\mathcal{A}, \beta)} := \{ \vec{a}\vec{m} \in A^{\ar(s_i,
    1)} \times \nats^{\ar(s_i, 2)}: \mathcal{A} \models \phi^D_i[\beta \cup
  \alpha^{\vec{a}, \vec{m}}_{\vec{x}, \vec{\mu}}] \}$. We say that $\mathcal{I}
  (\mathcal{A}, \beta)$ is defined if there exists $\mathcal{B} \in \fin[\tau]$
  with universe $B = \uplus_{i \in [q]} B_i$ and a sort preserving surjection $h
  : \uplus_{i \in [q]} (\phi^D_i)^{(\mathcal{A}, \beta)} \ra \uplus_{i \in [q]}
  B_i$ such that
    
  \begin{myenum}
  \item for each $i \in [q]$ the relation $\approx_i$ on $B_i$ defined such that
    for each $\vec{a}_1\vec{m}_1, \vec{a}_2\vec{m}_2 \in
    (\phi^D_i)^{(\mathcal{A}, \beta)}$ we have $\vec{a}_1\vec{m}_1 \approx_i
    \vec{a}_2\vec{m}_2$ if, and only if, $\mathcal{A} \models
    \phi^{\approx}_i[\beta \cup \alpha^{\vec{a}_1\vec{m}_1,
      \vec{a}_2\vec{m}_2}_{\vec{x}_1 \vec{\mu}_1, \vec{x}_2 \vec{\mu}_2}]$ is an
    equivalence relation;
  \item let $\approx = \uplus_{i \in [q]}\approx_i$ be an equivalence relation
    on $B$, then for $a , b \in \uplus_{i \in [q]}(\phi^D_i)^{(\mathcal{A},
      \beta)}$ if $a \approx b$ then $h(a) = h(b)$;
  \item for each $i \in [r]$ and $j \in [r_i]$ let $\vec{a}^i_j\vec{m}^i_j \in
    (\phi^D_{\zeta(R_i)(j)})^{(\mathcal{A}, \beta)}$ we have
    \begin{align*}
      h(\vec{a}^i_1 \vec{m}^i_1 , \ldots , \vec{a}^i_{r_i}\vec{m}^i_{r_i})
      \in R^{\mathcal{B}}_i \iff \mathcal{A} \models \phi_i
      [\beta, \alpha^{\vec{a}_1\vec{m}_1, \ldots, \vec{a}_{r_i} \vec{m}_{r_i}}_{\vec{x}^i_1 \vec{\mu}^i_1 , \ldots, \vec{x}^i_{r_i} \vec{\mu}^i_{r_i}}].
    \end{align*}
  \end{myenum}

  In this case we write $\mathcal{B} = \mathcal{I} (\mathcal{A}, \beta)$ and say
  that $\mathcal{I}$ interprets $\mathcal{B}$ in $(\mathcal{A}, \beta)$.
\end{definition}


% \section{A Framework for Operators and Quantifiers}


% Lindestrom quantifiers are defined by a class of structures, with the syntax of
% the quantifier is specified so as to define an interpretation, with the
% quantifier's evaluation depending on whether In the same way Lindestrom
% quantifiers are defined from a class of structures, generalised operators are
% similarly defined from a class of structures.


% We first need to generalise the notion of a many-sorted vocabulary in order to
% include function symbols that are intended to be interpreted as number-valued
% functions.

% \begin{definition}
%   Let $L$ be a logic ($\FO$ or $\FP$) and let $\tau$ be a many-sorted
%   vocabulary with function and constant symbols and $N: \fin \ra \fin[\tau]$
%   be a function that maps isomorphic structures to isomorphic structures. Let
%   $S = \{s_1, \ldots, s_q\}$ be the sorts in $\tau$. Let $\rho$ be a
%   vocabulary. Let $\tau \uplus \rho$ be the vocabulary formed by interpreting
%   $\rho$ as a many-sorted vocabulary with a single sort. We refer to the
%   single sort of $\rho$ as the \emph{element sort}. Let $L^{\tau, N}[\rho] =
%   L[\tau \uplus \rho]$ and for $\theta \in L^{\tau, N}[\rho]$ and $\mathcal{A}
%   \in \fin[\rho]$ let $\mathcal{A} \models_{L^{\tau, N}[\rho]} \theta$ if, and
%   only if, $(\mathcal{A} \uplus N(\mathcal{A})) \models_{L[\tau \uplus \rho]}
%   \theta$.
% \end{definition}

% Let $L^{\tau, N}[\rho]$ be a logic. We say $N$ is \emph{uniform} if for any
% $\mathcal{A}, \mathcal{B} \in \fin$ such that $\vert \mathcal{A} \vert = \vert
% \mathcal{B} \vert$ we have $N(\mathcal{A}) = N(\mathcal{B})$. In this case for
% each $n \in \nats$ we write $N(n)$ to abbreviate $N(\mathcal{A})$ for any
% $\mathcal{A} \in \fin$ of size $n$. We say $N$ is \emph{$\PT$-uniform} if $N$
% is uniform and $n \mapsto N(n)$ is computable in time polynomial in $n$.

% \begin{definition}
%   A \emph{many-sorted vocabulary with number-valued functions} is a tuple $\tau
%   := (R, F, S, \zeta)$ where $R$ is a sequence of relation symbols, $F$ is a
%   sequence of function symbols, $S$ is a sequence of sort symbols, and $\zeta$
%   is a function that assigns each $H \in R \cup F$ of arity $h$ to a tuple of
%   sort symbols $(s_1, \ldots, s_h)$.
% \end{definition}

% There is an obvious notion of a $\tau$-structure and isomorphism between
% $\tau$-structures.

% \begin{definition}
%   Let $L$ be a logic with a number sort. Let $\rho$ be a vocabulary. Let $\tau =
%   (R, F, S, \zeta)$ be a many-sorted vocabulary with number-valued functions,
%   where $R = \{R_1, \ldots, R_{r}\}$, $F = \{ F_1, \ldots, F_{f} \}$, and $S =
%   \{s_1, \ldots, s_{q}\}$. For each $i \in [r]$ and $j \in [f]$ let $r_i$ denote
%   the arity of $R_i$ and $f_j$ be the arity of $F_j$.

%   Let $\ar : S \times [2] \ra \nats$. A \emph{many-sorted $L [\rho,
%     \tau]$-interpretation} of \emph{width} $(\ar)$ is a sequence $\mathcal{I} :=
%   (\vec{\phi}^D, \vec{\phi}^{\approx}, (\phi_i)_{i \in [r]}, (\eta_j)_{j \in
%     [f]})$, where

%   \begin{myenum}
%   \item $\vec{\phi}^D = (\phi^D_1, \ldots, \phi^D_q)$, and for each $i \in [q]$,
%     $\phi^D_i(\vec{x}, \vec{\mu})$ where $\vert \vec{x} \vert = \ar(s_i, 1)$ and
%     $\vert \vec{\mu} \vert = \ar(s_i, 2)$,
%   \item $\vec{\phi}^{\approx} = (\phi^{\approx}_1, \ldots, \phi^{\approx}_q)$,
%     and for each $i \in [q]$ $\phi^{\approx}_i(\vec{x}_1\vec{\mu}_1, \vec{x}_2
%     \vec{\mu}_2)$ where $\vert \vec{x}_1 \vert = \vert \vec{x}_2 \vert =
%     \ar(s_i, 1)$ and $\vert \vec{\mu}_1 \vert = \vert \vec{\mu}_2 \vert =
%     \ar(s_i, 2)$,
%   \item for each $i \in [r]$, $\phi_i (\vec{x}^i_1 \vec{\mu}^i_1, \ldots,
%     \vec{x}^{i}_{r_i} \vec{\mu}^{i}_{r_i})$ where for each $j \in [r_i]$, $\vert
%     \vec{x}^i_j\vert = \ar(\zeta(R_i)(j), 1)$ and $\vert \vec{\mu}^i_j \vert =
%     \ar(\zeta(R_i)(j), 2)$, and
%   \item for each $i \in [f]$, $\nu_i (\vec{x}^i_1 \vec{\mu}^i_1, \ldots,
%     \vec{x}^{i}_{f_i} \vec{\mu}^{i}_{f_i})$ where for each $j \in [f_i]$, $\vert
%     \vec{x}^i_j \vert = \ar(\zeta(R_i)(j), 1)$ and $\vert \vec{\mu}^i_j \vert =
%     \ar(\zeta(R_i)(j), 2)$.
%   \end{myenum}

%   Let $\mathcal{A} \in \fin[\rho]$ and for all $i \in [q]$ let
%   $(\phi^D_i)^{\mathcal{A}} := \{ \vec{a}\vec{m} \in A^{\ar(s_i, 1)} \times
%   \nats^{\ar(s_i, 2)}: \mathcal{A} \models \phi^D_i[\vec{a}\vec{m}] \}$. We say
%   that $\mathcal{I} (\mathcal{A})$ is defined if there exists $\mathcal{B} \in
%   \fin[\tau]$ with universe $B = \uplus_{i \in [q]} B_i$ and a sort preserving
%   surjection $h : \uplus_{i \in [q]} (\phi^D_i)^{\mathcal{A}} \ra \uplus_{i \in
%     [q]} B_i$ such that:
    
%   \begin{myenum}
%   \item for each $i \in [q]$ the relation $\approx_i$ on $B_i$ defined for each
%     $\vec{a}_1\vec{m}_1, \vec{a}_2\vec{m}_2 \in (\phi^D_i)^{\mathcal{A}}$ such
%     that $\vec{a}_1\vec{m}_1 \approx_i \vec{a}_2\vec{m}_2$ if, and only if,
%     $\mathcal{A} \models \phi^{\approx}_i[\vec{a}_1\vec{m}_1,
%     \vec{a}_2\vec{m}_2]$ is an equivalence relation,
%   \item let $\approx = \uplus_{i \in [q]}\approx_i$ be an equivalence relation
%     $B$, then for $a , b \in \uplus_{i \in [q]}(\phi^D_i)^{\mathcal{A}}$ if $a
%     \approx b$ then $h(a) = h(b)$,
%   \item for each $i \in [r]$ and for all $j \in [r_i]$ and
%     $\vec{a}^j_i\vec{m}^j_i \in (\phi^D_i)$ $\vec{a}$$h(a) \in R^h_i =
%     \{(\vec{a}_1\vec{})\}$
%   \end{myenum}
% \end{definition}

\section{Generalised Operators}
We are now ready to define a generalised operator and what it means to extend a
logic by a generalised operator.

\subsection{Generalised Many-Sorted Operators}
Let $\tau = (R, F, S, \zeta)$ be a many-sorted vocabulary with number-valued
functions, where $R = \{R_1, \ldots, R_{r}\}$, $F = \{ F_1, \ldots, F_{f} \}$,
and $S = \{s_1, \ldots, s_{q}\}$. For each $i \in [r]$ and $j \in [f]$ let $r_i$
denote the arity of $R_i$ and $f_j$ be the arity of $F_j$.

Let $m \in \nats$ and let $E : \nats^{m} \times \fin[\tau] \ra \nats$. We say
that $E$ is \emph{closed under isomorphism} if for all $\mathcal{A}, \mathcal{B}
\in \fin[\tau]$, $\vec{p} \in \nats^{m}$ if $\mathcal{A} \simeq \mathcal{B}$
then $E(\vec{p}, \mathcal{A}) = E(\vec{p}, \mathcal{B})$. Let $\ar : S \times
[2] \ra \nats$ be a function. We associate with the pair $(E, \ar)$ a
\emph{number-valued generalised operator} $\Omega_{(E, \ar)}$. We say the
\emph{vocabulary} of the operator is $\tau$, the \emph{arity} is $\ar$, the
\emph{evaluation function} is $E$, and the \emph{parameter-width} is $m$.

For a logic $L$ the extension $L(\Omega_{E, \ar})$ is the closure of the set of
formulae in $L$ over the following formula formation rule:

Let $\vec{\pi}$ be a sequence of number terms in $L(\Omega_{E, \ar})$. Let
$\vec{\phi}^D$ and $\vec{\phi}^{\approx}$ be the $q$-sequences of formulas such
that for each $i \in [q]$, $\phi^D_i$ has exactly $\ar(s_i, 1)$ free element
,ariables and $\ar(s_i, 2)$ free number variables and $\phi^{\approx}_i$ has
exactly $2\cdot\ar(s_i, 1)$ free element variables and $2\cdot\ar(s_i, 2)$ free
number variables. Let $(\chi_i)_{i \in [r]}$ and $(\eta_i)_{i \in [f]}$ be
sequences of formulas and number-terms, respectively. For each $i \in [r]$ and
$l \in [r_i]$ let $s = \zeta (R_i)(l)$ and let $\vec{x}^i_l$ be an $\ar(s,
1)$-length tuple of element variables and $\vec{\mu}^i_l$ be an $\ar(s,
2)$-length tuple of number variables. For each $j \in [f]$ and $k \in [f_j]$ let
$s = \zeta (F_i)(k)$ and let $\vec{y}^j_k$ be an $\ar(s, 1)$-length tuple of
element variables and $\vec{\nu}^j_k$ be an $\ar(s, 2)$-length tuple of number
variables. The following is a number term:
\begin{align*}
  \gamma = \Omega_{E, \ar} [\vec{\pi}] [\vec{\phi}^D, \vec{\phi}^{\approx}]  [((\vec{x}^i_1 \vec{\mu}^i_1, \ldots, \vec{x}^i_{r_i} \vec{\mu}^i_{r_i})(\chi_i))_{i \in [r]}((\vec{y}^j_1 \vec{\nu}^j_1, \ldots, \vec{y}^j_{f_j} \vec{\nu}^j_{f_j})(\eta_j))_{j \in [f]}].
\end{align*}
The semantics of this operator is defined as follows: Let $\mathcal{I} :=
(\vec{\phi}^D, \vec{\phi}^{\approx}, \vec{\chi}, \vec{\eta})$ and
$\gamma^{\mathcal{A}} = E(\vec{\pi}^{\mathcal{A}}, \mathcal{I}(\mathcal{A}))$.

We can similarly define a \emph{Boolean-valued generalised operator}. The
definition of an operator is the same, except in this case $E : \nats^{m} \times
\fin[\tau] \ra \{0,1\}$. The extension of a logic by a Boolean-valued operator
is similar except in this case an application of a Boolean-valued generalised
operator is a formula rather than a number term.

Let $\setop$ be a set of operators. Let $L(\widetilde{\setop})$ be the set of
formulas and number-terms in $L(\setop)$ such that each application of a
generalised operator in $\setop$ that appears as a sub-formula or
sub-number-term has domain formulas that are valid and equality formulas that
simply test for equality. In other words, this interpretation involves no
restriction of the domain and no quotienting. In this case, since they have no
effect on the semantics, we often omit thee domain and equality formulas when
applying a generalised operator.

\begin{definition}
  We say a logic $L(\setop)$ is \emph{closed under operator quotients} if
  $L(\widetilde{\setop}) = L(\setop)$.
\end{definition}

It can be shown that many of the logics of interest are closed under operator
quotients.

\begin{lem}
  The logics $\FP^{\nats}$, $\FPC$, and $\FPR$ are closed under operator
  quotients.
  \label{lem:logics-operator-quotients}
\end{lem}
We are particularly interested in logics whose formulas cannot define numbers
that are super-polynomial in the size of the structure over which they are being
evaluated. In this case each number-term denotes a value that be held by a
fixed-length tuple of number variables.
\begin{definition}
  Let $L$ be a logic with a number sort. Let $\rho$ be a vocabulary. Let
  $\theta$ be a formula or number-term in $L[\rho]$. For each $n \in \nats$ let
  $e_{\theta}(n)$ be the maximal value denoted by a number-term in $\theta$ when
  evaluated for any given structure in $\fin[\rho, n]$ and assignment to the
  free variables in $\theta$ (and with free number-variables assigned to
  elements of $[n]$). We call the function $e_{\theta}$ the \emph{numeric bound}
  of $\theta$. We say $\theta$ has \emph{constant numeric-width} if there exists
  $k \in \nats$ such that $e_{\theta}(n) \leq n^k$ for all $n \in \nats$. In
  this case we call $k$ the \emph{numeric-width} of $\theta$. We say the logic
  $L$ has \emph{constant numeric-width} if for vocabulary $\rho$ each formula in
  $L[\rho]$ has constant numeric-width.
\end{definition}

It is easy to see that $\FPR$, $\FPC$, $\FP^{\nats}$, and many other logics of
interest have constant numeric-width.

We are restricting the set of operators to get a normal form.

\begin{definition}
  Let $\Omega$ be a generalised operator. We say $\Omega$ \emph{operates on
    formulas} if there are no function symbols in the vocabulary of $\Omega$. We
  say that $\Omega$ \emph{has no parameters} if the parameter-width is zero. We
  say $\Omega$ is \emph{numberless} if it operates on formulas and for all $s
  \in S$, $\ar(s, 2) = 0$. We call a numberless Boolean generalised operator a
  \emph{normal generalised quantifier}.
\end{definition}

\begin{lem}
  Let $L$ be a logic and let $\Omega_{E, \ar}$ be a generalised operator with
  constant numeric-width. There exists an operator $\Omega_{E', \ar'}$ that acts
  on formulas such that
  \begin{myenum}
  \item $L(\Omega_{E, \ar}) \leq L(\Omega_{E', \ar'})$,
  \item if $L$ is an extension of $\FP^{(\nats, \leq)}$ then $L(\Omega_{E',
      \ar'}) \leq L(\Omega_{E, \ar})$,
  \item if $L$ is an extension of $\FOC$ and $\ar^* : S' \times [2] \ra \nats$
    compatible with $\ar$ then $L(\Omega_{E', \ar^*}) \leq L(\Omega_{E, \ar})$,
    and
  \item if $L$ is an extension of $\FOC$ then $L(\Omega_{E}) \equiv
    L(\Omega_{E'})$.
  \end{myenum}
  \label{lem:op-to-op-on-formulas}
\end{lem}
\begin{proof}
  Let $\tau$ be the many-sorted vocabulary, $m \in \nats$ be the
  parameter-width, $\ar: S \times [2] \ra \nats$, and $E : \nats^m \times
  \fin[\tau] \ra \nats$ be the evaluation function of $\Omega$. Let $(R, F, S,
  \zeta):= \tau$, where $R = \{R_1, \ldots, R_{r}\}$, $F = \{ F_1, \ldots, F_{f}
  \}$, and $S = \{s_1, \ldots, s_{q}\}$. For each $i \in [r]$ and $j \in [f]$
  let $r_i$ denote the arity of $R_i$ and $f_j$ be the arity of $F_j$. Let
  $\tau' = (R', S', \zeta')$ such that
  \begin{myitemize}
  \item $R' = \{R_1, \ldots, R_r, R_{r+1}, \ldots, R_{r+f}\}$,
  \item $S' = (s_1, \ldots, s_q, s_{q+1})$, and
  \item for all $R_i \in R'$, $\zeta' (R_i) = \zeta(R_i)$ if $R_{i+r} \in R$ and
    $\zeta'(R_{i+r}) = (s_{i, 1}, \ldots, s_{i, f_i + 1})$, where $(s_{i, 1},
    \ldots, s_{i, f_i}) = \zeta(F_{i})$ and $s_{i, f_i+1} = s_{q+1}$.
  \end{myitemize}
  Let $r' = r + f$ and for all $i \in [r']$, $r_i'$ is the arity of $R_i'$. Let
  $\ar' : S' \times [2] \ra \nats$ be defined such that $\ar'(s, i) = \ar(s,i)$
  for all $(s, i) \in S \times [2]$, and $\ar'(s_{q+1}, 1) = 0$ and
  $\ar(s_{q+1}, 2) = t$. We associate with a structure $\mathcal{A} \in
  \fin[\tau']$ a structure $T(\mathcal{A}) \in \fin[\tau]$ defined by
  \begin{myitemize}
  \item $T(\mathcal{A})$ has universe $\uplus_{i \in [q]}A_i$,
  \item for all $R_i \in R$, $R^{T(\mathcal{A})}_i = R^{\mathcal{A}}_i$, and
  \item for all $F_j \in F$, $F^{T(\mathcal{A})}_j(a_1, \ldots, a_{f_i}) = \vert
    \{ a \in A_{q+1}: (a_1, \ldots, a_{f_i}, a) \in R^{\mathcal{A}}_i\} \vert$.
  \end{myitemize}
  Let $E': \nats^m \times \fin[\tau'] \ra \nats$ be defined such that
  $E'(\vec{a}, \mathcal{A}) = E(\vec{a}, T(\mathcal{A}))$ for all $(\vec{a},
  \mathcal{A}) \in \nats^m \times \fin[\tau']$.

  Let $\rho$ be a vocabulary. Let $\theta(\vec{x}) \in L(\Omega_{E,
    \ar})[\rho]$. We show by induction on the structure of the formula that for
  each sub-formula and sub-number-term of $\theta(\vec{x})$ there is a formula
  or number-term in $L(\Omega_{E', \ar'})$ that \emph{translates}
  $\theta(\vec{x})$, in that it has the same free variables and defines the same
  query. The only interesting case in this induction is an application of the
  operator. Let
  \begin{align*}
    \gamma = \Omega_{E, \ar} [\vec{\pi}] [\vec{\phi}^D, \vec{\phi}^{\approx}]  [((\vec{x}^i_1 \vec{\mu}^i_1, \ldots, \vec{x}^i_{r_i} \vec{\mu}^i_{r_i})(\chi_i))_{i \in [r]}((\vec{y}^j_1 \vec{\nu}^j_1, \ldots, \vec{y}^j_{f_j} \vec{\nu}^j_{f_j})(\eta_j))_{j \in [f]}].
  \end{align*}
  be a sub-formula or sub-number-term. We then define
  \begin{align*}
    \gamma' = \Omega_{E', \ar'}[\vec{\pi}'][\vec{\psi}^D, \vec{\psi}^{\approx}] ((\vec{x}^i_1 \vec{\mu}^i_1, \ldots, \vec{x}^i_{r_i'} \vec{\mu}^i_{r_i'})(\psi_i))_{i \in [r']}.
  \end{align*}
  where
  \begin{myitemize}
  \item $\vec{\pi}'$ is the translation of $\vec{\pi}$,
  \item $\vec{\psi}^D = (\psi^D_1, \ldots, \psi^D_{q}, \psi^D_{q+1})$, where for
    all $i \in [q]$, $\psi^D_i$ is the translation of $\phi^D_i$ and
    $\psi^D_{q+1}$ is a valid formula,
  \item $\vec{\psi}^{\approx} = (\psi^{\approx}_1, \ldots, \psi^{\approx}_q,
    \psi^{\approx}_{q+1})$, where for all $i \in [q]$, $\psi^{\approx}_i$ is the
    translation of $\phi^D_i$ and $\psi^{\approx}_{q+1}(\vec{\nu}_1,
    \vec{\nu}_2) := \vec{\nu}_1 = \vec{\nu}_2$,
  \item for all $i \in [r']$, and $j \in [r_i']$, $\vec{x}^i_j$ is a
    $\ar(\zeta(R_i')(j), 1)$-sequence of element variables $\vec{\mu}^i_j$ is a
    $\ar(\zeta(R_i')(j), 2)$ is a sequence of element variables, and
  \item for all $i \in [r']$, if $i \leq r$, then $\psi_i$ is the translation of
    $\chi_i$, otherwise let $\eta_{i - r}'$ be the translation of $\eta_{i - r}$
    and $\psi_i := \eta_{i-r}' < \vec{\mu}^i_{r_i'}$.
  \end{myitemize}
  It can be shown that $\gamma'$ translates $\gamma$. Statement (1) follows.

  We next prove statement (3). Let $\ar^*: S' \times [2] \ra \nats$ be
  compatible with $\ar$ and let $L$ be an extension of $\FOC$. Let
  $\theta'(\vec{x}) \in L(\Omega_{E', \ar'})$. We again define a translation by
  induction on the structure of the formula. The only interesting case is the
  application of an operator. Let
  \begin{align*}
    \gamma' = \Omega_{E', \ar^*}[\vec{\pi}'][\vec{\psi}^D, \vec{\psi}^{\approx}] ((\vec{x}^i_1 \vec{\mu}^i_1, \ldots, \vec{x}^i_{r_i'} \vec{\mu}^i_{r_i})(\psi_i))_{i \in [r']}.
  \end{align*}
  be a sub-formula of $\theta'(\vec{x})$. Then define
  \begin{align*}
    \gamma = \Omega_{E, \ar} [\vec{\pi}] [\vec{\phi}^D, \vec{\phi}^{\approx}]  [((\vec{x}^i_1 \vec{\mu}^i_1, \ldots, \vec{x}^i_{r_i} \vec{\mu}^i_{r_i})(\chi_i))_{i \in [r]}&((\vec{x}^j_1 \vec{\nu}^j_1, \ldots, \vec{y}^j_{f_j} \vec{\nu}^j_{f_j})(\eta_j))_{j \in [f]}].
  \end{align*}
  where
  \begin{myitemize}
  \item $\vec{\pi}$ is a translation of $\vec{\pi}'$,
  \item $\vec{\phi} = (\phi^D_1, \ldots, \phi^D_{q})$, where for all $i \in
    [q]$, $\phi^D_i$ is the translation of $\psi^D_i$,
  \item $\vec{\phi}^{\approx} = (\phi^{\approx}_1, \ldots, \phi^{\approx}_q)$,
    where for all $i \in [q]$, $\phi^{\approx}_i$ is the translation of
    $\psi^D_i$,
  \item for all $j \in [f]$, $k \in [f_j]$, $\vec{y}^j_k = \vec{x}^{r + j}_k$
    and $\vec{\nu}^j_k = \vec{x}^{r+j}_k$, and
  \item for all $j \in [f]$, let $\vec{x}_j = \vec{x}^{r+i}_{r+f}$ and
    $\vec{\mu}_j = \vec{\mu}^{r+j}_{r+f}$, and $\eta_j :=
    \#_{\vec{x}\vec{\mu}}[\phi^D_{r+f},\phi^\approx_{r+f}](\psi_{r+j})_{j \in
      [f]}$.
  \end{myitemize}

  It is easy to show that $\gamma'$ and $\gamma$ have the same free variables
  and the same semantics, and so statement (3) follows. Statement (2) follows
  from a very similar proof, except in this case we have that for each $j \in
  [f]$, and $R_{j+r}'$ the sort $\zeta'(R_{j+r})(f_i + 1)$ is defined using only
  numbers. It follows that we can replace the application of the counting
  operator above with a formula that includes a fixed-point operator. Statement
  (4) follows from statement (1) and (3).
\end{proof}

\begin{lem}
  Let $L$ be an extension of $\FOC$ and let $\setop$ be a set of generalised
  operators. Suppose $L(\setop)$ has constant number-width. It follows that
  there is a family of Boolean-valued generalised operators that act on formulas
  $\setop'$ such that $L(\setop) \equiv L(\setop')$.
\end{lem}

\begin{proof}
  From Lemma~\ref{lem:op-to-op-on-formulas} we may assume, without a loss of
  generality, that the operators in $\setop_E$ acts on formulas. Let $E' :
  \nats^{m + 1} \times \fin[\tau] \ra \{0,1\}$ be defined such that $E'(\vec{a},
  b, \mathcal{A}) = 1$ if, and only if, $E(\vec{a}, \mathcal{A}) = b$ for all
  $\vec{a} \in \nats^m$, $b \in \nats$, $\mathcal{A} \in \fin[\tau]$.

  We first show that $L(\setop_E) \leq L(\setop_{E'})$. Let $\theta(\vec{x}) \in
  L(\setop_E)$. We prove this result by induction on the structure of
  $\theta(\vec{x})$. Let $\gamma$ be a sub-formula or sub-number-term in
  $\theta(\vec{x})$. The interesting case is where $\gamma$ is an application of
  an operator in $\setop_E$. Suppose then
  \begin{align*}
    \gamma = \Omega_{E, \ar}[\vec{\pi}][\vec{\phi}^D, \vec{\phi}^{\approx}] ((\vec{x}^i_1 \vec{\mu}^i_1, \ldots, \vec{x}^i_{r_i} \vec{\mu}^i_{r_i})(\phi_i))_{i \in [r]}.
  \end{align*}
  is a number term, and let $\psi(\vec{x})$ be a formula that contains $\gamma$
  and such that $\psi$ does not strictly contains a formula that contains
  $\gamma$. It follows that $\psi$ is of the form $\gamma X \delta$, where
  $\delta$ is a number term and $X$ is either `$=$' or `$\leq$'. Let
  $\vec{\epsilon}$ be a $t$-length vector of variables not in $\theta$ and
  assume we have already defined a number-term $\delta' \in L(\setop_{E'})$,
  that translates $\delta$. Let $\psi' = \exists \vec{\epsilon} \, \gamma' \land
  \, \vec{\epsilon} X \delta'$, where $\gamma'$ is a formula defined such that
  \begin{align*}
    \gamma' = \Omega_{E', \ar}[\vec{\pi}', (\epsilon_1 \cdot M + \ldots + \epsilon_t \cdot M^{t-1})][\vec{\psi}^D, \vec{\psi}^{\approx}] ((\vec{x}^i_1 \vec{\mu}^i_1, \ldots, \vec{x}^i_{r_i} \vec{\mu}^i_{r_i})(\psi_i))_{i \in [r]}
  \end{align*}
  where
  \begin{myitemize}
  \item $\vec{\pi}$ is a translation of $\vec{\pi}'$,
  \item $M = \#_x (x = x)$ and $M^{i}$ is the $i$th power of $M$,
  \item $\vec{\psi} = (\psi^D_1, \ldots, \psi^D_{q})$, where for all $i \in
    [q]$, $\psi^D_i$ is the translation of $\phi^D_i$,
  \item $\vec{\psi}^{\approx} = (\psi^{\approx}_1, \ldots, \psi^{\approx}_q)$,
    where for all $i \in [q]$, $\psi^{\approx}_i$ is the translation of
    $\phi^D_i$, and
  \item for all $i \in [r]$, $\psi_i$ is the translation of $\phi_i$.
  \end{myitemize}
  It can be shown that $\psi'$ translates $\psi$. It remains to show that
  $L(\setop_{E'}) \leq L(\setop_{E})$. The idea is to simply replace any
  application of a (Boolean-valued) operator in $\setop_{E'}$ with formula that
  tests for the equality. We omit the details here.
\end{proof}

It follows from Lemmas~\ref{} and~\ref{} that we may assume, without a loss of
generality, that so long as we are interested in logics that are extensions of
$\FOC$, any given set of generalised operators $\setop$ is normal.

% We have defined the extension of a logic by an operator so as to allow for the
% inclusion of man the operator we may also specify formulas (e.g.\
% $\vec{\phi}^D, \vec{\phi}^{\approx}$) that may be used to restrict the domain
% and introduce a quotienting operator.


% The definition of extending a logic by a family of operators the possibility
% of defining a many-sort interpretation as part of the application of the
% operator. In many cases extension of a logic by a family of Lindstr\"{o}m
% quantifiers does not allow for the taking of interpretations. We can similarly
% define for a logic $L$ and set of operators $\setop$ an extension
% $L^{~}(\setop)$ that is defined similarly, but that does not allow for any
% definition of a

% \begin{definition}
%   We say a logic $L(\setop)$ is \emph{closed under operator quotients} if
%   $L^{~}(\setop) = L(\setop)$.
% \end{definition}

% We state the following easy to prove lemma.

% \begin{lem}
%   The logics $\FP^{\nats}$, $\FPC$, and $\FPR$ are closed under operator
%   quotients.
%   \label{lem:logics-operator-quotients}
% \end{lem}



% For each $i \in [q]$ let $B_i = \phi^D_i / \phi^{\approx}_i$. Let $i \in
% [l_1]$. For each $j \in [r_i]$ let $e_j \in B_{\zeta(R_i)(j)}$. We define
% $R^B_i$ by $(e_1, \ldots, e_{r_i}) \in R^{B}_i$ if, and only if, there exists
% $(\vec{a}_1 \vec{m}_1, \ldots , \vec{a}_{r_i} \vec{m}_{r_i})$ where $\vec{a}_j
% \vec{m}_j \in e_j$ for all $j \in [r_i]$ and $\mathcal{A} \models \theta_i
% [\vec{a}_1\vec{m}_1, \ldots, \vec{a}_{r_i} \vec{m}_{r_i}]$. We define
% $F^{B}_j$ such that $F^B_j (e_1, \ldots, e_{f_j}) = k$ there exists
% $(\vec{a}_1, \vec{m}_1, \ldots , \vec{a}_{r_i} \vec{m}_{r_i})$ such that
% $\vec{a}_j \vec{m}_j \in e_j$ for all $j \in [f_i]$ where $\mathcal{A} \models
% (k = \nu_i [\vec{a}_1, \vec{m}_1, \ldots , \vec{a}_{r_i} \vec{m}_{r_i}])$. Let
% $\mathcal{B}$ be the structure defined from the above. We have $\mathcal{A}
% \models (k = ...)$ if, and only if, $E(\vec{\pi}^{\mathcal{A}},
% \vec{\phi}^{\mathcal{A}}, \mathcal{B}) = k$.


% \begin{definition}
%   Let $\Omega_{E, \ar}$ be a many-sorted operator. We say $\Omega_{E, \ar}$
%   \emph{operates on formulas} if there are no function symbols in the
%   vocabulary of $\Omega_{E, \ar}$. We say that $\Omega_{E, \ar}$ \emph{has no
%   parameters} if the parameter-width is zero. We say $\Omega_{E, \ar}$ is
%   \emph{numberless} if it operates on formulas and for all $s \in S$, $\ar(s,
%   2) = 0$. We call a numberless Boolean operator a \emph{many-sorted
%   quantifier}.
% \end{definition}

% \begin{lem}
%   Let $L$ be a logic and $\setop$ be a set of operators. There is a set of
%   operators $\setop^*$ each of which operate on formulas such that $L(\setop)
%   \equiv L(\setop^*)$.
% \end{lem}
% \begin{proof}
  
% \end{proof}

% It follows that we may assume, without a loss of generality, that a given set
% of operators act on formulas.


% We say that $L(\setop)$ is \emph{closed under operator quotients} if each
% query in $L(\setop)$ can be expressed in

% $\equiv L(\setop^-)$.

% We

% \begin{definition}
%   Let $\Omega$ be an operator. Let $\setop$ be a set of operators. Let
%   $\setop^- = \{\Omega^- : \Omega \in \setop\}$. We say that $L(\setop)$ is
%   \emph{closed under operator quotients} if $L(\setop) \equiv L(\setop^-)$.
% \end{definition}

% \begin{lem}
%   $\FPR$ and $\FPC$ are closed under operator quotients.
% \end{lem}

% \begin{lem}
  
% \end{lem}

% \subsection{Boolean-Valued Operators}
% We can add a number term parameter and the operator evaluates to a value equal
% to this number term.

% \subsection{Operators on Formulas}
% We can replace each number term with a formula with an extra type. Summation
% over this extra type gets you back to the previous formula.

% We may thus assume, without a loss of generality, that the operators only
% operate on formulas and are Boolean valued.

\subsection{Extended Quantifiers}
Many of the logics that are of interest to us, including $\FPC$ and $\FPR$, can
be defined as extension of $\FP^{\nats}$ by generalised operators. The study of
these logics often reduces to the study of a corresponding bounded-variable
infinitary logic without number sorts (e.g.\ $\mathcal{C}^\omega$).

We aim to develop a general way of associating with each logic extended by a set
of generalised operators with an appropriate infinitary logic. Before we do so,
we need to show how to associate a set of operators with an appropriate set of
quantifiers that make no reference to a number sort. In this subsection we
introduce \emph{extended quantifiers}.

Let $\tau = (R, S, \zeta)$ be a many-sorted vocabulary where $R = \{R_1, \ldots,
R_{r}\}$, and $S = \{s_1, \ldots, s_{q}\}$. For each $i \in [r]$ let $r_i$
denote the arity of $R_i$. Let $m \in \nats$, $E : \nats^{m} \times \fin[\tau]
\ra \{0,1\}$ be isomorphism-closed, and let $\ar : S \times [2] \ra \nats$. We
associate with the pair $(E, \ar)$ and tuple $(\vec{p}, n) \in \nats^{m} \times
\nats$ an \emph{extended quantifier} $Q^{E, \ar}_{\vec{p},n}$. For a logic $L$
the extension $L(Q^{E, \ar}_{\vec{p},n})$ is defined by extending the
formula-formation rule for $L$ as follows:
\begin{textbox}[13.8cm]
  For each $i \in [r]$, $l \in [r_i]$ let $s^i_l = \zeta (R_i)(l)$ and let
  $c^i_l = \ar(s^i_l, 2)$. Let $T^i_n := [n]^{c^i_l} \times , \ldots, \times
  [n]^{c^i_{r_i}}$ and let $T_n$ be the function that maps $i \in [r]$ to
  $T^i_n$. Let $\Upsilon_i : T^i_n \ra L(Q^{E, \ar}_{\vec{p}, n})$ and let
  $\vec{x}^i_l$ be an $\ar(s^i_l, 1)$-length tuple of element variables. Then
  $\phi := Q^{E, \ar}_{\vec{p}, n} [(\vec{x}^i_1, \ldots, \vec{x}^i_{r_i}) \cdot
  \Upsilon_i]_{i \in [r]}$ is a formula in $L(Q^{E, \ar}_{\vec{p}, n})$. Let
  $\free{\phi} := \bigcup_{i \in [r], a \in
    \dom(\Upsilon_i)}(\free{\Upsilon_i(a)} \setminus (\bigcup_{j \in
    [r_i]}\vec{x}^i_j))$.
\end{textbox}
The semantics of the formula $\phi$ is defined for a structure $\mathcal{A} \in
\fin[\rho]$ and assignment $\alpha$ to the free variables in $\phi$ as follows:
\begin{textbox}[13.8cm]
  For each $i \in [r]$ let
  \begin{align*} \chi_i (\vec{x}^i_1 \vec{\mu}^i_1, \ldots, \vec{x}^i_{r_i}
    \vec{\mu}^i_{r_i}) = \bigvee_{\vec{a}_1, \ldots,
      \vec{a}_{r_i}}[(\bigwedge_{j \in [r_i]} \vec{\mu}^i_j= \vec{a}_j) \land
    \upsilon_i(\vec{a}_1, \ldots, \vec{a}_{r_i})(\vec{x}^i_1, \ldots,
    \vec{x}^i_{r_i})].
  \end{align*}
  Let $\mathcal{I}$ be the interpretation, with trivial domain and equality
  formulas, and with relations formulas $(\chi_i)_{i \in [r]}$. Let $\mathcal{A}
  \models \phi$ if, and only if, $E(\vec{p}, (\mathcal{I}(\mathcal{A}), \alpha))
  = 1$.
\end{textbox}

*****NEED TO ADD A NOTION OF AN INTERPRETATION WITH AN ASSIGNMENT

We may associate a set of operators with a set of extended quantifiers. Let
$\setop$ be a set of generalised operators. Let $\setquant$ be the set of
extended quantifiers such that $Q \in \setquant$ if, and only if, there exists
$\Omega_{E, \ar} \in \setop$ with parameter-width $m$, $\vec{p} \in \nats^m$,
and $n \in \nats$ such that $Q = Q^{E, \ar}_{\vec{p}, n}$. We say that
$\setquant$ is the set of extended quantifiers corresponding to $\setop$.


****SEPARABLE QUANTIFIERS

\section{Translating Formulas to Substitution Programs}
We now introduce the notion of a substitution program. A substitution program is
a finite sequence of formulas such that each formula may contain second-order
variables that correspond to formulas that come later in the sequence. The idea
is that a substitution program is a `compact' way of writing a formula, one that
allows formulas to repeatedly apply a formula without having to rewrite it each
time. We use substitution programs as a step towards our translation to
circuits. We also show that each substitution program can be translated to an
equivalent formula of an infinitary logic.

\begin{definition}
  Let $L$ be a logic and $\rho$ be a vocabulary. A $L[\rho]$-\emph{substitution
    program} is a sequence of formulas $\Phi := (\phi_1(\vec{y}_1, \vec{\mu}_1;
  \boldsymbol{V}_1), \ldots, \phi_k(\vec{y}_k, \vec{\mu}_i;\boldsymbol{V}_k))$,
  such that there is a sequence $V_1, \ldots, V_k$ of second-order variables and
  for each $i \in [k]$:
  \begin{myitemize}
  \item $\phi_i$ is a formula in $L[\rho]$,
  \item $\vec{y}_i$ and $\vec{\mu}_i$ are sequences of element and number
    variables, respectively,
  \item $V_i$ has the same type as $(\vec{y}_i, \vec{\mu}_i)$, and
  \item $\boldsymbol{V}_i \subseteq \{V_j : i < j \leq k\}$.
  \end{myitemize}
  For each $i \in [k]$ the \emph{flattening} of $\Phi$ at $i$ is defined
  recursively as follows. If $i = k$ then the flattening of $\Phi$ at $i$ is
  $\phi_i$. If $i < k$ the flattening of $\Phi$ at $i$ is defined by replacing
  each second-order variable $V_j$ appearing in $\phi_i$ with the flattening of
  $\Phi$ at $j$. The \emph{flattening} of $\Phi$ is the flattening of $\Phi$ at
  $1$.

  Let $\alpha$ be an assignment to the variables $\vec{y}_1$ and $\vec{\mu}_1$.
  Let $\mathcal{A} \in \fin[\tau]$. We write $\mathcal{A} \models \Phi[\alpha]$
  to abbreviate $\mathcal{A} \models \phi[\alpha]$, where $\phi$ is the
  flattening of $\Phi$. The \emph{formula-length} of a substitution program
  $\Phi$ is the maximal length (i.e.\ number of symbols) of the formulas in
  $\Phi$. The \emph{variable-width} of a substitution program is the maximum
  variable-width of a formula appearing in the program.
\end{definition}

We often work with substitution programs where the sequence is indexed by a
finite linearly ordered set, rather than an initial segment of the natural
numbers.

The proof of the following result follows from the standard `unrolling' of the
fixed-point operators. For more detail please see~\cite{}.

\begin{lem}
  Let $\setop$ be a set of generalised operators. Let $\setquant$ be the
  associated family of extended quantifiers. If a query can be defined in
  $\FP^{\nats} (\setop)$ then it can be defined by a $\PT$-uniform family of
  $\FO^{\nats} (\setop)$-substitution programs with a constant bound on the
  width and formula-length.
  \label{lem:unroll-fixed-point}
\end{lem}

% \begin{lem}
%   Let $\setop$ be a set of unrestricted operators. For each $n \in \nats$ let
%   $\setquant$ be the corresponding set of number domain non-uniform
%   number-domain non-quotienting quantifiers generated by $\setop$. Let
%   $\theta(\vec{x}, \vec{\mu}) \in \FO^{\nats} (\setop)$ be a formula defined
%   in terms of the mixed-sort relation variables $V_1, \ldots, V_k$. Let $m =
%   \vert \vec{\mu} \vert$ and for each $i \in [k]$ let $v_i$ be the arity of
%   the element sort and let $m_i$ be the arity of the number sort in $V_i$. Let
%   $n \in \nats$. For each $i \in [k]$ and $\vec{b} \in [n]^{m_i}$ we define an
%   element-sort relation variable $V_{\i, vec{b}}$ of arity $v_i$. There exists
%   a sequence of $\FO(\setquant)$ formulas $\{\theta_{\vec{a}}(\vec{x}) :
%   \vec{a} \in [n]^{m}\}$ where each $\theta_{\vec{a}}(\vec{x})$ is defined in
%   terms of the relation variables $\bigcup_{i \in [k]} \{V_{i, \vec{b}} :
%   \vec{b} \in [n]^{m_i}\}$.
% \end{lem}
% \begin{proof}
%   We say that a number term $\eta(\vec{y}, \vec{\nu})$ in
%   $\FO^{\nats}(\setop)$ has a \emph{translation for $n$} if there is a
%   sequence of $\FO(\setquant)$-formulas $(\phi^{\eta}_{n, k;
%   \beta}(\vec{y}))_{\beta \in [n]^{\vec{\nu}}}$ such that for each
%   $\tau$-structure $\mathcal{A}$ of size $n$ and for each assignment $\alpha
%   \in [n]^{\vec{y}}$ we have that $\mathcal{A} \models \phi^{\eta}_{n, k;
%   \beta}[\alpha]$ if, and only if, $\mathcal{a} \models (\eta = k)[\alpha \cup
%   \beta]$.

%   We say a formula $\psi(\vec{y}, \vec{\nu})$ in $\FO^{\nats}(\setop)$ has a
%   \emph{translation for $n$} if there is a sequence of
%   $\FO(\setquant)$-formulas $(\phi_{n; \beta} (\vec{y}))_{\beta \in
%   [n]^{\vec{\mu}}}$ such that each $\tau$-structure $\mathcal{A}$ of size $n$
%   and each assignment $\alpha \in [n]^{\vec{y}}$, we have that $\mathcal{A}
%   \models \phi_{n; \vec{\beta}}[\alpha]$ if, and only if, $\mathcal{A} \models
%   \phi[\alpha \cup \beta]$.

%   We aim to show that $\theta(\vec{x}, \vec{\mu})$ has a \emph{translation for
%   $n$} by induction on the structure of the formula. It is easy to see that if
%   $\eta(\vec{y}, \vec{\nu})$ is a number variable or a constant (i.e. $0$ or
%   $1$) then $\eta$ has a translation for $n$. In the case that $\psi(\vec{y})$
%   is a formula is a formula containing no number terms then there is an
%   obvious translation for $n$. If $psi(\vec{y}, \vec{\nu}) = V_i(\vec{y},
%   \vec{\nu})$ then the assignment $\psi_{n; \beta}(\vec{y}) = V_{i,
%   \beta(\vec{\nu})}$ defines a translation.

%   Let $\eta(\vec{y}, \vec{\nu})$ be a number term and suppose all sub-formulas
%   and sub-number-terms of $\eta$ have a translation for $n$. Let $\beta \in
%   [n]^{\vec{\nu}}$. Suppose $\eta(\vec{y}, \vec{\nu}) = \eta_1(\vec{y}_1,
%   \vec{\nu}_1) \cdot \eta_2(\vec{y}_2, \vec{\nu}_2)$. Let $\beta_1 \in
%   [n]^{\vec{nu}_1}$ and $\beta_2 \in [n]^{\vec{nu}_2}$ be assignments
%   compatible with $\beta$. Let $\psi^{\eta}_{n, k ; \beta} (\vec{y}) =
%   \underset{a, b \leq k, a \cdot b = k}{\bigvee}(\psi^{\eta)_1}_{n, a;
%   \beta}(\vec{x}) \land \psi^{\eta_2}_{n, b; \beta}(\vec{x}))$. The other
%   arithmetic cases are handled similarly.

%   We now consider the application of an operator. Let $\Omega \in \setop$ and
%   suppose

%   \begin{align*}
%   \eta(\vec{y}, \vec{\nu}) = \Omega_{E} [\vec{\pi}] [((\vec{x}^i_1 \vec{\mu}^i_1, \ldots, \vec{x}^i_{r_i} \vec{\mu}^i_{r_i})(\chi_i))_{i \in [l]}].
% \end{align*}
% For each $i \in [l]$ let $T_i = [n]^{c_1} \times , \ldots, \times
% [n]^{c_{r_i}}$. Let $T$ be a function that maps $i \in [l]$ to $T_i$. where
% $c_{j}$ is the arity of ...
% \begin{align*}
%   \psi^{\eta}_{n, k; \beta}(\vec{x}) := \bigvee_{p_1, \ldots, p_{\vert \vec{\pi} \vert}, e \leq M}[&(\bigwedge_{i \leq \vert \vec{\pi}\vert} \psi^{\pi_{i}}_{n, p_i; \beta}(\vec{x})) \\ &\land Q_{E, T ; (p_1, \ldots, p_n); e} ((\vec{x}^{j, \vec{t}}_1, \ldots, \vec{x}^{j, \vec{t}}_{r_i}) \cdot (\chi_{i, \vec{t}})_{ n; \beta})_{j \in [l], \vec{t} \in T_j}]
% \end{align*}
% \end{proof}

% \begin{lem}
%   Let $\setop$ be a set of non-quotienting operators. Let $\setquant :=
%   \setquant_{\setop}$ be the corresponding set of extended quantifiers. If a
%   query can be defined by a $\PT$-uniform family of $\FO^{\nats} (\setop)$
%   substitution programs formulas then it can be defined by a $\PT$-uniform
%   family of $\FO (\setquant)$ substitution programs
% \end{lem}
% \begin{proof}
%   Let $\rho$ be a vocabulary. Let $(f_n : [M_n] \ra \FO^{\nats} \ra
%   \FO^{\nats}(\setop)[\tau])_{n \in \nats}$ be a $\PT$-uniform family of
%   $\FO^{\nats}(\setop)[\tau]$ of substitution programs.
% \end{proof}

% The proof of the following result follows from the standard `unrolling' of the
% fixed-point operators. For more detail please see~\cite{}.

% \begin{lem}
%   Let $\setop$ be a set of operators. If a query can be defined in
%   $\FP^{\nats} (\setop)$ then it can be defined by a $\PT$-uniform family of
%   $\FO^{\nats} (\setquant)$ substitution programs with constant program-width
%   and relation-width.
% \end{lem}

Let $L$ be a logic with constant numeric-width. Let $\rho$ be a vocabulary. We
say that a $\PT$-uniform family of $L[\rho]$-substitution programs $(\Phi_n)_{n
  \in \nats}$ is \emph{bounded} if there is a constant $k \in \nats$ such that
for each $n \in \nats$ the numeric-width and length of each formula in $\Phi_n$
is less than $k$. We aim to show that every bounded $\PT$-uniform family of
$\exlogic{\FO^{\nats}}{\setop}$-substitution programs can be translated to
equivalent $\PT$-uniform families of $\FO(\setquant)$-substitution programs. We
define a translation for a bounded $\PT$-uniform family of
$\exlogic{\FO^{\nats}}{\setop}$-substitution programs $(\Theta_n)_{n \in \nats}$
in two parts. We show in Lemma~\ref{} that for each $n \in \nats$ we can expand
each formula in $\Theta_n$ into an equivalent sequence of
$\FO(\setquant)$-formulas. We then show in Lemma~\ref{} that by expanding each
formula in $\Theta_n$ and then concatenating the resultant sequence we can
define a $\PT$-uniform family of $\FO(\setquant)$-substitution programs that
defines the same query as $(\Theta_n)_{n \in \nats}$.

We first define what it means to expand a formula in a substitution program into
an equivalent sequence of formulas. Let $\rho$ be a vocabulary and let $\setop$
be a set of operators. Let $\setquant$ be the corresponding set of extended
quantifiers. Let $\theta (\vec{x}, \vec{\mu}; \vec{V})$ be a formula in
$\exlogic{\FO^{\nats}}{\setop}[\rho]$ defined in terms of a sequence of
mixed-sort second-order variables $\vec{V} = (V_1, \ldots, V_v)$. For each $i
\in [v]$ let $v_i$ and $m_i$ be the arity of the element sort and number sort of
$V_i$.

Let $n \in \nats$. Let $\boldsymbol{\theta}_n = \{\theta_{n; \beta}(\vec{x};
\vec{W}^n) : \beta \in [n]^{\vec{\mu}} \}$ in $\FO(\setquant)[\rho]$, where
$\vec{W}^n = \{ V^n_{i, \vec{a}} : i \in [v], \vec{a} \in [n]^{m_i}\}$, and for
each $i \in [v]$, $\vec{a} \in [n]^{m_i}$ we have that $V^n_{i, \vec{a}}$ is an
element-sort second-order variable with arity $v_i$. We say
$\boldsymbol{\theta}_n$ \emph{translates $\theta$ for $n$} if for all
\begin{myenum}
\item $\mathcal{A} \in \fin[\rho, n]$,
\item assignments $\alpha \in A^{\vec{x}}$ and $\beta \in [n]^{\vec{\mu}}$, and
\item assignments $\gamma$ and $\gamma'$ to the second-order variables, where
  $\gamma$ maps $V_i$ to a relation $V^{\gamma}_i \subseteq A^{v_i} \times
  [n]^{m_i}$ and $\gamma'$ maps $V^n_{i, \vec{c}}$ to the relation $V^{n,
    \gamma'}_{i, \vec{c}} \subseteq A^{v_i}$ such that $(\vec{a}, \vec{c}) \in
  V^{\gamma}_{i}$ if, and only if, $\vec{a} \in V^{n, \gamma'}_{i, \vec{c}}$
  then
\end{myenum}
\vspace{-2.8mm}
\begin{align*}
  \mathcal{A} \models \theta [\alpha , \beta, \gamma]  \iff \mathcal{A} \models \theta_{n;\beta} [\alpha, \gamma'].
\end{align*}

We say $\boldsymbol{\theta} = (\boldsymbol{\theta}_n)_{n \in \nats}$ translates
$\theta$ if $\boldsymbol{\theta}_n$ translates $\theta$ for all $n \in \nats$.
We similarly define a \emph{translation} for number-terms. Let $\eta(\vec{x},
\vec{\mu}; \vec{V})$ be a number term and let $n \in \nats$. We say that
$\boldsymbol{\theta}^{\eta}_n = \{\theta^{\eta}_{n, k; \beta}(\vec{x},
\vec{V}^n_1, \ldots, \vec{V}^n_{v}) : k \in \nats, \beta \in [n]^{\vec{\mu}} \}$
\emph{translates} $\eta$ for $n$ if for all $k \in \nats$ and all $\mathcal{A}$,
$\alpha$, $\beta$, $\gamma$, and $\gamma'$ defined as above we have
\begin{align*}
  \eta^{(\mathcal{A}; \alpha, \beta, \gamma)} = k \iff \mathcal{A} \models \theta^{\eta}_{n, k; \beta}[\alpha, \gamma'].
\end{align*}

We are now ready to prove that a formula that appears in a
$\exlogic{\FO^\nats}{\setop}$-substitution program may be translated to a
sequence of $\FO(\setquant)$-formulas, where $\setquant$ is the set of extended
quantifiers associated with $\setop$. Moreover, we show that this translation
can be constructed by an algorithm that runs within a prescribed time.

\begin{lem}
  \label{lem:translate-program-line}
  Let $\rho$ be a vocabulary. Let $\setop$ be a set of normal operators and let
  $\setquant$ be the corresponding set of extended quantifiers. Suppose
  $\exlogic{\FO^{\nats}}{\setop}$ has constant numeric-width. Let
  $\theta(\vec{x}, \vec{\mu}; \vec{V}) \in \exlogic{\FO^{\nats}}{\setop}[\rho]$
  be a formula defined in terms of the mixed-sort second-order variables
  $\vec{V} = (V_1, \ldots, V_v)$. Let $t$ be the numeric-width and $w$ be the
  number-variable width of $\theta$.
  
  There exists a set of formulas $\boldsymbol{\theta} = \{\theta_{n;
    \beta}(\vec{x}; \vec{W}^n) : n \in \nats, \beta \in [n]^{\vec{\mu}} \}$ in
  $\FO(\setquant)[\rho]$ that translates $\theta$. We also have that each
  formula in $\boldsymbol{\theta}$ has variable-width less than the
  element-variable width of $\theta$. Moreover, the function that maps $(n,
  \theta)$ to $(\theta_{n, \beta})_{\beta \in [n]^{\vec{\mu}}}$ is computable in
  time $c^{\vert \cl{\theta} \vert} \cdot n^{c \cdot \vert \cl{\theta} \vert
    \cdot (t + w)}$ for some constant $c$.
\end{lem}
\begin{proof}
  Let $n \in \nats$. We aim to recursively define a translation for $n$ for each
  formula and number-term in $\theta$. That is, for a sub-formula or
  sub-number-term $\gamma(\vec{y}, \vec{\nu} ; \vec{V})$ of $\theta$ we define a
  translation of $\gamma$ for $n$ given translations of the sub-formulas and
  sub-number-terms of $\gamma$ for $n$. More precisely, if $\gamma$ is a
  number-term we will define a sequence of $\FO(\setquant)[\rho]$-formulas
  $\boldsymbol{\phi}^{\gamma}_n = \{\phi^{\gamma}_{n, k; \beta}(\vec{y};
  \vec{W}^n) : \beta \in [n]^{\vec{\nu}}, k \leq n^t \}$ that translates
  $\gamma$ for $n$ and if $\gamma$ is a formula we will define a sequence
  $\boldsymbol{\phi}^{\gamma}_n = \{\phi^{\gamma}_{n ; \beta}(\vec{y};
  \vec{W}^n): \beta \in [n]^{\vec{\nu}} \}$ that translates $\gamma$ for $n$. In
  these formulas $\vec{W}^n = \{ V^n_{i, \vec{a}} : i \in [v], \vec{a} \in
  [n]^{m_i}\}$. We prove this result by considering cases. In each case we let
  $\gamma_1, \gamma_2, \ldots$ denote the sub-formulas or sub-number-terms of
  $\gamma$ and we let $\beta$ denote an arbitrary assignments to the free number
  variables in $\gamma$.

  We first consider the base cases. Suppose $\gamma$ is a number-variable or a
  constant (i.e.\ $0$ or $1$). If $\gamma$ evaluates to $k$ under the assignment
  $\beta$ let $\phi^{\gamma}_{n, k; \beta}$ be a tautology and otherwise let
  $\phi^{\gamma}_{n, k; \beta}$ be a contradiction. Suppose $\gamma$ is an
  atomic formula or (non-number) term. If the free variables in $\psi$ are all
  element variables then let $\phi^{\gamma}_{n; \beta} = \gamma$. If
  $\gamma(\vec{y}, \vec{\nu}; \vec{V}) = V_i(\vec{y}, \vec{\nu})$ for some $i
  \in [v]$ let $\phi^{\gamma}_{n; \beta}(\vec{y}) = V^n_{i, \beta(\vec{\nu})}$.
  This suffices to give a translation of $\gamma$ for $n$ in this case.

  We now consider the multiplication case. The other arithmetic functions may be
  handled similarly. Suppose $\gamma(\vec{y}, \vec{\nu}; \vec{V})$ is a number
  term and suppose $\gamma(\vec{y}, \vec{\nu}; \vec{V}) = \gamma_1(\vec{y}_1,
  \vec{\nu}_1; \vec{V}) \cdot \gamma_2(\vec{y}_2, \vec{\nu}_2; \vec{V})$. Let
  $\beta_1 \in [n]^{\vec{\nu}_1}$ and $\beta_2 \in [n]^{\vec{\nu}_2}$ be
  compatible with $\beta$, then for each $k \leq n^{t}$ let
  \begin{align*}
    \phi^{\gamma}_{n, k ; \beta} =
    \uset{a, b \leq k}{ a \cdot b = k}{\bigvee}(\phi^{\gamma_1}_{n, a;
    \beta_1} \land \phi^{\gamma_2}_{n, b; \beta_2}).
  \end{align*}

  We now handle the less-than relation for number-terms. The equality relation
  may be handled similarly. Suppose $\gamma(\vec{y}, \vec{\nu}; \vec{V}) =
  \gamma_1(\vec{y}_1, \vec{\nu}_1; \vec{V}) \leq \gamma_2 (\vec{y}_2,
  \vec{\nu}_2 ; \vec{V})$. Let $\beta_1 \in [n]^{\vec{\nu}_1}$ and $\beta_2 \in
  [n]^{\vec{\nu}_2}$ be compatible with $\beta$, and let
  \begin{align*}
    \phi^{\gamma}_{n; \beta} = \uset{a, b \leq n^t}{a \leq
    b}{\bigwedge} (\phi^{\gamma_1}_{n, a; \beta_1} \land \phi^{\gamma_2}_{n, b;
    \beta_2}).
  \end{align*}
  
  We now handle conjunction between formulas. The other logical connectives may
  be handled similarly. Suppose $\gamma(\vec{y}, \vec{\nu}; \vec{V}) =
  \gamma_1(\vec{y}_1, \vec{\nu} ; \vec{V}) \land \gamma_2 (\vec{y}_2,
  \vec{\nu}_2 ; \vec{V})$. Let $\beta_1 \in [n]^{\vec{\nu}_1}$ and $\beta_2 \in
  [n]^{\vec{\nu}_2}$ be compatible with $\beta$. Let $\phi^{\gamma}_{n ; \beta}
  = \phi^{\gamma_1}_{n ; \beta_1} \land \phi^{\gamma_2}_{n ; \beta_2}$.

  We now handle the operator case. Let $\Omega_{E, \ar} \in \setop$ and suppose
  \begin{align*}
    \gamma(\vec{y}, \vec{\nu}; \vec{V}) = \Omega_{E, \ar} [\vec{\pi}] [((\vec{x}^i_1 \vec{\mu}^i_1, \ldots, \vec{x}^i_{r_i} \vec{\mu}^i_{r_i})(\chi_i))_{i \in [r]}].
  \end{align*}
  For each $i \in [r]$, $l \in [r_i]$ let $s^i_l = \zeta(R_i)(l)$ and $c^i_l =
  \ar(s^i_l, 2)$. Let $T^i_n := [n]^{c^i_l} \times , \ldots, \times
  [n]^{c^i_{r_i}}$. Let $\Upsilon_i : T^n_i \ra \FO(\setquant)[\rho]$ be defined
  by $\Upsilon_i (\vec{m}_1, \ldots, \vec{m}_{r_i}) = \phi^{\chi_i}_{n ;
    \beta_i}$, where $\beta_i$ is an assignments to all the free number
  variables in $\chi_i$ such that (i) $\beta_i$ agrees with $\beta$ for each
  variable in $\vec{\nu}$ not bound by the operator and (ii) $\beta_i$ assigns
  $\vec{\mu}^i_{j}$ to $\vec{m}_j$ for each $j \in [r_i]$. For each $i \in
  [\vert \pi \vert]$ let $\alpha_i$ be an assignment to the free number
  variables in $\vec{\pi}(i)$ compatible with $\beta$. Let
  \begin{align*}
    \phi^{\gamma}_{n; \beta} := \bigvee_{p_1, \ldots, p_{\vert \vec{\pi} \vert} \leq n^t}[(\bigwedge_{i \in [\vert \vec{\pi}\vert]} \phi^{\pi(i)}_{n, p_i; \alpha_i}) \land Q^{E, \ar}_{n, (p_1, \ldots, p_{\vert \vec{\pi} \vert})} ((\vec{x}^{j}_1, \ldots, \vec{x}^{j}_{r_j}) \cdot \Upsilon_j)_{j \in [r]}].
  \end{align*}
  
  We can think of the $\forall$ and $\exists$ quantifiers as generalised
  operators, and so these case are subsumed by the operator case. There is an
  obvious recursive algorithm implementing the construction above. It can be
  shown by reviewing each of the above cases that the cost of computing the
  translation for a sub-formula or sub-number-term $\gamma (\vec{y}, \vec{\nu} ;
  \vec{V})$ is at most $(c_1 \cdot n^{c_2 \cdot (t_{\gamma} + w_{\gamma} )}
  \cdot X)$, where $c_1$ and $c_2$ are constants, $t_{\gamma}$ and $w_{\gamma}$
  are the numeric-width of $\gamma$ and number-variable width, respectively, and
  $X$ is the maximal cost of translating a sub-formula or sub-number-term of
  $\gamma$. It follow that for a given $n$ and $\theta$ we can construct a
  family of formulas $\boldsymbol{\theta}_n$ that translates $\theta$ for $n$ in
  at most $c^{\vert \cl{\theta} \vert}_1 \cdot n^{c_2 \vert \cl{\theta} \vert
    \cdot (t + w)}$. The result follows.
\end{proof}

% \begin{lem}
%   Let $\setop$ be a set of non-quotienting operators. Let $\setquant :=
%   \setquant_{\setop}$ be the corresponding set of extended quantifiers. If a
%   query can be defined by a $\PT$-uniform family of $\FO^{\nats} (\setop)$
%   substitution programs with constant program-width and relation-width then it
%   can be defined by a $\PT$-uniform family of $\FO (\setquant)$ substitution
%   programs.
% \end{lem}
% \begin{proof}
%   Let $\rho$ be a vocabulary. Let $I = (I_n : [M_n] \ra \FO^{\nats} \ra
%   \FO^{\nats}(\setop)[\tau])_{n \in \nats}$ be a $\PT$-uniform family of
%   substitution programs. Let $n \in \nats$. For each $w \in [M_n]$ let
%   $\theta_w (\vec{x}) := I_n(w)$. For each mixed-sort second-order variable
%   $V_i$ with arity $(v_i, m_i)$ we define a family of element-sort
%   second-order variables $\mathbb{V}_i = \{V_{i, \vec{a}} : \vec{a} \in
%   [n]^{m_i}\}$ such that $v_{i, \vec{a}}$ is the arity of $V_{i, \vec{a}}$.

%   Fix $w \in [M_n]$ and let $\theta_{n,w} (\vec{x}, \vec{\mu}) := I_n(w)$. We
%   aim to define for each number term $\eta(\vec{y}, \vec{\nu})$ appearing in
%   $\theta_w(\vec{x}, \vec{\mu})$ a sequence of $\FO(\setquant)[\rho]$-formulas
%   $\{\phi^{\eta}_{n, w; k; \beta}(\vec{y}))_{\beta \in [n]^{\vec{\nu}}}) :
%   \beta \in [n]^{\vec{\nu}}, k \in \nats\}$.
  
%   We say that a number term $\eta(\vec{y}, \vec{\nu})$ in
%   $\FO^{\nats}(\setop)$ has a \emph{translation for $n$} if there is a
%   sequence of $\FO(\setquant)$-formulas $(\phi^{\eta}_{n, k;
%   \beta}(\vec{y}))_{\beta \in [n]^{\vec{\nu}}}$ such that for each
%   $\rho$-structure $\mathcal{A}$ of size $n$ and for each assignment $\alpha
%   \in [n]^{\vec{y}}$ we have that $\mathcal{A} \models \phi^{\eta}_{n, k;
%   \beta}[\alpha]$ if, and only if, $\mathcal{a} \models (\eta = k)[\alpha \cup
%   \beta]$.

%   We say a formula $\psi(\vec{y}, \vec{\nu})$ in $\FO^{\nats}(\setop)$ has a
%   \emph{translation for $n$} if there is a sequence of
%   $\FO(\setquant)$-formulas $(\phi_{n; \beta} (\vec{y}))_{\beta \in
%   [n]^{\vec{\mu}}}$ such that each $\tau$-structure $\mathcal{A}$ of size $n$
%   and each assignment $\alpha \in [n]^{\vec{y}}$, we have that $\mathcal{A}
%   \models \phi_{n; \vec{\beta}}[\alpha]$ if, and only if, $\mathcal{A} \models
%   \phi[\alpha \cup \beta]$.

%   We aim to show that $\theta(\vec{x}, \vec{\mu})$ has a \emph{translation for
%   $n$} by induction on the structure of the formula. It is easy to see that if
%   $\eta(\vec{y}, \vec{\nu})$ is a number variable or a constant (i.e. $0$ or
%   $1$) then $\eta$ has a translation for $n$. We also have that if
%   $\psi(\vec{y})$ is a formula then $\psi(\vec{y})$ is its own translation for
%   $n$. If $psi(\vec{y}, \vec{\nu}) = V_i(\vec{y}, \vec{\nu})$ then the
%   assignment $\psi_{n; \beta}(\vec{y}) = V_{i, \beta(\vec{\nu})}$ defines a
%   translation for $n$.

%   Let $\eta(\vec{y}, \vec{\nu})$ be a number term and suppose all sub-formulas
%   and sub-number-terms of $\eta$ have a translation for $n$. Let $\beta \in
%   [n]^{\vec{\nu}}$. Suppose $\eta(\vec{y}, \vec{\nu}) = \eta_1(\vec{y}_1,
%   \vec{\nu}_1) \cdot \eta_2(\vec{y}_2, \vec{\nu}_2)$. Let $\beta_1 \in
%   [n]^{\vec{nu}_1}$ and $\beta_2 \in [n]^{\vec{nu}_2}$ be assignments
%   compatible with $\beta$. Let $\psi^{\eta}_{n, k ; \beta} (\vec{y}) =
%   \underset{a, b \leq k, a \cdot b = k}{\bigvee}(\psi^{\eta)_1}_{n, a;
%   \beta}(\vec{x}) \land \psi^{\eta_2}_{n, b; \beta}(\vec{x}))$. The other
%   arithmetic cases are handled similarly.

%   We now consider the application of an operator. Let $\Omega \in \setop$ and
%   suppose

%   \begin{align*}
%   \eta(\vec{y}, \vec{\nu}) = \Omega_{E} [\vec{\pi}] [((\vec{x}^i_1 \vec{\mu}^i_1, \ldots, \vec{x}^i_{r_i} \vec{\mu}^i_{r_i})(\chi_i))_{i \in [l]}].
% \end{align*}
% For each $i \in [l]$ let $T_i = [n]^{c_1} \times , \ldots, \times
% [n]^{c_{r_i}}$. Let $T$ be a function that maps $i \in [l]$ to $T_i$. where
% $c_{j}$ is the arity of ...x
% \begin{align*}
%   \psi^{\eta}_{n, k; \beta}(\vec{x}) := \bigvee_{p_1, \ldots, p_{\vert \vec{\pi} \vert}, e \leq M}[&(\bigwedge_{i \leq \vert \vec{\pi}\vert} \psi^{\pi_{i}}_{n, p_i; \beta}(\vec{x})) \\ &\land Q_{E, T ; (p_1, \ldots, p_n); e} ((\vec{x}^{j, \vec{t}}_1, \ldots, \vec{x}^{j, \vec{t}}_{r_i}) \cdot (\chi_{i, \vec{t}})_{ n; \beta})_{j \in [l], \vec{t} \in T_j}]
% \end{align*}


% \end{proof}
Let $L$ be a logic with constant numeric-width. Let $\rho$ be a vocabulary. We
say that a $\PT$-uniform family of $L[\rho]$-substitution programs $(\Phi_n)_{n
  \in \nats}$ is \emph{bounded} if there is a constant $k \in \nats$ such that
for each $n \in \nats$ the numeric-width and length of each formula in $\Phi_n$
is less than $k$.

\begin{lem}
  Let $\rho$ be a vocabulary and let $\setop$ be a set of normal operators such
  that $\FO^{\nats}(\setop)$ has constant numeric-width. Let $\setquant$ be the
  corresponding set of quantifiers. If a query can be defined by a bounded
  $\PT$-uniform family of $\FO^{\nats} (\setop)[\rho]$-substitution programs
  then it can be defined by a $\PT$-uniform family of $\FO
  (\setquant)[\rho]$-substitution programs with constant width.
  \label{lem:translate-sub-operators-to-queries}
\end{lem}
\begin{proof}
  Let $\Theta := (\Theta_{n})_{n \in \nats}$ be a bounded $\PT$-uniform family
  of $\FO^{\nats}(\setop)[\rho]$-substitution programs. For each $n \in \nats$
  let $M_n := \dom(\Theta_n)$. For each $n \in \nats$, $i \in M_n$ let
  $\vec{x}^n_i$ and $\vec{\mu}^n_i$ be the free element variables and number
  variables in $\Theta_n(i)$. Let $V^n_i$ be a mixed-sort second-order variable
  with type $(\vec{x}^n_i, \vec{\mu}^n_i)$. We think of $V^n_i$ as the
  second-order variable associated with the formula $\Theta_n(i)$. For each $n
  \in \nats$, $i \in M_n$, and $\vec{a} \in [n]^{\vert \vec{\mu}^n_i \vert}$ let
  $V^n_{i ; \vec{a}}$ be an element-sort second-order variable with arity $\vert
  \vec{x}^n_i\vert$. Let $\theta_{n, i}(\vec{x}^n_i, \vec{\mu}^n_i; \vec{V}^n_i)
  := \Theta_n(i)$, where $\vec{V}^n_i = (V^n_j)_{j > i}$. Let
  $\boldsymbol{\theta}_{n, i} := \{\theta_{n, i; \beta} (\vec{x}^n_i
  ,\vec{\mu}^n_i ; \vec{W}^n_{i}) : \vec{a} \in [n]^{\vert \vec{\mu}^n_i
    \vert}\}$ be a translation of $\theta_{n, i}$ for $n$, where $\vec{W}^n_i =
  \{V^{n}_{j; \vec{a}} : j \in M_n, j > i, \vec{a} \in [n]^{\vert \vec{\mu}^n_j
    \vert}\}$. Importantly, it follows from
  Lemma~\ref{lem:translate-program-line} and the fact that $\Theta$ is bounded
  that there is a an algorithm that runs in time polynomial in $n$ that takes a
  formula $\theta_{n, i}$ and outputs the translation $\boldsymbol{\theta}_{n,
    i}$ for $n$.

  Let $M_n' = \{(i, \vec{a}) : i \in M_n, \vec{a} \in [n]^{\vert \vec{\mu}^n_i
    \vert}\}$. Let $\Theta_n' : M_n' \ra \FO(\setquant)[\rho]$ be defined by
  $\Theta_n' (i, \vec{a}) := \theta_{n, i; \beta}$, where $\beta$ maps
  $\vec{\mu}^n_i$ to $\vec{a}$, for each $(i, \vec{a}) \in M_n'$. Let $\Theta' =
  (\Theta_n')_{n \in \nats}$. Since $M_n'$ is a product of two linearly ordered
  sets, we can define a lexographical order on $M_n'$. Let $i, j \in M_n$ and
  suppose $i > j$. Then, from the definition of a substitution program,
  $\theta_{n, i}$ does not contain the second-order variable $V^n_{j}$. It
  follows from the definition of a translation that for all $\vec{a} \in
  [n]^{\vert \vec{\mu}^n_i \vert}$ and $\vec{b} \in [n]^{\vert \vec{\mu}^n_j
    \vert}$ the translation $\theta_{n, i ; \beta}$, where $\beta$ is the
  assignment that maps $\vec{\mu}^n_i$ to $\vec{a}$, does not contain any
  second-order variable $V^n_{j ; \vec{b}}$. Moreover, for a fixed $n \in \nats$
  and $i \in M_n$ all of the formulas in $\boldsymbol{\theta}_{n, i}$ are
  defined in terms of the same set of second-order variables. It follows that
  for all $(i, \vec{a}), (j, \vec{b}) \in M_n'$, $(i, \vec{a}) < (j, \vec{b})$
  $\Theta_n'(i, \vec{a})$ does not contain the second-order variable $V^n_{j;
    \vec{b}}$. In other words, $\Theta_n'$ is a well-formed substitution
  program.

  \begin{claim}
    For each $n \in \nats$ let $\theta_n(\vec{x}^n_1, \vec{\mu}^n_1)$ be the
    flattening of $\Theta_n$ and for each $\vec{c} \in [n]^{\vert
      \vec{\mu}^n_1\vert}$ let $\theta_{n, \vec{c}}'(\vec{x}^n_1)$ be the
    flattening of $\Theta_n'$ at $(1, \vec{c})$. Let $\mathcal{A} \in \fin[\rho,
    n]$. Let $\alpha \in A^{\vec{x}^n_1}$ and let $\beta \in
    [n]^{\vec{\mu}^n_1}$. Then $\mathcal{A} \models \theta_n [\alpha, \beta]$
    if, and only if, $\mathcal{A} \models \theta_{n,
      \beta(\vec{\mu}^n_1)}[\alpha]$.
  \end{claim}
  \begin{proof}
    Let $n \in \nats$. For each $i \in M_n$ and $\vec{c} \in [n]^{\vert
      \vec{\mu}^n_i\vert}$ let $\phi_{n, i}$ be the flattening of $\Theta_n$ at
    $i$ and let $\phi_{n, (i, \vec{c})}'$ be the flattening of $\Theta_n'$ at
    $(i, \vec{a})$. Let $\mathcal{A} \in \fin[\rho, n]$. We will prove by
    backwards induction that for each $i \in M_n$, $\vec{c} \in [n]^{\vert
      \vec{\mu}^n_i\vert}$, and assignments $\alpha \in A^{\vec{x}^n_i}$ and
    $\beta$ that maps $\vec{\mu}^n_i$ to $\vec{c}$, we have that $\mathcal{A}
    \models \phi_{n, i}[\alpha, \beta]$ if, and only if, $\mathcal{A} \models
    \phi_{n, (i, \vec{c})}[\alpha]$.

    % Let $n \in \nats$, $(i, \vec{a}) \in M_n'$ let $M_{n, (i, \vec{a})}' =
    % \{(j,
    % \vec{b}) : (i, \vec{a}) \leq (j, \vec{b})\}$ and let $M_{n, i} = \{j : i
    % \leq j\}$. Let $\Psi_{n, (i, \vec{a})}'$ be the restriction of $\Theta_n'$
    % to $M_{n, (i, \vec{a})}$ and let $\Psi_{n, i}$ be the restriction of
    % $\Theta_n$ to $M_{n, i}$. We now prove that for each $(i, \vec{a}) \in
    % M_n'$
    % the flattening of $\Psi_{n, \vec(i, \vec{a})}'$ defines the same query as
    % the flattening of $\Psi_{n, i}$ with the assignment that $\vec{\mu}^n_i$
    % to
    % $\vec{a}$. The proof is by backwards induction. Let $i \in M_n$ and let
    % $\vec{a} \in [n]^{\vert \vec{\mu}^n_i \vert}$. Let $\phi_{n, i}$ be the
    % flattening of $\Phi_{n, i}$ and let $\phi_{n, (i, \vec{a})}$ be the
    % flattening of $\Phi_{n, (i, \vec{a})}$. Let $\mathcal{A} \in \fin[\rho,
    % n]$
    % and suppose we have assignments $\alpha \in A^{\vec{x}^n_i}$ and $\beta$
    % that maps $\vec{\mu}^n_i$ to $\vec{a}$.

    We begin with the base case. Suppose $i$ is the maximum value in $M_n$. Then
    no second-order variables appear in $\Theta_n(i)$ and hence, from the
    definition of a translation for $n$, no second-order variables in
    $\Theta_n'(i, \vec{c})$. It follows that $\phi_{n, i} = \Theta_n(i)$ and
    $\phi_{n, (i, \vec{c})}' = \Theta_{n}(i, \vec{c})$. Moreover, from the
    definition of a translation for $n$, we have that $\mathcal{A} \models
    \Theta_{n}(i)[\alpha, \beta]$ if, and only if, $\mathcal{A} \models
    \Theta_{n}'(i, \vec{c}) [\alpha, \beta]$. The base case follows.

    Suppose $i$ is not the maximal value in $M_n$ and the inductive proposition
    holds for all $j > i$. Let $\gamma$ an assignment that maps each mixed-sort
    second-order variable $V^n_j$ to the relation defined by $(\phi_{n,
      j})^{\mathcal{A}}$ for each $j > i$. Let $\gamma'$ be the assignment that
    maps each element-sort second-order variable $V^n_{j, \vec{c}}$ to the
    relation defined by $(\phi_{n, (j, \vec{c})})^{\mathcal{A}}$ for each $j >
    i$ and $\vec{c} \in [n]^{\vert \vec{\mu}^n_j \vert}$.

    Let $j > i$, $\vec{c}\in [n]^{\vert \vec{\mu}^n_j \vert}$, $\alpha \in
    A^{\vec{\mu}^n_j}$ and $\beta = \beta^{\vec{\mu}^n_j}_{\vec{c}}$. We notice
    the following. From the induction hypothesis we have that $\mathcal{A}
    \models \phi_{n, (j, \vec{c})}' [\alpha]$ if, and only if, $\mathcal{A}
    \models \phi_{n, j}[\alpha, \beta]$, where $\beta$ maps $\vec{\mu}^n_j$ to
    $\vec{c}$. It follows from the definition of $\gamma$ and $\gamma'$ that
    $(\vec{a}, \vec{c}) \in V^n_{j}$ if, and only if, $\vec{a} \in V^n_{j,
      \vec{c}}$. We have
    \begin{align*}
      \mathcal{A} \models \phi_{n, i} [\alpha, \beta] \iff \mathcal{A} \models \theta_{n, i}[\alpha, \beta, \gamma] \iff \mathcal{A} \models \theta_{n, i ; \beta}[\alpha, \gamma'] \iff \mathcal{A} \models \phi_{n, (i, \vec{c})}[\alpha],
    \end{align*}
    where the first and third equivalences follow from the definition of a
    flattening and the second equivalence follows from the definition of a
    translation for $n$. The result follows by induction.
  \end{proof}

  We have for each $n \in \nats$ that $\Theta_n(1)$ has no free number
  variables. Let $\theta_n$ and $\theta_n'$ be the flattening of $\Theta_n$ and
  $\Theta_n'$. It follows that no free number variables appear in $\theta_n$ and
  $\theta_n'$. Then, from the claim, we have that $\mathcal{A} \models
  \theta_n[\alpha]$ if, and only if $\mathcal{A} \models \theta_n'[\alpha]$ for
  each $\mathcal{A} \in \fin[\rho, n]$ and assignment $\alpha \in
  A^{\vec{x}^n_1}$. It follows that $\Theta'$ defines the same query as
  $\Theta$.

  Since $\Theta$ is $\PT$-uniform the function that maps $n$ to $\Theta_n$ is
  computable in time polynomial in $n$. Moreover, the function that maps any
  $\Theta_n(i)$ (for $n \in \nats$, $i \in M_n$) to the translation of
  $\Theta_n(i)$ for $n$ is computable in time polynomial in $n$. It follows that
  the function $n \mapsto \Theta_n'$ is computable in time polynomial in $n$. We
  also have from Lemma~\ref{lem:translate-program-line} that there is a constant
  bound on the width of $\Theta'$. In summary, we have that $\Theta'$ is a
  $\PT$-uniform family of $\FO(\setquant)[\rho]$-substitution programs with
  constant width and $\Theta'$ defines the same query as $\Theta$. The result
  follows.
\end{proof}

\subsection{Infinitary Logics}
Let $\setquant$ be a set of extended quantifiers. We define the extension of
$\mathcal{L}$ by $\setquant$ by extending the formula formation rules for
$\mathcal{L}$ with the following rule:

\begin{textbox}[14.0cm]
  Let $Q^{E, \ar}_{\vec{a}, n} \in \setquant$, $i \in [r]$, $l \in [r_i]$ let
  $c^i_l = \ar(\zeta(R_i)(l), 2)$ and let $T^i_n := [n]^{c^i_l} \times , \ldots,
  \times [n]^{c^i_{r_i}}$. Let $T_n$ be the function that maps $i \in [r]$ to
  $T^i_n$. Let $\Upsilon_i : T^i_n \ra L(Q^{E, \ar}_{\vec{p}, n})$. For each $i
  \in [r]$ and $l \in [r_i]$ let $s = \zeta (R_i)(l)$ and let $\vec{x}^i_l$ be
  an $\ar(s, 1)$-length tuple of element variables. Then $Q^{E, \ar}_{\vec{p},
    n} [(\vec{x}^i_1, \ldots, \vec{x}^i_{r_i}) \cdot \Upsilon_i]_{i \in [r]}$ is
  a formula in $\mathcal{L}(\setquant)$.
\end{textbox}

\begin{prop}
  Let $\rho$ be a vocabulary and $\setop$ be a set of operators. Let $\setquant$
  be the corresponding set of extended quantifiers. Then each query definable in
  $\FP^{\nats}(\setop)[\rho]$ can be defined by a bounded $\PT$-uniform family
  of $\FO(\setquant)[\rho]$ substitution programs and each query definable by a
  $\PT$-uniform family of $\FO(\setquant)[\rho]$ is definable by a formula of
  $\mathcal{C}^{\omega}(\setquant)[\rho]$.
\end{prop}
\begin{proof}
  The first part of the result follows from Lemma~\ref{}. Let $(\Phi_n)_{n \in
    \nats}$ be a $\PT$-uniform family of $\FO(\setquant)[\rho]$-substitution
  programs with constant width. Let $(\phi_n)_{n \in \nats}$ be such that
  $\phi_n$ is the flattening of $\Phi_n$ for all $n \in \nats$. We can rename
  variables such that there exists a sequence of element variables $\vec{x}$
  such that for each $n \in \nats$ the free element variables in $\phi_n$ are
  amoung those in $\vec{x}$. Let $\theta (\vec{x}) = \bigwedge_{n \in \nats}
  (\exists^{=n} \, x = x) \land \phi_n (\vec{x})$. It follows from the fact that
  $\Phi$ has constant width that $\theta \in \mathcal{L}^\omega(\setquant)$.
  Moreover, $\theta$ and $(\Phi_n)_{n \in \nats}$ define the same query. The
  second part of the result follows.
  % **problem, we need some sort of width condition
  % **problem, we need counting
  % **problem, each $\phi_n$ must have the same set of free variables
\end{proof}

We should work through two examples. Let $\setop_C$ be the set of all counting
operators.

\section{Translating Formulas to Circuits}

% \begin{lem}
%   Let $\setquant$ be a set of quantifiers. There is a function that maps a
%   $\FO(\setquant)$ formula $\theta(\vec{x})$ and $n \in \nats$ to a circuit
%   $C_n$ that translates $\theta(\vec{x})$ for $n$. This function can be
%   computed by an algorithm that runs in time $p(n^{\width(\theta)} \vert \cl
%   (\theta)\vert)$, for some polynomial $p$.
% \end{lem}

% \begin{prop}
%   Let $\rho$ be a vocabulary and let $\setquant$ be a family of quantifiers.
%   Let $\BB^{\setquant}$ be the basis associated with $\setquant$. If a query
%   can be defined by a $\PT$-uniform family of $\FO(\setquant)$ substitution
%   programs then it can be defined by a $\PT$-uniform family of transparent
%   symmetric $(\BB^{\setquant}, \rho)$-circuits that define $Q$.
% \end{prop}
% \begin{proof}
% \end{proof}
In this subsection we show that for each formula of a logic extended by a family
of operators may define a $\PT$-uniform family of transparent symmetric circuits
that defines the same query. We may associate a set of operators or
extended-quantifiers with a Boolean basis. We will show in this subsection that
a
\begin{definition}
  Let $\tau$ be a many-sorted vocabulary. Let $q$ be the number of sorts in
  $\tau$. Let $m \in \nats$. Let $E : \nats^m \times \fin[\tau] \ra \{0,1\}$ be
  an isomorphism-closed function and let $\vec{p} \in \nats^m$. Let $\BB_{E,
    \vec{p}} := \{F_{E, \vec{p}}: \{0,1\}^{\ind(\tau, \vec{d})}\ra \{0,1\} :
  \vec{d} \in \nats^{q}, \forall \mathcal{A} \in \str{\tau, \vec{d}}, F_{E,
    \vec{p}}(\mathcal{A}) = E(\vec{p}, \mathcal{A})\}$.

  Let $\setquant$ be a set of extended-quantifiers. Let $\BB_{\setquant}$ be the
  union of all bases $\BB_{E', \vec{p}'}$ such that for some arity function
  $\ar'$ and $n' \in \nats$ we have $Q^{E', \ar'}_{\vec{p}', n'} \in \setquant$.
  We say that $\BB_{\setquant}$ is the basis \emph{corresponding} to
  $\setquant$. Let $\setop$ be a set of operators and let $\setquant$ be the
  corresponding set of quantifiers. We let $\BB_{\setop} := \BB_{\setquant}$ and
  say that $\BB_{\setop}$ is the basis \emph{corresponding} to $\setop$.
\end{definition}

Let $\setquant$ be a set of quantifiers and let $\BB$ be the corresponding
Boolean basis. Let $\rho$ be a vocabulary and let $C$ be a $(\BB,
\rho)$-circuit. Let $\theta(\vec{x})$ be a $\FO(\setquant)[\rho]$-formula. We
say that $C$ \emph{translates} $\theta(\vec{x})$ for $n \in \nats$ if $C$ is a
transparent symmetric circuit of order $n$ and for all $\mathcal{A} \in
\fin[\rho, n]$, $\gamma in [n]^{\underline{A}}$, and $\alpha \in A^{\vec{x}}$ we
have that $\gamma (\alpha (\vec{x})) \in C[\gamma \mathcal{A}]$ if, and only if,
$\mathcal{A} \models \theta[\alpha]$. We say a family of circuits $(C_n)_{n \in
  \nats}$ \emph{translates} a $\theta(\vec{x})$ if $(C_n)_{n \in \nats}$ is
$\PT$-uniform and for all $n \in \nats$, $C_n$ translates $\theta(\vec{x})$ for
$n$.

\begin{lem}
  Let $\vec{V} := \{V_1, \ldots, V_v\}$ be a vocabulary. Let $\rho := \{R_1,
  \ldots, R_r\}$ be a distinct non-empty vocabulary and let $\rho^* = \vec{V}
  \cup \rho$. Let $\setquant$ be a set of extended quantifiers and let $\BB$ be
  the corresponding basis. There is a function that takes as input a number $n
  \in \nats$ and a $\FO(\setquant)[\rho^*]$-formula $\theta(\vec{x})$ and
  outputs a $(\BB, \rho^*)$-circuit $C$ that translates $\theta$ for $n$. This
  function is computable and there is a polynomial $p$ such that for an input
  $(n, \theta)$ the algorithm computing this function terminates in at most
  $p(\vert \cl{\theta} \vert (\vert \cl{\theta} \vert + \ewidth{\theta}) \cdot
  n^{\ewidth{\theta}})$ many steps.
  \label{lem:translating-FOquant-to-formulas}
\end{lem}
\begin{proof}

  % Suppose first that $\rho^*$ is empty, so all of the
  % atomic formulas in $\theta(\vec{x})$ are of the form $y_1 = y_2$, where
  % $y_1$
  % and $y_2$ are variables. It can be shown that for $\vec{a} \in n^k$, we can
  % evaluate $[n] \models \theta[\alpha^{\vec{x}}_{\vec{a}}]$ in time polynomial
  % in $n^{k} \vert \cl{\theta} \vert$. Let $f : [n]^k \rightarrow \{0,1\}$ be
  % defined such that $f (\vec{a}) = 1$ if, and only if, $[n] \models
  % \theta[\alpha^{\vec{x}}_{\vec{a}}]$. Let $C := \langle G, \Sigma, \Omega,
  % \Lambda, L \rangle$, where
  % \begin{myitemize}
  % \item $G = \{g_0, g_1\} \cup \{g_{\vec{a}} : \vec{a} \in [n]^{k}\}$,
  % \item $\Sigma (g_0) = 0$, $\Sigma (g_1) = 1$ and $\Sigma (g_{\vec{a}}) =
  %   \AND[1]$ for all $\vec{a} \in [n]^k$,
  % \item $\Omega : [n]^k \rightarrow G$ is defined by $\Omega(\vec{a}) =
  %   g_{\vec{a}}$ for all $\vec{a} \in [n]^k$,
  % \item $\Lambda = \emptyset$, and
  % \item for $\vec{a} \in [n]^k$, $L(g_{\vec{a}}) : [1] \rightarrow G$ is
  %   defined
  %   by $L(g_{\vec{a}})(1) = g_{f(\vec{a})}$.
  % \end{myitemize}
  
  % Since $C$ has symmetric gates, $C$ is transparent. Notice that, since
  % $\theta(\vec{x})$ defines a query on structures over the empty vocabulary,
  % we
  % have for all $\vec{a} \in [n]^k$, $\sigma \in \sym_n$, $f(\sigma \vec{a}) =
  % f(\vec{a})$. For $\sigma \in \sym_n$, let $\pi_\sigma : G \rightarrow G$ be
  % a
  % bijection that fixes the constant gates and such that $\pi_{\sigma}
  % g_{\vec{a}} = g_{\sigma \vec{a}}$ for all $\vec{a} \in [n]^k$. Then
  % $\pi_\sigma \Omega (\vec{a}) = \pi_\sigma g_{\vec{a}} = g_{\sigma \vec{a}} =
  % \Omega (\sigma \vec{a})$. It follows that $\pi_\sigma$ is an automorphism of
  % the circuit extending $\sigma$. We thus have that $C$ translates
  % $\theta(\vec{x})$ for $n$.

  Fix $n \in \nats$. We will first define from $\theta$ a formula $\lambda$ that
  defines the same query and acts as a normal form for $\theta$. We then define
  a circuit $C$ from $\lambda$ and prove that it translates $\theta$ for $n$. We
  conclude by showing that we can construct $C$ with within the time bounds.

  Before we define $\lambda$ we first define a few helper formulas. Let $T \in
  \rho$. For a variable $y$ let $\op{no-op}(y) := (T(y, y) \lor (\neg T(y, y)))$
  and let $\op{no-op-all}(y) = \forall y .\, \op{no-op}(y)$. Let $\vec{y}$ be a
  (possibly empty) sequence of variables. If $\vec{y}$ is non-empty let $(y_1,
  \ldots, y_m) := \vec{y}$ and let $\op{tag} (\vec{y}) := (\op{no-op}(y_1) \land
  (\op{no-op}(y_2) \land ( \op{no-op}(y_2) \land ( \ldots \land
  (\op{no-op}(y_m)) \ldots))\ldots ))$. If $\vec{y}$ is empty let $\op{tag}
  (\vec{y}) = \forall u .\, ((u = u) \land (u = u)$. We define a similar helper
  formula $\op{tag-num}$ for $e \in \nats$ and a variable $y$ by
  \begin{align*}
    \op{tag-num}(e, y) = (\bigwedge_{i \in [e]}\op{no-op}(y)) = \underbrace{(\op{no-op-all}(y) \land (\op{no-op-all}(y) \land (\ldots \land (\op{no-op-all}(y)) \ldots )))}_{e \text{ times}}.
  \end{align*}
  For a (possibly empty) sequence of natural numbers $\vec{e}$, if $\vec{e}$ is
  non-empty let $ (e_1, \ldots, e_w) := \vec{e}$ and
  \begin{align*}
    \op{tag-num}(\vec{e}, y) &= (\bigwedge_{j \in [w]}\op{tag-num}(e_j, y)) \\&= (\op{tag-num}(e_1 , y) \land (\op{tag-num}(e_2, y) \land ( \ldots  \land (\op{tag-num}(e_w, y)) \ldots ))), 
  \end{align*}
  and if $\vec{e}$ is empty let $\op{tag-num}(\vec{e}, y) = \exists y. \, (y =
  y) \lor (y = y)$. It is easy to see that $\op{tag}$ and $\op{tag-num}$ define
  tautologies for any variable, sequence of variables, or sequence of numbers.
  We define $\lambda$ from $\theta$ by recursively replacing sub-formulas of
  $\theta$ with an application of a quantifier at the head with a logically
  equivalent formula defined using these helper functions. Let $\psi (\vec{y})$
  be a sub-formula of $\theta (\vec{x})$ of the form $Q^{E, \ar}_{\vec{p}, n}
  [(\vec{y}^i_1, \ldots, \vec{y}^i_{l_i}) \Upsilon_i]_{i \in l}$, where $Q^{E,
    \ar}_{\vec{p}, n} \in \setquant$, and let $\tau = (R^{\tau}, S^{\tau},
  \zeta^{\tau})$ be the vocabulary of $Q^{E, \ar}_{\vec{p}, n}$, where $R^{\tau}
  = \{R^{\tau}_1, \ldots R^{\tau}_l\}$. For each $i \in [l]$ let $l_i$ be the
  arity of $R^{\tau}_i$ and let $\vec{y}^i := \vec{y}^i_1, \ldots,
  \vec{y}^i_{l_i}$.

  From $\psi$ we define the formula $\psi' (\vec{y}) := Q^{E, \ar}_{\vec{p}, n}
  [(\vec{y}^i_1, \ldots, \vec{y}^i_{l_i}) \cdot \Upsilon_i']_{i \in l}$ where
  for each $i \in [l]$, $\Upsilon_i'$ is a function with the same domain as
  $\Upsilon_i$ and such that for each $\vec{b} \in \dom(\Upsilon_i')$,
  $\Upsilon_i' (\vec{b}) = ((\forall u . u = u) \land \Upsilon_i(\vec{b})) \land
  (\op{tag-num}(\vec{b}, z) \land \op{tag}(\vec{y}^i))$. Since $\psi'$ is
  defined from $\psi$ by taking conjunctions with tautologies, it is easy to see
  that $\psi$ and $\psi'$ define the same query. Let $\lambda$ be defined from
  $\theta$ by recursively replacing each sub-formula $\psi$ in $\theta$ with the
  corresponding formula $\psi'$. It can be shown by induction that, since we
  always replace a sub-formula $\psi$ with a logically equivalent one, $\lambda
  (\vec{x})$ and $\theta (\vec{x})$ define the same query.

  % The intuition here is that $\op{tag}(\vec{y}_i)$ appends a tower of
  % conjunctions of tautologies, with each tautology referencing a unique
  % variable
  % from $\vec{y}_i$. When we construct the circuit, this tower of tautologies
  % will act to `tag' each input to the quantifier gate with a unique gadget. We
  % will show that the presence of these gadgets ensures that no two inputs to a
  % quantifier gate are syntactically-equivalent, and so the circuit is
  % transparent.

  It is easy to see that $\lambda(\vec{x})$ has the same variable-width as
  $\theta(\vec{x})$. It can be shown that $\vert \cl{\lambda}\vert \leq c_1
  \vert \cl{\theta} \vert (\vert \cl{\theta} \vert + \ewidth{\theta})$ for some
  constant $c_1$.
  

  % Moreover, for each $i \in [l]$ and $\vec{b} \in \dom(\Upsilon_i)$ we have
  % that $\vert \cl{Upsilon_i'(\vec{b})}\vert \leq \vert \cl{\leq Upsilon_i}
  % \vert + 3 + 1 + \vert \cl{(\op{tag-num}(\vec{b}, z)} \vert + 1 + \vert
  % \cl{\op{tag}(\vec{y}^i)} \vert$ and $\vert \cl{(\op{tag-num}(\vec{b},
  % z)}\vert \leq 1 + \sum_{b \in \vec{b}} (1 + b) + \vert
  % \cl{\op{no-op-all}(y)} \vert \leq 6 + \vert \vec{b} \vert +\sum_{b \in
  % \vec{b}} b \leq 6 + 2 \vert \dom (\Upsilon_i' (\vec{b})) \vert$. We also
  % have $\vert \cl{\op{tag}(\vec{y}^i)} \vert \leq 1 + \sum_{y \in \vec{y}^i}
  % \vert \op{no-op}(y) \vert \leq 1 + \vert {\vec{y}^i}\vert 5$. It follows
  % that $\vert \cl{Upsilon_i'(\vec{b})}\vert \leq 4 + \vert \cl{Upsilon_i}
  % \vert + 6 + 2 \vert \dom (\Upsilon_i' (\vec{b})) \vert + 1 + 5\vert
  % {\vec{y}^i}\vert \leq c_1 \vert \cl{\theta} \vert + c_2 \width{\theta}$ for
  % constants $c_1$ and $c_2$.

  % $\cl{\lambda}$ adds a constant number of sub-formuls of the sub-formulas in
  % $\lambda(\vec{x})$ is at most $\vert \cl{\theta} \vert (5 +
  % 4\width(\theta))$.
  % We now construct a circuit $C$ and show that $C$ translates
  % $\lambda(\vec{x})$
  % (and hence $\theta(\vec{x})$) at $n$.

  Let $\psi \in \FO(\setquant)[\rho^*]$ and $\alpha \in [n]^{\free {\psi}}$. We
  let $\psi[\alpha]$ be the result of substituting each occurrence of the free
  variable $y \in \free{\psi}$ in $\psi$ with $\alpha(y)$. If $\psi$ is of the
  form $y_1 = y_2$, then $\psi [\alpha] = 1$ if $\alpha(y_1) = \alpha (y_2)$ and
  $\psi[\alpha] = 0$ otherwise. We call $\psi[\alpha]$ a \emph{ground formula}.
  For each $\psi \in \cl{\lambda}$ let $G_\psi := \{g_{\psi[\alpha]} : \alpha
  \in [n]^{\free{\psi}} \}$. Let $G := \bigcup_{\psi \in \cl{\lambda}}
  G_{\psi}$. Let $g := g_{\psi[\alpha]} \in G$. We define $\Sigma$, $\Lambda$
  and $L$ as follows.
  \begin{myitemize}
  \item If $\psi[\alpha]$ is $0$ or $1$ let $\Sigma (g) = \psi[\alpha]$.
  \item If $\psi[\alpha] = T(\vec{a})$ for some $T \in \rho^*$ and $\vec{a} \in
    [n]^{\arty(T)}$, then $\Sigma (g) = T$ and $\Lambda_T (g) = \vec{a}$.
  \item Suppose $\psi = Q^{E, \ar}_{\vec{p}, n} [(\vec{y}^i_1, \ldots,
    \vec{y}^i_{l_i}) \cdot \Upsilon_i]_{i \in [l]}$ for some $Q^{E,
      \ar}_{\vec{p}, n}$. Let $\Sigma (g_{\psi[\alpha]}) = F^{E}_{\vec{p}, n}$.
    Let $\tau := (R^\tau, S^\tau, \zeta^\tau)$ be the vocabulary of $Q^{E,
      \ar}_{\vec{p}, n}$, where $R^\tau = {R^\tau_1, \ldots, R^\tau_l}$. For
    each $i \in [l]$ let $l_i$ be the arity of $R^\tau_i$. For each $i \in [l]$,
    $j \in [l_i]$, and $z \in [2]$ let $e^z_{i, j} := \ar
    (\zeta^\tau(R^\tau_i)(j), z)$ and let $D^z_i := [n]^{e^z_{i, 1}} \times
    \ldots \times [n]^{e^z_{i, l_i}}$. Let $(\vec{d}, R^\tau_i) \in \ind
    (F^{E}_{\vec{p}, n})$. There exists $\vec{a} = (\vec{a}_1, \ldots,
    \vec{a}_{l_i}) \in D^1_i$ and $\vec{b} = (\vec{b}_1, \ldots, \vec{b}_{l_i})
    \in D^2_i$ such that $\vec{d} = (\vec{a}_1 \vec{b}_1 , \ldots,
    \vec{a}_{l_i}\vec{b}_{l_i})$. Let $L(g)(\vec{d}, R^\tau_i) =
    g_{\Upsilon_i(\vec{b})[\beta]}$, where $\beta$ is the union of $\alpha$ and
    $\alpha^{\vec{a}}_{\vec{x}^i_1, \ldots, \vec{x}^i_{l_i}}$.
  \item If $\psi = Q z . \, \phi(\vec{y}, z)$, for $Q \in \{\forall, \exists\}$,
    then if $Q = \forall$ we let $\Sigma (g) = \AND[n]$, otherwise let $\Sigma
    (g) = \OR[n]$. Let $L(g) : [n] \rightarrow G$ be defined for $i \in [n]$ by
    $L(g)(i) = g_{\psi[\beta]}$, where $\beta \in [n]^{\free{\phi}}$ is the
    union of $\alpha$ and $\alpha^i_z$.
  \item If $\psi = \phi_1 \land \phi_2$, then let $\Sigma(g) = \AND[2]$ and
    $L(g) : [2] \rightarrow G$ be defined for $i \in [2]$ by $L(g)(i) =
    g_{\phi_i[\beta_i]}$ where $\beta_i \in [n]^{\free{\phi_i}}$ is the
    restriction of $\alpha$ to $\free{\phi_i}$. The same approach is used for
    the disjunctive case.
  \item If $\psi = \neg \phi$ let $\Sigma (g) = \NOT$ and $L(g): [1] \rightarrow
    G$ be defined by $L(g)(1) = g_{\phi[\alpha]}$.
  \end{myitemize}
  Let $q = \vert \vec{x} = \vert$. Let $\Omega : [n]^{q} \rightarrow G$ be
  defined for $\vec{a} \in [n]^q$ by $\Omega (\vec{a}) = g_{\lambda [\alpha]}$,
  where $\alpha := \alpha^{\vec{a}}_{\vec{x}}$. Let $\vec{a} \in [n]^q$. We have
  from our assumption that $\theta$ (and $\lambda$) contain at least one
  relation symbol, and so it follows that $\lambda[\alpha]$ cannot be $0$ or
  $1$. Let $\vec{b} \in [n]^q$ and suppose $\Omega (\vec{a}) = \Omega(\vec{b})$.
  But then $g_{\lambda[\alpha^{\vec{a}}_{\vec{x}}]} =
  g_{\lambda[\alpha^{\vec{b}}_{\vec{x}}]}$ and so
  $\lambda[\alpha^{\vec{a}}_{\vec{x}}]$ and
  $\lambda[\alpha^{\vec{b}}_{\vec{x}}]$ are equal as strings. But if
  $\lambda[\alpha^{\vec{a}}_{\vec{x}}]$ and
  $\lambda[\alpha^{\vec{b}}_{\vec{x}}]$ are equal then, since $\lambda$ contains
  at least one relation symbol, it follows that $\alpha^{\vec{a}}_{\vec{x}} =
  \alpha^{\vec{b}}_{\vec{x}}$ and so $\vec{a} = \vec{b}$. It follows that
  $\Omega$ is injective. Let $T \in \tau$, and let $g_{\psi[\alpha]},
  g_{\phi[\beta]} \in G$ be such that $\Lambda_T (g_{\psi[\alpha]}) = \Lambda_T
  (g_{\phi[\beta]})$. Then we have $\vec{a} \in [n]^{\arty(T)}$ and $\vec{b} \in
  [n]^{\arty(T)}$ such that $\psi[\alpha] = T(\vec{a})$ and $\phi[\beta] =
  T(\vec{b})$. But then $\vec{a} = \Lambda_T (g_{\psi[\alpha]}) = \Lambda_T
  (g_{\phi[\beta]}) = \vec{b}$, but then $\psi[\alpha] = \phi[\beta]$ and so
  $g_{\psi[\alpha]} = g_{\phi[\beta]}$. It follows that $\Lambda_T$ is
  injective. We thus have that $C := \langle G, \Omega, \Sigma, \Lambda, L
  \rangle$ is a circuit of order $n$.

\begin{claim}
  The circuit $C$ is symmetric.
\end{claim}
\begin{proof}
  Let $\sigma \sym_n$. Let $\pi_\sigma : G \rightarrow G$ be defined such that
  $\pi_{\sigma} g_{\psi [\alpha]} = g_{\psi[\sigma \alpha]}$ for each
  $g_{\psi[\alpha]} \in G$ . It is easy to see that $\pi_\sigma$ is a bijection.
  We now show that $\pi_{\sigma}$ is an automorphism of $C$ extending $\sigma$.
  We prove this by induction on the structure of the circuit.

  Suppose $g_{\psi[\alpha]}$ is an input gate. If $g_{\psi[\alpha]}$ is a
  constant gate then $\psi (y_1, y_2) = (y_1 = y_2)$. We have $\psi[\alpha] = 1$
  if, and only if, $\alpha(y_1) = \alpha(y_2)$ if, and only if, $\sigma \alpha
  (y_1) = \sigma \alpha(y_2)$ if, and only if, and $\psi[\sigma \alpha] = 1$. It
  follows that $\pi_\sigma g_{\psi[\alpha]} = g_{\psi[\sigma \alpha]} =
  g_{\psi[\alpha]}$. If $g_{\psi[\alpha]}$ is a relational gate, then $\psi
  (\vec{y}) = T(\vec{y})$ for some relation symbol $T$ and $\Lambda_T
  (g_{\psi[\alpha]}) = \alpha (\vec{y})$. It follows that $\Lambda_T
  (\pi_{\sigma} g_{\psi[\alpha]}) = \Lambda_T( g_{\psi[\sigma \alpha]}) = \sigma
  \alpha (\vec{y}) = \sigma \Lambda_T (g_{\psi[\alpha]})$.
  
  Suppose $g_{\psi[\alpha]}$ is an internal gate. Suppose $g_{\phi[\beta]}$ is a
  child of $g_{\psi[\alpha]}$. From the definition of $C$ we have that $\phi$ is
  an immediate sub-formula of $\psi$ and $\beta$ must be an assignment to the
  free variables in $\phi$ compatible with $\alpha$. It follows that
  $\sigma\beta$ and $\sigma\alpha$ are compatible and so $\pi_\sigma
  g_{\phi[\beta]} = g_{\phi[\sigma\beta]}$ is a child of
  $\pi_{\sigma}g_{\psi[\alpha]} = g_{\psi[\sigma \alpha]}$. We thus have
  $H_{\pi_\sigma g_{\psi[\alpha]}} = \pi_\sigma H_{g_{\psi[\alpha]}}$. If
  $g_{\psi[\sigma]}$ is a symmetric gate, then this is sufficient to conclude
  that $\pi_{\sigma} L(g_{\psi [\alpha]})$ is isomorphic to $L(\pi_{\sigma}
  g_{\psi[\alpha]})$.

  Suppose $g$ is a non-symmetric gate. Then $\psi (\vec{y}) = Q^{E,
    \ar}_{\vec{p}, n} [(\vec{y}^i_1, \ldots, \vec{y}^i_{l_i}) \cdot
  \Upsilon_i]_{i \in [l]}$ for some $Q^{E, \ar}_{\vec{p}, n} \in \setquant$. Let
  $\tau := (R^\tau, S^\tau, \zeta^\tau)$ be the vocabulary of $Q^{E,
    \ar}_{\vec{p}, n}$, where $R^\tau = {R^\tau_1, \ldots, R^\tau_l}$. For each
  $i \in [l]$ let $l_i$ be the arity of $R^\tau_i$. For each $i \in [l]$ let
  $D^1_i$ and $D^2_i$ be defined as given in the definition of the circuit. For
  each $(\vec{d}, R^{\tau}_i) \in \ind(g_{\psi[\alpha]})$ let $\vec{a} :=
  (\vec{a}^i_1, \ldots, \vec{a}^i_{l_i}) \in D^1_i$ and $\vec{b} = (\vec{b}^i_1,
  \ldots, \vec{b}^i_{l_i}\in D^i_i$ be given as in the definition of $C$, i.e.\
  such that $\vec{d} = (\vec{a}^i_1\vec{b}^i_1, \ldots,
  \vec{a}^i_{l_i}\vec{b}^i_{l_i})$. Let $\delta : \ind(g_{\psi[\alpha]}) \ra
  \ind(g_{\psi[\alpha]})$ be defined such that $\delta (\vec{d}, R^\tau_i) =
  (\vec{d}', R^{\tau}_i)$, where $\vec{d}' = ((\sigma \vec{a}^i_1)\vec{b}^i_1 ,
  \ldots, (\sigma \vec{a}^i_{l_i}) \vec{b}^i_{l_i})$. The idea is that $\vec{d}$
  is constructed from the sequences $\vec{a}$ and $\vec{b}$, where $\vec{a}$ is
  an assignment to the variables bound by the operator and $\vec{b}$ is the
  input to the function $\Upsilon_i$. The function $\delta$ maps $\vec{d}$ by
  applying $\sigma$ only to $\vec{a}$, i.e.\ to those tuples that denote
  assignment to the variables. It can be shown that $\delta \in
  \aut(g_{\psi[\alpha]})$. We have that $\pi_{\sigma}
  L(g_{\psi[\alpha]})(\vec{d}, R^\tau_i) = g_{\Upsilon_i (\vec{b})[\sigma
    \beta]}$ where $\beta$ is the union of $\alpha$ and
  $\alpha^{\vec{a}}_{\vec{y}^i_1, \ldots, \vec{y}^i_{l_i}}$. But then $\sigma
  \beta$ is the union of $\sigma \alpha$ and $\alpha^{\sigma
    \vec{a}}_{\vec{y}^i_1, \ldots, \vec{y}^i_{l_i}}$, and so $g_{\Upsilon_i
    (\vec{b})[\sigma \beta]} = L(g_{\psi[\sigma \alpha]})(\vec{d}', R^\tau_i) =
  L(\pi_\sigma g_{\psi [\sigma \alpha]})(\delta (\vec{d}, R^\tau_i))$.

  Suppose $g_{\psi[\alpha]}$ is an output gate. Then $\psi = \lambda (\vec{x})$
  and let $\vec{a} := \alpha(\vec{x})$. From the definition of $C$ we have that
  $g_{\psi[\alpha]} = \Omega(\vec{a})$. It follows that $\pi_{\sigma} \Omega
  (\vec{a}) = \pi_\sigma g_{\psi[\alpha]} = g_{\psi[\sigma \alpha]} =
  \Omega(\sigma \vec{a})$. This concludes the proof of the claim.

\end{proof}

\begin{claim}
  Let $\mathcal{A} \in \fin[\rho, n]$ be a structure and let $\gamma \in
  [n]^{\underline{A}}$. For each $g_{\psi[\alpha]} \in G$ we have $\mathcal{A}
  \models \psi[\gamma^{-1}\alpha]$ if, and only if, $C[\gamma
  \mathcal{A}](g_{\psi[\alpha]}) = 1$.
\end{claim}
\begin{proof}
  We prove this claim by structural induction on the circuit. Let
  $g_{\psi[\alpha]} \in G$. If $g_{\psi[\alpha]}$ is an input gate then $\psi$
  has at its head a relation symbol or equality. It is easy to prove the claim
  in both cases. Suppose $g_{\psi[\alpha]}$ is an internal gate and the claim
  holds for each child of $g_{\psi[\alpha]}$. It is easy to handle the case
  where $g_{\psi[\alpha]}$ is labelled by an element of $\BS$. The only
  non-trivial case is where $g_{\psi[\alpha]}$ is labelled by an element of
  $\BB$. Suppose $g_{\psi[\alpha]}$ is an internal gate and $\psi (\vec{y}) :=
  Q^{E, \ar}_{\vec{p}, n} [(\vec{y}^i_1, \ldots, \vec{y}^i_{l_i}) \cdot
  \Upsilon_i]_{i \in [l]}$ for some $Q^{E, \ar}_{\vec{p}, n} \in \setquant$. Let
  $\tau := (R^\tau, S^\tau, \zeta^\tau)$ be the vocabulary of $Q^{E,
    \ar}_{\vec{p}, n}$, where $R^\tau = {R^\tau_1, \ldots, R^\tau_l}$. For each
  $i \in [l]$ let $l_i$ be the arity of $R^\tau_i$. Let $(\vec{d}, R^\tau_i) \in
  \ind(g_{\psi[\alpha]})$. As in the definition of the circuit we associate with
  $\vec{d}$ a unique pair of tuples $(\vec{a}, \vec{b})$. Let $\beta$ be the
  union of $\alpha$ and $\alpha^{\vec{a}}_{\vec{y}^i_1, \ldots,
    \vec{y}^i_{l_i}}$. It follows from the inductive hypothesis that $L^{\gamma
    \mathcal{A}}(g_{\psi[\alpha]})(\vec{d}, R^\tau_i) = 1$ if, and only if,
  $C[\gamma \mathcal{A}](g_{\Upsilon_i(\vec{b})[\beta]}) = 1$ if, and only if,
  $\mathcal{A} \models \Upsilon_i (\vec{b})[\gamma^{-1} \beta]$. Let
  $\mathcal{B} := \mathcal{I}(\mathcal{A}, \gamma^{-1} \alpha)$, where
  $\mathcal{I}$ is the interpretation defined by $Q^{E, \ar}_{\vec{p}, n}$.

  From the definition of the interpretation and the above observation we have
  that $\vec{d} \in (R^\tau_i)^{\mathcal{B}}$ if, and only if, $\mathcal{A}
  \models \Upsilon_i (\vec{b})[\gamma^{-1} \beta]$ if, and only if, $L^{\gamma
    \mathcal{A}}(g_{\psi[\alpha]})(\vec{d}, R^{\tau}_i) = 1$. It follows that
  $\mathcal{B}$ is isomorphic to the structure defined by $L^{\gamma
    \mathcal{A}}(g_{\psi [\alpha]})$, and so $\mathcal{A} \models
  \psi[\gamma^{-1} \alpha]$ if, and only if, $E(\vec{p}, \mathcal{B}) = 1$ if,
  and only if, $F^E_{\vec{p}, n}(L^{\gamma \mathcal{A}}(g_{\psi[\alpha]})) = 1$
  if, and only if, $C[\gamma \mathcal{A}] (g_{\psi[\alpha]}) = 1$.
\end{proof}
Let $\mathcal{A} \in \fin[\rho, n]$, let $\gamma \in [n]^{\underline{A}}$ and
let $\alpha \in A^{\vec{x}}$. Then $g_{\lambda [\gamma \alpha]} = \Omega (\gamma
\alpha (\vec{x}))$. It follows from the above claim that $C [\gamma \mathcal{A}]
(\Omega( \gamma \alpha (\vec{x})) = 1$ if and only if, $\mathcal{A} \models
\lambda [\alpha]$. In other words, $C$ and $\theta(\vec{x})$ compute the same
query for structures of size $n$.


% Let $\mathcal{A}$ be a $\tau$-structure with universe $A$ of size $n$ and let
% $\gamma \in [n]^{\underline{A}}$. It can be shown that for all $\psi \in
% \cl{\lambda}$ and $\alpha \in A^{\free{\psi}}$, $\mathcal{A} \models
% \psi[\alpha]$ if, and only if, $C[\gamma \mathcal{A}](g_{\psi[\gamma\alpha]})
% = 1$. It follows that $\vec{a} \in A^k$ is an element of $Q(\theta(\vec{x}))$
% if, and only if, $\mathcal{A} \models \lambda[\alpha^{\vec{x}}_{\vec{a}}]$ if,
% and only if, $C[\gamma \mathcal{A}](g_{\lambda[\alpha^{\vec{x}}_{\gamma
% \vec{a}}]}) = 1$. Since $g_{\lambda [\alpha^{\vec{x}}_{\gamma \vec{a}}]} =
% \Omega (\gamma \vec{a})$, we have that $C$ computes the query defined by
% $\theta(\vec{x})$ on structures of size $n$.

\begin{claim}
  The circuit $C$ is transparent.
  \label{claim:circuit-translation-transparent}
\end{claim}
\begin{proof}
  If $C$ has symmetric gates then $C$ is transparent. Suppose $C$ contains a
  non-symmetric gate. Then there exists $g_{\psi[\alpha]}\in G$ where $\psi
  (\vec{y}) := Q^{E, \ar}_{\vec{p}, n} [(\vec{y}^i_1, \ldots, \vec{y}^i_{l_i}) .
  \Upsilon_i]_{i \in [l]}$. Let $\vec{y}^i := \vec{y}^i_1, \ldots,
  \vec{y}^i_{l_i}$. Let $\tau := (R^\tau, S^\tau, \zeta^\tau)$ be the vocabulary
  of $Q^{E, \ar}_{\vec{p}, n}$, where $R^\tau = {R^\tau_1, \ldots, R^\tau_l}$.
  For each $i \in [l]$ let $l_i$ be the arity of $R^\tau_i$. Let $i \in [l]$ and
  let $(\vec{d})_1, R^\tau_i), (\vec{d}_2, R^{\tau}_i) \in
  \ind(g_{\psi[\alpha]})$. Let $D^1_i$ and $D^2_i$ be given as in the definition
  of $C$. For $z \in [2]$, let $\vec{a}^z = (\vec{a}^z_1, \ldots,
  \vec{a}^z_{l_i}) \in D^1_i$ and $\vec{b}^z = (\vec{b}^z_1, \ldots,
  \vec{b}^z_{l_i}) \in D^2_i$ such that $\vec{d}_z = (\vec{a}^z_1\vec{b}^z_1,
  \ldots, \vec{a}^z_{l_i}\vec{b}^z_{l_i})$. Let $\beta_z$ be the union of
  $\alpha^{\vec{a}^z}_{\vec{y}^i}$ and $\alpha$. For each $z \in [2]$ let $h_z
  := L(g_{\psi[\alpha]})(\vec{d})_z, R^\tau_i) =
  g_{\Upsilon_i(\vec{b}^z)[\beta_z]}$. Suppose $h_1 \equiv h_2$.


  From the definition of $\lambda$ we have for any $\vec{b} \in
  \dom(\Upsilon_i)$ that $\Upsilon_i(\vec{b}) = \kappa^1_{ \vec{b}} \land
  \kappa^2_{\vec{b}}$, where $\kappa^1_{\vec{b}} = ((\forall u. u = u) \land
  \Upsilon_i'(\vec{b}))$, and $\kappa^2_{\vec{b}} = (\op{tag}(\vec{y}^i) \land
  \op{tag-num}(\vec{b}, u))$, for some variable $u$. Let $z \in [2]$. Then
  $H_{h_z} = \{g_{\kappa^1_{\vec{b}^z}[\beta_z]},
  g_{\kappa^2_{\vec{b}^z}[\beta_z]}\}$. Let $w \in [2]$. We note that $u = u$ is
  a sub-formula of an immediate sub-formula of $\kappa^1_{ \vec{b}^z}$ while
  there is no sub-formula of an immediate sub-formula of $\kappa^2_{ \vec{b}^w}$
  that has equality at the head of the formula. It follows that a child of a
  child of $g_{\kappa^1_{ \vec{b}^z}[\beta_z]}$ is a constant gate while no
  child of a child of $g_{\kappa^2_{ \vec{b}^w}[\beta_w]}$ is a constant gate.
  We thus have that $g_{\kappa^1_{ \vec{b}^z}[\beta_z]} \not\equiv g_{\kappa^2_{
      \vec{b}^2}[\beta_w]}$. It then follows from $h_1 \equiv h_1$ that
  $g_{\kappa^1_{ \vec{b}^1}[\beta_1]} \equiv g_{\kappa^1_{ \vec{b}^2}[\beta_2]}$
  and $g_{\kappa^2_{ \vec{b}^1} [\beta_1]} \equiv g_{\kappa^2_{ \vec{b}^2}
    [\beta_2]}$.

  Let $z \in [2]$. Let $\epsilon^1_z = \op{tag}(\vec{y})[\beta_z]$ and
  $\epsilon^2_z = \op{tag-num}(\vec{b}^z, u))$. Let $w \in [2]$. We notice that
  if $\vec{y}$ is empty then $g_{\epsilon^1_z}$ is an $\AND$-gate and has a
  grandchild that is a constant gate. In contrast we have that
  $g_{\epsilon^2_w}$ is either an $\OR$-gate or has no grandchildren that are
  constant gates. In either case $g_{\epsilon^1_z} \not\equiv g_{\epsilon^2_w}$.
  If $\vec{y}$ is non-empty then $g_{\epsilon^1_z}$ has a relational gate as a
  grandchild, while $g_{\epsilon^2_w}$ does not. It follows that
  $g_{\epsilon^1_z} \not\equiv g_{\epsilon^2_w}$. Then, from the fact that
  $g_{\kappa^2_{ \vec{b}^1} [\beta_1]} \equiv g_{\kappa^2_{ \vec{b}^2}
    [\beta_2]}$ we have that $g_{\epsilon^1_1} \equiv g_{\epsilon^1_2}$ and
  $g_{\epsilon^2_1} \equiv g_{\epsilon^2_2}$. But from the definition of
  $\op{tag}$ and the fact that $g_{\epsilon^1_1} \equiv g_{\epsilon^1_2}$ we
  have that $\vec{beta}_1 = \vec{\beta}_2$. From the definition of
  $\op{tag-num}$ and the fact that $g_{\epsilon^2_1} \equiv g_{\epsilon^2_2}$ we
  have that $\vec{b}^1 = \vec{b}^2$. It follows that $h_1 =
  g_{\Upsilon_i(\vec{b}^1)[\beta_1]} = g_{\Upsilon_i(\vec{b}^2)[\beta_2]} =
  h_2$.

  It follows that $L(g_{\psi[\alpha]})$ is injective and no two gates in the
  same relation of $g_{\psi[\alpha]}$ are syntactically-equivalent. We conclude
  that $g_{\psi[\alpha]}$ has unique labels, and so $C$ is transparent.
\end{proof}

We have already shown that $\vert \cl{\lambda} \vert \leq c_1\vert \cl{\theta}
\vert (\vert \cl{\theta} \vert + \ewidth{\theta})$ and that $\ewidth{\lambda} =
\ewidth{\theta}$. We can construct the set of gates by recursing over the
sub-formulas of $\lambda$ and for each sub-formula $\psi$ and assignment $\alpha
\in [n]^{\free{\psi}}$ defining a gate $g_{\psi[\alpha]}$. This can be done in
time polynomial in $\vert \cl{\lambda} \vert \cdot \ewidth{\lambda}$, and we
have $\vert \cl{\lambda} \vert \cdot \ewidth{\lambda} \leq \vert \cl{\theta}
\vert (\vert \cl{\theta} \vert + \ewidth{\theta}) \cdot n^{\ewidth{\theta} +
  1}$. We can define the rest of circuit $C$ in time polynomial in the number of
gates, and so in time polynomial in $\vert \cl{\theta} \vert (\vert \cl{\theta}
\vert + \ewidth{\theta}) \cdot n^{\ewidth{\theta} + 1}$.

We have from the above three claims that $C$ translates $\lambda$ (and hence
$\theta$) for $n$ and from the above argument that we may construct $C$ within
the required time bounds. The result follows.


% Moreover, it can be seen that $\lambda$ may be recursively constructed from
% $\theta$ in time polynomial in $c_1\vert \cl{\theta} \vert (\vert \cl{\theta}
% \vert + \width{\theta}) \cdot n^{\width{\theta}}$. By recursion on the
% structure of the formula $\lambda$ we can define the set of gates in time
% polynomial in $c_1\vert \cl{\theta} \vert (\vert \cl{\theta} \vert +
% \width{\theta}) \cdot n^{\width{\theta}}$. We can define the circuit $C$ in
% time polynomial in the set of gates, and hence in time polynomial in $c_1\vert
% \cl{\theta} \vert (\vert \cl{\theta} \vert + \width{\theta}) \cdot
% n^{\width{\theta}}$.

% It is easy to see that $\lambda$ and $\theta$ have the same variable-width.
% Moreover, we have defined $\lambda$ by adjo Moreover, from the construction of
% $\lambda$ the

% We note that $\vert C \vert \leq n^{\width(\lambda)} \vert \cl{\lambda} \vert
% \leq n^{\width(\theta)} \vert \cl{\theta} \vert (5 + \width(\theta))$. It is
% not hard to see that $C$ (and $\lambda(\vec{x})$) can be constructed in a
% number of steps bounded by a polynomial in the size of $C$. This completes the
% proof of the result.
\end{proof}

\begin{prop}
  Let $\rho$ is a vocabulary. Let $\setquant$ be a set of extended quantifiers
  and let $\BB$ be the associated basis. Each query definable by a $\PT$-uniform
  family of $\FO(\setquant)[\rho]$-substitution programs with a constant bound
  on width is definable by a $\PT$-uniform family of transparent symmetric
  $(\BB, \rho)$-circuits $(C_n)_{n \in \nats}$.
  \label{prop:translating-programs-to-circuits}
\end{prop}
\begin{proof}
  Let $\Theta := (\Theta_n)_{n \in \nats}$ be a $\PT$-uniform family of
  $\FO(\setquant)[\rho]$ substitution programs. Let $n \in \nats$. Let
  $(\theta_{n, 1}, \ldots, \theta_{n, k}) := \Theta_n$. For each $i \in [k]$ let
  $\vec{V}_i$ be the element-sort second-order variables that appear in
  $\theta_{n, i}$. We treat these second-order variables as relation symbols,
  and suppose, without a loss of generality, that the symbols that appear in
  $\vec{V}_i$ do not appear in $\rho$ for each $i \in [k]$. For each $i \in [k]$
  let $\rho_i = \vec{V}_i \cup \rho$. From
  Lemma~\ref{lem:translating-FOquant-to-formulas} we may construct for each $i
  \in [k]$ a $(\BB, \rho_i)$-circuit $C_{n, i}$ such that $C_{n, i}$ translates
  the formula $\theta_{n, i}$ for $n$. We assume, without a loss of generality,
  that in each of these circuits none of the input gates are also output gates
  (if this is not the case, we can alter the circuit by adding in single-input
  $\AND$-gates with each gate that is both an input and output gate taken as
  input to the $\AND$-gate, and the $\AND$-gate then assigned to be an output
  gate).

  For each $i \in [k]$ let $\theta_{n, i}'$ be the flattening of $\Theta_n$ at
  $i$. We now recursively define for each $i \in [k]$ a $(\BB, \rho)$-circuit
  $C_{n, i}'$ that translates $\theta_{n, i}'$ for $n$. We recall that for each
  $i \in [k]$ the formula $\theta_{n, i}'$ is defined from $\theta_{n, i}$ by
  replacing each symbol $V_j$, for $j > i$, with the formula $\theta_{n, j}'$.
  Similarly, we define $C_{n, i}'$ from $C_{n, i}$ by replacing each input gate
  labelled by the symbol $V_j$ and tuple $\vec{a}$ with the output gate of
  $C_{n, j}'$ labelled by $\vec{a}$.

  For each $i \in [k]$ let $(G_i, \Omega_i, \Sigma_i, \Lambda_i, L_i) := C_{n,
    i}$. We define the $(\BB,\rho)$-circuit $C_{n, i}' := (G_i', \Omega_i',
  \Sigma_i', \Lambda_i', L_i')$ from the circuits $\{C_{n, j}' : j > i \}$ as
  follows.
  \begin{myitemize}
  \item Let $G_i' = \{g \in G_i : \Sigma (g) \not\in \vec{V}_i\} \cup
    \bigcup{V_{j} \in \vec{V}} G_{j}'$. We identify input gates labelled by the
    same relation symbol and tuple so that there are no duplicate input gates.
    
  \item Let $\Lambda_i$ be defined for each relation symbol $T \in \rho$, tuple
    $\vec{a} \in [n]^{\arty(T)}$, and $g \in G_i'$ such that $(\Lambda_i')_{T}
    (g) = \vec{a}$ if, and only if, (i) $g \in G_i$ and $(\Lambda_i)_{T}(g) =
    \vec{a}$ or (ii) there exists $j > i$ such that $g \in G_j'$ and
    $(\Lambda_j')_T (g) = \vec{a}$.
    
  \item For each $g \in G_i'$ if $g \in G_i$ let $\Sigma_i' (g) = \Sigma_i (g)$
    and otherwise there exists $j > i$ such that $g \in G_j'$ and let $\Sigma_i'
    (g) = \Sigma_j' (g)$.
    
  \item Let $q$ be the arity of the query decided by $C_{n, i}$. Let $\Omega_i'$
    be an injection from $[n]^{q}$ to $G_i'$ defined such that $\Omega_i'
    (\vec{a}) = \Omega_i(\vec{a})$.
    % if, and only if, $g \in G_i$, $\Omega_i (\vec{a}) = g$ and $\Sigma_i(g)
    % \not\in \vec{V}_i$, or (ii) there exists $j > i$ such that $g \in G_j'$,
    % $\Sigma_i (\Omega_i (\vec{a})) = V_j$, $(\Lambda_i)_{V_j}(\Omega_i
    % (\vec{a})) = \vec{a}$, and $\Omega_j'(\vec{a}) = g$.
    
  \item Let $g \in G_i'$. We define $L_i' (g) : \dom (L_i(g)) \ra G_i$ for $a
    \in \dom(L_i(g))$ as follows. Suppose $g \in G_i$. Let $L_i' (g)(a) =
    L_i(g)(a)$ if $\Sigma_i (L_i (g)(a)) \not\in \vec{V}_i$ and otherwise let
    $L_i'(g)(a) = \Omega_j((\Lambda_i)_{V_j}(L_i(g)(a)))$, where $j > i$ and
    $\Sigma_i (L_i (g)(a)) = V_j$. Suppose $g \not\in G_i$. Then there exists $j
    > i$ such that $g \in G_j'$ and let $L_i(g) = L_j'(g)$.
  \end{myitemize}
  
  We now show by backwards induction that for all $i \in [k]$, $C_{n, i}'$
  translates $\theta_{n, i}'$ for $n$. It is easy to see from the above
  definition that $C_{n, k}' = C_{n, k}$, and so $C_{n, k}$ translates
  $\theta_{n, k}' = \theta_{n, k}$ for $n$. Let $i \in [k]$ and suppose for each
  $j > i$ we have that $C_{n, j}'$ translates $\theta_{n, j}'$ for $n$. We now
  show that $C_{n, i}'$ translates $\theta_{n, i}'$ for $n$.

\begin{claim}
  The circuit $C_{n, i}'$ is symmetric.
\end{claim}
\begin{proof}
  Let $\sigma \in \sym_n$. We have automorphisms $\pi_i$ of $C_{n, i}$ extending
  $\sigma$ and for each $j > i$ an automorphism $\pi_j'$ of $C_{n, j}'$
  extending $\sigma$. We define $\pi_i'$ for each $g \in G_i'$ as follows. If $g
  \in G_i$ let $\pi_i' (g) := \pi_i (g)$, and otherwise let $\pi_i' (g) :=
  \pi_j' (g)$, where $j > i$ such that $g \in G_j'$. It remains to show that
  $\pi_i'$ is a valid automorphism. We note that for $j > i$ if $g$ is an input
  gate in $C_{n, i}'$ then $\pi_i' (g) = \pi_i (g) = \pi_j'(g) = \sigma(g)$.

  Let $g \in G_i'$. Suppose $g \not\in G_i$. Then $g \in G_j'$ for some $j > i$.
  Since $C_{n, j}'$ is symmetric there exists $\lambda \in \aut(g)$ such that
  $\pi_j' L_j'(g) = L_j(\pi_j' g)\lambda$. Then for all $a \in \ind(g)$ we have
  $\pi_i' L_i' (g)(a) = \pi_j' L_j' (g)(a) = L_j'(\pi_j' g) (\lambda a) =
  L_i'(\pi_i' g)(\lambda a)$. Suppose $g \in G_i$. Since $C_{n, i}$ is symmetric
  there exists $\lambda \in \aut (g)$ such that $\pi_i L_i (g) = L_i (\pi_i g)
  \lambda$. Let $a \in \ind(g)$. If $L_i' (g)(a) \in G_i$ and so $\pi_i L_i
  (g)(a) \in G_i$ and $\pi_i' L_i' (g)(a) = \pi_i L_i (g)(a) = L_i (\pi_i g)
  (\lambda a) = L_i' (\pi_i' g)(\lambda a)$. If $L_i'(g)(a) \not\in G_i$ then
  $L_i'(g)(a) = \Omega_j'((\Lambda_i)_{V_j}(L_i(g)(a)))$ for some $j > i$ and
  $\pi_i' L_i'(g)(a) = \pi_j' \Omega_j'((\Lambda_i)_{V_j}(L_i(g)(a))) =
  \Omega_j' (\sigma (\Lambda_i)_{V_j}(L_i(g)(a))) = \Omega_j'
  ((\Lambda_i)_{V_j}(\pi_i L_i(g)(a)) = \Omega_j'(((\Lambda_i)_{V_j}(L_i(\pi_i
  g)(\lambda a))) = L_i'(\pi_i g)(\lambda a) = L_i'(\pi_i' g)(\lambda a)$. The
  first second to last equities follows from the definition of the circuit. It
  is easy to check the remaining requirements for the symmetry of $C_{n, i}'$.
\end{proof}

\begin{claim}
  The circuit $C_{n, i}'$ is transparent.
\end{claim}
\begin{proof}
  We now show that $C_{n, i}'$ is transparent. Let $g \in G_i'$ be a
  non-symmetric gate. Suppose $g \in G_j'$ for some $j > i$. Then from the fact
  that $C_{n, j}'$ is transparent if follows that $g$ has unique labels. Suppose
  $g \in G_i$. From the translation in
  Lemma~\ref{lem:translating-FOquant-to-formulas} (and from the proof of
  Claim~\ref{claim:circuit-translation-transparent}) it follows that $g$ has
  unique labels.
\end{proof}

\begin{claim}
  Let $\mathcal{A} \in \fin[\rho, n]$ let $\vec{a}$ be an assignment to the free
  variables in $\theta_{n,i}'$ and let $\gamma \in [n]^{\underline{U}}$. Then
  $C_{n, i}'[\gamma \mathcal{A}](\Omega_i' (\gamma \vec{a})) = 1$ if, and only
  if, $\mathcal{A} \models \theta_{n, i}' [\vec{a}]$.
\end{claim}
\begin{proof}
  We first prove by induction on the structure of the circuit $C_{n, i}$ that
  for each $g \in G_{n, i}$ if $g$ is an input gate labelled by some $V_j$ with
  $j > i$ then $C_{n, i}[\gamma \mathcal{A}^*](g) = C_{n, j}'[\gamma
  \mathcal{A}](\Omega_j'((\Lambda_i)_{V_j})(g))$ and otherwise $C_{n, i}[\gamma
  \mathcal{A}^*](g) = C_{n, i}'[\gamma \mathcal{A}](g)$. Let $g \in G_{n, i}$.
  Suppose $g$ is an input gate labelled by $V_j$ for some $j > i$ and let
  $\vec{b} := (\Lambda_i)_{V_j}(g)$. Then we have that $C_{n, i}[\gamma
  \mathcal{A}^*](g) = 1$ if, and only if, $\gamma^{-1} \vec{b} \in
  V^{\mathcal{A}^*}_j$ if, and only if, $\mathcal{A} \models \theta_{n,
    j}'[\gamma^{-1} \vec{b}]$ if, and only if, $C_{n, j}'[\gamma
  \mathcal{A}](\Omega_j'(\vec{b})$. Clearly if $g$ is an input gate not labelled
  by $V_j$ for some $j > i$ then $C_{n, i}[\gamma \mathcal{A}^*](g) = C_{n, i}'
  [\gamma \mathcal{A}^*](g)$. Suppose $g$ is not an input gate and suppose the
  inductive hypothesis holds for all children of $g$. Let $a \in \ind(g)$. If
  $L_i'(g)(a) \in G_i$ then $L_i' (g) (a) = L_i (g)(a)$ and so $(L_i')^{\gamma
    \mathcal{A}}(g)(a) = L^{\gamma \mathcal{A}}_i(g)(a)$. If $L_i'(g)(a) \not\in
  G_i$ then $L_i(g)(a)$ is an input gate labelled by some $V_j$ for $j > i$. It
  follows from the definition of the circuit that $L_i'(g)(a) =
  \Omega_j'((\Lambda_i)_{V_j})(L_i(g)(a))$, from the inductive hypothesis,
  $C_{n, i}[\gamma \mathcal{A}^*](L_i'(g)(a)) = C_{n, j}'[\gamma
  \mathcal{A}](\Omega_j'((\Lambda_i)_{V_j})(L_i(g)(a))) = C_{n, i}'[\gamma
  \mathcal{A}](L_i'(g)(a))$. It follows that $(L_i')^{\gamma \mathcal{A}}(g) =
  L^{\mathcal{A}^*}_i(g)$.

  Let $\vec{a}$ be an assignment to the free variables in $\theta_{n, i}'$. But
  $\theta_{n, i}$ has the same free first-order variables as $\theta_{n, i}$ and
  so $\vec{a}$ is a $q$-tuple, where $q$ is the arity of the query decided by
  $C_{n, i}$. It follows from the definition of the circuit that
  $\Omega_i'(\gamma \vec{a}) = \Omega_i(\gamma \vec{a})$ and from the above
  inductive argument that $C_{n, i}'[\gamma \mathcal{A}](\Omega_i'(\gamma
  \vec{a})) = C_{n, i} [\gamma \mathcal{A}^*] (\Omega_i(\gamma \vec{b}))$. It
  follows that $C_{n, i}' [\gamma \mathcal{A}](\Omega_i' (\gamma \vec{a})) = 1$
  if, and only if, $C_{n, i} [\gamma \mathcal{A}^*] (\Omega_i(\gamma \vec{a})) =
  1$ if, and only if, $\mathcal{A}^* \models \theta_{n, i} [\vec{a}]$ if, and
  only if, $\mathcal{A} \models \theta_{n, i}'[\vec{a}]$. The final equivalence
  follows from the definition of a flattening.


  % It follows that $C_{n, i}' [\gamma \mathcal{A}](\Omega_i (\gamma \vec{a})) =
  % 1$ if, and only if,$ $i}[\gamma \mathcal{A}^*](\Omega_i)$C

  


  % Suppose $\Omega_i'(\vec{b}) \not\in G_i$. Then $\Omega_i(\vec{b})$ is an
  % input gate in $C_{n, i}$ labelled by some $V_j$ for $j > i$ and $\Omega_i'
  % (\vec{b}) = \Omega_j'(\vec{b})$. We also have have from the above inductive
  % argument that $C_{n, i}[\gamma \mathcal{A}^*](\Omega_i (\vec{b})) = C_{n,
  % j}'[\gamma \mathcal{A}](\Omega_j'((\Lambda_i)_{V_j})(\Omega_i (\vec{b}))) $

  



  
  % From the inductive hypotheses we have that the claim holds for each $j > i$.
  % Let $\mathcal{A}^*$ be the $\rho_i$-structure defined by extending
  % $\mathcal{A}$ by mapping each relation symbol $V_j \in \vec{V}_i$ to the
  % relation $V^{\mathcal{A}^*}_j = (\theta_{n, j}')^{\mathcal{A}}$. Then
  % $\vec{b} \in V^{\mathcal{A}^*}$ if, and only if, $\mathcal{A} \models
  % \theta_{n, j}'[\vec{b}]$ if, and only if, $C[\gamma
  % \mathcal{A}](\Omega_j'(\gamma \vec{b})) = 1$. From the definition of a
  % flattening and the fact that $C_{n, i}$ translates $\theta_{n, i}$ for $n$
  % (where we think of $\theta_{n, i}$ as a $\rho_i$-formula) it follows that
  % $\mathcal{A} \models \theta_{n, i}'[\vec{a}]$ if, and only if,
  % $\mathcal{A}^* \models \theta_{n, i} [\vec{a}]$ if, and only if, $C_{n,
  % i}[\gamma \mathcal{A}^*](\Omega_i (\gamma \vec{a}))$. Thus to prove the
  % claim it suffices to show that $C_{n, i}[\gamma \mathcal{A}^*](\Omega_i
  % (\vec{b})) = C_{n, i}' [\gamma \mathcal{A}](\Omega_i' (\vec{b}))$ for all
  % $\vec{b} \in \dom(\Omega_i)$.

  % Let $\vec{b} \in \dom(\Omega_i)$. Suppose $\Omega_i'(\vec{b}) \not\in G_i$.
  % Then $\Omega_i(\vec{b})$ is an input gate in $C_{i, j}$ such that
  % $(\Lambda_i)_{V_j}(\Omega_i(\vec{b})) = \vec{b}$ for some $j > i$ and
  % $\Omega_i'(\vec{b}) = \Omega_j'(\vec{b})$. Then $C_{n, i}[\gamma
  % \mathcal{A}^*](\Omega_i (\vec{b})) = 1$ if, and only if,
  % $\gamma^{-1}(\vec{b}) \in (V_j)^{\mathcal{A}^*}$ if, and only if,
  % $\mathcal{A} \models \theta_{n, j}' [\gamma^{-1} \vec{b}]$ if, and only if,
  % $C_{n, j}'[\gamma \mathcal{A}](\Omega_j'(\vec{b})) = 1 $ if, and only if,
  % $C_{n, i}' [\gamma \mathcal{A}](\Omega_i'(\vec{b}))$. Suppose $\Omega_i'
  % (\vec{b}) \in G_i$. Then $\Omega_i' (\vec{b}) = \Omega_i (\vec{b})$.

  % We now show by induction on the structure of the circuit that for each $g
  % \in G_i \cap G_i'$ if $g$ is an input gate in $C_{n, i}'$ then we have
  % $L^{\gamma \mathcal{A}^*}_i = L^{\gamma \mathcal{A}^*}C_{n,
  % i}[\gamma\mathcal{A}^*](g) = C_{n, i}' [\gamma\mathcal{A}](g)$. We prove
  % this by induction. Suppose $g$ is a gate such that all of its children in
  % $C_{n, i}$ are input gates. From the definition of the circuit we have that
  % for each $a \in \ind(g)$ if $L_i(g)(a)$ is an input gate labelled by the
  % relation symbol $V_j$ for some $j > i$ and tuple $\vec{c}$ then $L_i'(g)(a)
  % = \Omega_j' (\vec{c})$. But then $C_{n, j}'[\gamma \mathcal{A}](\Omega_j
  % (\vec{b})) = C_{n, i}[\gamma \mathcal{A}^*]((\Lambda_{i})^{-1}(\vec{b}))$.
  % Then for each $a \in \ind(g)$ if $L_i(g)(a)$ is an input gate in $C_{n, i}$
  % labelled by $V_j$ and the tuple $\vec{b}$ then $(L_i')^{\gamma \mathcal{A}}
  % (g)(a) = C_{n, i}'[\gamma \mathcal{A}](L_i'(g)(a)) = C_{n, j}' [\gamma
  % \mathcal{A}] (\Omega_j' (\vec{b}) = C_{n, i}[\gamma
  % \mathcal{A}^*]((\Lambda_{i})^{-1}(\vec{b})) = C_{n, i}[\gamma
  % \mathcal{A}^*](L_i(g)(a))$. It follows that $(L_i')^{\gamma \mathcal{A}} (g)
  % = L^{\gamma \mathcal{A}^*}_i (g)$ and so $C_{n, i}[\gamma \mathcal{A}^*] (g)
  % = C_{n, i}' [\gamma \mathcal{A}](g)$. We can extend this result from gates
  % with children that are input gates to all gates via a straight-forward
  % inductive argument.

\end{proof}

For each $n \in \nats$ let $C_{n}' := C_{n, 1}'$ and let $\theta_n'$ be the
flattening of $\Theta_n'$. It follows from the above three claims that $C_n'$
translates $\theta_{n}'$ for $n$. It remains to argue that the construction of
$C_n'$ for each $n$ may be completed in time polynomial in $n$. This follows
from three polynomial bounds. First, since $\Theta$ is $\PT$-uniform there
exists a polynomial $p_1$ such that the function $n \mapsto \Theta_n$ is
computable in time $p_1(n)$. Second, it follows from
Lemma~\ref{lem:translating-FOquant-to-formulas} that there is a polynomial $p_2$
such that the function that maps $\Theta_n$ to the sequence of circuits $C_{n,
  1}, \ldots, C_{n, \vert \Theta_n \vert}$ is computable in time $\sum_{i \in
  [\vert \Theta_n \vert]} p_2(\vert \cl{\theta_{n, i}} \vert (\vert
\cl{\theta_{n, i}} \vert + \ewidth{\theta_{n, i}}) \cdot n^{\ewidth{\theta_{n,
      i}}}) \leq p_2(n) \cdot p_2(p_1 (n) (p_1(n) + c_1) n^{c_1})$, where $c_1$
is the constant bound on the variable-width of the formulas of $\Theta$.
Thirdly, the construction of $C_n'$ from $C_{n, 1}, \ldots, C_{n, \vert \Theta_n
  \vert}$ works by constructing each $C_{n, i}'$ from $C_{n, j}$, for all $j >
i$, by replacing the appropriate input gates of $C_{n, i}$ with the output gates
of $C_{n, j}'$. It can be shown that this algorithm runs in time polynomial in
the combined size of these circuits. The function that maps $n \mapsto C_n'$ is
the composition of these three functions and so can be computed in time
polynomial in $n$.


% Suppose there exists $h \in G_{n, i}$ such that $h$ is a child of $g$,
% $\Sigma_i (h) = V_j$ for some $j > i$ and let $\vec{a} :=
% (\Lambda_i)_{V_j}(g)$. From the definition of $C_{n, i}'$ we have that
% $L_i'(g)(a) = \Omega_j'(\vec{a})$ for each $a \in \ind(g)$ such that $L_i
% (g)(a) = h$. From the inductive hypothesis it follows that $C_{}[\gamma
% \mathcl]$


% From the inductive hypothesis we have for each $j > i$ that $C_{n, j}'$
% translates $\theta_{n, j}'$ for $n$. Let $g$ be an input gate in $C_{n, i}$
% such that $\Sigma_i (g) = V_j$ for some $j > i$ and let $\vec{a} :=
% (\Lambda_i)_{V_j}(g)$. We have that $C_{n, i}[\gamma\mathcal{A}^*](g) = 1$ if,
% and only if, $C_{n, i}$


% It follows from the fact that $C_{n, j}'$ translates $\theta_{n, j}'$ for all
% $j > i$ and the fact that $C_{n, i}$ translates $\theta_{n, i}$

% Let $\mathcal{A}^*$ be the $\rho_i$-structure defined by extending
% $\mathcal{A}$ by mapping each relation symbol $V_j \in \vec{V}_i$ to the
% relation $V^{\mathcal{A}^*}_j = (\theta_{n, j})^{\mathcal{A}}$. We have that
% $\mathcal{A}^* \models \Theta_{n}(i)[\alpha]$ if, and only if, $\mathcal{A}
% \models \theta_{n, i}[\alpha]$. Let $\gamma \in [n]^{\underline{U}}$. We have
% that $C_{n, i}[\gamma \mathcal{A}^*] = vec{a}$ if, and only if, $\mathcal{A}^*
% \models theta_{n, j}[\vec{a}]$. But $C_{n, i}$ $\mathcal{A}^* \models
% \theta_{n, i}[\alpha]$


% $\mathcal{A} \models \$


% $If none of the children of $g$ in $C_{n, i}$ are in input gates labelled by a
% relation symbol in $\vec{V}_i$ then, from the transparency of $C_{n, i}$, it
% follows $g$ has unique labels.


% rom the definition of a translation in Lemma~\ref{}, it follows that

% It is easy too see that for each $\mathcal{A} \in \fin[\rho, n]$

% as follows. let denote the circuit corresponding the kth circuit and replace
% each
  
\end{proof}
We now present the main theorem of this chapter: that each extension of a
fixed-point logic by generalised operators may be translated to an equivalent
$\PT$-uniform family of transparent symmetric circuits. This result follows
almost immediately from Lemma~\ref{lem:unroll-fixed-point},
Lemma~\ref{lem:translate-sub-operators-to-queries}, and
Proposition~\ref{prop:translating-programs-to-circuits}.

\begin{thm}
  Let $\rho$ is a vocabulary. Let $\setop$ be a set of operators and let $\BB$
  be the associated basis. Every query definable in $\FP(\setop)[\rho]$ is
  definable by a $\PT$-uniform family of transparent symmetric $(\BB,
  \rho)$-circuits $(C_n)_{n \in \nats}$.
  \label{thm:translating-FP-formulas-to-circuits}
\end{thm}
\begin{proof}
  Let $\theta(\vec{x}) \in \FP^{\nats}(\setop)[\rho]$. From
  Lemma~\ref{lem:unroll-fixed-point} there exists a $\PT$-uniform family of
  $\FO^{\nats}(\setop)[\rho]$-substitution programs $\Theta^1 := (\Theta^1_n)_{n
    \in \nats}$ with constant bounds on width and formula-length and such that
  $\Theta^1$ decides the same query as $\theta$. Let $\setquant$ be the set of
  quantifiers correpsonding to $\setop$. From
  Lemma~\ref{lem:translate-sub-operators-to-queries} there exists a
  $\PT$-uniform family of $\FO(\setquant)[\rho]$-substitution programs $\Theta^2
  := (\Theta^2_n)_{n \in \nats}$ with a constant bound on width such that
  $\Theta^1$ and $\Theta^2$ define the same query. From
  Proposition~\ref{prop:translating-programs-to-circuits} there is a
  $\PT$-uniform family of transparent symmetric $(\BB, \rho)$-circuits $(C_n)_{n
    \in \nats}$ that decides the same query as $\Theta^2$. The result follows.
\end{proof}

% \section{rough nonsense}

% \subsection{Generalised Many-Sorted Mixed-Type Quantifiers}
% Let $\tau = (\{R_1, \ldots, R_{l_1}\},\{F_1, \ldots, F_{l_2} \}, \{s_1,
% \ldots, s_{q}\}, \nu)$ be a many-sorted vocabulary with number-valued function
% symbols. Let $\mathcal{K}$ be a class of $\tau$-structures.

% \begin{align*}
%   Q_{\mathcal{K}}[\vec{\phi}^D, \vec{\phi}^{\approx}]  [&((\vec{x}^i_1 \vec{\mu}^i_1, \ldots, \vec{x}^i_{r_i} \vec{\mu}^i_{r_i})(\theta_i))_{i \in [l_1]}\\&((\vec{y}^j_1 \vec{\nu}^j_1, \ldots, \vec{y}^j_{f_j} \vec{\nu}^j_{f_j})(\eta_i))_{i \in [l_2]}]
% \end{align*}

% Let $B_1, \ldots , B_{q}$ be a sequence of sets such that $B_i = \phi^D_i /
% \phi^{\approx}_i$. Let $R^{B}_i$ be defined such that $(e_1, \ldots , e_{r_i})
% \in R^{B}_i$ if, and only if, there exists $(\vec{a}_1 \vec{m}_1, \ldots ,
% \vec{a}_{r_i} \vec{m}_{r_i})$ such that $\vec{a}_j \vec{m}_j \in e_j$ for all
% $j \in [r_i]$ and $\mathcal{A} \models \theta_i [\vec{a}_1\vec{m}_1, \ldots,
% \vec{a}_{r_i} \vec{m}_{r_i}]$. Let $F^{B}_i$ be defined such that $F^B_i (e_1,
% \ldots, e_{f_i}) = k$ there exists $(\vec{a}_1, \vec{m}_1, \ldots ,
% \vec{a}_{r_i} \vec{m}_{r_i})$ such that $\vec{a}_j \vec{m}_j \in e_j$ for all
% $j \in [f_i]$ where $\mathcal{A} \models (k = \nu_i [\vec{a}_1, \vec{m}_1,
% \ldots , \vec{a}_{r_i} \vec{m}_{r_i}])$. Let $\mathcal{B}$ be the structure
% defined from the above. We have $\mathcal{A} \models Q_{\mathcal{K}}...$ if,
% and only if, $\mathcal{B} \in \mathcal{K}$.

% We

% \subsection{Restriction to Operators on Formulas}
% Let $\tau = (\{R_1, \ldots, R_{l_1}\}, \{s_1, \ldots, s_{q}\}, \nu)$ be a
% many-sorted vocabulary. Let $\mathcal{K}$ be a class of $\tau$-structures.

% \begin{lem}
%   Let $\Omega$ be a family of generalised many-sorted operators. Let
%   $L(\Omega)$ be a logic extended by $\Omega$. There is a family of
%   generalised many-sorted operators $\Omega^*$ with no number-terms being
%   operated on such that $L(\Omega)$ and $L(\Omega^*)$ have the same expressive
%   power.
% \end{lem}
% \begin{proof}
%   Let $\oper_E \in \Omega$ where $\dom (E) = \nats^{m_1} \times \{0,1\}^{m_2}
%   \times \fin[\tau]$ and $\tau = (\{R_1, \ldots, R_{l_1}\},\{F_1, \ldots,
%   F_{l_2} \}, \{s_1, \ldots, s_{q}\}, \zeta)$. Let $S^* = S \uplus
%   \{s_{q+1}\}$. For each $F_i$ define a relation symbol $R^*_i$ with arity
%   $\iota (R^^)$ Let $\tau^* = (\{R^{*}_1, \ldots, R^*_{l_1}\})$ where $r^*_i =
%   r$


%   $E^{*} : \dom (E) \ra $, $\oper_{E^{*}}$ be defined as follows.
% \end{proof}

% \subsection{Restriction to Boolean-Valued Operators}


% \begin{align*}
%   Q_{\mathcal{K}}[\vec{\phi}^D, \vec{\phi}^{\approx}]  [&((\vec{x}^i_1 \vec{\mu}^i_1, \ldots, \vec{x}^i_{r_i} \vec{\mu}^i_{r_i})(\theta_i))_{i \in [l_1]}\\&((\vec{y}^j_1 \vec{\nu}^j_1, \ldots, \vec{y}^j_{f_j} \vec{\nu}^j_{f_j})(\eta_i))_{i \in [l_2]}]
% \end{align*}


% \subsection{From Operators to Families of Quantifiers}

% \subsection{Quantifiers}
% Let $\tau = (\{R_1, \ldots, R_{l}\}, \{s_1, \ldots, s_{q}\}, \nu)$. Let
% $\mathcal{K}$ be a class of $\tau$-structures. Let $L$ be an extension of
% first-order logic with a number sort. A many-sorted Lindst\"{o}m quantifier
% $Q_{\mathcal{K}}$ of arity $\arty : S \ra \nats^2$ is defined such that

% \begin{align*}
%   Q_{\mathcal{K}}((\phi^D_{s_1}, \ldots, \phi^D_{s_q}), (\vec{t}_{s_1} \ldots \vec{t}_{s_q}), (\phi^{\equiv}_{s_1},
%   \ldots, \phi^{\equiv}_{s_q})) ([\vec{x}^i_1 \vec{\mu}^i_1, \ldots, \vec{x}^i_{r_i} \vec{\mu}^{i}_{r_i} \cdot \theta_i)_{i \in [l]}
% \end{align*}

% such that for each $i \in [l]$, $j \in [r_i]$ if $\nu (R_i)(j) = s$ then
% $\arty(s) = (\vert \vec{x}^i_j \vert, \vert \vec{\mu}^i_j \vert$.

% The semantics of the quantifier is defined as follows. For each $s \in S$ let
% \begin{align*}
%   D_s := \{\vec{a} \vec{m} : \vec{a} \in A^{\arty(s)(1)}, \vec{a} \in \phi^D_s, \vec{m} \in \nats^{\arty(s)(2)} and \vec{m} \leq
%   (t_s)^{\mathcal{A}}\}.
% \end{align*}

% The formulas $\Phi^{\equiv}$ We may define an equivalence relation on $D_s$ by


% Let $A_s := $


% For each $i \in [l]$ let $R^A_i$

\section{Particular Cases: Counting and Rank}

\end{document}

→
%  LocalWords:  correpsonding
