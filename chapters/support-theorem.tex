%!TEX root = ../main/thesis.tex
\documentclass[../main/thesis.tex]{subfiles}
\begin{document}

\section{Supports and Supporting Partitions}
We reproduce the definitions of a support and supporting partition from Anderson
and Dawar~\cite{AndersonD17} below.

\begin{definition}
  Let $G \leq \sym_n$ and let $S \subseteq [n]$. Then $S$ is a \emph{support}
  for $G$ if $\stab_n(S) \leq G$.
\end{definition}

\begin{definition}
  Let $G \leq \sym_n$ and $\mathcal{P}$ be a partition of $[n]$. Then
  $\mathcal{P}$ is a \emph{supporting partition} for $G$ if
  $\stab_n(\mathcal{P}) \leq G$.
\end{definition}

Notice that if $\mathcal{P}$ is a supporting partition for $G$ and $P \in
\mathcal{P}$ then $\stab([n] \setminus P) \leq \stab(\mathcal{P}) \leq G$, i.e.\
$[n] \setminus P$ is a support for $G$.

Let $\mathcal{P}, \mathcal{P}'$ be partitions of $[n]$. We say that
$\mathcal{P}'$ is as \emph{coarse} as $\mathcal{P}$ (denoted by $\mathcal{P}
\preceq \mathcal{P}'$) if for all $x \in \mathcal{P}$ there exists $y \in
\mathcal{P}'$ such that $x \subseteq y$. Anderson and Dawar~\cite{AndersonD17}
define an operation $\mathcal{E}$ on pairs of partitions of $[n]$, where
$\mathcal{E} (\mathcal{P}, \mathcal{P}')$ is the partition of $[n]$ consisting
of the equivalence classes of the transitive closure of the relation $\sim$ on
$[n]$ defined by $a \sim b$ if, and only if, there exists $P \in \mathcal{P}
\cup \mathcal{P}'$ such that $a,b \in P$.

Anderson and Dawar show that $\mathcal{E}$ preserves supporting partitions, and
that $\mathcal{E}(\mathcal{P}, \mathcal{P}')$ is as coarse as both $\mathcal{P}$
and $\mathcal{P}'$. We state this result formally in
Proposition~\ref{prop:combining-supporting-patitions}.

\begin{prop}[{\cite[Proposition 2]{AndersonD17}}]
  \label{prop:combining-supporting-patitions}
  Let $G \leq \sym_n$ be a group and let $\mathcal{P}$ and $\mathcal{P}'$ be
  supporting partitions of $G$. Then $\mathcal{P} \preceq
  \mathcal{E}(\mathcal{P}, \mathcal{P}')$, $ \mathcal{P}' \preceq
  \mathcal{E}(\mathcal{P}, \mathcal{P}')$ and $\mathcal{E}(\mathcal{P},
  \mathcal{P}')$ is a supporting partition of $G$.
\end{prop}

Anderson and Dawar, using Proposition~\ref{prop:combining-supporting-patitions},
show that every group $G \leq \sym_n$ has a unique coarsest supporting
partition. We call this partition the \emph{canonical supporting partition}, and
denote it by $\SP (G)$. We now define the notion of a \emph{canonical support}.

\begin{definition}
  Let $G \leq \sym_n$. Let $\| \SP(G) \| = \min \{\vert [n] \setminus P \vert :
  P \in \SP(G) \}$. We say that $G$ has \emph{small support} if $\| \SP(G) \| <
  \frac{n}{2}$.
\end{definition}

Note that if $G$ has small support then there is a part in its canonical
supporting partition of size greater than $\frac{n}{2}$, and this part is the
unique largest part in the partition.

\begin{definition}
  Let $G \leq \sym_n$ such that $G$ has small support. Let $\consp(G) = [n]
  \setminus P$, where $P$ is the largest element of $\SP(G)$. We call
  $\consp(G)$ the \emph{canonical support} of $G$.
\end{definition}

\begin{lem}\label{lem:subgroup-coarse}
  Let $G_1, G_2 \leq \sym_n$ with $G_1 \leq G_2$. Then $\SP(G_1) \preceq
  \SP(G_2)$
\end{lem}
\begin{proof}
  We have that $\stab(\SP(G_1)) \leq G_1 \leq G_2$, and so $\SP(G_1)$ supports
  $G_2$. Since $\SP(G_2)$ is the coarsest supporting partition of $G_2$ we have
  that $\SP(G_1) \preceq \SP(G_2)$.
\end{proof}

\begin{lem}
  Let $G_1, G_2 \leq \sym_n$ such that $G_1 \leq G_2$ and $G_1$ has small
  support. Then $G_2$ has small support and $\consp(G_2) \subseteq \consp(G_1)$.
  \label{lem:support-containment}
\end{lem}
\begin{proof}
  By Lemma~\ref{lem:subgroup-coarse}, we have that $\SP(G_1) \preceq \SP(G_2)$.
  Let $P_1$ be the largest element of $\SP(G_1)$, and note that since $G_1$ has
  small support, $\vert P_1 \vert > \frac{n}{2}$. Then there exists a (unique)
  $P_2 \in \SP(G_2)$ such that $P_1 \subseteq P_2$. Thus $G_2$ has small
  support, and $\consp(G_2) = [n] \setminus P_2 \subseteq [n] \setminus P_1 =
  \consp(G_1)$.
\end{proof}

\begin{lem}
  Let $G, H, K \leq \sym_n$ such that $G$ has small support and $G = H \cap K$.
  Then $H$ and $K$ have small support and $\consp(G) = \consp(H) \cup
  \consp(K)$.
  \label{lem:row-column-supports-well-behaved}
\end{lem}
\begin{proof}
  The fact that $H$ and $K$ have small support follows from
  Lemma~\ref{lem:support-containment}.

  Let $\mathcal{Q} := \{P_H \cap P_K : P_H \in \SP(H), P_K \in \SP(K) \text{ and
  } \exists P \in \SP(G), P \subseteq P_H \cap P_K \} $. We first show that
  $\mathcal{Q}$ is a supporting partition of $G$. We have that for any $P_H \cap
  P_K, P_H' \cap P_K' \in \mathcal{Q}$, $P_H \cap P_K \cap P_H' \cap P_K' \neq
  \emptyset$ if, and only if, $P_H = P_H'$ and $P_K = P_K'$ if, and only if,
  $P_H \cap P_K = P_H' \cap P_K'$. By Lemma~\ref{lem:subgroup-coarse}, we have
  that $\SP(G) \preceq \SP(H)$ and $\SP(G) \preceq \SP(K)$. It follows that for
  each $P \in \SP(G)$ there exists $P_K \in \SP(K)$ and $P_H \in \SP(H)$ such
  that $P \subseteq P_H \cap P_K$. So, for each $a \in [n]$ there is $P_a \in
  \SP(G)$, $P_H \in \SP(H)$ and $P_K \in \SP(K)$ such that $a \in P_a \subseteq
  P_H \cap P_K$. It follows that $\mathcal{Q}$ is a partition.

  Moreover, we note that $\stab(\mathcal{Q}) \leq \stab(\SP(H))$ and
  $\stab(\mathcal{Q}) \leq \stab(\SP(K))$, and thus $\stab(\mathcal{Q}) \leq
  \stab(\SP(H)) \cap \stab(\SP(K)) \leq H \cap K = G$. It follows that
  $\mathcal{Q}$ is a supporting partition of $G$, and so by definition of the
  canonical supporting partition $\mathcal{Q} \preceq \SP (G)$. Since $G$ has
  small support, there is a part $P_G$ in $\SP(G)$ with $|P_G| > \frac{n}{2}$.
  Then there exists $P_H \in \SP(H)$ and $P_K \in \SP(K)$ such that $P_G
  \subseteq P_H \cap P_K$, and so $\vert P_H \cap P_K \vert > \frac{n}{2}$. Thus
  $P_H \cap P_K$ is the unique largest element in $\mathcal{Q}$. It follows from
  $\mathcal{Q} \preceq \SP(G)$ that $P_G \subseteq P_H \cap P_K \subseteq P_G$.
\end{proof}

We state the following two results proved by Anderson and
Dawar~\cite{AndersonD17}.

\begin{lem}
  \label{lem:SP-conjugation}
  Let $G \leq \sym_n$ and $\sigma \in \sym_n$ then $\sigma \SP (G) = \SP(\sigma
  G \sigma^{-1})$.
\end{lem}

\begin{lem}
  For any $G \leq \sym_n$ we have that $\stab (\SP (G)) \leq G \leq
  \setstab(\SP(G))$.
\end{lem}

\section{Group Action on Supports}
\label{subsec:group-actions-on-supports}
In this paper, unlike in the case of Anderson and Dawar \cite{AndersonD17}, we
often wish to speak of supports and group actions on elements of the circuit
other than gates, particularly elements of the universe or index of a gate. In
this subsection we develop theory and terminology for dealing with group actions
and supports in this more general setting.

\begin{definition}
  Let $X$ be a set on which a left group action of $G \leq \sym_n$ is defined.
  We denote the \emph{canonical supporting partition} of $x \in X$ by $\SP_G (x)
  = \SP (\stab_G(x))$. Similarly we let $\| \SP_G (x) \| = \| \SP (\stab_G(x))
  \|$. We say that $x \in X$ has \emph{small support} if $\stab_G(x)$ has small
  support. We say that $X$ has small supports if each $x \in X$ has small
  support. If $x \in X$ has small support (i.e.\ $\| \SP_G(x) \| <
  \frac{n}{2}$), we denote the \emph{canonical support} of $x$ by $\consp_G(x) =
  \consp (stab_G(x))$. We refer to $\SP_G$ (resp.\ $\consp_G(g)$) as the
  canonical supporting partition (resp.\ canonical support) of $x$
  \emph{relative to $G$}.
\end{definition}

If the subgroup $G \leq \sym_n$ is obvious from context we omit the subscript in
the canonical support and canonical support partition without the subscript. The
group action most often referenced in this paper is the action of the entire
group $\sym_n$ on the gates of a symmetric circuit with unique extensions given
by extending a permutation to an automorphism of the circuit. In this case we
omit the subscript.

We are often interested in group actions of some set $G \lneq \sym_n$ on $X$,
where $G$ is the stabiliser group of some object (or set of objects) and $X$ is
some set of gates or the universe of a gate.

We are often interested in defining a group action of some stabiliser group on
some part of the circuit. In this case we make use of supports relative to some
$G \lneq \sym_n$, where $G$ is the stabiliser group of some object or set of
objects. In this case we make use of the following abbreviations in order to
avoid complex subscripts. Let $X_1$ and $X_2$ be sets on which a group action of
$\sym_n$ is defined, and let $x \in X_1$ and $S \subseteq X_2$. Let $\SP_S(x)$
and $\consp_S(x)$ abbreviate $\SP_{\stab(S)}(x)$ and $\consp_{\stab(S)}(x)$,
respectively. Similarly, we let $\orb_S(x)$ and $\stab_S(x)$ abbreviate
$\orb_{\stab(S)}(x)$ and $\stab_{\stab(S)}(x)$, respectively. In the event that
$S$ is a singleton we omit the set braces in the subscript.

\begin{lem}
  \label{lem:stab-conjugation}
  Let $X$ be a set on which a left group action of $G \leq \sym_n$ is defined
  and let $\sigma \in G$. Then for any $x \in X$, $\sigma \stab_G (x)
  \sigma^{-1} = \stab_G(\sigma x)$.
\end{lem}

\begin{proof}
  Let $\pi \in \stab_G(x)$, then $\sigma \pi \sigma^{-1}(\sigma x) = \sigma \pi
  x = \sigma x$, and so $\sigma \pi \sigma^{-1} \in \stab_G(\sigma x)$. Let $\pi
  \in \stab_G(\sigma x)$ then $\pi (\sigma x) = \sigma x$ and so $\sigma^{-1}
  \pi \sigma x = x$. It follows that $\sigma^{-1} \pi \sigma \in \stab_G(x)$ and
  so $\pi = \sigma (\sigma^{-1} \pi \sigma) \sigma ^{-1} \in \sigma \stab_G(x)
  \sigma^{-1}$.
\end{proof}

\begin{lem}
  \label{lem:support-mapping}
  Let $X$ be a set on which a left group action of $G \leq \sym_n$ is defined
  and let $\sigma \in G$. Then for any $x \in X$ it follows that $\sigma \SP_G
  (x) = \SP_G (\sigma x)$ and, if $x$ has small support, $\sigma \consp_G (x) =
  \consp_G (\sigma x)$.
\end{lem}
\begin{proof}
  We have Lemmas~\ref{lem:SP-conjugation} and~\ref{lem:stab-conjugation} that
  $\sigma \SP_G (x) = \sigma \SP(\stab_G (x)) = \SP(\sigma \stab_G(x)
  \sigma^{-1}) = \SP (\stab_G(\sigma x)) = \SP_G (\sigma x)$, proving the first
  part of the statement.

  From the fact that $\| \SP_G(x) \| < \frac{n}{2}$ it follows there exists a
  unique $P \in \SP_G(x)$ such that $\vert P \vert > \frac{n}{2}$ and
  $\consp_G(x) = [n] \setminus P$. But then $\sigma \consp_G (x) = \sigma([n]
  \setminus P) = [n] \setminus (\sigma P)$. We note that $\sigma P \in \sigma
  \SP_G (x) = \SP_G(\sigma x)$ and $\vert \sigma P \vert > \frac{n}{2}$. Thus
  $\sigma P$ is the unique largest part in $\SP_G(\sigma x)$, and so
  $\consp_G(\sigma x) = [n] \setminus (\sigma P) = \sigma \consp_G(x)$.
\end{proof}

\begin{lem}
  Let $X$ be a set on which a left group action of $G \leq \sym_n$ is defined.
  Let $x \in X$ be such that $x$ has small support and let $\sigma, \sigma' \in
  G$. If $\sigma (a) = \sigma' (a)$ for all $a \in \consp_G(x)$ then $\sigma (x)
  = \sigma'(x)$.
  \label{lem:support-determine-action}
\end{lem}
\begin{proof}
  We have that $(\sigma')^{-1}\sigma \in \stab(\consp_G(x)) \subseteq
  \stab_G(x)$ and so $(\sigma')^{-1} \sigma (x) = x$ and thus $\sigma (x) =
  \sigma' (x)$.
\end{proof}

\section{The Support Theorem}
The support theorem of Anderson and Dawar~\cite{AndersonD17} gives us an upper
bound on the size of the support of a gate in a symmetric circuit with symmetric
gates in terms of the size of the circuit. In particular, their result implies
that if $(C_n)_{n \in \nats}$ is a polynomial-size family of symmetric circuits
with symmetric gates then for large enough $n$ every gate in $C_n$ has a
constant-size canonical support. In this section we generalise the support
theorem to $(\mathbb{B}, \rho)$-circuits with unique labels defined over
arbitrary bases.

The proof of this result follows a strategy broadly similar to the one used
in~\cite{AndersonD17}, and makes use of two lemmas from there. The first of
these lemmas gives us that if the index of a group $G \leq \sym_n$ is small then
$\SP(G)$ either has very few or very many parts. The second lemma gives us that
for $G \leq \sym_n$, if $\SP(G)$ has very few parts then it must have a single
very large part (and hence a small canonical support). These two results allow
us to conclude that a gate in a symmetric circuit has a small canonical support
if it has a canonical supporting partition with very few parts. We then prove by
structural induction that the canonical supporting partition of every gate has
few parts. To be precise, we show that if $g$ is the first gate in the circuit
with a canonical supporting partition with too many parts, then the size of its
orbit would exceed the bound on the size of the circuit.

We begin by first building up a few observations about the size of the
stabiliser group, and hence the orbit, of a gate. We first introduce some useful
notation needed in this subsection. Let $C$ be a symmetric circuit of order $n$
with unique labels and let $g$ be a gate in the circuit. Let $\vec{a}_R \in
\ind(g)$ and $\sigma \in \sym_n$ be such that $\sigma L(g)(\vec{a}_R) \in H_g$.
We let $\sigma \cdot \vec{a}_R := L(g)^{-1}\sigma L(g)(\vec{a}_R)$. We note that
$\sigma \cdot \vec{a}_R$ may not be defined for every $\sigma \in \sym_n$, and
so this operation does not define a group action of $\sym_n$ on $\ind(g)$.
However, we note that this operation does define a group action of
$\setstab_n(H_g)$ on $\ind(g)$.

Moreover, we notice that, for $\sigma \in \sym_n$, we have that $\sigma H_g =
H_g$ and $L(g)$ is isomorphic to $\sigma L(g)$ if, and only if, $\sigma \in
\stab_n(g)$ if, and only if, $\sigma H_g = H_g$ and there exists $\pi \in
\aut(g)$ such that $\sigma \cdot \vec{a}_R = \pi (\vec{a}_R)$ for all $\vec{a}_R
\in \ind(g)$. The final equivalence expresses the fact that $\sigma \in \sym_n$
fixes $g$ if, and only if, $\sigma$ acts like an element of $\aut(g)$ on
$\ind(g)$.

We now use this observation to establish a useful characterisation of the
stabiliser group of a gate. We first prove, more generally, that if $\sigma \in
\sym_{\ind(\tau, D)}$ (for some vocabulary $\tau$ and universe $D$), then
$\sigma$ acts on $\ind(\tau, D)$ like an element of $\aut(\tau, D)$ if, and only
if, for every pair of elements in $\ind(\tau, D)$, $\sigma$ acts like a partial
automorphism of $\str{\tau, D}$ on this pair of elements. We then apply this
result to circuits, proving that if $\sigma$ extends to an automorphism of the
circuit, then $\sigma$ fixes a gate $g$ if, and only if, $\sigma$ acts like a
partial automorphism on the labels associated with every pair of gates in $H_g$.
We now define what it means for a permutation to act like a partial automorphism
and prove these two results.

\begin{definition}
  Let $\tau$ be a relational vocabulary and $D$ be $\tau$-sorted set. Let $X :=
  \ind(\tau, D)$, $\sigma \in \sym_X$ and $\vec{a}_R, \vec{b}_{R'} \in X$. We
  say that $\sigma$ \emph{acts like a partial automorphism} on $(\vec{a}_R,
  \vec{b}_{R'})$ if (i) $\sigma \vec{a}_R$ has tag $R$ and $\sigma \vec{b}_{R'}$
  has tag $R'$ and (ii) for all $i \in [\arty(R)]$ and $j \in [\arty(R')]$,
  $\vec{a}(i) = \vec{b}(j)$ if, and only if, $(\sigma \vec{a}_R)(i) = (\sigma
  \vec{b}_{R'})(j)$.
\end{definition}

We now define an analogous notion for pairs of gates in a circuit.

\begin{definition}
  Let $C$ be a symmetric circuit of order $n$ with unique labels and let $g$ be
  a gate in the circuit. Let $\sigma \in \sym_n$ and $h, h' \in H_g$. Let
  $\vec{a}_R:= L(g)^{-1}(h)$ and $\vec{b}_{R'} := L(g)^{-1}(h')$. We say that
  $(h, h')$ is \emph{compatible} with $\sigma$ if $\sigma h, \sigma h' \in H_g$
  and $\sigma$ acts like a partial automorphism on $(\vec{a}_R, \vec{b}_{R'})$
\end{definition}
  
We now show that a permutation acts like an element of $\aut(\tau,D)$ if, and
only if, it acts like a partial automorphism on every pair of elements in the
domain.

\begin{lem}
  Let $\tau$ be a relational vocabulary and $D$ be $\tau$-sorted set. Let $X :=
  \ind(\tau, D)$, $\sigma \in \sym_X$ and $\vec{a}_R, \vec{b}_{R'} \in X$. Then
  $\sigma$ acts like an element of $\aut (\tau, D)$ if, and only if, for all
  $\vec{a}_R, \vec{b}_{R'} \in X$, $\sigma$ acts like a partial automorphism on
  $(\vec{a}_R, \vec{b}_{R'})$.
  \label{lem:aut-partial}
\end{lem}
\begin{proof}
  `$\Rightarrow$': Suppose there exists $\pi \in \aut(\tau, D)$ such that
  $\sigma \vec{a}_R = \pi \vec{a}_R$ for all $\vec{a}_R \in X$. It follows that
  the action of $\sigma$ on $X$ preserves tag. Let $\vec{a}_R, \vec{b}_{R'} \in
  X$, $i \in [\arty (R)]$,and $j \in [\arty (R')]$. Then $(\sigma \vec{a}_R) (i)
  = (\pi \vec{a}_R)(i) = (\pi \vec{b}_{R'})(j) = (\sigma \vec{b}_{R'})(j)$ if,
  and only if, $(\pi \vec{a}_R)(i) = \pi (\vec{a}(i)) = \pi (\vec{b}(j)) =
  (\pi\vec{b}_{R'})(j)$ if, and only if, $\vec{a}(i) = \vec{b}(j)$. The final
  equivalence follows from the injectivity of $\pi$.
  
  `$\Leftarrow$': Suppose for all $\vec{a}_R, \vec{b}_{R'} \in X$, $\sigma$ acts
  like a partial automorphism on $(\vec{a}_R, \vec{b}_{R'})$. We now define
  $\pi$ such that $\pi \in \aut(\tau, D)$ and $\sigma \vec{a}_R = \pi \vec{a}_R$
  for all $\vec{a}_R \in X$. Let $a \in D$ and $\vec{a}_R \in X$ be an element
  containing $a$. Let $\pi (a) = (\sigma \vec{a}_R) (\vec{a}^{-1}_R(a))$. Since
  $\sigma$ acts like a partial automorphism on all pairs of elements of $X$, the
  definition of $\pi$ is independent of the particular choice of $\vec{a}_R$ and
  for all $a \in D$, $a$ and $\pi (a)$ must be of the same sort.
  
  Since $\sigma$ acts like a partial automorphism on each pair $(\vec{a}_R,
  \vec{b}_{R'})$ it is easy to see that $\pi$ is an injection. Let $c \in D$.
  Suppose $c$ appears in $\vec{a}_R \in X$ and let $i \in [\arty(R)]$ be such
  that $\vec{a}_R(i) = c$. Since the action of $\sigma$ is a permutation, there
  exists $\vec{b}_R \in X$ such that $\sigma \vec{b}_R = \vec{a}_R$. It follows
  that $\pi (\vec{b}_R(i)) = (\sigma \vec{b}_R) (\vec{b}^{-1}_R(\vec{b}_R(i))) =
  \vec{a}_R(i) = c$. Thus $\pi$ surjective, and so bijective, and $\pi \in
  \aut(\tau, D)$.

  Clearly $\pi \vec{a}_R = \sigma \vec{a}_R$ for all $\vec{a}_R \in X$, and the
  result follows.

\end{proof}

We now prove an analogous result in the context of circuits, giving us a useful
characterisation of the stabiliser group of a gate.

\begin{lem}
  Let $C$ be a symmetric circuit of order $n$ with injective labels, $g$ be a
  gate in the circuit, and $\sigma \in \sym_n$. Then $\sigma \in \stab_n(g)$ if,
  and only if, $\sigma H_g = H_g$ and $\sigma L(g)$ is isomorphic to $L(g)$ if,
  and only if, for all $h, h' \in H_g$, $(h, h')$ is compatible with $\sigma$.
  \label{lem:isostab-compatible}
\end{lem}

\begin{proof}
  The result follows from the following equivalences.
  \begin{align*}
    & \sigma \in \stab_n\\
    & \Leftrightarrow \text{ $\sigma H_g = H_g$ and $\sigma L(g)$ is
      isomorphic to $L(g)$}\\
    & \Leftrightarrow \text{$\sigma H_g = H_g$ and $\sigma H_g$ acts like an
      element of $\aut (g)$ on $\ind(g)$} \\
    & \Leftrightarrow \text{$\sigma H_g = H_g$ and for all $\vec{a}_R,
      \vec{b}_{R'} \in \ind(g)$, $\sigma$ acts like a partial automorphism on
      $(\vec{a}_R, \vec{b}_{R'})$}\\
    & \Leftrightarrow \text{for all $h, h' \in H_g$, $(h, h')$ is compatible with
      $\sigma$}
  \end{align*}

  The second equivalence follows from the fact that $C$ has injective labels and
  the third equivalence follows from Lemma~\ref{lem:aut-partial}.

\end{proof}

In the proof of the support theorem we aim to show that if we a gate $g$ has a
canonical support that is too big then the orbit of $g$ is also necessarily
large. We do this by establishing the existence of a large set of permutations
that each take $g$ to a different gate. To construct this set, we define a set
of triples of the form $(\sigma, h, h')$ where $\sigma \in \sym_n$ and $h,h' \in
H_g$. Each of these triples is useful (formally defined below) in a sense that
guarantees that $\sigma$ moves $g$. Moreover, the triples are pairwise
independent which means that we can compose them in arbitrary combinations to
generate new permutations moving $g$, while guaranteeing that each such
combination gives us a different element in the orbit of $g$. We now define
these terms formally and prove the result.

\begin{definition}
  Let $C$ be a circuit with injective labels of order $n$ and $g$ be a gate in
  $C$. We say that $(\sigma, h, h') \in \sym_n \times H^2_g$ is \emph{useful} if
  $(h, h')$ is not compatible with $\sigma$.

  We say that two distinct pairs $(\sigma_1, h_1, h_1'), (\sigma_2, h_2, h_2')
  \in \sym_n\times H^2_g$ are \emph{mutually independent} if
  \begin{myitemize}
  \item $\sigma_2 h_1 = h_1$,
  \item $\sigma_2 \sigma_1 h_1 = \sigma_1 h_1$,
  \item $\sigma_2 h_1' = h_1'$, and
  \item $\sigma_2 \sigma_1 h_1' = \sigma h_1'$.
  \end{myitemize}
  We say that a set $S \subseteq \sym_n \times H^2$ is \emph{useful} (at $g$) if
  each pair in it is useful. We say that $S$ is \emph{independent} (at $g$) if
  every two distinct elements of $S$ are mutually independent.
\end{definition}

We now prove the upper-bound on the powerset of useful and independent sets used
in the proof of the support theorem.

\begin{lem}
  \label{lem:useful-independant-set}
  Let $C$ be a circuit of order $n$ and let $g$ be a gate in that circuit. If
  $S$ is a useful and independent set at $g$ then $[\sym_n : \stab_n(g)] \geq
  2^{\vert S \vert}$.
\end{lem}
\begin{proof}
  We have that $\sigma \in \stab_n(g)$ if, and only if, $H_g = \sigma H_g$ and
  $L(g)$ is isomorphic to $\sigma L(g)$. For any $R \subseteq S$ define
  $\sigma_R = \Pi_{(\sigma, h, h') \in R} \sigma$ (with some arbitrary linear
  order assumed on $S$). We prove that the mapping $R \mapsto \sigma_R
  \in\aut(C)$ is injective. Let $R$ and $Q$ be distinct subsets of $S$ and,
  without loss of generality, let $\vert R \vert \geq \vert Q \vert$.

  We prove that if $\sigma^{-1}_Q \sigma_R H_g = H_g$ then $\sigma^{-1}_Q
  \sigma_R L(g)$ is not isomorphic to $L(g)$.

  Pick any $(\sigma, h, h') \in R \setminus Q \neq \emptyset$. From independence
  we have $\sigma_R h = \sigma h$, $\sigma_R h' = \sigma h'$, $\sigma_Q \sigma h
  = \sigma h $, and $\sigma_Q \sigma h = \sigma h$. Thus $\sigma^{-1}_Q \sigma_R
  h = \sigma^{-1}_Q \sigma h = \sigma h$, and similarly $\sigma^{-1}_Q\sigma_R
  h' = \sigma h'$. Thus, since $(h, h')$ is incompatible with $\sigma$, it
  follows $(h, h')$ is incompatible with $\sigma^{-1}_Q\sigma_R$. So Lemma
  \ref{lem:isostab-compatible} gives us that $\sigma^{-1}_Q \sigma_R L(g)$ is
  not isomorphic to $L(g)$ and the result follows.
\end{proof}

The following two lemmas are proved by Anderson and Dawar \cite{AndersonD17},
and are both useful for proving the support theorem.
Lemma~\ref{lem:big-or-small} is used to establish a size bound on the supporting
partition of a gate. In particular, it shows that for a partition $\mathcal{P}$
of $[n]$, if the index of $\setstab (\mathcal{P})$ in $\sym_n$ is small enough,
then $\mathcal{P}$ either contains very few or very many parts.

\begin{lem}[{\cite[Lemma 5]{AndersonD17}}]
  \label{lem:big-or-small}
  For any $\epsilon$ and $n$ such that $0 < \epsilon < 1$ and $\log n \geq
  \frac{4}{\epsilon}$, if $\mathcal{P}$ is a partition of $[n]$ with $k$ parts,
  $s = [\sym_n : \setstab (\mathcal{P})]$ and $n \leq s \leq
  2^{n^{1-\epsilon}}$, then $\min \{k, n-k\} \leq \frac{8}{\epsilon} \frac{\log
    s}{\log n}$.
\end{lem}

Lemma~\ref{lem:small-means-support} gives us that, under the same assumptions as
Lemma~\ref{lem:big-or-small}, if the number of parts in $\mathcal{P}$ is less
then $\frac{n}{2}$ then $\mathcal{P}$ contains a very large part. This lemma is
used both to establish that the stabiliser group of a gate has small support,
and hence a canonical support, but also that this canonical support has bounded
size (and is in fact constant for polynomial-size circuits).

\begin{lem}[{\cite[Lemma 6]{AndersonD17}}]
  \label{lem:small-means-support}
  For any $\epsilon$ and $n$ such that $0 < \epsilon < 1$ and $\log n \geq
  \frac{8}{\epsilon^2}$, if $\mathcal{P}$ is a partition of $[n]$ with $\vert
  \mathcal{P} \vert \leq \frac{n}{2}$, $s:= [\sym_n : \setstab (\mathcal{P})]$
  and $n \leq s \leq 2^{n^{1-\epsilon}}$, then $\mathcal{P}$ contains a part $P$
  with at least $n - \frac{33}{\epsilon} \cdot \frac{\log s} {\log n}$ elements.
\end{lem}

We are now ready to prove the support theorem for symmetric circuits with
non-symmetric gates.

\begin{thm}
  \label{thm:support-thm}
  For any $\epsilon$ and $n$ such that $\frac{2}{3} \leq \epsilon \leq 1$ and $n
  \geq \frac{128}{\epsilon^2}$, if $C$ is a symmetric circuit of order $n$ with
  unique labels and $s := \max_{g \in C} \vert \orb (g)\vert \leq
  2^{n^{1-\epsilon}}$, then, $\SP(C) \leq \frac{33}{\epsilon}\frac{\log s}{\log
    n}$.
\end{thm}

\begin{proof}
  First we note that if $1 \leq s < n$, then $C$ cannot have a relational gate,
  as the orbit of a relational gate has at least $n$ elements. Since $C$ has no
  relational gates, the only input gates in the circuit are the constant gates.
  Since constant gates are fixed by all permutations, it follows that any gate
  $g$ whose children are constant gates must similarly be fixed under all
  permutations. Furthermore, from the fact that $C$ has unique labels, and hence
  unique extensions, this property inductively extends to the rest of the
  circuit. Thus for each gate $g$ in $C$ the partition $\{[n]\}$ supports $g$,
  and since this is trivially the coarsest such partition it follows that $\|
  \SP(g) \| = 0 = \SP(C)$. We therefore assume that $s \geq n$.

  If $g$ is a gate in $C$ then $\stab (g) \leq \setstab(\SP(g))$, and so $s \geq
  \vert \orb(g) \vert = [\sym_n : \stab(g)] \geq [\sym_n : \setstab(\SP(g))]$.
  Thus if $\vert \SP(g) \vert \leq \frac{n}{2}$, then from Lemma
  \ref{lem:small-means-support}, we have $\| \SP (g) \| \leq \frac{33}{\epsilon}
  \cdot \frac{\log s} {\log n}$. The result thus follows from showing that for
  each $g$ in $C$ we have that $\vert \SP (g) \vert \leq \frac{n}{2}$.
  
  If $g$ is a constant gate, then as argued above, it follows $\vert \SP(g)
  \vert = 0 < \frac{n}{2}$. If $g$ is a relational gate, then $g$ is fixed by a
  permutation $\sigma \in \sym_n$ if, and only if, $\sigma$ fixes all elements
  that appear in $\Lambda(g)$. It follows that $\{a\} \in \SP(g)$ for each $a$
  appearing in $\Lambda(g)$ and all other elements of $[n]$ are contained in a
  single part of $\SP(g)$. But suppose $\vert \SP(g) \vert > \frac{n}{2}$. Then
  the number of singletons in $\SP(g)$ must be larger than $\frac{n}{2}$, which
  in turn, from the orbit-stabiliser theorem, gives us that $ s \geq \vert \orb
  (g) \vert \geq \frac{n!}{(n-\vert \Lambda (g) \vert)!} \geq
  \frac{n!}{{(\frac{n}{2}})!} \geq 2^{\frac{n}{4}} > 2^{n^{1- \epsilon}} $. This
  is a contradiction, and so $\vert \SP(g) \vert \leq \frac{n}{2}$.

  We now consider the internal gate case. Let $g$ be the topologically first
  internal gate with $\vert \SP(g) \vert > \frac{n}{2}$. Let $k' := \lceil
  \frac{8 \log s}{\epsilon \log n} \rceil$. From the assumptions on $s$, $ n$
  and $\epsilon$ we have that $k' \leq \frac{1}{4}n^{1-\epsilon} < \frac{n}{2}$.
  We note that Lemma~\ref{lem:big-or-small} implies that $n - \vert \SP(g) \vert
  \leq k'$.
  
  We now construct a sufficiently large useful and independent set of triples
  which, using Lemma~\ref{lem:useful-independant-set}, allows us to place a
  lower bound on the orbit size of $g$. Divide $[n]$ into $\lfloor \frac{n}{k' +
    2} \rfloor$ disjoint sets $S_i$ of size $k' + 2$ and ignore the elements
  left over. It follows that for each $i$ there is a permutation $\sigma_i$
  which fixes $[n] \setminus S_i$ pointwise but moves $g$. Suppose there was no
  such $\sigma_i$, but then every permutation that fixes $[n]\setminus S_i$
  pointwise fixes $g$. Thus the partition of all the singletons in $[n]\setminus
  S_i$ and $S_i$ is a supporting partition of $g$. As $\SP(g)$ is the coarsest
  such partition it follows that $\vert \SP(g) \vert \leq n - (k'+2) + 1 = n -
  k' - 1$, which contradicts the inequality $n - \vert \SP(g) \vert \leq k'$.

  Since $g$ is moved by each $\sigma_i$, and $C$ has unique labels, it follows
  from Lemma~\ref{lem:isostab-compatible}, and the fact that the circuit has
  unique labels, that there exists $(h_i, h_i') \in H_g$ such that $(h_i, h_i')$
  is not consistent with $\sigma_i$, and so the triple $(\sigma_i, h_i, h_i')$
  is useful.

  Note that our choice of $g$ guarantees that for all $h \in H_g$, $\vert \SP(h)
  \vert \leq \frac{n}{2}$, and so, from Lemma \ref{lem:small-means-support} and
  the hypothesis of this theorem, $h$ has small support. Let $Q_i = \consp(h_i)
  \cup \consp(\sigma_i h_i) \cup \consp(h_i') \cup \consp(\sigma_i h_i')$. Then
  note that if $\sigma_j$ fixes $Q_i$ pointwise then by construction we have
  that $\sigma_j \in \stab(\SP(h_i)) \cap \stab(\SP(\sigma_i h_i)) \cap
  \stab(\SP(h_i')) \cap \stab(\SP(\sigma_i h_i'))$

  Define a directed graph $K$ with vertices given by the sets $S_i$ and an edge
  from $S_i$ to $S_j$ (with $i \neq j$) if, and only if, $Q_i \cap S_j \neq
  \emptyset$. It follows then that if there is no edge from $S_i$ to $S_j$ then
  $Q_i \subseteq [n]\setminus S_j$, and so $\sigma_j$ fixes $Q_i$ pointwise,
  giving us that $(\sigma_i, h_i, h_i')$ and $(\sigma_j, h_j, h_j')$ are
  mutually independent. It remains to argue that $K$ has a large independent
  set. This is possible as the out-degree of $S_i$ in $K$ is bounded by
  \begin{align*}
    \vert Q_i \vert \leq \|\SP(h_i) \| + \|\SP(\sigma_i h_i) \| + \|\SP(h_i') \| + \|\SP(\sigma_i h_i') \leq 4 \cdot \frac{33\log s}{\epsilon \log n}.
  \end{align*}
  These inequalities follow from the fact that the sets $S_i$ are disjoint and
  we may apply Lemma \ref{lem:small-means-support} to each of the child gates.
  From these inequalities we have that the average total degree (in + out
  degree) of $K$ is at most $2 \cdot \vert Q_i \vert \leq 34 \cdot k'$. Now
  greedily select a maximal independent set in $K$ by repeatedly selecting $S_i$
  with the lowest total degree and eliminating it and its neighbours. This
  action does not affect the bound on the average total degree of $K$ and hence
  determines an independent set $I$ in $K$ of size at least
  \begin{align*}
    \frac{\lfloor \frac{n}{k' + 2} \rfloor}{34k' + 1} \geq \frac{n - (k'+2)}{34k'+1k'+2} \geq \frac{n\frac{7}{16}}{34k'^2 + 69k' +2} \geq \frac{n}{(16k')^2}.
  \end{align*}

  Take $S = \{(\sigma_i, h_i, h_i') : S_i \in I \}$. Then from the above
  argument we have that $S$ is useful and independent.
  
  Moreover, from Lemma~\ref{lem:useful-independant-set}, we have that $s \geq
  \vert \orb(g) \vert \geq 2^{\vert S \vert} \geq 2^{\frac{n}{(16k')^2}}$ then
  $n^{1-\epsilon} \geq \log s \geq n \cdot (\frac{128}{\epsilon}\frac{\log
    s}{\log n})^{-2} > n \cdot (n^{1-\epsilon})^{-2} = n^{2\epsilon -1} \geq
  n^{1-\epsilon}$. This is a contradiction, and the result follows.
\end{proof}
 
Let $\mathcal{C} = (C_n)_{n \in \nats}$ be a polynomial-size family of symmetric
circuits with unique labels. Then $s(n) := \max_{g \in C_n} \vert \orb (g)\vert$
must be polynomially bounded, and so the theorem implies that there exists $k
\in \nats$ such that for all $n$ large enough, for all $g \in C_n$, $\vert
\consp(g) \vert \leq k$.

\begin{cor}
  Let $\mathcal{C} := (C_n)_{n \in \nats}$ be a polynomial-size family of
  symmetric circuits with unique labels. There is a $k$ such that $\SP(C_n) \leq
  k$ for all $n$.
  \label{cor:constant-size-support}
\end{cor}

\subsection{Supports on Indexes}
We have shown that families of polynomial-size symmetric circuits with unique
labels have constant-size supports. In this subsection we introduce a group
action on the elements of the universe of a gate, and show that the support
theorem can be extended to apply to the (relative) supports of these elements as well.

Let $C$ be a symmetric circuit of order $n$ with unique labels and $g$ be a gate
in the circuit. Since $C$ is symmetric and has unique labels, we have from
Proposition~\ref{prop:unique-children-unique-extensions} that each permutation in $\sym_n$ extends uniquely to an
automorphism of $C$. We have defined an action of $\setstab(H_g) \leq \sym_n$ on
$\ind(g)$, and noted that $\sigma \in \stab(g)$ if, and only if, $\sigma \in
\setstab(H_g)$ and there exists $\pi_{\sigma} \in \aut(g)$ such that $\sigma
\vec{a}_R = \pi_{\sigma} \vec{a}_R$ for all $\vec{a}_R \in \ind(g)$.

We now define a group action of $\stab(g)$ on $\universe{g}$ by $\sigma \cdot a
:= (\pi_{\sigma}\vec{a}_R) (\vec{a}^{-1}_R(a))$, for $\sigma \in \stab(g)$ and
$a \in \universe{g}$, and where $\vec{a}_R \in \ind(g)$ contains the element
$a$. Since $\pi_{\sigma}$ is an automorphism uniquely determined by $\sigma$
this action is well-defined.

Since we have a group action of $\stab(g)$ on $\universe{g}$, but not $\sym_n$
on $\universe{g}$, we must speak of the support of $a \in \universe{g}$ relative
to $\stab(g)$.  We now prove an analogue of the support theorem, showing 
that for any polynomial-size family $(C_n)_{n \in \nats}$ of symmetric circuits
with unique labels, we have for large enough $n$ that for each $g$ in $C_n$ the
sizes of the supports of $a \in \universe{g}$ relative to the subgroups $\stab(g)$ and
$\spstab{g}$ are bounded by a constant. In the proof of the following result we
make use of the abbreviations introduced in Section~\ref{subsec:group-actions-on-supports} for relative supports.

\begin{lem}
  \label{lem:row-column-supports}
  Let $(C_n)_{n \in \nats}$ be a polynomial-size family of symmetric circuits
  with unique labels. There exists $n_0, k \in \nats$ such that for all $n >
  n_0$, $g$ a gate $C_n$ and $a \in \universe{g}$
  \begin{myitemize}
  \item $\stab_n(g)$, $\stab_{\consp(g)}(a)$ and $\stab_{g}(a)$ have small
    supports,
  \item if $h \in H_g$ and $a$ appears in $L(g)^{-1}(h)$ then $\consp_g(a)
    \subseteq \consp_{\consp(g)}(a) \subseteq \consp(g) \cup \consp(h)$, and
  \item $\vert \consp(g) \leq k$ and $\vert \consp_g(a) \vert \leq \vert
    \consp_{\consp(g)}(a) \vert \leq 2k$.
  \end{myitemize}
\end{lem}

\begin{proof}
  We have from Corollary~\ref{cor:constant-size-support} that there exists $k,
  n_0' \in \nats$ such that for all $n \geq n_0'$, $\SP(C_n) \leq k$. Let $n_0 =
  \max(n_0', 4k + 1)$. Let $g$ be a gate in $C_n$, $a \in \universe{g}$, and $h
  \in H_g$ be such that $a \in \vec{a}_R := (a_1, \ldots , a_r)_R :=
  L(g)^{-1}(h)$.

  Since $\SP(C_n) \leq k$ and $n > 2k$, we have that $g$ and $h$ have small
  supports of size at most $k$. Moreover, we have that $\stab(\consp(h) \cup
  \consp(g)) = \stab(\consp(h)) \cap \stab(\consp(g)) \leq \stab(h) \cap
  \spstab{g} = \stab_{\consp(g)}(h)$. It follows that $\consp(h) \cup \consp(g)$
  supports $\stab_{\consp(g)}(h)$ and so, since $\vert \consp(h) \cup \consp(g)
  \vert \leq 2k < \frac{n}{2}$, $\stab_{\consp(g)}(h)$ has small support and
  $\consp_{\consp(g)}(h) \subseteq \consp(h) \cup \consp(g)$.

  We also have that $\stab_{\consp(g)}(h) = \bigcap_{i \in
    [r]}\stab_{\consp(g)}(a_i)$, and so by repeated application of
  Lemma~\ref{lem:row-column-supports-well-behaved}, we have that
  $\consp_{\consp(g)}(a) \subseteq \bigcup_{i \in [r]}\consp_{\consp(g)}(a_i) =
  \consp_{\consp(g)}(h) \subseteq \consp(h) \cup \consp(g)$. We thus also have
  that $\vert \consp_{\consp(g)}(a) \vert \leq 2k$.

  Lastly, we note that $\stab_{\consp(g)}(a) \subseteq \stab_g(a)$, and so
  $\stab_g(a)$ has small support and $\consp_g(a) = \subseteq
  \consp_{\consp(g)}(a)$. The result follows.
\end{proof}

\end{document}