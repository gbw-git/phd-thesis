% !TEX root = ../main/thesis.tex
\documentclass[../main/thesis.tex]{subfiles}
\begin{document}

In this chapter we show that formulas of fixed-point logic extended by a set of
generalised operators can be translated into $\PT$-uniform families of
transparent symmetric circuits. More precisely, we first show that each set of
generalised operators $\setop$ defines a Boolean basis $\BB_{\setop}$. We then
show that if a query is definable in $\FP(\setop)[\rho]$ then it is definable by
a $\PT$-uniform family of transparent symmetric $(B_{\setop}, \rho)$-circuits
that defines the same query.

\begin{definition}
  Let $\tau$ be a many-sorted vocabulary. Let $q$ be the number of sorts in
  $\tau$. Let $C$ be a class of $\tau$-structures. For each $q$-sequence of
  finite sets $\vec{X}$ let
  \begin{align*}
    \BB^{\vec{X}}_{C} := \{F^{\vec{X}}_{C}: \{0,1\}^{\ind(\tau, \vec{X})}\ra
    \{0,1\} : \text{ for all }\mathcal{A} \in \fin{\tau,
    \vec{X}},\, F^{\vec{X}}_{C}(\mathcal{A}) = \id_C (\mathcal{A})\}.
  \end{align*}
  Let $Q := Q^n_{C, \ar}$ be an extended quantifier. We call $\BB(Q) :=
  \{\BB^{\vec{X}}_C : \vec{X} = (X_1, \ldots, X_q) \text{ is a sequence of
    finite sets } \}$ the basis \emph{corresponding to} $Q$. Let $\setquant$ be
  a set of extended quantifiers. We call $\BB_{\setquant} := \cup_{Q \in
    \setquant} \BB(Q)$ the basis \emph{corresponding to} $\setquant$. Let
  $\setop$ be a set of operators and let $\setquant$ be the corresponding set of
  quantifiers. We call $\BB_{\setop} := \BB_{\setquant}$ the basis
  \emph{corresponding} to $\setop$.
\end{definition}

% \begin{definition}
%   Let $\tau$ be a many-sorted vocabulary. Let $q$ be the number of sorts in
%   $\tau$. Let $m \in \nats$, let $E : \nats^m \times \fin{\tau} \ra \{0,1\}$
%   be an isomorphism-closed function, and let $\vec{p} \in \nats^m$. Let
%   \begin{align*}
%     \BB_{E, \vec{p}} := \{F_{E, \vec{p}}: \{0,1\}^{\ind(\tau, \vec{d})}\ra
%     \{0,1\} : \vec{d} \in \nats^{q}, \text{ and }\forall \mathcal{A} \in \fin{\tau,
%     \vec{d}},\, F_{E, \vec{p}}(\mathcal{A}) = E(\vec{p}, \mathcal{A})\}.
%   \end{align*}

%   Let $\setquant$ be a set of extended quantifiers. Let $\BB_{\setquant}$ be
%   the union of all $\BB_{E', \vec{p}'}$ such that for some arity function
%   $\ar'$ and $n' \in \nats$ we have $Q^{n', \vec{p}'}_{E', \ar'} \in
%   \setquant$. We say that $\BB_{\setquant}$ is the basis \emph{corresponding}
%   to $\setquant$. Let $\setop$ be a set of operators and let $\setquant$ be
%   the corresponding set of quantifiers. We let $\BB_{\setop} :=
%   \BB_{\setquant}$ and say that $\BB_{\setop}$ is the basis
%   \emph{corresponding} to $\setop$.
% \end{definition}

Let $\setquant$ be a set of extended quantifiers and let $\BB$ be the
corresponding Boolean basis. Let $\rho$ be a vocabulary and let $C$ be a $(\BB,
\rho)$-circuit. Let $\theta(\vec{x})$ be a $\FO(\setquant)[\rho]$-formula. We
say that $C$ \emph{translates} $\theta(\vec{x})$ for $n \in \nats$ if $C$ is a
transparent symmetric circuit of order $n$ and for all $\mathcal{A} \in
\fin{\rho, n}$, $\gamma \in [n]^{\underline{A}}$, and $\alpha \in A^{\vec{x}}$
we have that $\gamma (\alpha (\vec{x})) \in C[\gamma \mathcal{A}]$ if, and only
if, $\mathcal{A} \models \theta[\alpha]$. We say a family of circuits $(C_n)_{n
  \in \nats}$ \emph{translates} $\theta(\vec{x})$ if $(C_n)_{n \in \nats}$ is
$\PT$-uniform and for all $n \in \nats$, $C_n$ translates $\theta(\vec{x})$ for
$n$.

\begin{lem}
  Let $\vec{V} := \{V_1, \ldots, V_v\}$ be a vocabulary. Let $\rho := \{R_1,
  \ldots, R_r\}$ be a non-empty vocabulary with $\vec{V} \cap \rho = \emptyset$
  and let $\rho^* = \vec{V} \cup \rho$. Let $\setquant$ be a set of extended
  quantifiers and let $\BB$ be the corresponding basis. There is a function that
  takes as input a number $n \in \nats$ and a $\FO(\setquant)[\rho^*]$-formula
  $\theta(\vec{x})$ and outputs a $(\BB \cup \BS, \rho^*)$-circuit $C$ that
  translates $\theta$ for $n$. This function is computable and there is a
  polynomial $p$ such that for an input $(n, \theta)$ the algorithm computing
  this function terminates in at most $p(\vert \cl{\theta} \vert (\vert
  \cl{\theta} \vert + \ewidth{\theta}) \cdot n^{\ewidth{\theta}})$ many steps.
  \label{lem:translating-FOquant-to-formulas}
\end{lem}
\begin{proof}
  We assume, without a loss of generality, that at least one relation symbol
  from $\rho^*$ appears in $\theta$. If this is not the case we take a
  conjunction of $\theta$ with a tautology of the form $\forall \vec{x}. \,
  (T(\vec{x}) \lor \neg T(\vec{x}))$ for some $T \in \rho$.

  Fix $n \in \nats$. The structure of this proof is as follows. We first define
  from $\theta$ a formula $\lambda$, which defines the same query and should be
  thought of as a normal form of $\theta$. We then define a circuit $C$ from
  $\lambda$ and prove that it translates $\theta$ for $n$. We conclude by
  showing that we can construct $C$ with within the time bounds.

  Before we define $\lambda$ we first define a few helper formulas. Let $T \in
  \rho$. For a variable $y$ let $\op{no-op}(y) := (T(y \, \dots, y) \lor (\neg
  T(y , \ldots, y)))$ and let $\op{no-op-all}(y) = \forall y .\, \op{no-op}(y)$.
  Let $\vec{y}$ be a (possibly empty) sequence of variables. If $\vec{y}$ is
  non-empty let $(y_1, \ldots, y_m) := \vec{y}$ and let $\op{tag} (\vec{y}) :=
  (\op{no-op}(y_1) \land (\op{no-op}(y_2) \land ( \op{no-op}(y_2) \land ( \ldots
  \land (\op{no-op}(y_m)) \ldots))\ldots ))$. If $\vec{y}$ is empty let
  $\op{tag} (\vec{y}) = \forall u .\, ((u = u) \land (u = u)$. We define a
  similar helper formula $\op{tag-num}$ for $e \in \nats$ and a variable $y$ by
  \begin{align*}
    \op{tag-num}(e, y) = (\bigwedge_{i \in [e]}\op{no-op}(y)) = \underbrace{(\op{no-op-all}(y) \land (\op{no-op-all}(y) \land (\ldots \land (\op{no-op-all}(y)) \ldots )))}_{e \text{ times}}.
  \end{align*}
  For a (possibly empty) sequence of natural numbers $\vec{e}$, if $\vec{e}$ is
  non-empty let $ (e_1, \ldots, e_w) := \vec{e}$ and
  \begin{align*}
    \op{tag-num}(\vec{e}, y) &= (\bigwedge_{j \in [w]}\op{tag-num}(e_j, y)) \\&= (\op{tag-num}(e_1 , y) \land (\op{tag-num}(e_2, y) \land ( \ldots  \land (\op{tag-num}(e_w, y)) \ldots ))), 
  \end{align*}
  and if $\vec{e}$ is empty let $\op{tag-num}(\vec{e}, y) = \exists y. \, (y =
  y) \lor (y = y)$. It is easy to see that $\op{tag}$ and $\op{tag-num}$ define
  tautologies for any variable, sequence of variables, or sequence of numbers.
  We define $\lambda$ from $\theta$ by recursively replacing sub-formulas of
  $\theta$ with an application of a quantifier at the head with a logically
  equivalent formula defined using these helper functions. Let $\psi (\vec{y})$
  be a sub-formula of $\theta (\vec{x})$ of the form $Q\,[(\vec{y}^i_1, \ldots,
  \vec{y}^i_{l_i}) \Upsilon_i]_{i \in l}$, where $Q \in \setquant$, and let
  $\tau^Q := (R^{Q}, S^{Q}, \zeta^{Q})$ be the vocabulary of $Q$, where $R^{Q} =
  \{R^{Q}_1, \ldots R^{Q}_l\}$. For each $i \in [l]$ let $l_i$ be the arity of
  $R^{Q}_i$ and let $\vec{y}^i := \vec{y}^i_1, \ldots, \vec{y}^i_{l_i}$.

  From $\psi$ we define the formula $\psi' (\vec{y}) := Q \, [(\vec{y}^i_1,
  \ldots, \vec{y}^i_{l_i}) \cdot \Upsilon_i']_{i \in l}$ where for each $i \in
  [l]$, $\Upsilon_i'$ is a function with the same domain as $\Upsilon_i$ and
  such that for each $\vec{b} \in \dom(\Upsilon_i')$, $\Upsilon_i' (\vec{b}) =
  ((\forall u . u = u) \land \Upsilon_i(\vec{b})) \land (\op{tag-num}(\vec{b},
  z) \land \op{tag}(\vec{y}^i))$. Since $\psi'$ is defined from $\psi$ by taking
  conjunctions with tautologies, it is easy to see that $\psi$ and $\psi'$
  define the same query. Let $\lambda$ be defined from $\theta$ by recursively
  replacing each sub-formula $\psi$ in $\theta$ with the corresponding formula
  $\psi'$. It can be shown by induction that, since we always replace a
  sub-formula $\psi$ with a logically equivalent one, $\lambda (\vec{x})$ and
  $\theta (\vec{x})$ define the same query.

  It is easy to see that $\lambda(\vec{x})$ has the same variable-width as
  $\theta(\vec{x})$. It can be shown that $\vert \cl{\lambda}\vert \leq c_1
  \vert \cl{\theta} \vert (\vert \cl{\theta} \vert + \ewidth{\theta})$ for some
  constant $c_1$.

  Let $\psi \in \FO(\setquant)[\rho^*]$ and $\alpha \in [n]^{\free {\psi}}$. We
  let $\psi[\alpha]$ be the result of substituting each occurrence of the free
  variable $y \in \free{\psi}$ in $\psi$ with $\alpha(y)$. If $\psi$ is of the
  form $y_1 = y_2$, then $\psi [\alpha] = 1$ if $\alpha(y_1) = \alpha (y_2)$ and
  $\psi[\alpha] = 0$ otherwise. We call $\psi[\alpha]$ a \emph{ground formula}.
  For each $\psi \in \cl{\lambda}$ let $G_\psi := \{g_{\psi[\alpha]} : \alpha
  \in [n]^{\free{\psi}} \}$. Let $G := \bigcup_{\psi \in \cl{\lambda}}
  G_{\psi}$. Let $g := g_{\psi[\alpha]} \in G$. We define $\Sigma$, $\Lambda$
  and $L$ as follows.
  \begin{myitemize}
  \item If $\psi[\alpha]$ is $0$ or $1$ let $\Sigma (g) = \psi[\alpha]$.
  \item If $\psi[\alpha] = T(\vec{a})$ for some $T \in \rho^*$ and $\vec{a} \in
    [n]^{\arty(T)}$, then $\Sigma (g) = T$ and $\Lambda_T (g) = \vec{a}$.
  \item Suppose $\psi = Q \, [(\vec{y}^i_1, \ldots, \vec{y}^i_{l_i}) \cdot
    \Upsilon_i]_{i \in [l]}$ for some $Q \in \setquant$. Let $\tau := (R^Q, S^Q,
    \zeta^Q)$ be the vocabulary of $Q$, where $R^{Q} = \{R^{Q}_1, \ldots
    R^{Q}_l\}$ and $S^Q = \{s_1, \ldots, s_q\}$. For each $i \in [l]$ let $l_i$
    be the arity of $R^{Q}_i$. Let $D$ be the class of structures and $\ar$ be
    the arity of $Q$. For each $i \in [q]$ let $X_i = [n]^{\ar(s_i, 1)} \times
    \numdompow{n}{\ar(s_i, 2)}$. Let $\Sigma (g) = F^{(X_1, \ldots, X_q)}_{D}$.

    Let $T := (T_1, \ldots, T_l)$ be the number domain of $Q$. Let $i \in [l]$
    and $j \in [l_i]$ let $d^i_j := \ar (\zeta(R^Q_i)(j), 1)$ and $e^i_j := \ar
    (\zeta(R^Q_i)(j), 2)$. Let $(\vec{c}, R^Q_i) \in \dom (\Sigma(g))$. Then
    $\vec{c} \in ([n]^{d^i_1} \times \numdompow{n}{e^i_1}) \times \ldots \times
    ([n]^{d^i_{l_i}} \times \numdompow{n}{e^i_{l_i}})$ and so for all $j \in
    [l_i]$ there exists $\vec{a}_j \in [n]^{d^i_j}$ and $\vec{b}_j \in
    \numdompow{n}{e^i_j}$ such that $\vec{c} = ((\vec{a}_1 \vec{b}_1), \ldots,
    (\vec{a}_{l_i} \vec{b}_{l_i}))$. Let $\vec{a} := (\vec{a}_1, \ldots,
    \vec{a}_{l_i})$ and $\vec{b} := (\vec{b}_1, \ldots, \vec{b}_{l_i})$. Let
    $\beta := \ext{\alpha}{\alpha^{\vec{a}_1, \ldots,
        \vec{a}_{l_i}}_{\vec{y}^i_1, \ldots, \vec{y}^i_{l_i}}}$ and
    $L(g)(\vec{c} , R^Q_i) = g_{\Upsilon_i(\vec{b})[\beta]}$.
    
  \item If $\psi = Q z . \, \phi(\vec{y}, z)$, for $Q \in \{\forall, \exists\}$,
    then if $Q = \forall$ we let $\Sigma (g) = \AND[n]$, otherwise let $\Sigma
    (g) = \OR[n]$. Let $L(g) : [n] \rightarrow G$ be defined for $i \in [n]$ by
    $L(g)(i) = g_{\psi[\beta]}$, where $\beta := \ext{\alpha}{\alpha^i_z}$.
  \item If $\psi = \phi_1 \land \phi_2$, then let $\Sigma(g) = \AND[2]$ and
    $L(g) : [2] \rightarrow G$ be defined for $i \in [2]$ by $L(g)(i) =
    g_{\phi_i[\beta_i]}$ where $\beta_i \in [n]^{\free{\phi_i}}$ is the
    restriction of $\alpha$ to $\free{\phi_i}$. The same approach is used for
    the disjunctive case.
  \item If $\psi = \neg \phi$ let $\Sigma (g) = \NOT$ and $L(g): [1] \rightarrow
    G$ be defined by $L(g)(1) = g_{\phi[\alpha]}$.
  \end{myitemize}
  It is easy to check for each case that for $g \in G$, $L(g)$ is an injection.
  Let $q = \vert \vec{x} \vert$. Let $\Omega : [n]^{q} \rightarrow G$ be defined
  for $\vec{a} \in [n]^q$ by $\Omega (\vec{a}) = g_{\lambda [\alpha]}$, where
  $\alpha := \alpha^{\vec{a}}_{\vec{x}}$. Let $\vec{a} \in [n]^q$. We have
  assumed that $\theta$ (and so $\lambda$) contain at least one relation symbol,
  and so $\lambda[\alpha]$ cannot be $0$ or $1$. Let $\vec{b} \in [n]^q$ and
  suppose $\Omega (\vec{a}) = \Omega(\vec{b})$. But then
  $g_{\lambda[\alpha^{\vec{a}}_{\vec{x}}]} =
  g_{\lambda[\alpha^{\vec{b}}_{\vec{x}}]}$ and so
  $\lambda[\alpha^{\vec{a}}_{\vec{x}}]$ and
  $\lambda[\alpha^{\vec{b}}_{\vec{x}}]$ are equal as strings. But if
  $\lambda[\alpha^{\vec{a}}_{\vec{x}}]$ and
  $\lambda[\alpha^{\vec{b}}_{\vec{x}}]$ are equal then, since $\lambda$ contains
  at least one relation symbol, it follows that $\alpha^{\vec{a}}_{\vec{x}} =
  \alpha^{\vec{b}}_{\vec{x}}$ and so $\vec{a} = \vec{b}$. It follows that
  $\Omega$ is injective. Let $T \in \rho^*$, and let $g_{\psi[\alpha]},
  g_{\phi[\beta]} \in G$ be such that $\Lambda_T (g_{\psi[\alpha]}) = \Lambda_T
  (g_{\phi[\beta]})$. Then we have $\vec{a} \in [n]^{\arty(T)}$ and $\vec{b} \in
  [n]^{\arty(T)}$ such that $\psi[\alpha] = T(\vec{a})$ and $\phi[\beta] =
  T(\vec{b})$. But then $\vec{a} = \Lambda_T (g_{\psi[\alpha]}) = \Lambda_T
  (g_{\phi[\beta]}) = \vec{b}$, but then $\psi[\alpha] = \phi[\beta]$ and so
  $g_{\psi[\alpha]} = g_{\phi[\beta]}$. It follows that $\Lambda_T$ is
  injective. We conclude that $C := \langle G, \Omega, \Sigma, \Lambda, L
  \rangle$ is a circuit with injective labels of order $n$.

\begin{claim}
  The circuit $C$ is symmetric.
\end{claim}
\begin{proof}
  Let $\sigma \in \sym_n$. Let $\pi_\sigma : G \rightarrow G$ be defined such
  that $\pi_{\sigma} g_{\psi [\alpha]} = g_{\psi[\sigma \alpha]}$ for each
  $g_{\psi[\alpha]} \in G$ . It is easy to see that $\pi_\sigma$ is a bijection.
  We now show that $\pi_{\sigma}$ is an automorphism of $C$ extending $\sigma$.
  We prove this by induction on the structure of the circuit. Let
  $g_{\psi[\alpha]} \in G$.

  Suppose $g_{\psi[\alpha]}$ is an input gate. If $g_{\psi[\alpha]}$ is a
  constant gate then $\psi (y_1, y_2) = (y_1 = y_2)$. We have $\psi[\alpha] = 1$
  if, and only if, $\alpha(y_1) = \alpha(y_2)$ if, and only if, $\sigma \alpha
  (y_1) = \sigma \alpha(y_2)$ if, and only if, and $\psi[\sigma \alpha] = 1$. It
  follows that $\pi_\sigma g_{\psi[\alpha]} = g_{\psi[\sigma \alpha]} =
  g_{\psi[\alpha]}$. If $g_{\psi[\alpha]}$ is a relational gate, then $\psi
  (\vec{y}) = T(\vec{y})$ for some relation symbol $T$ and $\Lambda_T
  (g_{\psi[\alpha]}) = \alpha (\vec{y})$. It follows that $\Lambda_T
  (\pi_{\sigma} g_{\psi[\alpha]}) = \Lambda_T( g_{\psi[\sigma \alpha]}) = \sigma
  \alpha (\vec{y}) = \sigma \Lambda_T (g_{\psi[\alpha]})$.
  
  Suppose $g_{\psi[\alpha]}$ is an internal gate. It can be shown by considering
  cases that $\pi_\sigma H_{\psi[\alpha]} = H_{\pi_\sigma \psi[\alpha]}$. If
  $g_{\psi[\sigma]}$ is a symmetric gate, then this is sufficient to conclude
  that $\pi_{\sigma} L(g_{\psi [\alpha]})$ is isomorphic to $L(\pi_{\sigma}
  g_{\psi[\alpha]})$. Suppose $g_{\psi[\alpha]}$ is a non-symmetric gate. Then
  $\psi (\vec{y}) = Q \, [(\vec{y}^i_1, \ldots, \vec{y}^i_{l_i}) \cdot
  \Upsilon_i]_{i \in [l]}$ for some $Q \in \setquant$. Let $\tau := (R^Q, S^Q,
  \zeta^Q)$ be the vocabulary of $Q$, where $R^{Q} = \{R^{Q}_1, \ldots
  R^{Q}_l\}$ and $S^Q = \{s_1, \ldots, s_q\}$. For each $i \in [l]$ let $l_i$ be
  the arity of $R^{Q}_i$. Let $D$ be the class of structures and $\ar$ be the
  arity of $Q$. Let $\delta : \ind(g_{\psi[\alpha]}) \ra \ind(g_{\psi[\alpha]})$
  be defined as follows. Let $(\vec{c}, R^Q_i) \in \ind(g)$. Then, as in the
  definition of the $C$, we have for all $j \in [l_i]$ there exists $\vec{a}_j
  \in [n]^{d^i_j}$ and $\vec{b}_j \in \numdompow{n}{e^i_j}$ such that $\vec{c} =
  ((\vec{a}_1 \vec{b}_1), \ldots, (\vec{a}_{l_i} \vec{b}_{l_i}))$. Let $\vec{a}
  := (\vec{a}_1, \ldots, \vec{a}_{l_i})$ and $\vec{b} := (\vec{b}_1, \ldots,
  \vec{b}_{l_i})$. Let $\beta = \ext{\alpha}{\alpha^{\vec{a}_1, \ldots,
      \vec{a}_{l_i}}_{\vec{y}^i_1, \ldots, \vec{y}^i_{l_i}}}$. Let $\delta
  (\vec{c}, R^Q_i) := (\vec{c}', R^{Q}_i)$, where $\vec{c}' = (((\sigma
  \vec{a}^i_1)\vec{b}^i_1) , \ldots, ((\sigma \vec{a}^i_{l_i})
  \vec{b}^i_{l_i}))$. In other words $\delta$ applies the permutation $\sigma$
  only to those sequences in $\vec{c}$ that denote assignments to element
  variables. The can be shown that $\delta \in \aut(g_{\psi[\alpha]})$. It
  follows that $\sigma \beta = \ext{\sigma \alpha} {\alpha^{\sigma \vec{a}_1,
      \ldots, \sigma \vec{a}_{l_i}}_{\vec{y}^i_1, \ldots, \vec{y}^i_{l_i}}}$,
  and so $\pi_{\sigma} L(g_{\psi[\alpha]})(\vec{c}, R^Q_i) = \pi_{\sigma}
  g_{\Upsilon_i (\vec{b})[\beta]} = g_{\Upsilon_i (\vec{b})[\sigma \beta]} =
  L(g_{\psi[\sigma \alpha]})(\vec{c}', R^Q_i) = L(\pi_\sigma g_{\psi
    [\alpha]})(\delta (\vec{c}, R^Q_i))$.

  % Let $\tau := (R^\tau, S^\tau, \zeta^\tau)$ be the vocabulary of $Q^{E,
  % \ar}_{\vec{p}, n}$, where $R^\tau = {R^\tau_1, \ldots, R^\tau_l}$. For each
  % $i \in [l]$ let $l_i$ be the arity of $R^\tau_i$. For each $i \in [l]$ let
  % $D^1_i$ and $D^2_i$ be defined as given in the definition of the circuit.
  % For
  % each $(\vec{d}, R^{\tau}_i) \in \ind(g_{\psi[\alpha]})$ let $\vec{a} :=
  % (\vec{a}^i_1, \ldots, \vec{a}^i_{l_i}) \in D^1_i$ and $\vec{b} =
  % (\vec{b}^i_1,
  % \ldots, \vec{b}^i_{l_i}\in D^i_i$ be given as in the definition of $C$,
  % i.e.\
  % such that $\vec{d} = (\vec{a}^i_1\vec{b}^i_1, \ldots,
  % \vec{a}^i_{l_i}\vec{b}^i_{l_i})$. Let $\delta : \ind(g_{\psi[\alpha]}) \ra
  % \ind(g_{\psi[\alpha]})$ be defined such that $\delta (\vec{d}, R^\tau_i) =
  % (\vec{d}', R^{\tau}_i)$, where $\vec{d}' = ((\sigma \vec{a}^i_1)\vec{b}^i_1
  % ,
  % \ldots, (\sigma \vec{a}^i_{l_i}) \vec{b}^i_{l_i})$. The idea is that
  % $\vec{d}$
  % is constructed from the sequences $\vec{a}$ and $\vec{b}$, where $\vec{a}$
  % is
  % an assignment to the variables bound by the operator and $\vec{b}$ is the
  % input to the function $\Upsilon_i$. The function $\delta$ maps $\vec{d}$ by
  % applying $\sigma$ only to $\vec{a}$, i.e.\ to those tuples that denote
  % assignment to the variables. It can be shown that $\delta \in
  % \aut(g_{\psi[\alpha]})$. We have that $\pi_{\sigma}
  % L(g_{\psi[\alpha]})(\vec{d}, R^\tau_i) = g_{\Upsilon_i (\vec{b})[\sigma
  % \beta]}$ where $\beta$ is the union of $\alpha$ and
  % $\alpha^{\vec{a}}_{\vec{y}^i_1, \ldots, \vec{y}^i_{l_i}}$. But then $\sigma
  % \beta$ is the union of $\sigma \alpha$ and $\alpha^{\sigma
  % \vec{a}}_{\vec{y}^i_1, \ldots, \vec{y}^i_{l_i}}$, and so $g_{\Upsilon_i
  % (\vec{b})[\sigma \beta]} = L(g_{\psi[\sigma \alpha]})(\vec{d}', R^\tau_i) =
  % L(\pi_\sigma g_{\psi [\sigma \alpha]})(\delta (\vec{d}, R^\tau_i))$.

  Suppose $g_{\psi[\alpha]}$ is an output gate. Then $\psi = \lambda (\vec{x})$
  and let $\vec{a} := \alpha(\vec{x})$. From the definition of $C$ we have that
  $g_{\psi[\alpha]} = \Omega(\vec{a})$. It follows that $\pi_{\sigma} \Omega
  (\vec{a}) = \pi_\sigma g_{\psi[\alpha]} = g_{\psi[\sigma \alpha]} =
  \Omega(\sigma \vec{a})$. This concludes the proof of the claim.

\end{proof}

\begin{claim}
  Let $\mathcal{A} \in \fin{\rho, n}$ be a structure and let $\gamma \in
  [n]^{\underline{A}}$. For each $g_{\psi[\alpha]} \in G$ we have $\mathcal{A}
  \models \psi[\gamma^{-1}\alpha]$ if, and only if, $C[\gamma
  \mathcal{A}](g_{\psi[\alpha]}) = 1$.
\end{claim}
\begin{proof}
  We prove this claim by structural induction on the circuit. Let
  $g_{\psi[\alpha]} \in G$. If $g_{\psi[\alpha]}$ is an input gate then $\psi$
  has at its head a relation symbol or equality. It is easy to prove the
  inductive claim in either case. Suppose $g_{\psi[\alpha]}$ is an internal gate
  and the claim holds for each child of $g_{\psi[\alpha]}$. If
  $g_{\psi[\alpha]}$ is labelled by an element of $\BS$, then $\psi$ has at its
  head either a Boolean connective, or an existential or universal quantifier.
  It is not hard to show in each of these cases that $\mathcal{A} \models
  \psi[\gamma^{-1}\alpha]$ if, and only if, $C[\gamma
  \mathcal{A}](g_{\psi[\alpha]}) = 1$.

  Suppose $g_{\psi[\alpha]}$ is labelled by an element of $\BB$. Then $\psi
  (\vec{y}) = Q \, [(\vec{y}^i_1, \ldots, \vec{y}^i_{l_i}) \cdot \Upsilon_i]_{i
    \in [l]}$ for some $Q \in \setquant$. Let $\tau := (R^Q, S^Q, \zeta^Q)$ be
  the vocabulary of $Q$, where $R^{Q} = \{R^{Q}_1, \ldots R^{Q}_l\}$ and $S^Q =
  \{s_1, \ldots, s_q\}$. For each $i \in [l]$ let $l_i$ be the arity of
  $R^{Q}_i$. Let $(\vec{c}, R^Q_i) \in \ind(g_{\psi[\alpha]})$. Let $D$ be the
  class of structures and $\ar$ be the arity of $Q$. For each $i \in [q]$ let
  $X_i = [n]^{\ar(s_i, 1)} \times \numdompow{n}{\ar(s_i, 2)}$. As above, in the
  definition of the circuit $C$, we associate with $\vec{c}$ a unique pair of
  tuples $(\vec{a}, \vec{b})$, where $\vec{a} = (\vec{a}_1, \ldots,
  \vec{a}_{l_i})$ and $\vec{b} = (\vec{b}_1, \ldots, \vec{b}_{l_i})$. Let $\beta
  := \ext{\alpha}{\alpha^{\vec{a}_1, \ldots, \vec{a}_{l_i}}_{\vec{y}^i_1,
      \ldots, \vec{y}^i_{l_i}}}$.
  
  It follows from the induction hypothesis that
  $\mathcal{A}(g_{\psi[\alpha]})(\vec{c}, R^Q_i) = 1$ if, and only if, $C[\gamma
  \mathcal{A}](g_{\Upsilon_i(\vec{b})[\beta]}) = 1$ if, and only if,
  $\mathcal{A} \models \Upsilon_i (\vec{b})[\gamma^{-1} \beta]$. Let
  $\mathcal{I}$ be the interpretation defined by $\psi$. Let $\mathcal{B} :=
  \mathcal{I}(\mathcal{A}, \gamma^{-1} \alpha)$. From the definition of
  $\mathcal{I}$ we have that $\vec{c} \in (R^Q_i)^{\mathcal{B}}$ if, and only
  if, $\mathcal{A} \models \Upsilon_i (\vec{b})[\gamma^{-1} \beta]$. From these
  two sets of equivalences it follows that $\vec{c} \in (R^Q_i)^{\mathcal{B}}$
  if, and only if, $\mathcal{A}(g_{\psi[\alpha]})(\vec{c}, R^Q_i) = 1$. As a
  consequence we have that $\mathcal{B}$ and the structure defined by $L^{\gamma
    \mathcal{A}}(g_{\psi [\alpha]})$ are isomorphic. We thus have that
  $\mathcal{A} \models \psi[\gamma^{-1} \alpha]$ if, and only if, $\mathcal{B}
  \in D$ if, and only if, $F^{(X_1, \ldots, X_q)}_D(L^{\gamma
    \mathcal{A}}(g_{\psi [\alpha]})) = 1$ if, and only if,
  $\Sigma(g_{\psi[\alpha]})(L^{\gamma \mathcal{A}}(g_{\psi [\alpha]})) = 1$ if,
  and only if, $C[\gamma \mathcal{A}] (g_{\psi[\alpha]}) = 1$.
  

  % (\mathcal{B})

  % if, and only if, $L^{\gamma \mathcal{A}}(g_{\psi[\alpha]})(\vec{c},
  % R^{Q}_i)
  % = 1$
  

  % Let $\mathcal{B} := \mathcal{I}(\mathcal{A}, \gamma^{-1} \alpha)$, where
  % $\mathcal{I}$ is the interpretation defined by $Q^{E, \ar}_{\vec{p},
  % n}$.

  % From the definition of the interpretation and the above observation we
  % have
  % that $\vec{d} \in (R^\tau_i)^{\mathcal{B}}$ if, and only if,
  % $\mathcal{A}
  % \models \Upsilon_i (\vec{b})[\gamma^{-1} \beta]$ if, and only if,
  % $L^{\gamma
  % \mathcal{A}}(g_{\psi[\alpha]})(\vec{d}, R^{\tau}_i) = 1$. It follows
  % that
  % $\mathcal{B}$ is isomorphic to the structure defined by $L^{\gamma
  % \mathcal{A}}(g_{\psi [\alpha]})$, and so $\mathcal{A} \models
  % \psi[\gamma^{-1} \alpha]$ if, and only if, $E(\vec{p}, \mathcal{B}) = 1$
  % if,
  % and only if, $F^E_{\vec{p}, n}(L^{\gamma \mathcal{A}}(g_{\psi[\alpha]}))
  % =
  % 1$ if, and only if, $C[\gamma \mathcal{A}] (g_{\psi[\alpha]}) = 1$.
\end{proof}
Let $\mathcal{A} \in \fin{\rho, n}$, let $\gamma \in [n]^{\underline{A}}$ and
let $\alpha \in A^{\vec{x}}$. Then $g_{\lambda [\gamma \alpha]} = \Omega (\gamma
\alpha (\vec{x}))$. It follows from the above claim that $C [\gamma \mathcal{A}]
(\Omega( \gamma \alpha (\vec{x})) = 1$ if and only if, $\mathcal{A} \models
\lambda [\alpha]$. In other words, $C$ and $\theta(\vec{x})$ express the same
query for structures of size $n$.


\begin{claim}
  The circuit $C$ is transparent.
  \label{claim:circuit-translation-transparent}
\end{claim}
\begin{proof}
  If every gate in $C$ is symmetric then $C$ is transparent. Suppose there
  exists there a non-symmetric gate $g_{\psi[\alpha]}\in G$. Then $\psi
  (\vec{y}) = Q \, [(\vec{y}^i_1, \ldots, \vec{y}^i_{l_i}) . \Upsilon_i]_{i \in
    [l]}$ for some $Q \in \setquant$. Let $\vec{y}^i := \vec{y}^i_1, \ldots,
  \vec{y}^i_{l_i}$. Let $\tau := (R^Q, S^Q, \zeta^Q)$ be the vocabulary of $Q$,
  where $R^Q = {R^Q_1, \ldots, R^Q_l}$. For each $i \in [l]$ let $l_i$ be the
  arity of $R^Q_i$. Let $i \in [l]$ and let $(\vec{c})_1, R^Q_i), (\vec{c}_2,
  R^{Q}_i) \in \ind(g_{\psi[\alpha]})$. Let $D^1_i$ and $D^2_i$ be given as in
  the definition of $C$. For $z \in [2]$, let $\vec{a}^z = (\vec{a}^z_1, \ldots,
  \vec{a}^z_{l_i})$ and $\vec{b}^z = (\vec{b}^z_1, \ldots, \vec{b}^z_{l_i})$ be
  such that $\vec{c}_z = (\vec{a}^z_1\vec{b}^z_1, \ldots,
  \vec{a}^z_{l_i}\vec{b}^z_{l_i})$. Let $\beta_z :=
  \ext{\alpha}{\alpha^{\vec{a}^z}_{\vec{y}^i}}$. For each $z \in [2]$ let $h_z
  := L(g_{\psi[\alpha]})(\vec{c}_z), R^Q_i) =
  g_{\Upsilon_i(\vec{b}^z)[\beta_z]}$. Suppose $h_1 \equiv h_2$.


  From the definition of $\lambda$ we have for any $\vec{b} \in
  \dom(\Upsilon_i)$ that $\Upsilon_i(\vec{b}) = \kappa^1_{ \vec{b}} \land
  \kappa^2_{\vec{b}}$, where $\kappa^1_{\vec{b}} = ((\forall u. u = u) \land
  \Upsilon_i'(\vec{b}))$, and $\kappa^2_{\vec{b}} = (\op{tag}(\vec{y}^i) \land
  \op{tag-num}(\vec{b}, u))$, for some variable $u$. Let $z \in [2]$. Then
  $H_{h_z} = \{g_{\kappa^1_{\vec{b}^z}[\beta_z]},
  g_{\kappa^2_{\vec{b}^z}[\beta_z]}\}$. Let $w \in [2]$. We note that $u = u$ is
  a sub-formula of an immediate sub-formula of $\kappa^1_{\vec{b}^z}$ while
  there is no sub-formula of an immediate sub-formula of $\kappa^2_{\vec{b}^w}$
  that has equality at the head of the formula. It follows that a child of a
  child of $g_{\kappa^1_{ \vec{b}^z}[\beta_z]}$ is a constant gate while no
  child of a child of $g_{\kappa^2_{ \vec{b}^w}[\beta_w]}$ is a constant gate.
  We thus have that $g_{\kappa^1_{ \vec{b}^z}[\beta_z]} \not\equiv g_{\kappa^2_{
      \vec{b}^2}[\beta_w]}$. It then follows from $h_1 \equiv h_1$ that
  $g_{\kappa^1_{ \vec{b}^1}[\beta_1]} \equiv g_{\kappa^1_{ \vec{b}^2}[\beta_2]}$
  and $g_{\kappa^2_{ \vec{b}^1} [\beta_1]} \equiv g_{\kappa^2_{ \vec{b}^2}
    [\beta_2]}$.

  Let $z \in [2]$. Let $\epsilon^1_z = \op{tag}(\vec{y})[\beta_z]$ and
  $\epsilon^2_z = \op{tag-num}(\vec{b}^z, u))$. Let $w \in [2]$. We notice that
  if $\vec{y}$ is empty then $g_{\epsilon^1_z}$ is an $\AND$-gate and has a
  grandchild that is a constant gate. In contrast we have that
  $g_{\epsilon^2_w}$ is either an $\OR$-gate or has no grandchildren that are
  constant gates. In either case $g_{\epsilon^1_z} \not\equiv g_{\epsilon^2_w}$.
  If $\vec{y}$ is non-empty then $g_{\epsilon^1_z}$ has a relational gate as a
  grandchild, while $g_{\epsilon^2_w}$ does not. It follows that
  $g_{\epsilon^1_z} \not\equiv g_{\epsilon^2_w}$. Then, from the fact that
  $g_{\kappa^2_{ \vec{b}^1} [\beta_1]} \equiv g_{\kappa^2_{ \vec{b}^2}
    [\beta_2]}$ we have that $g_{\epsilon^1_1} \equiv g_{\epsilon^1_2}$ and
  $g_{\epsilon^2_1} \equiv g_{\epsilon^2_2}$. But from the definition of
  $\op{tag}$ and the fact that $g_{\epsilon^1_1} \equiv g_{\epsilon^1_2}$ we
  have that $beta_1 = \beta_2$. From the definition of $\op{tag-num}$ and the
  fact that $g_{\epsilon^2_1} \equiv g_{\epsilon^2_2}$ we have that $\vec{b}^1 =
  \vec{b}^2$. It follows that $h_1 = g_{\Upsilon_i(\vec{b}^1)[\beta_1]} =
  g_{\Upsilon_i(\vec{b}^2)[\beta_2]} = h_2$.

  It follows that $L(g_{\psi[\alpha]})$ is injective and no two child gates of
  $g_{\psi[\alpha]}$ labelled by the same relation are syntactically-equivalent.
  We conclude that $g_{\psi[\alpha]}$ has unique labels, and $C$ is transparent.
\end{proof}

We have already shown that $\vert \cl{\lambda} \vert \leq c_1\vert \cl{\theta}
\vert (\vert \cl{\theta} \vert + \ewidth{\theta})$ and that $\ewidth{\lambda} =
\ewidth{\theta}$. We can construct the set of gates by recursing over the
sub-formulas of $\lambda$ and for each sub-formula $\psi$ and assignment $\alpha
\in [n]^{\free{\psi}}$ defining a gate $g_{\psi[\alpha]}$. This can be done in
time polynomial in $\vert \cl{\lambda} \vert \cdot n^{\ewidth{\lambda}}$, and we
have $\vert \cl{\lambda} \vert \cdot n^{\ewidth{\lambda}} \leq \vert \cl{\theta}
\vert (\vert \cl{\theta} \vert + \ewidth{\theta}) \cdot n^{\ewidth{\theta}}$.
Since the rest of the circuit can be constructed in time polynomial in the
number of gates, it follows that the construction of $C$ can be completed in
time polynomial in $\vert \cl{\theta} \vert (\vert \cl{\theta} \vert +
\ewidth{\theta}) \cdot n^{\ewidth{\theta}}$.

We have from the above three claims that $C$ translates $\lambda$ (and hence
$\theta$) for $n$, and from the above argument we can construct $C$ within the
required time bounds. The result follows.
\end{proof}

\begin{prop}
  Let $\rho$ is a vocabulary. Let $\setquant$ be a set of extended quantifiers
  and let $\BB$ be the associated basis. Each query definable by a $\PT$-uniform
  family of $\FO(\setquant)[\rho]$-substitution programs with a constant bound
  on width is definable by a $\PT$-uniform family of transparent symmetric
  $(\BB, \rho)$-circuits $(C_n)_{n \in \nats}$.
  \label{prop:translating-programs-to-circuits}
\end{prop}
\begin{proof}
  Let $\Theta := (\Theta_n)_{n \in \nats}$ be a $\PT$-uniform family of
  $\FO(\setquant)[\rho]$ substitution programs. Let $n \in \nats$. Let
  $(\theta_{n, 1}, \ldots, \theta_{n, k}) := \Theta_n$. For each $i \in [k]$ let
  $\vec{V}_i$ be the element-sort second-order variables that appear in
  $\theta_{n, i}$. We treat these second-order variables as relation symbols,
  and suppose, without a loss of generality, that the symbols that appear in
  $\vec{V}_i$ do not appear in $\rho$ for each $i \in [k]$. For each $i \in [k]$
  let $\rho_i = \vec{V}_i \cup \rho$. From
  Lemma~\ref{lem:translating-FOquant-to-formulas} we may construct for each $i
  \in [k]$ a $(\BB, \rho_i)$-circuit $C_{n, i}$ such that $C_{n, i}$ translates
  the formula $\theta_{n, i}$ for $n$. We assume, without a loss of generality,
  that in each of these circuits none of the input gates are also output gates
  (if this is not the case, we can alter the circuit by adding in single-input
  $\AND$-gates with each gate that is both an input and output gate taken as
  input to the $\AND$-gate, and the $\AND$-gate then assigned to be an output
  gate).

  For each $i \in [k]$ let $\theta_{n, i}'$ be the flattening of $\Theta_n$ at
  $i$. We recall that for each $i \in [k]$ the formula $\theta_{n, i}'$ is
  defined by replacing each appearance of a symbol $V_j$ in $\theta_{n, i}$ with
  the formula $\theta_{n, j}'$. Notice, this definition is a (backwards)
  recursive definition. We will similarly define $C_{n, i}'$ from $C_{n, i}$ by
  replacing each input gate labelled by the symbol $V_j$ and tuple $\vec{a}$
  with the output gate of $C_{n, j}'$ labelled by $\vec{a}$. For each $i \in
  [k]$ let $(G_i, \Omega_i, \Sigma_i, \Lambda_i, L_i) := C_{n, i}$. We now
  define the $(\BB,\rho)$-circuit $C_{n, i}' := (G_i', \Omega_i', \Sigma_i',
  \Lambda_i', L_i')$ recursively. If $i = k$ then $C_{n, i}' = C_{n, i}$. If $i
  < k$ then we define $C_{n, i}' $ from the circuits $\{C_{n, j}' : j > i \}$ as
  follows.
  \begin{myitemize}
  \item Let $G_i' = \{g \in G_i : \Sigma (g) \not\in \vec{V}_i\} \cup
    \bigcup_{V_{j} \in \vec{V}} G_{j}'$. We identify input gates labelled by the
    same relation symbol and tuple so that there are no duplicate input gates.
    
  \item Let $\Lambda_i$ be defined for each $T \in \rho$, $\vec{a} \in
    [n]^{\arty(T)}$, and $g \in G_i'$ such that $(\Lambda_i')_{T} (g) = \vec{a}$
    if, and only if, (i) $g \in G_i$ and $(\Lambda_i)_{T}(g) = \vec{a}$ or (ii)
    there exists $j > i$ such that $g \in G_j'$ and $(\Lambda_j')_T (g) =
    \vec{a}$.
    
  \item For each $g \in G_i'$ if $g \in G_i$ let $\Sigma_i' (g) = \Sigma_i (g)$
    and otherwise there exists $j > i$ such that $g \in G_j'$ and let $\Sigma_i'
    (g) = \Sigma_j' (g)$.
    
  \item Let $q$ be the arity of the query decided by $C_{n, i}$. Let $\Omega_i'$
    be an injection from $[n]^{q}$ to $G_i'$ defined such that $\Omega_i'
    (\vec{a}) = \Omega_i(\vec{a})$.
    
  \item Let $g \in G_i'$. Suppose $g \in G_i$. Let $L_i' (g) : \dom (L_i(g)) \ra
    G_i'$ be defined for $a \in \dom(L_i(g))$ such that $L_i' (g)(a) =
    L_i(g)(a)$ if $\Sigma_i (L_i (g)(a)) \not\in \vec{V}_i$ and otherwise let
    $L_i'(g)(a) = \Omega_j((\Lambda_i)_{V_j}(L_i(g)(a)))$, where $j > i$ and
    $\Sigma_i (L_i (g)(a)) = V_j$. Suppose $g \not\in G_i$. Then there exists $j
    > i$ such that $g \in G_j'$. Let $L_i(g) = L_j'(g)$.
  \end{myitemize}
  
  We next show, by backwards induction, that for all $i \in [k]$, $C_{n, i}'$
  translates $\theta_{n, i}'$ for $n$. Since $C_{n, k}' = C_{n, k}$ and
  $\theta_{n, k}' = \theta_{n, k}$, clearly $C_{n, k}'$ translates $\theta_{n,
    k}'$ for $n$. Let $i \in [k]$ and suppose for each $j > i$ we have that
  $C_{n, j}'$ translates $\theta_{n, j}'$ for $n$. We now show that $C_{n, i}'$
  translates $\theta_{n, i}'$ for $n$. We split the proof over the following
  three claims.

\begin{claim}
  The circuit $C_{n, i}'$ is symmetric.
\end{claim}
\begin{proof}
  Let $\sigma \in \sym_n$. We an automorphism $\pi_i$ of $C_{n, i}$ extending
  $\sigma$ and for each $j > i$ an automorphism $\pi_j'$ of $C_{n, j}'$
  extending $\sigma$. We define $\pi_i'$ for each $g \in G_i'$ as follows. If $g
  \in G_i$ let $\pi_i' (g) := \pi_i (g)$, and otherwise let $\pi_i' (g) :=
  \pi_j' (g)$, where $j > i$ such that $g \in G_j'$. It remains to show that
  $\pi_i'$ is a valid automorphism. We note that for $j > i$ if $g$ is an input
  gate in $C_{n, i}'$ then $\pi_i' (g) = \pi_i (g) = \pi_j'(g) = \sigma(g)$.

  Let $g \in G_i'$. Suppose $g \not\in G_i$. Then $g \in G_j'$ for some $j > i$.
  Since $C_{n, j}'$ is symmetric there exists $\lambda \in \aut(g)$ such that
  $\pi_j' L_j'(g) = L_j(\pi_j' g)\lambda$. Then for all $a \in \ind(g)$ we have
  $\pi_i' L_i' (g)(a) = \pi_j' L_j' (g)(a) = L_j'(\pi_j' g) (\lambda a) =
  L_i'(\pi_i' g)(\lambda a)$. Suppose $g \in G_i$. Since $C_{n, i}$ is symmetric
  there exists $\lambda \in \aut (g)$ such that $\pi_i L_i (g) = L_i (\pi_i g)
  \lambda$. Let $a \in \ind(g)$. If $L_i' (g)(a) \in G_i$ and so $\pi_i L_i
  (g)(a) \in G_i$ and $\pi_i' L_i' (g)(a) = \pi_i L_i (g)(a) = L_i (\pi_i g)
  (\lambda a) = L_i' (\pi_i' g)(\lambda a)$. If $L_i'(g)(a) \not\in G_i$ then
  $L_i'(g)(a) = \Omega_j'((\Lambda_i)_{V_j}(L_i(g)(a)))$ for some $j > i$ and
  $\pi_i' L_i'(g)(a) = \pi_j' \Omega_j'((\Lambda_i)_{V_j}(L_i(g)(a))) =
  \Omega_j' (\sigma (\Lambda_i)_{V_j}(L_i(g)(a))) = \Omega_j'
  ((\Lambda_i)_{V_j}(\pi_i L_i(g)(a)) = \Omega_j'(((\Lambda_i)_{V_j}(L_i(\pi_i
  g)(\lambda a))) = L_i'(\pi_i g)(\lambda a) = L_i'(\pi_i' g)(\lambda a)$. The
  first and the fifth equities follow from the definition of the circuit. It is
  easy to check the remaining requirements for the symmetry of $C_{n, i}'$.
\end{proof}

\begin{claim}
  The circuit $C_{n, i}'$ is transparent.
\end{claim}
\begin{proof}
  If all of the gates in $C_{n, i}'$ are symmetric then $C_{n, i}'$ is
  transparent. Suppose there exists a non-symmetric gate $g \in G_i'$. If $g \in
  G_j'$ for some $j > i$ then, since $C_{n, j}'$ is transparent, $g$ has unique
  labels. Otherwise $g \in G_i$. It follows from a very similar argument as for
  Claim~\ref{claim:circuit-translation-transparent} in
  Lemma~\ref{lem:translating-FOquant-to-formulas} that $g$ has unique labels. It
  follows that $C_{n, i}'$ is transparent.
\end{proof}

\begin{claim}
  Let $\mathcal{A} \in \fin{\rho, n}$, let $\vec{x}$ be the free variables in
  $\theta_{n,i}'$, and let $\alpha \in A^{\vec{x}}$. Let $\gamma \in
  [n]^{\underline{A}}$. Then $C_{n, i}'[\gamma \mathcal{A}](\Omega_i' (\gamma(
  \alpha (\vec{x})))) = 1$ if, and only if, $\mathcal{A} \models \theta_{n, i}'
  [\alpha]$.
\end{claim}
\begin{proof}
  Let $\mathcal{A}^*$ be the $\rho^*$-structure with the same universe as
  $\mathcal{A}$ and such that for each $R \in \rho$, $R^{\mathcal{A}^*} =
  R^{\mathcal{A}}$ and for all $V_j \in \vec{V}_i$, $V^{\mathcal{A}^*}_j =
  (\theta_{i, j}')^{\mathcal{A}}$. It follows from the definition of a
  substitution program that $\mathcal{A} \models \theta_{n, i}'[\alpha]$ if, and
  only if, $\mathcal{A}^* \models \theta_{n, i}[\alpha]$ if, and only if, $C_{n,
    i}[\gamma \mathcal{A}^*](\Omega_i (\gamma (\alpha(\vec{x}))))$. It thus
  suffices to show that $C_{n, i}[\gamma \mathcal{A}^*](\Omega_i (\gamma(
  \alpha(\vec{x})))) = C_{n, i}'[\gamma \mathcal{A}](\Omega_i' (\gamma (\alpha
  (\vec{x}))))$.

  We prove by induction on the structure of $C_{n, i}$ that for each $g \in
  G_{n, i}$ if $g$ is an input gate in $C_{n, i}$ labelled by $V_j$ for some $j
  > i$ then $C_{n, i}[\gamma \mathcal{A}^*](g) = C_{n, j}'[\gamma
  \mathcal{A}](\Omega_j'((\Lambda_i)_{V_j}(g)))$ and otherwise $C_{n, i}[\gamma
  \mathcal{A}^*](g) = C_{n, i}'[\gamma \mathcal{A}](g)$. Let $g \in G_{n, i}$.
  Suppose $g$ is an input gate. If $g$ is labelled by $V_j$ for some $j > i$
  then $C_{n, i}[\gamma \mathcal{A}^*](g) = 1$ if, and only if, $\gamma^{-1}
  ((\Lambda_i)_{V_j}(g)) \in V^{\mathcal{A}^*}_j$ if, and only if, $\mathcal{A}
  \models \theta_{n, j}'[\gamma^{-1} (((\Lambda_i)_{V_j}(g)))]$ if, and only if,
  $C_{n, j}'[\gamma \mathcal{A}](\Omega_j'(((\Lambda_i)_{V_j}(g))))$. Otherwise,
  $g$ is an input gate not labelled by $V_j$ for any $j > i$, and, from the
  definition of $C_{n, i}'$, we have $C_{n, i}[\gamma \mathcal{A}^*](g) = C_{n,
    i}' [\gamma \mathcal{A}](g)$. Suppose $g$ is not an input gate in $C_{n, i}$
  and suppose the inductive hypothesis holds for each child of $g$ in $C_{n,
    i}$. Let $a \in \ind(g)$. If $L_i(g)(a)$ is not an input gate labelled by
  $V_j$ for any $j > i$ then, from the definition of $C_{n, i}'$ we have $L_i'
  (g) (a) = L_i (g)(a)$ and, from the induction hypothesis, $(L_i')^{\gamma
    \mathcal{A}}(g)(a) = C_{n, i}' [\gamma \mathcal{A}](L_i'(g)(a)) = C_{n,
    i}[\gamma \mathcal{A}^*](L_i(g)(a)) = L^{\gamma \mathcal{A}^*}_i(g)(a)$.
  Otherwise, $L_i(g)(a)$ is an input gate labelled by $V_j$ for some $j > i$
  and, from the definition of $C_{n, i}'$, we have $L_i'(g)(a) =
  \Omega_j'((\Lambda_i)_{V_j})(L_i(g)(a))$. It follows that that
  $L^{\mathcal{A}^*}_i(g)(a) = C_{n, i}[\gamma \mathcal{A}^*](L_i(g)(a)) = C_{n,
    j}'[\gamma \mathcal{A}](\Omega_j'((\Lambda_i)_{V_j})(L_i(g)(a))) = C_{n,
    i}'[\gamma \mathcal{A}](L_i'(g)(a)) = (L_i')^{\gamma \mathcal{A}}(g)(a)$.
  The third equivalence follows from the definition of $C_{n, i}'$. It follows
  that $(L_i')^{\gamma \mathcal{A}}(g) = L^{\mathcal{A}^*}_i(g)$, and so $C_{n,
    i}[\gamma \mathcal{A}^*](g) = C_{n, i}'[\gamma \mathcal{A}](g)$. This
  completes the inductive argument. From the definitions of the circuits $C_{n,
    i}$ and $C_{n, i}'$ we have that $\Omega_i' = \Omega_i$ and that no output
  gate in either circuit is also an input gate. In particular, we have $C_{n,
    i}'[\gamma \mathcal{A}](\Omega_i' (\gamma (\alpha (\vec{x})))) = C_{n,
    i}[\gamma \mathcal{A}^*](\Omega_i ((\gamma \alpha (\vec{x}))))$, and the
  claim follows.

  % C_{n, i}[\gamma
  % \mathcal{A}^*]$

  


  % We now show by induction on the structure of the circuit that if $g$ is an
  % internal gate in $C_{n, i}$ then $C_{n, i}[\gamma mathcal{A}^*](g) = C_{n,
  % i}'[\gamma \mathcal{A}](g)$.




  % Suppose $g$ is not an input gate and suppose the inductive hypothesis holds
  % for all children of $g$. Let $a \in \ind(g)$. If $L_i'(g)(a) \in G_i$ then
  % $L_i' (g) (a) = L_i (g)(a)$ and so $(L_i')^{\gamma \mathcal{A}}(g)(a) =
  % L^{\gamma \mathcal{A}}_i(g)(a)$. If $L_i'(g)(a) \not\in G_i$ then
  % $L_i(g)(a)$
  % is an input gate labelled by some $V_j$ for $j > i$. It follows from the
  % definition of the circuit that $L_i'(g)(a) =
  % \Omega_j'((\Lambda_i)_{V_j})(L_i(g)(a))$, from the inductive hypothesis,
  % $C_{n, i}[\gamma \mathcal{A}^*](L_i'(g)(a)) = C_{n, j}'[\gamma
  % \mathcal{A}](\Omega_j'((\Lambda_i)_{V_j})(L_i(g)(a))) = C_{n, i}'[\gamma
  % \mathcal{A}](L_i'(g)(a))$. It follows that $(L_i')^{\gamma \mathcal{A}}(g) =
  % L^{\mathcal{A}^*}_i(g)$.




  % We first show that


  % We now show by induction on the structure of the circuit that if $g$ is an
  % internal gate in $C_{n, i}$ then $C_{n, i}[\gamma mathcal{A}^*](g) = C_{n,
  % i}'[\gamma \mathcal{A}](g)$. Let $g$ be an internal gate in $C_{n, i}$.
  % Suppose all the children of $g$ input gates. Let $a \in \dom (L_i(g)))$. If
  % $L_i(g)(a)$ is labelled by $V_j$ for some $j > i$ then $L^{\gamma
  % \mathcal{A}^*}_i(g)(a) = C_{n, i}[\gamma mathcal{A}^*](L(g)(a)) = 1$ if, and
  % only if, $\gamma^{-1} ((\Lambda_i)_{V_j}(L_i(g)(a))) \in
  % V^{\mathcal{A}^*}_j$
  % if, and only if, $\mathcal{A} \models \theta_{n, j}'[\gamma^{-1}
  % ((\Lambda_i)_{V_j}(L_i(g))]$ if, and only if, $C_{n, j}'[\gamma
  % \mathcal{A}](\Omega_j ((\Lambda_i)_{V_j}(L_i(g)(a)))) = (L_i')^{\gamma
  % \mathcal{A}} (g)(a) = 1$. If $L_i(g)(a)$ is not labelled by an element of
  % $\vec{V}_j$ then $L_i'(g)(a) = L_i(g)(a)$. It follows that $(L_i')^{\gamma
  % \mathcal{A}} (g) = L_i)^{\gamma \mathcal{A}^*} (g)$ and so $C_{n, i}

  

  % It thus suffices to show that for each internal in $C_{n, i}$

  % Notice that the outWe prove by induction on the structure of the circuit
  % that
  % (i) for all $g \in G_i$ if $g$ is not labelled by an element of $\vec{V}_i$
  % then $g \in G_i'$ and $C_{n, i}[\gamma mathcal{A}^*](g) = C_{n, i}[\gamma
  % mathcal{A}]()$ and (ii) if $g$ is labelled by an input gate labelled by an
  % element of $\vec{V}_i$ then $C_{n, i}[\gamma mathcal{A}^*](g)$

  
  % Let $g \in G_i'$ be an input gate. Then $g \in G_i$ or $g \in G_j'$ for some
  % $j > i$.

  

  % Let $g \in G_{i}$ be an input gate in $C_{n, i}$. Suppose $g$ is labelled by
  % some $V_j$ with $j > i$ and let $\vec{a} := (\Lambda_i)_{V_j}(g)$. Then
  % $C_{n,
  % i}[\gamma \mathcal{A}^*](g) = 1$ if, and only if, $\gamma^{-1} \vec{a} \in
  % V^{\mathcal{A}^*}_j$ if, and only if, $\gamma^{-1} \vec{a} \in (\theta_{i,
  % j}')^{\mathcal{A}}$ if, and only if, $C_{n,
  % j}'[\mathcal{A}](\Omega_j'(\gamma ^{-1}(\vec{a}))$. Suppose instead that $g$
  % is not labelled by some $V_j$ for $j > i$ then $g$ is a constant gate or is
  % labelled by a relation $R \in rho$. In ether case we have x
  

  
  % We first prove by induction on the structure of the circuit $C_{n, i}$ that
  % for each $g \in G_{n, i}$ if $g$ is an input gate labelled by some $V_j$
  % with
  % $j > i$ then $C_{n, i}[\gamma \mathcal{A}^*](g) = C_{n, j}'[\gamma
  % \mathcal{A}](\Omega_j'((\Lambda_i)_{V_j})(g))$ and otherwise $C_{n,
  % i}[\gamma
  % \mathcal{A}^*](g) = C_{n, i}'[\gamma \mathcal{A}](g)$. Let $g \in G_{n, i}$.
  % Suppose $g$ is an input gate labelled by $V_j$ for some $j > i$ and let
  % $\vec{b} := (\Lambda_i)_{V_j}(g)$. Then we have that $C_{n, i}[\gamma
  % \mathcal{A}^*](g) = 1$ if, and only if, $\gamma^{-1} \vec{b} \in
  % V^{\mathcal{A}^*}_j$ if, and only if, $\mathcal{A} \models \theta_{n,
  % j}'[\gamma^{-1} \vec{b}]$ if, and only if, $C_{n, j}'[\gamma
  % \mathcal{A}](\Omega_j'(\vec{b})$. Clearly if $g$ is an input gate not
  % labelled
  % by $V_j$ for some $j > i$ then $C_{n, i}[\gamma \mathcal{A}^*](g) = C_{n,
  % i}'
  % [\gamma \mathcal{A}^*](g)$. Suppose $g$ is not an input gate and suppose the
  % inductive hypothesis holds for all children of $g$. Let $a \in \ind(g)$. If
  % $L_i'(g)(a) \in G_i$ then $L_i' (g) (a) = L_i (g)(a)$ and so $(L_i')^{\gamma
  % \mathcal{A}}(g)(a) = L^{\gamma \mathcal{A}}_i(g)(a)$. If $L_i'(g)(a) \not\in
  % G_i$ then $L_i(g)(a)$ is an input gate labelled by some $V_j$ for $j > i$.
  % It
  % follows from the definition of the circuit that $L_i'(g)(a) =
  % \Omega_j'((\Lambda_i)_{V_j})(L_i(g)(a))$, from the inductive hypothesis,
  % $C_{n, i}[\gamma \mathcal{A}^*](L_i'(g)(a)) = C_{n, j}'[\gamma
  % \mathcal{A}](\Omega_j'((\Lambda_i)_{V_j})(L_i(g)(a))) = C_{n, i}'[\gamma
  % \mathcal{A}](L_i'(g)(a))$. It follows that $(L_i')^{\gamma \mathcal{A}}(g) =
  % L^{\mathcal{A}^*}_i(g)$.

  % Let $\vec{a}$ be an assignment to the free variables in $\theta_{n, i}'$.
  % But
  % $\theta_{n, i}$ has the same free first-order variables as $\theta_{n, i}$
  % and
  % so $\vec{a}$ is a $q$-tuple, where $q$ is the arity of the query decided by
  % $C_{n, i}$. It follows from the definition of the circuit that
  % $\Omega_i'(\gamma \vec{a}) = \Omega_i(\gamma \vec{a})$ and from the above
  % inductive argument that $C_{n, i}'[\gamma \mathcal{A}](\Omega_i'(\gamma
  % \vec{a})) = C_{n, i} [\gamma \mathcal{A}^*] (\Omega_i(\gamma \vec{b}))$. It
  % follows that $C_{n, i}' [\gamma \mathcal{A}](\Omega_i' (\gamma \vec{a})) =
  % 1$
  % if, and only if, $C_{n, i} [\gamma \mathcal{A}^*] (\Omega_i(\gamma \vec{a}))
  % =
  % 1$ if, and only if, $\mathcal{A}^* \models \theta_{n, i} [\vec{a}]$ if, and
  % only if, $\mathcal{A} \models \theta_{n, i}'[\vec{a}]$. The final
  % equivalence
  % follows from the definition of a flattening.

\end{proof}

For each $n \in \nats$ let $C_{n}' := C_{n, 1}'$ and let $\theta_n'$ be the
flattening of $\Theta_n'$. It follows from the above three claims that $C_n'$
translates $\theta_{n}'$ for $n$. It remains to argue that the construction of
$C_n'$ for each $n$ may be completed in time polynomial in $n$. This follows
from three polynomial bounds. First, since $\Theta$ is $\PT$-uniform there
exists a polynomial $p_1$ such that the function $n \mapsto \Theta_n$ is
computable in time $p_1(n)$. Second, it follows from
Lemma~\ref{lem:translating-FOquant-to-formulas} that there is a polynomial $p_2$
such that the function that maps $\Theta_n$ to the sequence of circuits $C_{n,
  1}, \ldots, C_{n, \vert \Theta_n \vert}$ is computable in time $\sum_{i \in
  [\vert \Theta_n \vert]} p_2(\vert \cl{\theta_{n, i}} \vert (\vert
\cl{\theta_{n, i}} \vert + \ewidth{\theta_{n, i}}) \cdot n^{\ewidth{\theta_{n,
      i}}}) \leq p_2(n) \cdot p_2(p_1 (n) (p_1(n) + c_1) n^{c_1})$, where $c_1$
is the constant bound on the variable-width of the formulas of $\Theta$.
Thirdly, the construction of $C_n'$ from $C_{n, 1}, \ldots, C_{n, \vert \Theta_n
  \vert}$ works by constructing each $C_{n, i}'$ from $C_{n, j}$, for all $j >
i$, by replacing the appropriate input gates of $C_{n, i}$ with the output gates
of $C_{n, j}'$. It can be shown that this algorithm runs in time polynomial in
the combined size of these circuits. The function that maps $n \mapsto C_n'$ is
the composition of these three functions, each of which is computable in time
polynomial in $n$, and so $n \mapsto C_n'$ can be computed in time polynomial in
$n$.

  
\end{proof}
We now present the main theorem of this chapter: that each extension of a
fixed-point logic by generalised operators may be translated to an equivalent
$\PT$-uniform family of transparent symmetric circuits defined over the
corresponding basis. This result follows almost immediately from
Lemma~\ref{lem:unroll-fixed-point},
Lemma~\ref{lem:translate-sub-operators-to-queries}, and
Proposition~\ref{prop:translating-programs-to-circuits}.

\begin{thm}
  Let $\rho$ is a vocabulary. Let $\setop$ be a set of operators and let $\BB$
  be the associated basis. Every query definable in $\FP(\setop)[\rho]$ is
  definable by a $\PT$-uniform family of transparent symmetric $(\BB,
  \rho)$-circuits $(C_n)_{n \in \nats}$.
  \label{thm:translating-FP-formulas-to-circuits}
\end{thm}
\begin{proof}
  Let $\theta(\vec{x}) \in \FP^{\nats}(\setop)[\rho]$. From
  Lemma~\ref{lem:unroll-fixed-point} there exists a $\PT$-uniform family of
  $\FO^{\nats}(\setop)[\rho]$-substitution programs $\Theta^1 := (\Theta^1_n)_{n
    \in \nats}$ with constant bounds on width and formula-length and such that
  $\Theta^1$ decides the same query as $\theta$. Let $\setquant$ be the set of
  quantifiers corresponding to $\setop$. From
  Lemma~\ref{lem:translate-sub-operators-to-queries} there exists a
  $\PT$-uniform family of $\FO(\setquant)[\rho]$-substitution programs $\Theta^2
  := (\Theta^2_n)_{n \in \nats}$ with a constant bound on width such that
  $\Theta^1$ and $\Theta^2$ define the same query. From
  Proposition~\ref{prop:translating-programs-to-circuits} there is a
  $\PT$-uniform family of transparent symmetric $(\BB, \rho)$-circuits $(C_n)_{n
    \in \nats}$ that decides the same query as $\Theta^2$, and hence the same
  query as $\Theta^1$ and $\theta$. The result follows.
\end{proof}


\end{document}
